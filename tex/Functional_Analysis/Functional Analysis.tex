\documentclass[lang=cn,10pt]{gorgeousnbook}
\usepackage{geometry}
%\geometry{landscape}%,margin=2in,legalpaper,
%标题、题记、作者、组织、日期、版本、邮箱
\title{Functional Analysis}                                  
\subtitle{课程笔记}
\author{Fir1247}
\institute{USTC}
\date{\today}
%\version{1.1}
\bioinfo{联系方式}{fa1247@mail.ustc.edn.cn或者QQ:3105292483}
%封面下方的那行话
\extrainfo{泛函分析}
%目录显示级数
\setcounter{tocdepth}{2}
%封面右下角小方块头像
\logo{logo.jpg}
%封面
\cover{cover.jpg}
%本文档命令
\usepackage{array}
\usepackage{amsmath}
\usepackage[ruled,vlined]{algorithm2e}
\numberwithin{equation}{section}%公式按节编号
\numberwithin{figure}{section}%图表按节编号

\newcommand{\ccr}[1]{\makecell{{\color{#1}\rule{1cm}{1cm}}}}
% 修改标题页的橙色带
\definecolor{customcolor}{RGB}{245, 250, 246}
\colorlet{coverlinecolor}{customcolor}
\usepackage{zhlipsum}
\usepackage{longtable}
%习题集模板
\usepackage{ocgx2}
\usepackage{color}
\usepackage{framed}
\definecolor{shadecolor}{RGB}{241, 241, 255}
\newcounter{exnum}
\newcounter{solnum}
\newenvironment{ex}{\begin{shaded}\stepcounter{exnum}\par\noindent\textbf{题目\arabic{exnum}.}}{\end{shaded}\par}
%注意,ocgx2宏包的切换解答显示对pdf阅读器有很高要求,如果你正在使用的阅读器无法正常使用,推荐Adobe Acrobat DC,或者使用下面这行代码。
%\newenvironment{solve}{\par\noindent\textbf{解答:}\it}{\par}
\newenvironment{solve}{\par\noindent\textbf{解答:}\stepcounter{solnum}\it\switchocg{\arabic{solnum}}{\textbf{\rm Show answer}}\begin{ocg}{Ex}{\arabic{solnum}}{1}\par}{\end{ocg}\par}
%\newenvironment{note}{\par\noindent\textbf{注. }}{\par}
%表格相关
\usepackage{float}
%自己加的宏定义
\def\p{\partial}
\def\d{{\rm d}}
\def\Q{\mathbb{Q}}
\def\R{\mathbb{R}}
\def\C{\mathbb{C}}
\def\K{\mathbb{K}}
\def\F{\mathbb{F}}
\def\E{\mathbb{E}}
\def\N{\mathbb{N}}
\def\Z{\mathbb{Z}}
\def\D{\mathbb{D}}
\def\e{ {\rm e} }
\def\i{ {\rm i} }
\def\L{\mathcal{L}}


\def\defeq{ \mathop{=}\limits^{\rm def} }
\def\eq#1{ \mathop{=}\limits^{#1} }
\def\Ra#1{ \mathop{\Rightarrow}\limits^{#1} }
\def\ve#1{ \textit{\textbf{#1}} }
\def\sve#1{ \boldsymbol{#1} }
\newcommand*{\ag}[1]{
        \langle #1 \rangle
}
\newcommand*{\agl}[2]{
        \langle #1,#2 \rangle
}
\newcommand*{\fun}[2]{
	\mathop{\rm #1}\limits_{#2}
}
\newcommand*{\ms}[1]{
	\left|\left|#1\right|\right|
}
\def\wto{ \mathop{\rightarrow}\limits^{w} }
\def\w*to{ \mathop{\rightarrow}\limits^{w^*} }

\usepackage{ocgx2}

%开始!
\begin{document}
%主字体
\setmainfont{Palatino Linotype}
%目录的封面
\chapterimage{empty.jpg}
%绘制标题
\maketitle
%前言

\frontmatter
\thispagestyle{empty}
\newpage
\begin{center}
	\textbf{\LARGE 前言}
\end{center}

    本文档为中国科学技术大学
	刘聪文老师2023年秋季学期泛函分析课程\footnote{教材:林源渠、张恭庆.泛函分析讲义(上)[M],第二版,北京大学出版社,2021}
	的笔记,主要基于讲义、课堂板书和助教习题课。涵盖范围大致为课本3.3节之前。
	由于课程内容、顺序、侧重点并不和课本完全一致,所以章节和小节标题是按照我个人喜好划分的,
	比如我将有界算子的谱相关内容挪到了第三章,因为和第二章其他内容没什么联系。课本上的习题都会标注课本原题号,
	所有习题答案来自个人解答、网上资料,不能保证完全正确,笔记里也可能会有typo,还望读者斧正。

	泛函分析和实分析是我目前学的最酣畅淋漓的两门数学课,这两门带给我的感受,既不是
	复分析、微分几何的直观之美(以及与考试计算大赛对比带来的的割裂感),也不是统计课对于实际问题的巧妙求解。
	套用李小龙的一句话:“Don't think, feel.”很多定理、题目的证明,都可以靠一个这样的过程:
	要从$A$证明$D$,我先感受条件和结论的联系,从直观上找到可能位于中间的两个$B$和$C$,
	猜想这道题的证明方法就是$A\Rightarrow B\Rightarrow C\Rightarrow D$,
	然后完善证明细节。例如证明$(\ell^p)^*=\ell^q$时,
	难点无非就是怎么构造等距同构,回忆证明$(L^p)^*=L^q$的过程,那时构造的等距同构是函数相乘再积分,
	这和$L^2$上的内积非常像,姑且就叫它“内积”好了。所以猜测$(\ell^p)^*=\ell^q$的等距同构也是通过数列的内积构造的。
	然后再去完善其它细节。实分析很多也有类似的题目,印象比较深刻的比如
	关于紧支函数的那一部分,我就把紧支函数想象成一个“小函数”,
	它在整个实数空间里只占据了有限大的空间,然后用这些“小函数”去逼近全体函数。
	无论是泛函还是实分析,我都没有刻意去掌握做题技巧之类的东西,
	而是整理好整门课程的思路和脉络,最终不仅获得还算不错的成绩,
	也让自己切实感受到有所收获。
	不过,也有可能是我根本没学明白复分析和微分几何(笑),这些都是我个人观点。
	我相信讨厌泛函和实分析、喜欢其他课程的同学不在少数,所以上述讨论仅图一乐,请大家务必不要上纲上线。

	最后是一些格式上的说明:
	非文本类型的字母均为斜体,{\rm i}、{\rm e}等特殊量为正体;
	某些引理的证明会写在引理的框架内部,目的是与定理的证明相区分,单独引理的证明还是会写在框架外面;
	对于步骤很多的证明,暂时没有找到好用的排版方式,目前大部分采用分段然后标注粗体Step;
	${\rm Ran}(A)$、${\rm R}(A)$和${\rm Im}(A)$都代表算子$A$的值域,
	${\rm D}(A)$和${\rm Dom}(A)$代表算子$A$的定义域。
	\begin{flushright}
		最后更新:\today
	\end{flushright}
\frontmatter
%编译目录
\frontmatter
\thispagestyle{fancy}
\tableofcontents
%第一部分
%\partsimage{empty.jpg}
%\part{课程笔记}
\mainmatter	
	%第一章
	\chapterimage{empty.jpg}
	\chapter{度量空间}
%\begin{center}
%    “泛函分析心泛寒”,说明泛函是一门很难的课,但是泛函其实并没有比数分、线代、实分析更难。
%    这是一门很抽象的课,由于前面的课没学好,很多同学学到这里学崩了。泛函是压死骆驼的最后一根稻草。
%\end{center}
%\rightline{——刘聪文老师,2023.09.04}
%\vspace{-5pt}
%\begin{center}
%    \pgfornament[width=0.36\linewidth,color=lsp]{88}
%\end{center}
\section{压缩映射原理}
        \begin{definition}
            $X$是一个非空集合,映射$d:X\times X\rightarrow \mathbb{R}$满足:
            \begin{enumerate}[(1)]
                \item 唯一性:$d(x,y)=0\Leftrightarrow x=y$.
                \item 非负性:$\forall x,y\in X,d(x,y)\geqslant 0$.
                \item 对称性:$\forall x,y\in X,d(x,y)=d(y,x)$.
                \item 三角不等式:$\forall x,y,z\in X,d(x,z)\leqslant d(x,y)+d(y,z)$.
            \end{enumerate}
            则称$d$是$X$上的一个距离函数(度量),$(X,d)$称为一个度量空间,度量空间里的元素称为“点”。

            对于度量空间$X$,$Y\subset X$,
            限制在$Y$上的$d$,记作$d|_Y$,是$Y$上的度量,$(Y,d|_Y)$称为$(X,d)$的子度量空间。
        \end{definition}
        \begin{remark}
            实际上$(2)(3)(4)\Rightarrow (1)$:
            \begin{equation*}
                2d(x,y)=d(x,y)+d(y,x)\geqslant d(x,x)=0
            \end{equation*}
        \end{remark}
        \begin{example}
            $\mathbb{R}^n$上定义:$x=(x_1,\cdots,x_n),y=(y_1,\cdots,y_n)$,对于$1\leqslant p<\infty$,
            \begin{equation*}
                d_p(x,y)\mathop{=}\limits^{\rm def}\sqrt[p]{\sum_{k=1}^n |x_k-y_k|^p}
            \end{equation*}
            是距离函数。

            $p=\infty$,取$d_p(x,y) \mathop{=}\limits^{\rm def} \mathop{\rm max}\limits_{1\leqslant k\leqslant n}|x_k-y_k|$,也是距离函数。
        \end{example}
        \begin{example}
            数组空间
            \begin{equation*}
                \ell^p(\mathbb{F})\mathop{=}\limits^{\rm def}\{ (x_k):x_k\in\mathbb{F},k=1,2,\cdots,\sum_{k=1}^\infty |x_k|^p<\infty \}
            \end{equation*}
            一般$\mathbb{F}$取$\mathbb{R},\mathbb{C}$,对于$1\leqslant p<\infty$,
            \begin{equation*}
                d_p(x,y)\mathop{=}\limits^{\rm def}\sqrt[p]{\sum_{k=1}^\infty |x_k-y_k|^p}
            \end{equation*}
            是距离函数。

            $p=\infty$,
            \begin{equation*}
                \ell^\infty(\mathbb{F})=\{ (x_k):\mathop{\rm sup}_k |x_k|<\infty \}
            \end{equation*}
            \begin{equation*}
                d_\infty(x,y)=\mathop{\rm sup}_k|x_k-y_k|
            \end{equation*}
        \end{example}
        \begin{example}
            离散度量:
            \begin{equation*}
                d(x,y)=\left\{ \begin{array}{ll}0&,x=y\\1&,x\neq y\end{array} \right.
            \end{equation*}
        \end{example}
        \begin{example}
            积度量空间:对于度量空间$(X,d)$,$(Y,\rho)$,
            \begin{equation*}
                X\times Y=\{ (x,y):x\in X,y\in Y \}
            \end{equation*}
            取
            \begin{equation*}
                d_{X\times Y}( (a,b),(c,d) )\mathop{=}\limits^{\rm def}d(a,c)+\rho(b,d)
            \end{equation*}
            是距离函数。
        \end{example}
            约定一些记号:度量空间$(X,d)$上,对于$x_0\in X,r>0$,
            \begin{equation*}
                B(x_0,r)\mathop{=}\limits^{\rm def}
                \{ x\in X:d(x,x_0)<r \}
            \end{equation*}
            称为$x_0$的一个$r$邻域(球形邻域)。此外,记
            \begin{equation*}
                \overline{B}(x_0,r)\mathop{=}\limits^{\rm def}\overline{B(x_0,r)}=\{ x\in X:d(x,x_0)\leqslant r \}
            \end{equation*}
            \begin{equation*}
                S(x_0,r)\mathop{=}\limits^{\rm def}\{ x\in X:d(x,x_0) = r \}
            \end{equation*}
            对于$A\subset X$,
            \begin{equation*}
                {\rm diam}(A)\mathop{=}\limits^{\rm def}\mathop{\rm sup}\limits_{x,y\in A}d(x,y)
            \end{equation*}
            那么如果${\rm diam}(A)$有限,称$A$有界,等价条件为$\exists B(x_0,R)\supset A$.
        \begin{definition}
            度量空间$(X,d)$上,称$\{x_n\}_{n=1}^\infty$收敛,是指存在
            $x_0\in X$使得
            \begin{equation*}
                d(x_n,x_0)\rightarrow 0{\rm\ as\ }n\rightarrow\infty
            \end{equation*}
        \end{definition}
        \begin{corollary}
            度量空间$(X,d)$上的收敛列的极限唯一,且收敛列有界。
        \end{corollary}
        \begin{proof}
            设$\{x_n\}_{n=1}^\infty$是$X$上的收敛列,

            极限唯一:假设$\{x_n\}_{n=1}^\infty$有两个极限$a,b$且$a\neq b$,即
            \begin{equation*}
                d(x_n,a)\rightarrow 0,d(x_n,b)\rightarrow 0
            \end{equation*}
            取$\varepsilon=\frac{1}{2}d(a,b)>0$,存在$N$使得$n>N$时$d(x_n,a)<\varepsilon$,于是
            \begin{align*}
                d(a,b)&\leqslant d(x_n,a)+d(x_n,b)\\
                \Rightarrow d(x_n,b)&\geqslant d(a,b)-d(x_n,a)>\varepsilon-\frac{1}{2}\varepsilon=\frac{1}{2}\varepsilon
            \end{align*}
            这与$d(x_n,b)\rightarrow 0$矛盾。
        
            收敛列有界:设$\{x_n\}_{n=1}^\infty$收敛到$x_0$,$\varepsilon>0$,存在$N$使得$n>N$时
            $d(x_n,x_0)<\varepsilon$,令
            \begin{equation*}
                R_0=\mathop{\rm max}\limits_{1\leqslant n\leqslant N}d(x_n,x_0),R={\rm max}\{ R_0,\varepsilon \}
            \end{equation*}
            则$\{ x_n \}_{n=1}^\infty\subset B(x_0,R)$.            
        \end{proof}
        \begin{example}
            $C[0,1]$为$[0,1]$上连续函数全体,定义度量:
            \begin{equation*}
                d(f,g)\mathop{=}\limits^{\rm def} \mathop{\rm max}\limits_{x\in[0,1]}|f(x)-g(x)|
            \end{equation*}
            于是
            \begin{equation*}
                d(f_n,f)\rightarrow 0\Leftrightarrow f_n\rightrightarrows f
            \end{equation*}
            若定义:
            \begin{equation*}
                \rho_1(f,g)\mathop{=}\limits^{\rm def}
                \int_0^1|f(t)-g(t)|{\rm d}t
            \end{equation*}
            则$\rho_1$也是$C[0,1]$上的一个度量,设
            \begin{equation*}
                f_k(t)=\left\{ \begin{array}{ll}
                    -k^3(t-\frac{1}{k^2})&,t\in [0,\frac{1}{k^2}]\\
                    0&,t\in [\frac{1}{k^2},1]
                \end{array} \right.
            \end{equation*}
            那么$\rho_1(f_k,0)=\frac{1}{2}k\cdot \frac{1}{k^2}\rightarrow 0{\rm\ as\ }k\rightarrow \infty$,
            但是$d(f_k,0)=k\nrightarrow 0$.

            这个例子说明不同度量下的点列的收敛情况可能不同。
        \end{example}
        \begin{definition}
            度量空间$(X,d)$,称$X$中的集合$E$是开集是指:
            $\forall x\in E$,$\exists r>0$使得$B(x,r)\subset E$,即$\forall x\in E$是$E$的内点。

            开集的余集称为闭集。\footnote{实际上这是一般拓扑度量空间上闭集的最原始定义,如果该拓扑是由度量诱导的,
            即所有点都是内点的集合作为开集构成拓扑(见定理1.1.1),则闭集的定义有其他等价表述(见推论1.1.2)}
        \end{definition}
        \begin{theorem}
            记$X$上所有的开集为$\tau$,$\tau$满足拓扑的定义:
            \begin{enumerate}
                \item $\varnothing\in\tau$,$X\in \tau$.
                \item $\tau$对于任意并封闭。
                \item $\tau$对于有限交封闭。
            \end{enumerate}
        \end{theorem}
        \begin{definition}
            度量空间$(X,d)$,$E\subset X$满足:存在$x_0\in X$使得
            \begin{enumerate}
                \item $\forall \varepsilon>0$,$B(x_0,\varepsilon)\cap E\neq \varnothing$,则$x_0$为$E$的接触点;
                \item $\forall \varepsilon>0$,$B(x_0,\varepsilon)\cap (E\backslash \{x_0\})\neq \varnothing$,则$x_0$为$E$的聚点。
            \end{enumerate}
            $E$的接触点全体称为$E$的闭包,记作$\overline{E}$.
        \end{definition}
        \begin{corollary}
            度量空间$(X,d)$,$E\subset X$,下列命题等价:
            \begin{enumerate}
                \item $E$是闭集;
                \item $E=\overline{E}$;
                \item $\forall \{x_n\}_{n=1}^\infty\subset E$,如果$x_n\rightarrow x_0$,则$x_0\in E$.
            \end{enumerate}
        \end{corollary}
        \begin{proof}
            证明:
            $(1)\Rightarrow (2)$:$E$闭$\Rightarrow E^c$开,如果存在
            $x_0\in \overline{E}$且$x_0\notin E$,则$x_0\in E^c$,存在邻域$B=B(x_0,\varepsilon_0)\subset E^c\Rightarrow B\cap E=\varnothing$,
            这与$x_0$是接触点矛盾,故$\overline{E}\subset E$,进而$\overline{E}=E$.
        
            $(2)\Rightarrow (1)$:$\overline{E}=E\Rightarrow \forall y\in E^c,y\notin \overline{E}\Rightarrow $存在邻域$B(y,\varepsilon)\cap E=\varnothing\Rightarrow
            B(y,\varepsilon)\subset E^c\Rightarrow E^c$开$\Rightarrow E$闭。
        
            $(2)\Rightarrow (3)$:任取收敛列$\{x_n\}_{n=1}^\infty\subset E$,$x_n\rightarrow x_0$,则$\forall \varepsilon>0$,存在
            $N$使得$n>N$时$x_n\in B(x_0,\varepsilon)$,于是
            $x_{N+1}\in B(x_0,\varepsilon)\cap E\neq \varnothing\Rightarrow x_0\in \overline{E}$.
        
            $(3)\Rightarrow (2)$:设$x_0\in \overline{E}$,取$\varepsilon_1>0$,存在$x_1\in B(x_0,\varepsilon_1)\cap E$;
            取$0<\varepsilon_2<\varepsilon_1$使得
            存在$x_2\in B(x_0,\varepsilon_2)\cap E$且$x_2\neq x_1$;以此类推,并可以要求$\varepsilon_n\rightarrow 0$,否则
            $B(x_0,\mathop{\rm inf}\limits_{n}\{\varepsilon_n\} )\cap E=\varnothing$,这与$x_0\in\overline{E}$矛盾。最后得到点列$\{x_n\}$,且
            $x_n\rightarrow x_0$,由$(3)$可知$x_0\in E$,所以$\overline{E}\subset E$,进而$\overline{E}=E$.
        \end{proof}

        \begin{definition}
            度量空间$(X,d)$,$E\subset X$,$x_0\in E$,如果
            $\forall \varepsilon>0$,$ B(x_0,\varepsilon)\backslash\{x_0\}\cap E=\varnothing $,
            等价于存在点列$\{x_n\}\subset E$使得$x_n\rightarrow x_0$,则称
            $x_0$是$E$的聚点或者极限点。

            记$E'$为$E$的聚点全体,称为$E$的导集。

            记$\overline{E}=E\cup E'$,称为$E$的闭包。如果$\overline{E}=X$,称$E$在$X$中稠密,记作$E\mathop{\subset}\limits^{\rm dense} X$或者$E\subset\subset X$.
            如果$X$有一个可数稠密子集,称$X$可分。
        \end{definition}

        \begin{example}
            $\mathbb{Q}\mathop{\subset}\limits^{\rm dense}\mathbb{R}$,
            多项式全体$P[a,b]\mathop{\subset}\limits^{\rm dense} C[a,b]$.
        \end{example}

        \begin{example}
            $C[a,b]$可分。
        \end{example}
        \begin{proof}
            记$Q[a,b]$为$[a,b]$上全体有理系数多项式,这是一个可数集;$P[a,b]$是$[a,b]$上全体实系数多项式。设
            \begin{equation*}
                r(x)=r_nx^n+\cdots+r_0\in P[a,b],\forall r_i\in \mathbb{R}
            \end{equation*}
            因为$\mathbb{Q}\mathop{\subset}\limits^{\rm dense}\mathbb{R}$,$\forall \varepsilon>0$,对于每个$r_i$,存在
            $q_i\in \mathbb{Q}{\rm\ s.t.\ }|r_i-q_i|<\varepsilon$,于是令
            \begin{equation*}
                q(x)=q_nx^n+\cdots+q_0\in Q[a,b]
            \end{equation*}
            \begin{align*}
                \Rightarrow d(r,q)&=\mathop{\rm sup}\limits_{x\in [a,b]}
                | (r_n-q_n)x^n+\cdots+(r_0-q_0) |\\
                &\leqslant \mathop{\rm sup}\limits_{x\in [a,b]}\varepsilon(|x^n|+\cdots+|x|+1)\\
                &=C\varepsilon\rightarrow 0{\rm\ as\ }\varepsilon\rightarrow 0
            \end{align*}
            因此$Q[a,b]\mathop{\subset}\limits^{\rm dense}P[a,b]$,而$P[a,b]\mathop{\subset}\limits^{\rm dense}C[a,b]$,$\forall f\in C[a,b]$,设
            $d(r,f)<\frac{1}{2}\varepsilon$,$d(q,r)<\frac{1}{2}\varepsilon$,
            \begin{equation*}
                d(q,f)\leqslant d(r,f)+d(q,r)<\varepsilon
            \end{equation*}
            所以$C[a,b]$有可数稠密子集$Q[a,b]$.
        \end{proof}

        \begin{definition}
            $(X,d)$,$(Y,\rho)$是两个度量空间,设映射:
            \begin{equation*}
                T:X\rightarrow Y
            \end{equation*}
            在$x_0\in X$处连续是指:$\forall \varepsilon>0$,存在$\delta>0$使得
            \begin{equation*}
                d(x,x_0)<\delta\Rightarrow \rho(T(x),T(x_0))<\varepsilon
            \end{equation*}
            如果$T$在$X$中的每一点都连续,则称$T$是连续映射。
        \end{definition}

        \begin{theorem}
            $(X,d)$,$(Y,\rho)$是两个度量空间,
            映射$ T:X\rightarrow Y $连续$\Leftrightarrow $任取$Y$上的开集$U$,
            $T^{-1}(U)$是$X$上的开集。
        \end{theorem}
        \begin{proof}
            记$B_X(x_0,r)=\{ x\in X|d(x,x_0)<r \}$,$B_Y(y_0,r)=\{ y\in Y|\rho(y,y_0)<r \}$.

            $(\Rightarrow )$:$\forall y_0\in U$,$B_Y(y_0,\varepsilon)\subset U$,设$x_0\in T^{-1}(U)$,
            因为$T$连续,存在$\delta>0$使得$x\in B_X(x_0,\delta)\Rightarrow T(x)\in B_Y(y_0,\varepsilon)\subset U$,所以
            $B_X(x_0,\delta)\subset T^{-1}(U)$,由于$y_0$的任意性,$x_0$能取遍整个$T^{-1}(U)$,故$T^{-1}(U)$为开集。
        
            $(\Leftarrow)$:对于$x_0\in X$,设$y_0=T(x_0)$,$U=B_Y(y_0,\varepsilon)$是$Y$上的开集,
            且$x_0\in T^{-1}(U)$,所以存在$B_X(x_0,\delta)\subset T^{-1}(U)$,则
            $x\in B_X(x_0,\delta)\Rightarrow T(x)\in U=B_Y(y_0,\varepsilon)$.由$\varepsilon$的任意性,$T$是连续映射。
        \end{proof}

        \begin{theorem}[Heine]
            $T$在$x_0$处连续$\Leftrightarrow$
            任取$X$上收敛到$x_0$的点列$\{x_n\}$,都有$T(x_n)\rightarrow T(x_0)$.
        \end{theorem}
        \begin{proof}
            $(\Rightarrow )$:$\forall \varepsilon>0,\exists \delta>0{\rm\ s.t.\ }x\in B_X(x_0,\delta)\Rightarrow T(x)\in B_Y(y_0,\varepsilon)$,
            存在$N$使得$n>N$时$x_n\in B_X(x_0,\delta)\Rightarrow T(x_n)\in B_Y(T(x_0),\varepsilon)\Rightarrow T(x_n)\rightarrow T(x_0)$.
        
            $(\Leftarrow)$:假设$T$在$x_0$处不连续,即存在$\varepsilon>0$,$\forall \delta>0$,存在$x\in B_X(x_0,\delta){\rm\ s.t.\ }T(x)\notin B_Y(y_0,\varepsilon)$,
            取$x_n\rightarrow x_0$,每个$T(x_n)\notin B_Y(y_0,\varepsilon)$,则$T(x_n)\nrightarrow T(x_0)$,矛盾。
        \end{proof}

        \begin{definition}
            度量空间$(X,d)$,称$\{x_n\}$是一个基本列(或者叫Cauthy列)是指:
            $\forall \varepsilon>0$,存在$N$使得:
            \begin{equation*}
                d(x_m,x_n)<\varepsilon,\forall m,n\geqslant N
            \end{equation*}
            如果$(X,d)$中任意基本列都收敛,则称$(X,d)$完备。完备的度量空间称为Banach空间,
        \end{definition}

        \begin{example}
            $(\mathbb{R},d)$完备,$(\mathbb{Q},d)$不完备;$L^p[0,1]$完备。
        \end{example}
        \begin{example}
            例1.1.5中的$(C[0,1],d)$完备。
        \end{example}
        \begin{proof}
            设$\{f_n\}_{n=1}^\infty$是$C[0,1]$中任一基本列,于是
            \begin{equation*}
                \forall \varepsilon,\exists N{\rm\ s.t.\ }
                \mathop{\rm max}\limits_{t\in[0,1]}|f_m(t)-f_n(t)|<\varepsilon,\forall m,n\geqslant N
            \end{equation*}
            于是对于每个固定的$t\in [0,1]$,$|f_m(t)-f_n(t)|<\varepsilon$,从而可知
            $\{f_n(t)\}_{n=1}^\infty$是$\mathbb{R}$中的基本列,于是存在极限$f(t)=\mathop{\rm lim}\limits_{n\rightarrow\infty}f_n(t)$.
            在(1.1.1)式中令$m\rightarrow\infty$,于是
            \begin{align*}
                &\mathop{\rm max}\limits_{t\in[0,1]}|f_m(t)-f(t)|\leqslant \varepsilon,\forall n\geqslant N\\
                \Rightarrow &f_n\rightrightarrows f\\
                \Rightarrow &f\in C[0,1],d(f_n,f)\rightarrow 0
            \end{align*}
        \end{proof}

        \begin{example}
            例1.1.5中的$(C[0,1],\rho_1)$不完备。
        \end{example}
        \begin{proof}
            令:
            \begin{equation*}
                f_n(t)\mathop{=}\limits^{\rm def}
                \left\{ \begin{array}{ll}
                    0&,t\in[0,\frac{1}{2}-\frac{1}{n}]\\
                    nt-\frac{n}{2}+1&,t\in [\frac{1}{2}-\frac{1}{n},\frac{1}{2}]\\
                    1&,t\in[\frac{1}{2},1]
                \end{array} \right.
            \end{equation*}
            于是
            \begin{equation*}
                \rho_1(f_n,f_m)=\frac{1}{2}\left| \frac{1}{n}-\frac{1}{m} \right|\rightarrow 0{\rm\ as\ }n,m\rightarrow\infty
            \end{equation*}
            因此$\{f_n\}$是基本列。下面证明它不收敛,假设存在$f\in C[0,1]$使得
            $\rho_1(f_n,f)\rightarrow 0$,
            \begin{align*}
                \rho_1(f_n,f)&=\int_0^1 |f_n(t)-f(t)|{\rm d}t\\
                &=\int_0^{\frac{1}{2}-\frac{1}{n}}|f(t)|{\rm d}t+
                \int_{\frac{1}{2}-\frac{1}{n}}^{\frac{1}{2}}
                |f_n(t)-f(t)|{\rm d}t+\int_{\frac{1}{2}}^1 |1-f(t)|{\rm d}t
            \end{align*}
            令$n\rightarrow\infty$可得
            \begin{equation*}
                \int_0^{\frac{1}{2}}|f(t)|{\rm d}t+\int_{\frac{1}{2}}^1 |1-f(t)|{\rm d}t=0
            \end{equation*}
            所以
            \begin{equation*}
                f(t)=\left\{ \begin{array}{ll}
                    0&,t\in (0,\frac{1}{2})\\
                    1&,t\in (\frac{1}{2},1)
                \end{array} \right.
            \end{equation*}
            和$f$连续矛盾。
        \end{proof}

        \begin{example}
            离散度量空间完备。
        \end{example}
        \begin{proof}
            任取$X$上的柯西列$\{x_n\}_{n=1}^\infty$,取$\varepsilon=\frac{1}{2}$,
            存在$N$使得$\forall n,m>N$,$d(x_n,x_m)<\frac{1}{2}\Rightarrow d(x_n,x_m)=0$,即
            $\forall n>N,x_n=x_{N+1}$,所以$x_n\rightarrow x_{N+1}\in X$.故离散度量空间完备。
        \end{proof}

        \begin{definition}
            度量空间$(X,d)$,映射$T:X\rightarrow X$,如果存在
            $x^*\in X$使得$T(x^*)=x^*$,则称$x^*$是$T$的一个不动点。

            如果存在$\alpha\in (0,1)$,使得
            \begin{equation*}
                d(T(x),T(y))\leqslant \alpha d(x,y),\forall x,y\in X
            \end{equation*}
            则称$T$是一个压缩映射。
        \end{definition}
        \begin{theorem}[Banach不动点定理、压缩映射原理]
            完备度量空间到自身的压缩映射有唯一的不动点。
        \end{theorem}
        \begin{proof}
            存在性:这个证明方法叫做Picard迭代,任取$x_0\in X$,定义迭代序列:
            \begin{equation*}
                x_{n+1}=T(x_n),n=0,1,\cdots
            \end{equation*}
            \begin{align*}
                \Rightarrow d(x_{n+1},x_n)&=d( T(x_n),T(x_{n-1}) )\\
                &\leqslant \alpha d(x_n,x_{n-1})=\alpha d( T(x_{n-1}),T(x_{n-2}) )\\
                &\leqslant \alpha^2 d(x_{n-1},x_{n-2})\\
                &\leqslant\cdots\leqslant \alpha^n d(x_1,x_0)
            \end{align*}
            利用三角不等式,
            \begin{align*}
                \Rightarrow d( x_{n+p},x_n )&\leqslant \sum_{k=1}^p d(x_{n+k},x_{n+k-1})\\
                &\leqslant \sum_{k=1}^p \alpha^{n+k-1}d(x_1,x_0)\\
                &\leqslant \frac{\alpha^n}{1-\alpha}d(x_1,x_0)<\varepsilon,n\mbox{充分大时},\forall p
            \end{align*}
            \begin{align*}
                \Rightarrow& \{x_n\}_{n=1}^\infty \mbox{是$X$中的基本列}\\
                X\mbox{完备}\Rightarrow& \exists x^*\in X{\rm\ s.t.\ }d(x_n,x^*)\rightarrow 0{\rm\ as\ }n\rightarrow\infty\\
                \Rightarrow& d( T(x^*),x^* )\leqslant d( T(x^*),T(x_n) )+d( T(x_n),x_n )+
                d(x_n,x^*)\\
                &\leqslant \alpha d(x^*,x_n)+d(x_{n+1},x_n)+d(x_n,x^*)\rightarrow 0{\rm\ as\ }n\rightarrow\infty\\
                \Rightarrow& T(x^*)=x^*
            \end{align*}

            唯一性:设$y^*$是另一个不动点,则
            \begin{equation*}
                d(x^*,y^*)=d(T(x^*),T(y^*))\leqslant \alpha d(x^*,y^*)
            \end{equation*}
            因此只能$d(x^*,y^*)=0\Rightarrow x^*=y^*$.
        \end{proof}
        \begin{example}
            完备的条件不可去,例如$X=(0,1)$不完备,
            度量$d(x,y)=|x-y|$,压缩映射$T(x)=\frac{1}{2}x$,无不动点。
        \end{example}
    
\section{完备化}
        \begin{definition}
            $(X_1,d_1)$和$(X_2,d_2)$是两个度量空间,如果映射$T:X_1\rightarrow X_2$保持距离不变,即
            \begin{equation*}
                d_1(x,y)=d_2(T(x),T(y)),\forall x,t\in X_1
            \end{equation*}
            则称$T$是等距映射。如果存在从$X_1$到$X_2$的既单又满的等距映射,则称
            $(X_1,d_1)$和$(X_2,d_2)$是等距同构的。
        \end{definition}
        \begin{definition}
            如果$(X_1,d_1)$和$(X_2,d_2)$的某个子空间等距同构,则称
            $(X_1,d_1)$可等距嵌入$(X_2,d_2)$,记作
            \begin{equation*}
                (X_1,d_1) \hookrightarrow (X_2,d_2)
            \end{equation*}
            在此意义下,称$X_1$是$X_2$的子空间。
        \end{definition}
        \begin{definition}
            对于度量空间$(X,d)$,如果存在完备的度量空间
            $(\tilde{X},\tilde{d})$,它的某个稠密子空间$X_0$和$X$等距同构,则称
            $(\tilde{X},\tilde{d})$是$(X,d)$的一个完备化。
        \end{definition}
        \begin{example}
            \begin{enumerate}
                \item $\mathbb{R}$是$\mathbb{Q}$的完备化。
                \item $L^1[a,b]$是$(C[a,b],\rho_1)$的完备化。
                \item $C[a,b]$是$( P[a,b],d )$的完备化。
            \end{enumerate}
        \end{example}
        \begin{theorem}
            任何度量空间都有完备化,且完备化在等距同构意义下唯一,
            即如果$(\tilde{X},\tilde{d})$和$(X',d')$都是$(X,d)$的完备化,则二者等距同构。
        \end{theorem}
        我们的证明思路是:
        \begin{enumerate}
            \item 构造$(\tilde{X},\tilde{d})$.
            \item 构造稠密子空间$(X_0,\tilde{d})$和等距同构。
            \item 证明$(\tilde{X},\tilde{d})$完备。
            \item 证明等距同构意义下的唯一性。
        \end{enumerate}
        \begin{proof}
            \begin{enumerate}
                \item $\mathcal{F}$定义为$(X,d)$上基本列全体,记$\xi=\{x_n\}_{n=1}^\infty$,
                $\eta=\{y_n\}_{n=1}^\infty$,
                在$\mathcal{F}$上引入等价关系:
                \begin{equation*}
                    \xi\sim\eta\mathop{\Leftrightarrow}\limits^{\rm def}
                    \mathop{\rm lim}\limits_{n\rightarrow\infty}d(x_n,y_n)=0
                \end{equation*}
                $\tilde{X}\mathop{=}\limits^{\rm def}\mathcal{F}/_\sim$,定义
                $\tilde{X}$上的度量:
                \begin{equation*}
                    \tilde{d}( [\xi],[\eta] )
                    \mathop{=}\limits^{\rm def}
                    \mathop{\rm lim}\limits_{n\rightarrow\infty}d(x_n,y_n)
                \end{equation*}
                这里$\{x_n\}_{n=1}^\infty$是$[\xi]$中任一代表元,
                $\{y_n\}_{n=1}^\infty$是$[\eta]$中任一代表元,
                    \begin{enumerate}
                        \item $\tilde{d}$良定:\begin{enumerate}
                            \item $\mathop{\rm lim}\limits_{n\rightarrow\infty}d(x_n,y_n)$存在:由三角不等式:
                            \begin{align*}
                                | d(x_n,y_n)-d(x_m,y_m) |=&| d(x_n,y_n)-d(y_n,x_m)+d(x_m,y_n)-d(x_m,y_m) |\\
                                \leqslant & |d(x_n,y_n)-d(y_n,x_m)|+|d(x_m,y_n)-d(x_m,y_m)|\\
                                \leqslant & d(x_n,x_m)+d(y_n,y_m) \rightarrow 0{\rm\ as\ }n,m\rightarrow\infty
                            \end{align*}
                            所以$\{ d(x_n,y_n) \}_{n=1}^\infty$是$\mathbb{R}$中基本列,再由$\mathbb{R}$的完备性可知极限存在。
                        \item $\tilde{d}([\xi],[\eta])$不依赖于$[\xi],[\eta]$的代表元选取:
                            设$\xi^{(1)}=\{ x_n^{(1)} \}_{n=1}^\infty,\xi^{(2)}=\{ x_n^{(2)} \}_{n=1}^\infty\in[\xi]$,
                            根据定义有$\mathop{\rm lim}\limits_{n\rightarrow\infty}d( x_n^{(1)}, x_n^{(2)} )=0$,
                            \begin{align*}
                                &|d(x_n^{(1)},y_n)-d( x_n^{(2)},y_n )|\leqslant d(x_n^{(1)},x_n^{(2)})\rightarrow 0{\rm\ as\ }n\rightarrow\infty\\
                                \Rightarrow &
                                \mathop{\rm lim}\limits_{n\rightarrow\infty}d(x_n^{(1)},y_n)=\mathop{\rm lim}\limits_{n\rightarrow\infty}d(x_n^{(2)},y_n)
                            \end{align*}
                            $[\eta]$同理。
                        \end{enumerate}
                        \item $\tilde{d}$是度量:平凡。
                    \end{enumerate}
                \item 对$x\in X$,记$\xi_x\mathop{=}\limits^{\rm def}(x,x,\cdots)$,称为常驻点列,当然也是一个基本列。设
                    \begin{equation*}
                        X_0\mathop{=}\limits^{\rm def}
                        \{ [\xi_x]:x\in X \}\subset\tilde{X}
                    \end{equation*}
                    设$T:X\rightarrow X_0,x\mapsto [\xi_x]$,则$T$是
                    $X$到$X_0$的等距同构。任取$[\xi]\in \tilde{X}$,任取代表元
                    $\{x_n\}_{n=1}^\infty$,根据$\tilde{d}$的定义有(常驻点列$\xi_{x_n}$的第$k$项是$x_n$,点列$\xi$的第$k$项为$x_k$)
                    \begin{equation*}
                        \mathop{\rm lim}\limits_{n\rightarrow\infty}
                        \tilde{d}( [\xi_{x_n}],[\xi] )
                        =\mathop{\rm lim}\limits_{n\rightarrow\infty}
                        \mathop{\rm lim}\limits_{k\rightarrow\infty}
                        d(x_n,x_k)
                    \end{equation*}
                    而$\xi=\{x_n\}_{n=1}^\infty$是基本列,所以上式等于$0$,即
                    柯西列$[\xi_{x_n}]\rightarrow [\xi]$,可得$\overline{X_0}=\tilde{X}$.
                \item $(\tilde{X},\tilde{d})$完备:设$ \{ [\xi^{(k)}] \}_{k=1}^\infty $是
                    $(\tilde{X},\tilde{d})$中任一基本列,这里$\xi^{(k)}=
                    \{x_n^{(k)}\}_{n=1}^\infty$,
                    由上一步的结论,取常驻点列$\xi_{x_{n}^{(k)}}=(x_{n}^{(k)},\cdots)$,
                    于是对每个$k$,$[\xi_{x_{n}^{(k)}}]\rightarrow [\xi^{(k)}]$
                    取充分大的$n_k$使得:
                    \begin{equation*}
                        \tilde{d}( [\xi^{(k)}],[\xi_{x_{n_k}^{(k)}}] )<\frac{1}{k}
                    \end{equation*}
                    可得
                    \begin{align*}
                        \tilde{d}( [\xi_{x_{n_k}^{(k)}}],[\xi_{x_{n_j}^{(j)}}] )
                        &\leqslant \tilde{d}( [\xi_{x_{n_k}^{(k)}}],[\xi^{(k)}] )
                        +\tilde{d}( [\xi^{(k)}],[\xi^{(j)}] )+
                        \tilde{d}( [\xi^{(j)}],[\xi_{x_{n_j}^{(j)}}] )\\
                        &< \frac{1}{k}+\tilde{d}( [\xi^{(k)}],[\xi^{(j)}] )
                        +\frac{1}{j}\rightarrow 0{\rm\ as\ }j,k\rightarrow\infty
                    \end{align*}
                    令$\xi'\mathop{=}\limits^{\rm def}
                    \{ x_{n_k}^{(k)} \}_{k=1}^\infty \in \mathcal{F}$,于是$[\xi']\in\tilde{X}$,
                    根据$\tilde{d}$的定义有(常驻点列$\xi_{x_{n_k}^{(k)}}$的第$j$项是$\xi_{x_{n_k}^{(k)}}$,
                    点列$\xi'$的第$j$项为$x_{n_j}^{(j)}$)
                    \begin{equation*}
                        \mathop{\rm lim}\limits_{k\rightarrow\infty}
                        \tilde{d}( [\xi_{x_{n_k}^{(k)}}],[\xi'] )
                        =\mathop{\rm lim}\limits_{k\rightarrow\infty}
                        \mathop{\rm lim}\limits_{j\rightarrow\infty}
                        d(x_{n_k}^{(k)},x_{n_j}^{(j)})=0
                    \end{equation*}
                    最后
                    \begin{equation*}
                        \tilde{d}( [\xi^{(k)}],[\xi'] )
                        \leqslant \mathop{\tilde{d}( [\xi^{(k)}],[\xi_{x_{n_k}^{(k)}}] )}\limits_{<\frac{1}{k}\rightarrow 0}
                        +\mathop{\tilde{d}( [\xi'],[\xi_{x_{n_k}^{(k)}}] )}\limits_{\rightarrow 0}
                    \end{equation*}
                    所以$[\xi^{(k)}]\rightarrow [\xi']$,完备性得证。
                \item 唯一性:
                设$(X',d')$也是$(X,d)$的完备化,即$(X,d)$等距同构于
                $(X',d')$的一个稠密子空间$(X_0',d')$,设
                $T':X\rightarrow X_0'$是等距同构,则
                $\varphi=T'\circ T^{-1}$是$(X_0,\tilde{d})$
                到$(X_0',d')$的等距同构,下面把$\varphi$延拓为
                $(\tilde{X},\tilde{d})$到$(X',d')$的等距同构。
                \begin{equation*}
                    \forall [\xi]\in \tilde{X}
                    ,\exists [\xi^n]\in X_0,n=1,2,\cdots{\rm\ s.t.\ }
                    \tilde{d}( [\xi^n],[\xi] )\rightarrow 0\mbox{(由稠密性)}
                \end{equation*}
                因为$\varphi$是等距映射,所以$\varphi( [\xi^{(n)}] )$是$(X',d')$中的基本列。$X'$完备,所以
                存在$y\in X'$使得$d'(\varphi[\xi^{(n)}],y)\rightarrow 0$,定义映射:
                $\Phi:\tilde{X}\rightarrow X',[\xi]\mapsto y$,接下来验证$\Phi$是等距同构(作业):

                任取$[\xi^{(1)}],[\xi^{(2)}]\in\tilde{X}$,设$\Phi([\xi^{(1)}])=y_1,\Phi([\xi^{(2)}])=y_2$,
                \begin{align*}
                    &d'(T'(x_n^{(1)}),T'(x_n^{(2)}))-d'(y_1,y_2)\\
                    &\leqslant ( d'(T'(x_n^{(1)}),y_2)+d'(y_2,T'(x_n^{(2)})) )-( d(T'(x_n^{(1)}),y_2)-d'(T'(x_n^{(1)}),y_1) )\\
                    &\leqslant d'(T'(x_n^{(1)}),y_1)+d'(T'(x_n^{(2)}),y_2)\rightarrow 0
                \end{align*}
                而$d'(T'(x_n^{(1)}),T'(x_n^{(2)}))=\tilde{d}( [\xi_{x_n^{(1)}}],[\xi_{x_n^{(2)}}] )$,同理
                \begin{align*}
                    \tilde{d}( [\xi_{x_n^{(1)}}],[\xi_{x_n^{(2)}}] )-\tilde{d}( [\xi^{(1)}],[\xi^{(2)}] )
                    \leqslant \tilde{d}( [\xi_{x_n^{(1)}}],[\xi^{(1)}] )+\tilde{d}( [\xi_{x_n^{(2)}}],[\xi^{(2)}] )\rightarrow 0
                \end{align*}
                因此
                \begin{equation*}
                    \tilde{d}( [\xi^{(1)}],[\xi^{(2)}] )=\mathop{\rm lim}\limits_{n\rightarrow\infty}\tilde{d}( [\xi_{x_n^{(1)}}],[\xi_{x_n^{(2)}}] )
                    =\mathop{\rm lim}\limits_{n\rightarrow\infty}d'(T'(x_n^{(1)}),T'(x_n^{(2)}))
                    =d'(y_1,y_2)
                \end{equation*}
                这就证明了$\Phi$是等距同构。
            \end{enumerate}
        \end{proof}
    
\section{紧性推理}
        \begin{definition}
            度量空间$(X,d)$,$A\subset X$,
            \begin{enumerate}
                \item 如果一族开集$\{G_\alpha\}_{\alpha\in \Lambda}$使得
                    \begin{equation*}
                        \bigcup_{\alpha\in\Lambda}G_\alpha\supset A
                    \end{equation*}
                    则称$\{G_\alpha\}_{\alpha\in \Lambda}$是$A$的一个开覆盖。
                \item 如果$A$的任一开覆盖$\{G_\alpha\}_{\alpha\in \Lambda}$都有有限子覆盖,即
                    存在$\alpha_1,\alpha_2,\cdots,\alpha_N\in \Lambda$使得
                    \begin{equation*}
                        \bigcup_{k=1}^N G_{\alpha_k}\supset A
                    \end{equation*}
                    则称$A$紧。
                \item 如果$A$中任一点列都有在$X$中收敛的子列,则称$A$列紧。
                \item 如果$A$中任一点列都有在$A$中收敛的子列,则称$A$自列紧。
                \item 如果空间$X$自身列紧,则称$X$为列紧空间。
            \end{enumerate}
        \end{definition}
        \begin{corollary}
            度量空间上,自列紧集等价于列紧闭集。
        \end{corollary}
        \begin{example}
            $\mathbb{R}^n$中,列紧集等价于有界集,自列紧集等价于有界闭集等价于紧集。
        \end{example}
        \begin{example}
            一般度量空间中,有界集不一定列紧,如无穷维线性空间和欧式度量构成的度量空间,设
            $e_n$为第$n$个分量为$1$,其余为$0$的向量,无穷点列$\{e_n\}_{n=1}^\infty$是有界的,
            但是$d(e_n,e_m)=\sqrt{2},\forall n\neq m$,无收敛子列。
        \end{example}
        \begin{proposition}
            列紧空间中任一集合都列紧,任一闭集都自列紧。
        \end{proposition}
        \begin{proposition}
            列紧空间一定完备。
        \end{proposition}
        \begin{proof}
            设$\{x_n\}_{n=1}^\infty$是$(X,d)$中基本列,
            $(X,d)$列紧$\Rightarrow $有子列$x_{n_k}\rightarrow x_0\in X$,
            \begin{equation*}
                \Rightarrow d(x_n,x_0)\leqslant d(x_n,x_{n_k})+d(x_{n_k},x_0)\rightarrow 0{\rm\ as\ }n,k\rightarrow\infty
            \end{equation*}
        \end{proof}
        \begin{definition}
            度量空间$(X,d)$,$A\subset X$,$\varepsilon>0$,
            \begin{enumerate}
                \item 称$N_\varepsilon\subset A$是$A$的一个$\varepsilon$网是指\begin{equation*}
                    \forall x\in A,\exists y\in N_\varepsilon{\rm\ s.t.\ }d(x,y)<\varepsilon
                \end{equation*}
                等价于
                \begin{equation*}
                    A\subset \bigcup_{y\in N_\varepsilon}B(y,\varepsilon)
                \end{equation*}
                即由半径为$\varepsilon$的开球组成的$A$的开覆盖。
                \item 如果$\forall \varepsilon>0$,都有$A$的一个元素有限的$\varepsilon$网,则称$A$完全有界。即可以选取有限个半径为$\varepsilon$的开球作为$A$的开覆盖。
            \end{enumerate}
        \end{definition}
        \begin{proposition}
            完全有界$\Rightarrow $有界
        \end{proposition}
        \begin{proof}
            有$A$的有限$1$网$N_1:=\{ y_1,\cdots,y_m \}$,于是令
            $R=\sum_{k=2}^m d(y_k,y_1)+1$,则
            \begin{equation*}
                A\subset \bigcup_{k=1}^m B(y_k,1)\subset B(y_1,R)
            \end{equation*}
        \end{proof}
        \begin{example}
            有界并不一定完全有界,考虑例$1.3.2$,
            $A=\{e_n\}_{n=1}^\infty$没有有限的$\frac{1}{2}$网,因为每个$\frac{1}{2}$球只能覆盖球心。
        \end{example}
        \begin{theorem}[Hausdorff]
            \begin{enumerate}
                \item 列紧$\Rightarrow$完全有界
                \item 完备度量空间中,列紧$\Leftrightarrow $完全有界。
            \end{enumerate}
        \end{theorem}
        \begin{proof}
            \begin{enumerate}
                \item 假设$A$列紧不完全有界,即存在$\varepsilon_0>0$使得有限个
                    半径为$\varepsilon_0$的球不能覆盖$A$,按照以下方式选取出一列点:
                    \begin{align*}
                        x_1&\in A\\
                        x_2&\in A\backslash B(x_1,\varepsilon_0)\\
                        x_3&\in A\backslash \bigcup_{k=1}^2 B( x_k,\varepsilon_0 )\\
                        &\vdots\\
                        x_n&\in A\backslash \bigcup_{k=1}^n B( x_k,\varepsilon_0 )\\
                        &\vdots
                    \end{align*}
                    那么序列$\{x_n\}_{n=1}^\infty \subset A$使得
                    \begin{equation*}
                        x_n\notin \bigcup_{i=1}^{n-1}B(x_i,\varepsilon_0)
                    \end{equation*}
                    因此$d(x_n,x_m)\geqslant \varepsilon_0,\forall n\neq m$,说明$A$并不列紧,矛盾。
                \item 只需证明:$X$完备并且$A$完全有界$\Rightarrow A$列紧。设$\{x_n\}_{n=1}^\infty\subset A$,
                    对于$\varepsilon=1$,有$A$的有限$1$网$N_1=\{ y_1^{(1)},\cdots,y_{m_1}^{(1)} \}$,于是
                    \begin{equation*}
                        \{ x_n \}_{n=1}^\infty\subset A\subset \bigcup_{k=1}^{m_1} B( y_k^{(1)},1 )
                    \end{equation*}
                    因此,存在某个$k$使得$B(y_k^{(1)},1)$包含$\{ x_n \}_{n=1}^\infty$中的无穷多项,记作$\{ x_n^{(1)} \}_{n=1}^\infty$,
                    并记$y_k^{(1)}=y^{(1)}$.
                    同理,$\exists y^{(2)}\in N_{\frac{1}{2}}$使得有$\{ x_n^{(1)} \}_{n=1}^\infty$的子列
                    $\{x_n^{(2)}\}_{n=1}^\infty\subset B(y^{(2)},\frac{1}{2})$,以此类推:
                    \begin{equation*}
                        \begin{matrix}
                            x_1^{(1)}&,x_2^{(1)}&,x_3^{(1)}&,\cdots&\in B(y^{(1)},1)\\
                            x_1^{(2)}&,x_2^{(2)}&,x_3^{(2)}&,\cdots&\in B(y^{(2)},\frac{1}{2})\\
                            \vdots&\vdots&\vdots&&\\
                            x_1^{(n)}&,x_2^{(n)}&,x_3^{(n)}&,\cdots&\in B(y^{(n)},\frac{1}{n})\\
                            \vdots&\vdots&\vdots&&
                        \end{matrix}
                    \end{equation*}
                    而且
                    \begin{equation*}
                        \{ x_n^{(1)} \}_{n=1}^\infty\supset \{ x_n^{(2)} \}_{n=1}^\infty\supset\cdots\supset\{ x_n^{(n)} \}_{n=1}^\infty\supset \cdots
                    \end{equation*}
                    取对角线子列:
                    \begin{equation*}
                        x_n^{(n)}\in \bigcap_{k=1}^n B(y^{(k)},\frac{1}{k}),n=1,2,\cdots
                    \end{equation*}
                    所以$\forall n,p$,由于$x_{n+p}^{(n+p)},x_n^{(n)}\in B(y^{(n)},\frac{1}{n})$,因此
                    \begin{equation*}
                        d( x_{n+p}^{(n+p)},x_n^{(n)} )\leqslant \frac{2}{n}\rightarrow 0
                    \end{equation*}
                    所以$\{x_n^{(n)}\}$是基本列,由$X$完备可知$\{x_n^{(n)}\}$收敛,这就是$\{x_n\}_{n=1}^\infty$的收敛子列,于是$A$列紧。
            \end{enumerate}
        \end{proof}
        \begin{theorem}
            度量空间中,紧$\Leftrightarrow $自列紧。
        \end{theorem}
        \begin{proof}
            必要性:\begin{enumerate}
                \item 紧集是闭集。设$A$是紧集,希望证明$X\backslash A$是开集,任取
                    $x\in X\backslash A$,取开覆盖
                    \begin{equation*}
                        \bigcup_{y\in A} B( y,\frac{1}{3}d(x,y) )\supset A
                    \end{equation*}
                    存在子覆盖
                    \begin{equation*}
                        \bigcup_{k=1}^m B(y_k,\frac{1}{3}d(x,y_k))\supset A
                    \end{equation*}
                    令$\delta=\mathop{\rm min}\limits_{1\leqslant k\leqslant m}\frac{1}{3}d(x,y_k)$,
                    \begin{align*}
                        &\Rightarrow B(x,\delta)\cap \bigcup_{k=1}^m B( y_k,\frac{1}{3}d(x,y_k) )=\varnothing\\
                        &\Rightarrow B(x,\delta)\subset X\backslash A
                    \end{align*}
                    所以$x$是内点,进而$X\backslash A$是开集。
                \item 紧集是列紧集。假设$A$紧而不列紧,则存在$\{x_n\}_{n=1}^\infty\subset A$没有收敛子列,不妨假设$x_n$互不相同,令
                    \begin{equation*}
                        S_n=\{x_k\}_{k=1}^\infty \backslash {x_n}
                    \end{equation*}
                    则$S_n$是闭集\footnote{$S_n$没有聚点,也符合闭集定义。},
                    $X\backslash S_n$是开集。而
                    \begin{equation*}
                        \bigcup_{n=1}^\infty (X\backslash S_n)=X\backslash \left( \bigcap_{n=1}^\infty S_n \right)=X\supset A
                    \end{equation*}
                    这就是$A$的一个开覆盖,存在$N$使得
                    \begin{equation*}
                        \bigcup_{n=1}^N (X\backslash S_n)\supset A
                    \end{equation*}
                    但是
                    \begin{equation*}
                        \bigcup_{n=1}^N (X\backslash S_n)=
                        X\backslash \left( \bigcap_{n=1}^N S_n \right)
                        =X\backslash \{ x_n \}_{n=N+1}^\infty
                    \end{equation*}
                    不能是$A$的覆盖,矛盾。
            \end{enumerate}

            充分性:假设$A$自列紧但不紧,即存在$A$的一个开覆盖
            $\{G_\alpha\}_{\alpha\in\Lambda}$使得任取有限个$G_\alpha$都不能覆盖$A$,
            $A$自列紧,所以完全有界,对于$\forall n$,取有限$\frac{1}{n}$网:
            \begin{equation*}
                N_{\frac{1}{n}}=\{ y_1^{(n)},\cdots,y_{m_n}^{(n)} \},\ A\subset \bigcup_{k=1}^{m_n}B(y_k,\frac{1}{n})
            \end{equation*}
            那么,对于每个$n$,存在$y_k^{(n)}\in N_{\frac{1}{n}}$使得$B(y_k^{(n)},\frac{1}{n})$
            不能被有限个$G_\alpha$覆盖\footnote{否则,每个$B(y_k^{(n)},\frac{1}{n})$都能被有限覆盖,取这些有限覆盖的并就是$A$的有限覆盖,矛盾。},记$y_k^{(n)}=y^{(n)}$,
            得到点列$\{y^{(n)}\}_{n=1}^\infty$,
            因为$A$自列紧,所以有收敛子列$\{y^{(n_k)}\}_{k=1}^\infty$,并设其收敛到$y_0\in A$.
            
            设$y_0\in G_{\alpha_0}$,则存在$\delta>0$使得
            $B(y_0,\delta)\subset G_{\alpha_0}$,而
            $y^{(n_k)}\rightarrow y_0$,当$k$充分大时,
            $n_k>\frac{2}{\delta}$且$d( y^{(n_k)},y_0 )<\frac{\delta}{2}$,则$\forall y\in B(y^{(n_k)},\frac{1}{n_k})$,
            \begin{equation*}
                d( y,y_0 )\leqslant d(y_0,y^{(n_k)})+d( y,y^{(n_k)} )\leqslant \frac{2}{\delta}+\frac{1}{n_k}\leqslant \delta 
            \end{equation*}
            于是
            \begin{equation*}
                B(y_{n_k},\frac{1}{n_k})\subset B(y_0,\delta)\subset G_{\alpha_0}
            \end{equation*}
            和$B(y_{n_k},\frac{1}{n_k})$不能被有限个$G_\alpha$覆盖矛盾。
        \end{proof}
%        \begin{example}
%            Hilbert长方体:
%            \begin{equation*}
%                \{ x=(x_1,x_2,\cdots,x_n,\cdots)\in \ell^2:|x_n|\leqslant \frac{1}{2^n},n=1,2,\cdots \}
%            \end{equation*}
%            是$\ell^2$上的列紧集。
%
%            这里$\ell^p(1\leqslant p<\infty)$是数列空间,每个元素都是一个数列,度量为:
%            \begin{equation*}
%                d( \{x_n\},\{y_n\} )=\left(\sum_{k=1}^\infty |x_n-y_n|^p\right)^{\frac{1}{p}}
%            \end{equation*}
%        \end{example}
%        \begin{proof}
%            作业。
%        \end{proof}
        \begin{property}
            \begin{equation*}
                \mbox{有界闭}\mathop{\Leftrightarrow}\limits_{X=\mathbb{R}^n}^{} 
                \mbox{紧}\mathop{\Leftrightarrow}\limits_{}^{} 
                \mbox{自列紧}\mathop{\leftrightarrows}\limits_{}^{\mbox{闭}} 
                \mbox{列紧}\mathop{\leftrightarrows}\limits_{}^{X\mbox{完备}} 
                \mbox{完全有界}\mathop{\leftrightarrows}\limits_{}^{X=\mathbb{R}^n} 
                \mbox{有界}
            \end{equation*}
        \end{property}
        \begin{theorem}
            列紧空间可分。
        \end{theorem}
        \begin{proof}
            列紧$\Rightarrow $完全有界$\Rightarrow \forall n$,存在有限的
            $\frac{1}{n}$网$N_{\frac{1}{n}}$,然后对所有的$n$取并得到一个可数集:
            \begin{equation*}
                \bigcup_{n=1}^\infty N_{\frac{1}{n}}\mathop{\subset }\limits^{\rm dense} X
            \end{equation*}
            这是因为$\forall x\in X$,$\forall n$,存在$x_n\in N_{\frac{1}{n}}$使得$d(x_n,x)<\frac{1}{n}$,从而$x_n\rightarrow x$.
        \end{proof}
        \begin{proposition}
            $(M,\rho)$是紧度量空间,$C(M)$为$M$上的连续函数全体,定义
            \begin{equation*}
                d(f,g)\mathop{=}\limits^{\rm def}
                \mathop{\rm sup}\limits_{x\in M}|f(x)-g(x)|
            \end{equation*}
            则$d$是$C(M)$上的度量,且$(C(M),d)$完备。(作业)
        \end{proposition}
        \begin{proof}
            先证明$d$是$C(M)$上的度量:
    \begin{enumerate}[$(1)$]
        \item 唯一性:$d(f,g)=\fun{sup}{x\in M}|f(x)-g(x)|=0\Leftrightarrow f(x)=g(x){\rm\ on\ }M$.
        \item 非负性:绝对值非负,故$d(f,g)$非负。
        \item 对称性:$|f-g|=|g-f|$.
        \item 三角不等式:
            \begin{align*}
                d(f,g)+d(g,h)&=
                \fun{sup}{x\in M} |f(x)-g(x)|+\fun{sup}{x\in M} |g(x)-h(x)|\\
                &\geqslant \fun{sup}{x\in M} (|f(x)-g(x)|+|g(x)-h(x)|)\\
                &\geqslant \fun{sup}{x\in M} |f(x)-h(x)|=d(f,h)
            \end{align*}
    \end{enumerate}

    再证明$(C(M),d)$完备:任取$C(M)$上的柯西列$\{ f_n \}_{n=1}^\infty$,即
    \begin{equation*}
        \forall \varepsilon>0,\exists N{\rm\ s.t.\ }\forall m,n\geqslant N,d(f_n,f_m)=\fun{sup}{x\in M}|f_n(x)-f_m(x)|<\varepsilon
    \end{equation*}
    则固定$x\in M$,$\{f_n(x)\}_{n=1}^\infty$是$\R$上的柯西列,
    进而收敛,设其收敛到$f_0(x)$,于是
    \begin{equation*}
        \forall x\in M,\forall \varepsilon>0,\exists N>0{\rm\ s.t.\ }
        n\geqslant N\Rightarrow \fun{sup}{x\in M}|f_n(x)-f_0(x)|<\varepsilon\tag*{(1)}
    \end{equation*}
    $f_n$连续,所以对于$\varepsilon$,
    \begin{equation*}
        \exists \delta{\rm\ s.t.\ }\forall y\in B(x,\delta),|f_n(x)-f_n(y)|<\varepsilon\tag*{(2)}
    \end{equation*}
    则
    \begin{align*}
        |f_0(x)-f_0(y)|&\leqslant \mathop{|f_0(x)-f_n(x)|}\limits_{(1)}+
        \mathop{|f_n(x)-f_n(y)|}\limits_{(2)}+
        \mathop{|f_n(y)-f_0(x)|}\limits_{(1)}\\
        &\leqslant \varepsilon+\varepsilon+\varepsilon=3\varepsilon\rightarrow 0
    \end{align*}
    所以$f_0$连续,即$f_0\in C(M)$,于是$(C(M),d)$完备。
        \end{proof}
        \begin{definition}
            $(M,\rho)$是紧度量空间,$C(M)$为$M$上的连续函数全体,称$C(M)$中的一族函数$\mathcal{F}$等度连续是指:$\forall \varepsilon>0$,存在$\delta>0$使得
            \begin{equation*}
                |\varphi(x')-\varphi(x'')|<\varepsilon,\forall x',x''\in M{\rm\ with\ }\rho(x',x'')<\delta,\forall \varphi\in\mathcal{F}
            \end{equation*}
        \end{definition}
        \begin{theorem}[Argela-Ascoli]
            $\mathcal{F}$列紧当且仅当$\mathcal{F}$作为函数族一致有界、等度连续。
        \end{theorem}
        \begin{proof}
            必要性:$\mathcal{F}$列紧$\Rightarrow $完全有界$\Rightarrow $有界,
            \begin{equation*}
                d(f,0)\leqslant R,\forall f\in \mathcal{F}\Rightarrow 
                \mathop{\rm sup}\limits_{x\in M}|f(x)|\leqslant R,\forall f\in\mathcal{F}
            \end{equation*}
            $\Rightarrow $一致有界。下证等度连续:$\forall \varepsilon>0$,存在
            $N_{\frac{\varepsilon}{3}}=\{ \varphi_{1},\cdots,\varphi{m} \}$使得
            \begin{equation*}
                \bigcup_{k=1}^m B(\varphi_k,\frac{\varepsilon}{3})\supset \mathcal{F}\tag*{(1)}
            \end{equation*}
            因为$\varphi_k$一致连续,对每个$k$,存在$\delta_k>0$使得
            \begin{equation*}
                |\varphi_k(x')-\varphi_k(x'')|<\frac{\varepsilon}{3},\forall x',x''\in M{\rm\ with\ }\rho(x',x'')<\delta_k
            \end{equation*}
            令$\delta={\rm min}\{\delta_1,\cdots,\delta_m\}$,则
            \begin{equation*}
                |\varphi_k(x')-\varphi_k(x'')|<\frac{\varepsilon}{3},\forall x',x''\in M{\rm\ with\ }\rho(x',x'')<\delta,\forall k\tag*{(2)}
            \end{equation*}
            由(1),
            \begin{equation*}
                \forall \varphi\in\mathcal{F},\exists k{\rm\ s.t.\ }d(\varphi,\varphi_k)<\frac{\varepsilon}{3}
            \end{equation*}
            于是$\forall x,|\varphi(x)-\varphi_k(x)|\leqslant
            \mathop{\rm sup}\limits_{x\in M}|\varphi(x)-\varphi_k(x)|=d(\varphi,\varphi_k)\leqslant \frac{\varepsilon}{3}$.
            而当$\rho(x',x'')<\delta$时,由(2),$|\varphi_k(x')-\varphi_k(x'')|<\frac{\varepsilon}{3}$,所以
            \begin{equation*} 
                |\varphi(x')-\varphi(x'')|\leqslant 
                |\varphi(x')-\varphi_k(x')|+|\varphi_k(x')-\varphi_k(x'')|+|\varphi_k(x'')-\varphi(x'')|<\varepsilon
            \end{equation*}

            充分性:$\mathcal{F}$等度连续,$\forall \varepsilon>0$,存在$\delta>0$,使得
            \begin{equation*}
                |\varphi(x')-\varphi(x'')|<\varepsilon,
                \forall x',x''\in M{\rm\ with\ }\rho(x',x'')<\delta,\forall \varphi\in\mathcal{F}
            \end{equation*}
            $M$紧,所以有有限$\delta$网$N_\delta=\{x_1,\cdots,x_n\}$,定义映射:
            \begin{equation*}
                T:\mathcal{F}\rightarrow\mathbb{R}^n,\varphi\mapsto ( \varphi(x_1),\cdots,\varphi(x_n) )
            \end{equation*}
            $\mathcal{F}$一致有界,所以可令
            \begin{equation*}
                R\mathop{=}\limits^{\rm def}
                \mathop{\rm sup}\limits_{\varphi\in\mathcal{F}}
                \mathop{\rm sup}\limits_{x\in M}|\varphi(x)|<\infty
            \end{equation*}
            则
            \begin{equation*}
                \left[ \sum_{i=1}^n |\varphi(x_i)|^2 \right]^{\frac{1}{2}}\leqslant \sqrt{n}R,\forall \varphi\in\mathcal{F}
            \end{equation*}
            所以$T(\mathcal{F})$是$\mathbb{R}^n$中有界集,故列紧,设
            $T(\mathcal{F})$的有限$\frac{\varepsilon}{3}$网为
            \begin{equation*}
                M_{\frac{\varepsilon}{3}}=\{ T(\varphi_1),\cdots,T(\varphi_m) \}
            \end{equation*}
            Claim:$\{\varphi_1,\cdots,\varphi_m\}$是$\mathcal{F}$的$\varepsilon$网。
            \begin{equation*}
                \forall \varphi\in\mathcal{F},\exists k{\rm\ s.t.\ }d_{\mathbb{R}^n}(T(\varphi),T(\varphi_m))<\frac{\varepsilon}{3}
            \end{equation*}
            于是$|\varphi(x_i)-\varphi_k(x_i)|\leqslant d_{\mathbb{R}^n}(T(\varphi),T(\varphi_m))<\frac{\varepsilon}{3}$.
            同时,$\forall x\in M,\exists x_i\in N_\delta{\rm\ s.t.\ }\rho(x_i,x)<\delta$,由(1.3.3),
            \begin{equation*}
                |\varphi(x)-\varphi(x_i)|,|\varphi_k(x_i)-\varphi_k(x)|<\frac{\varepsilon}{3}
            \end{equation*}
            所以
            \begin{equation*}
                |\varphi(x)-\varphi_k(x)|\leqslant |\varphi(x)-\varphi(x_i)|+|\varphi(x_i)-\varphi_k(x_i)|+|\varphi_k(x_i)-\varphi_k(x)|<\varepsilon
            \end{equation*}
            所以$d(\varphi,\varphi_k)\leqslant\varepsilon$.
        \end{proof}
            
            $L^p$空间中列紧集是怎样的?
        \begin{theorem}[Riesz-Frechet-kolmogorov]
            设$1\leqslant p<\infty$,$\mathcal{F}\subset L^p(\mathbb{R}^n)$列紧当且仅当:
            \begin{enumerate}
                \item $\mathcal{F}$有界,即$\mathop{\rm sup}\limits_{f\in\mathcal{F}}||f||_p<\infty$.
                \item $\forall \varepsilon>0$,$\exists R>0{\rm\ s.t.}$
                    \begin{equation*}
                        \int_{|x|>R}|f(x)|^p{\rm d}x<\varepsilon^p,\forall f\in\mathcal{F}
                    \end{equation*}
                \item $\forall \varepsilon$,$\exists\delta>0{\rm\ s.t.}$
                    \begin{equation*}
                        ||\tau_h f-f||_p<\varepsilon,\forall h\in\mathbb{R}^n{\rm\ with\ }|h|<\delta,\forall f\in\mathcal{F}
                    \end{equation*}
                    其中$(\tau_h f)(x)=f(x+h)$.
            \end{enumerate}
        \end{theorem}
        \begin{example}
            $A$是$\ell^2$上的列紧集$\Leftrightarrow $
            \begin{enumerate}
                \item $A$有界。
                \item $\forall \varepsilon>0$,$\exists N$使得\begin{equation*}
                    \sum_{k=N+1}^\infty |x_k|^2<\varepsilon,\forall x=(x_1,x_2,\cdots)\in A
                \end{equation*}
            \end{enumerate}
            (作业)
        \end{example}
        \begin{proof}
            $(\Rightarrow)$:假设$A$无界,则能取发散点列$\{a_n\}_{n=1}^\infty$满足
    $d(a_n,0)\rightarrow\infty $,与$A$列紧矛盾;
    $A$列紧则完全有界,$\forall\varepsilon>0$,
    取$A$的有限$\varepsilon/2$网$\{a^{(i)}\}_{i=1}^n$,
    其中每个$a^{(i)}=(a^{(i)}_1,\cdots,a^{(i)}_j,\cdots)$,
    根据定义有:
    \begin{equation*}
        \sum_{j=1}^\infty |a_j^{(i)}|^2<+\infty,\ i=1,2,\cdots,n
    \end{equation*}
    所以
    \begin{equation*}
        \exists N_i{\rm\ s.t.\ } 
        \sum_{j=N_i+1}^\infty |a_j^{(i)}|^2<\varepsilon/2,
        \ i=1,2,\cdots,n
    \end{equation*}
    取$N=\fun{max}{i=1,2,\cdots,n}N_i$,因为$\{a^{(i)}\}_{i=1}^n$是$\varepsilon$网,
    \begin{equation*}
        \forall x\in A,\exists a^{(i)}{\rm\ s.t.\ }d(x,a^{(i)})=\sum_{j=1}^\infty | a_j^{i}-x_i |^2<\varepsilon/2
    \end{equation*}
    所以
    \begin{equation*}
        \sum_{i=N+1}^\infty |x_i|^2\leqslant \sum_{i=N+1}^\infty 
        ( |a_i^{(k)}|^2+|x_i-a_i^{(k)}|^2 )\leqslant \varepsilon
    \end{equation*}

    $(\Leftarrow)$:$\forall \varepsilon>0$,存在$N$使得
    \begin{equation*}
        \sum_{i=N+1}^\infty |x_i|^2\leqslant \varepsilon,\forall x\in A
    \end{equation*}
    取连续映射$\varphi:A\rightarrow\R^N,x\mapsto (x_1,\cdots,x_N)$,
    $A$有界$\Rightarrow \varphi(A)$有界$\Rightarrow \varphi(A)$完全有界,
    对于$\delta=\frac{\varepsilon^2}{2}$,存在
    $\varphi(A)$上的有限$\delta$网$\{ \varphi(x^{(i)}) \}_{i=1}^n$,
    记$X=\{ x^{(i)} \}_{i=1}^n\subset A$.
    \begin{equation*}
        \forall y\in A,\exists x^{(i)}\in X{\rm\ s.t.\ }
        d_{\R^N}(\varphi(y),\varphi(x^{(i)}))<\frac{\varepsilon^2}{2}
    \end{equation*}
    \begin{align*}
        \Rightarrow d(y,x^{(i)})^2&=
        \sum_{j=1}^N |y_j-x_j^{(i)}|^2+\sum_{j=N+1}^\infty |y_j-x_j^{(i)}|^2\\
        &<d_{\R^N}(\varphi(y),\varphi(x^{(k)}))
        +2\sum_{j=N+1}^\infty ( |y_j|^2+|x_j^{(i)}|^2 )<\varepsilon^2
    \end{align*}
    于是$X$为$A$的有限$\varepsilon$网,$A$完全有界$\Rightarrow A$列紧。
        \end{proof}
    
\section{赋范线性空间}
        \subsection{Banach空间}
        \begin{definition}
            $X$是非空集合,$\mathbb{K}$表示$\mathbb{C}$或者$\mathbb{R}$,如果能在$X$上定义两种封闭的运算:
            \begin{enumerate}
                \item 加法:$X\times X\rightarrow X,(x,y)\mapsto x+y$.满足:
                    \begin{enumerate}[(i)]
                        \item 结合律
                        \item 交换律
                        \item 零元
                        \item 负元
                    \end{enumerate}
                \item 乘法:$\mathbb{K}\times X\rightarrow X,(\lambda,x)\mapsto \lambda x$.满足:
                    \begin{enumerate}
                        \item[(v)] $1x=x$.
                        \item[(vi)] $\alpha(\beta x)=(\alpha\beta)x$.
                        \item[(vii)] $(\alpha+\beta)x=\alpha x+\beta y$.
                        \item[(viii)] $\alpha(x+y)=\alpha x+\alpha y$.
                    \end{enumerate}
            \end{enumerate}
            则称$X$是$\mathbb{K}$上的向量空间,线性空间。$X$中的元素称为向量。

            如果向量空间$X$的子集$Y$,如果对同一数域$\mathbb{K}$上的加法和乘法
            构成向量空间,则称之为$X$的向量子空间,也等价于$Y$关于加法和乘法封闭。
        \end{definition}

        约定一些记号:
            \begin{align*}
                x+E&:=\{ x+y:y\in E \}\\
                \lambda E&:= \{ \lambda y:y\in E \}\\
                E+F&:=\{ x+y:x\in E,y\in F \}
            \end{align*}
            \begin{equation*}
                {\rm span}(E):= \left\{
                    \sum_{k=1}^n \lambda_k x_k,x_k\in E,\lambda_k\in\mathbb{K},n\in \mathbb{N}
                \right\}
            \end{equation*}
            称为$E$张成的子空间。

            如果$E$线性无关且${\rm span}(E)=X$,则称$E$是$X$的Hamel基(代数基,线性基)。
        
        \begin{theorem}
            任一向量空间一定有Hamel基。
        \end{theorem}

        如果Hamel基是有限集,则定义${\rm dim\ }X=\# E$,否则记${\rm dim\ }X=\infty$.
        \begin{definition}
            $\mathbb{K}$是$\mathbb{C}$或者$\mathbb{R}$,$X$是$\mathbb{K}$上的向量空间,
            如果函数$||\cdot ||:X\rightarrow\mathbb{R}$满足:
            \begin{enumerate}
                \item 正定性
                \item 齐次性
                \item 三角不等式
            \end{enumerate}
            则称之为$X$上的一个范数。$(X,||\cdot||)$称为一个赋范空间。定义
            \begin{equation*}
                d(x,y)\mathop{=}\limits^{\rm def}||x-y||
            \end{equation*}
            称为范数诱导的度量,也叫典则度量。

            如果$(X,||\cdot ||)$在此度量下完备,则称之为Banach空间。
        \end{definition}
        \begin{example}
            Banach空间的一些例子:
            
            函数空间$L^p,L^\infty,C(M)$;
            数列空间$\ell^p,\ell^\infty$(有界数列空间,$||x||_\infty \mathop{=}\limits^{\rm def} \mathop{\rm sup}\limits_{k\geqslant 1}|x_k|$),
            $C$(收敛数列空间),$C_0$(收敛到零的数列全体).
        \end{example}
        \begin{example}
            $\Omega$是$\mathbb{R}^n$中的有界域,
            $C^k(\overline{\Omega})$是$\overline{\Omega}$上$k$次连续可微的函数全体,定义
            \begin{equation*}
                ||u||_{k,p}\mathop{=}\limits^{\rm def}
                \left( \sum_{|\alpha|\leqslant k}\int_\Omega |\partial^\alpha u|^p \right)^{\frac{1}{p}}
            \end{equation*}
            这是$C^k(\overline{\Omega})$上的一个范数,
            \begin{equation*}
                S\mathop{=}\limits^{\rm def}
                \left\{ u\in C^k(\overline{\Omega}):||u||_{k,p}<\infty\right\}
            \end{equation*}
            的完备化称为Sobolev空间,记作$H^{k,p}(\Omega)$.
        \end{example}
        \subsection{范数等价}
        \begin{definition}
            $X$是向量空间,$||\cdot||_1$和$||\cdot||_2$是$X$上的两个范数,称
            $||\cdot||_2$强于$||\cdot||_1$是指:$\forall \{x_n\}_{n=1}^\infty \subset X$,
            \begin{equation*}
                ||x_n||_2\rightarrow 0\Rightarrow ||x_n||_1\rightarrow 0
            \end{equation*}
            记作$||\cdot||_1\lesssim ||\cdot||_2$.如果$||\cdot||_2$强于$||\cdot||_1$,同时$||\cdot||_1$强于$||\cdot||_2$,则称二者是等价范数。
        \end{definition}
        \begin{proposition}
            $||\cdot||_2$强于$||\cdot||_1\Leftrightarrow \exists C>0$
            使得$||x||_1\leqslant C||x||_2,\forall x\in X$.
        \end{proposition}
        \begin{proof}
            充分性显然,下证必要性:假设不然,则$\forall n$,存在$x_n\in X$使得
            $||x_n||_1\geqslant m||x_n||_2$,令$y_n=\frac{x_n}{||x_n||_1}$,则
            \begin{equation*}
                ||y_n||_2\leqslant \frac{1}{n}\rightarrow 0{\rm\ as\ }n\rightarrow\infty
            \end{equation*}
            $||\cdot||_2$强于$||\cdot||_1$,所以$||y_n||_1\rightarrow 0$,
            但是$||y_n||_1$恒等于$1$,矛盾。
        \end{proof}
        \begin{corollary}
            $||\cdot||_1$和$||\cdot||_2$等价,当且仅当存在
            $C_1,C_2>0$,使得 
            \begin{equation*}
                C_1||x||_1\leqslant ||x||_2\leqslant C_2||x||_1,\forall x\in X
            \end{equation*}
        \end{corollary}
        \begin{example}
            $\mathbb{R}^n$上$||\cdot ||_p(1\leqslant p\leqslant \infty)$彼此等价,因为
            \begin{equation*}
                ||x||_\infty\leqslant ||x||_p \leqslant n^{\frac{1}{p}}||x||_\infty
            \end{equation*}
        \end{example}
        \begin{theorem}
            有限维空间上所有范数都等价。
        \end{theorem}
        \begin{proof}
            设${\rm dim\ }X=n$,$\{e_1,\cdots,e_n\}$是一组基,
            $\forall x\in X$有唯一表示
            \begin{equation*}
                x=\sum_{i=1}^n \xi_i e_i,\xi_i\in\mathbb{K}
            \end{equation*}
            定义映射:$T\rightarrow \mathbb{K}^n,x\mapsto (\xi_1,\xi_2,\cdots,\xi_n)$,
            则$T$是$X$到$\mathbb{K}$的代数同构,设
            \begin{equation*}
                |\xi|=\left( \sum_{i=1}^n |\xi_i|^2 \right)^{\frac{1}{2}},\xi\in\mathbb{K}^n
            \end{equation*}
            令$||x||_T=|T(x)|$,则$||\cdot||_T$是$X$上的范数。任取一个$X$上的范数$||\cdot||$,下面证明
            $||\cdot||_T$和$||\cdot||$等价。

            定义函数
            \begin{equation*}
                p:\mathbb{K}^n\rightarrow \mathbb{R},\xi\mapsto || \sum_{i=1}^n \xi_i e_i ||
            \end{equation*}
            \begin{enumerate}
                \item $p(\xi)=|\xi|\cdot p(\frac{\xi}{|\xi|}),\forall \xi\neq 0$.
                \item $p$在$\mathbb{K}$上连续:\begin{align*}
                    |p(a)-p(b)|&=\left| ||\sum_{i=1}^n a_i e_i||-||\sum_{i=1}^n b_i e_i|| \right|\\
                    &\leqslant || \sum_{i=1}^n (a_i-b_i) e_i ||\\
                    &\leqslant \sum_{i=1}^n |a_i-b_i| e_i\\
                    &\leqslant \left( \sum_{i=1}^n ||e_i||^2 \right)^{\frac{1}{2}}|a-b|
                \end{align*}
                最后一步是Caurhy-Schwarz不等式。
            \end{enumerate}
            令$S_1=\{ \xi\in\mathbb{K}^n:|\xi|=1 \}$,则$S_1$是紧集,故$p$在$S_1$上存在最小值和最大值,分别记作
            $C_1$和$C_2$,从而
            \begin{align*}
                &\Rightarrow C_1\leqslant p(\frac{\xi}{|\xi|})\leqslant C_2,\forall \xi\neq 0\\
                &\Rightarrow C_1|\xi|\leqslant p(\xi)\leqslant C_2|\xi|,\forall \xi\\
                &\Rightarrow C_1|T(x)|\leqslant p(T(x))\leqslant C_2|T(x)|,\forall x\in X\\
                &\Leftrightarrow C_1||x||_T\leqslant ||x||\leqslant C_2||x||_T
            \end{align*}
            只需证明$C_1>0$,假设$C_1=0$,则存在$\xi^*\in S_1$使得
            \begin{equation*}
                ||\sum_{i=1}^n \xi_i^* e_i||=p(\xi^*)=0 \Rightarrow \sum_{i=1}^n \xi_i^* e_i=0\Rightarrow \xi^*=0
            \end{equation*}
            这与$\xi^*\in S_1$矛盾。
        \end{proof}
        \begin{corollary}
            同维数的两个有限维赋范空间(作为向量空间)代数同构且
            (作为拓扑空间)同胚。
        \end{corollary}
        \begin{proof}
            \begin{equation*}
                T:X\rightarrow \mathbb{K}^n,x=\sum_{i=1}^n \xi_i e_i\mapsto \xi
            \end{equation*}
            是一个代数同构,满足
            \begin{equation*}
                C_1|T(x)|\leqslant ||x||\leqslant C_2|T(x)|,\forall x\in X
            \end{equation*}
            第一个不等号得到$T$连续,第二个不等号得到$T^{-1}$连续,故$T$也是同胚。
        \end{proof}
        \begin{corollary}
            有限维赋范空间一定是Banach空间。
        \end{corollary}
        \begin{proof}
            $C_1|T(x)|\leqslant ||x||\leqslant C_2|T(x)|$,设$\{x_n\}_{n=1}^\infty$是$X$中基本列,
            则$\{ T(x_n) \}_{n=1}^\infty$是$\mathbb{K}^n$中基本列,设$T(x_k)\rightarrow \xi$,
            \begin{equation*}
                ||x_n-T^{-1}(\xi)||\leqslant C_2|T(x_n)-\xi|\rightarrow 0
            \end{equation*}
        \end{proof}
        \begin{corollary}
            任何赋范空间的有限维子空间一定是闭子空间。(作业)
        \end{corollary}
        \begin{proof}
            任何赋范空间的有限维子空间是有限维赋范空间,所以是Banach空间,
            所有的收敛列都是柯西列,所有的柯西列都收敛到子空间内某点,所以是闭子空间。
        \end{proof}
        \begin{theorem}
            赋范空间$(X,||\cdot||)$,$X$中单位球面列紧
            $\Leftrightarrow {\rm dim\ }X<\infty$.
        \end{theorem}
        \begin{lemma}[Riesz]
            赋范空间$(X,||\cdot||)$,$(Y,||\cdot||)$是$X$的闭子空间,$Y\neq X$,则
            $\forall \varepsilon>0$,$\exists e\in X{\rm\ with\ }||e||=1$,使得
            \begin{equation*}
                {\rm dist}(e,Y):=\fun{inf}{y\in Y} ||e-y|| \geqslant 1-\varepsilon
            \end{equation*}
            \begin{proof}
                取$x\in X\backslash Y$,令
                \begin{equation*}
                    d_i={\rm dist}(x,y)=\mathop{\rm inf}\limits_{y\in Y}||x-y||
                \end{equation*}
                $Y$是闭集,$d>0$,且存在$y_0\in Y$使得
                \begin{equation*}
                    d\leqslant ||x-y_0||\leqslant \frac{d}{1-\varepsilon}
                \end{equation*}
                令
                \begin{equation*}
                    e\mathop{=}\limits^{\rm def}
                    \frac{x-y_0}{||x-y_0||}\Rightarrow ||e||=1,e\notin Y
                \end{equation*}
                $\forall \zeta\in Y$,
                \begin{align*}
                    ||e-\zeta||&=|| \frac{x-y_0}{||x-y_0||}-\zeta ||\\
                    &=\frac{1}{||x-y_0||}\cdot ||x-\mathop{( y_0+||x-y_0||\zeta )}\limits_{\mbox{这一堆}\in Y}||\\
                    &\geqslant \frac{1-\varepsilon}{d}\cdot d=1-\varepsilon
                \end{align*}
                所以${\rm dist}(e,Y)\geqslant 1-\varepsilon$.
            \end{proof}
        \end{lemma}
        \begin{proof}
            充分性:
            \begin{equation*}
                T:X\rightarrow \mathbb{K}^n,x=\sum \xi_i e_i\mapsto \xi
            \end{equation*}
            满足
            \begin{equation*}
                C_1|T(x)|\leqslant ||x||\leqslant C_2|T(x)|
            \end{equation*}
            于是$T(S_1)$有界,故列紧,
            \begin{align*}
                \Rightarrow& \forall \{x_n\}_{n=1}^\infty\subset S_1,\exists T(x_{n_k})\rightarrow y\in\mathbb{K}^n\\
                \Rightarrow& x_{n_k}\rightarrow T^{-1}(y)\in X
            \end{align*}
            必要性:假设${\rm dim\ }X=\infty$,则存在一列向量$\{e_n\}_{n=1}^\infty $线性无关,令子空间
            \begin{equation*}
                X_n\mathop{=}\limits^{\rm def} \left< e_1,e_2,\cdots,e_n \right>,n=1,2,\cdots
            \end{equation*}
            于是$X_n\subsetneqq X_{n+1}$且是闭子空间,由引理,取$\varepsilon=\frac{1}{2}$,
            $\forall n\in\mathbb{N}$,存在$x_n\in X_n{\rm\ with\ }||x_n||=1$,使得
            \begin{equation*}
                {\rm dist}(x_n,X_{n-1})\geqslant \frac{1}{2}
            \end{equation*}
            \begin{align*}
                \Rightarrow & ||x_n-x_m||\geqslant \frac{1}{2},\forall n\neq m\\
                \Rightarrow & \{ x_n \}_{n=1}^\infty\mbox{没有收敛子列}
            \end{align*}
            这与$S_1$列紧矛盾。
        \end{proof}
        \subsection{商空间}
        \begin{definition}[商空间]
            赋范空间$(X,||\cdot||)$,$(X_0,||\cdot||)$是$X$的闭子空间。
            在$X$中定义
            \begin{equation*}
                x\sim y\mathop{\Leftrightarrow}\limits^{\rm def} x-y\in X_0
            \end{equation*}
            $[x]\mathop{=}\limits^{\rm def} x$所在的等价类,
            \begin{equation*}
                X/X_0\mathop{=}\limits^{\rm def}\{ [x]:x\in X \}
            \end{equation*}
            定义运算:
            \begin{equation*}
                [x]+[y]=[x+y],\lambda [x]=[\lambda x]
            \end{equation*}
            则$X/X_0$成为向量空间。并定义:
            \begin{equation*}
                ||[x]||_*\mathop{=}\limits^{\rm def} \mathop{\rm inf}\limits_{y\in [x]}||y||
            \end{equation*}
        \end{definition}
        \begin{remark}
            注意这里必须要求$X_0$是闭子空间。
        \end{remark}
        \begin{theorem}
            $||\cdot ||_*$是$X/X_0$上的范数。
        \end{theorem}
        \begin{proof}
            \begin{enumerate}
                \item 正定性:$\forall y\in X,||y||\geqslant 0\Rightarrow ||[x]||_*\geqslant 0$,
                    \begin{align*}
                        ||[x]||_*=0 &\Rightarrow \exists \{x_n\}_{n=1}^\infty \subset [x]{\rm\ s.t.\ }||x_n||\rightarrow 0\\
                        &\Rightarrow x_n\rightarrow 0
                    \end{align*}
                    $[x]=x+X_0$是闭集,所以$0\in [x]$,进而可得$[x]=[0]$.
                \item 齐次性:显然。
                \item 三角不等式:由下确界的定义,$\forall \varepsilon$,
                    \begin{align*}
                        \exists x'\in [x]{\rm\ s.t.\ }||x'||<||[x]||_*+\frac{\varepsilon}{2}\\
                        \exists y'\in [y]{\rm\ s.t.\ }||y'||<||[y]||_*+\frac{\varepsilon}{2}
                    \end{align*}
                    \begin{align*}
                        \Rightarrow & ||x'+y'||\leqslant ||x'||+||y'||\leqslant 
                        ||[x]||_*+||[y]||_*+\varepsilon\\
                        \mathop{\Rightarrow}\limits^{x'+y'\in [x+y]}&
                        ||[x+y]||_*\leqslant ||[x]||_*+||[y]||_*+\varepsilon \\
                        \mathop{\Rightarrow}\limits^{\varepsilon\rightarrow 0}& 
                        ||[x+y]||_*\leqslant ||[x]||_*+||[y]||_*
                    \end{align*}
            \end{enumerate}
        \end{proof}
        \begin{theorem}
            $(X,||\cdot||)$完备,则$(X/X_0,||\cdot||_*)$也完备。
        \end{theorem}
        \begin{lemma}
            $X$完备$\Leftrightarrow \forall \{x_n\}_{n=1}^\infty \subset X$,
            \begin{equation*}
                \sum_{n=1}^\infty ||x_n||<\infty \Rightarrow \sum_{n=1}^\infty x_n\mbox{收敛}
            \end{equation*}
            \begin{proof}
                习题1.4.7.
            \end{proof}
        \end{lemma}
        \begin{proof}
            由引理,只需证明$X/X_0$中任一绝对收敛级数都收敛。设
            \begin{equation*}
                \sum_{n=1}^\infty ||[x_n]||_*<\infty
            \end{equation*}
            对每个$n$,
            \begin{align*}
                &\exists y_n\in X_0{\rm\ s.t.\ }||x_n+y_n||\leqslant ||[x_n]||_*+\frac{1}{2^n}\\
                \Rightarrow &
                \sum_{n=1}^\infty ||x_n+y_n||\leqslant \sum_{n=1}^\infty ||[x_n]||_*+1<\infty\\
                \mathop{\Leftrightarrow}\limits^{X\mbox{完备}}&
                \exists x\in X{\rm\ s.t.\ }
                ||\sum_{k=1}^n (x_k+y_k)-x||\rightarrow 0{\rm\ as\ }n\rightarrow\infty\\
                \Rightarrow &
                ||\sum_{k=1}^n [x_j]-[x]||_*
                \mathop{=}\limits^{y_k\in X_0}
                ||[ \sum_{k=1}^n (x_k+y_k)-x ]||_*
                \leqslant ||\sum_{k=1}^n (x_k+y_k)-x||\rightarrow 0{\rm\ as\ }n\rightarrow\infty
            \end{align*}
        \end{proof}
    
\section{内积空间}
        \subsection{Hilbert空间}
        \begin{definition}
            $X$是$\mathbb{K}$上的向量空间,如果函数
            \begin{equation*}
                \left< \cdot,\cdot \right>:X\times X\rightarrow \mathbb{K}
            \end{equation*}
            满足:
            \begin{enumerate}
                \item 对第一变元线性:$\left< \alpha x_1+\beta x_2,y \right>=\alpha
                \left<x_1,y\right>+\beta\left<x_2,y\right>$.
                \item 对第二变元共轭线性:
                $\left< x,\alpha y_1+\beta y_2 \right>=\overline{\alpha}
                \left<x,y_1\right>+\overline{\beta}\left<x,y_2\right>$.
                \item 共轭对称:$\overline{\left<x,y\right>}=\left<y,x\right>$.
                \item $\left<x,x\right>\geqslant 0,\forall x\in X$.等号成立当且仅当$x=0$.
            \end{enumerate}
            则称$\left< \cdot,\cdot \right>$是$X$上的一个内积,
            $(X,\left< \cdot,\cdot \right>)$称为内积空间。
        \end{definition}
        \begin{lemma}[Cauthy-Schwarz]
            $(X,\left< \cdot,\cdot \right>)$是内积空间,令
            \begin{equation*}
                ||x||\mathop{=}\limits^{\rm def}\sqrt{ \left<x,x\right> },x\in X
            \end{equation*}
            则$|\left<x,y\right>|\leqslant ||x||||y||,\forall x,y\in X$,等号成立当且仅当
            存在$\lambda\in\mathbb{K}$,使得$x=\lambda y$.
            \begin{proof}
                不妨设$y\neq 0$,则$\forall \lambda\in\mathbb{K}$,
                \begin{align*}
                    0&\leqslant \left< x+\lambda y,x+\lambda y \right>\\
                    &=\left<x,x\right>+\lambda \left<y,x\right>+\overline{\lambda}\left<x,y\right>+
                    |\lambda|^2\left<y,y\right>\\
                    &=||x||^2+2{\rm Re}\{ \overline{\lambda}\left<x,y\right> \}+|\lambda|^2 ||y||^2
                \end{align*}
                这里取$\lambda=-\frac{\left<x,y\right>}{||y||^2}$,
                \begin{equation*}
                    0\leqslant ||x||^2-2{\rm Re}\{ \frac{ |\left<x,y\right>|^2 }{||y||^2} \}+
                    \frac{ |\left<x,y\right>|^2 }{||y||^4}\cdot ||y||^2
                    =||x||^2-\frac{ |\left<x,y\right>|^2 }{||y||^2}
                \end{equation*}
                于是得证。
            \end{proof}
        \end{lemma}
        \begin{proposition}
            Cauthy-Schwarz引理中的$||x||$是一个范数。
        \end{proposition}
        \begin{proof}
            只需验证三角不等式:
            \begin{align*}
                ||x+y||^2&=||x||^2+2{\rm Re}\left<x,y\right>+||y||^2\\
                &\leqslant ||x||^2+2||x||||y||+||y||^2
            \end{align*}
        \end{proof}
        \begin{definition}
            如果一个内积空间在其内积诱导范数下是一个Banach空间,则称之为Hilbert空间。
        \end{definition}
        \begin{example}
            $\ell^2$是一个Hilbert空间。(作业)
        \end{example}
        \begin{proof}
            只需证明$\ell^2$完备。任取$\ell^2$上基本列$\{x^{(n)}\}_{n=1}^\infty$,
            \begin{equation*}
                \forall \varepsilon>0,\exists N{\rm\ s.t.\ }
                ||x^{(n)}-x^{(m)}||_2<\varepsilon,\forall n,m\geqslant N
            \end{equation*}
            即
            \begin{align*}
                &\sum_{k=1}^\infty |x_k^{(n)}-x_k^{(m)}|^2<\varepsilon,\forall n,m\geqslant N\tag*{(1)}\\
                \Rightarrow& \forall \mbox{固定}k,|x_k^{(n)}-x_k^{(m)}|<\varepsilon,\forall n,m\geqslant N\\
                \Rightarrow& \{x_k^{n}\}_{n=1}^\infty \mbox{是$\R$中基本列}\\
                \Rightarrow& \exists x_k\in\R{\rm\ s.t.\ }x_k^{(n)}\rightarrow x_k{\rm\ as\ }n\rightarrow\infty
            \end{align*}
            令$x\defeq \{x_k\}_{k=1}^\infty$,Claim:$x\in \ell^2$且$||x^{(n)}\rightarrow x||_2\rightarrow 0$.
            由(1),$\forall p\in \N$
            \begin{align*}
                &\sum_{k=1}^p |x_k^{(n)}-x_k^{(m)}|^2<\varepsilon^2,\forall n,m\geqslant N\\
                \mathop{\Rightarrow}\limits^{m\rightarrow\infty}&\forall p,
                \sum_{k=1}^p |x_k^{(n)}-x_k|^2\leqslant \varepsilon^2,\forall n\geqslant N\\
                \mathop{\Rightarrow}\limits^{p\rightarrow\infty}&
                \sum_{k=1}^\infty |x_k^{(n)}-x_k|^2\leqslant \varepsilon^2,\forall n\geqslant N\tag*{(2)}\\
                \Rightarrow& x-x^{(N)}\in \ell^2\\
                \Rightarrow& x=x-x^{(N)}+x^{(N)}\in \ell^2
            \end{align*}
            而且(2)就是
            \begin{equation*}
                ||x^{(n)}-x||_2\leqslant \varepsilon,\forall n\geqslant N
                \Rightarrow ||x^{(n)}-x||_2\rightarrow 0{\rm\ as\ }n\rightarrow\infty
            \end{equation*}
        \end{proof}

        \begin{proposition}[极化恒等式]
            $(X,\left< \cdot,\cdot \right>)$是内积空间,内积诱导范数$||x||$,
            \begin{enumerate}
                \item $\mathbb{K}=\mathbb{R}$,则\begin{equation*}
                    \left<x,y\right>=\frac{1}{2}( ||x+y||^2-||x||^2-||y||^2 )
                \end{equation*}
                \item $\mathbb{K}=\mathbb{C}$,则\begin{equation*}
                    \left<x,y\right>=\frac{1}{4}\sum_{k=0}^3 {\rm i}^3 ||x+{\rm i}^ky||^2
                \end{equation*}
            \end{enumerate}
            (作业)
        \end{proposition}
        \begin{proof}
            $\K=\R$,
        \begin{align*}
            \frac{1}{2}( ||x+y||^2-||x||^2-||y||^2 )
            &=\frac{1}{2}( \agl{x+y}{x+y}-\agl{x}{x}-\agl{y}{y} )\\
            &=\frac{1}{2}( \agl{x+y}{x}+\agl{x+y}{y}-\agl{x}{x}-\agl{y}{y} )\\
            &=\frac{1}{2}( \agl{y}{x}+\agl{x}{y} )\\
            &=\agl{x}{y}
        \end{align*}
        $\K=\C$,
        \begin{align*}
            \frac{1}{4}\sum_{k=0}^3 {\rm i}^k||x+{\rm i}^ky||^2
            &=\frac{1}{4}( 
            \agl{x+y}{x+y}
            +{\rm i}\agl{x+{\rm i}y}{x+{\rm i}y}
            -\agl{x-y}{x-y}
            -{\rm i}\agl{x-{\rm i}y}{x-{\rm i}y} )\\
            &=\frac{1}{2}(
            \agl{x}{y}+\agl{y}{x}+{\rm i}\agl{x}{{\rm i}y}+{\rm i}\agl{{\rm i}y}{x})\\
            &=\frac{1}{2}(
            \agl{x}{y}+\agl{y}{x}+\agl{x}{y}-\agl{y}{x})\\
            &=\agl{x}{y}
        \end{align*}
        \end{proof}
        \begin{proposition}[平行四边形法则,P.L.]
            $(X,\left< \cdot,\cdot \right>)$是内积空间,内积诱导范数$||\cdot ||$,
            \begin{equation*}
                ||x+y||^2+||x-y||^2=2( ||x||^2+||y||^2 )
            \end{equation*}
            (作业)
        \end{proposition}
        \begin{proof}
        \begin{align*}
            ||x+y||^2+||x-y||^2&=\agl{x+y}{x+y}+\agl{x-y}{x-y}\\
            &=2\agl{x}{x}+2\agl{y}{y}
            +\agl{x}{y}+\agl{y}{x}+
            +\agl{x}{-y}+\agl{-y}{x}\\
            &=2\agl{x}{x}+2\agl{y}{y}
            +\agl{x}{y}+\agl{y}{x}
            -\agl{x}{y}-\agl{y}{x}\\
            &=2(||x||^2+||y||^2)
        \end{align*}
        \end{proof}
        \begin{theorem}[Frechet-von Neumann]
            $(X,||\cdot ||)$是赋范空间,$||\cdot ||$可由某个内积诱导出
            $\Leftrightarrow ||\cdot ||$满足P.L.(作业)
        \end{theorem}
        \begin{proof}
            必要性显然,只说明充分性:
        先考虑$\K=\R$,令
        \begin{equation*}
            \agl{x}{y}=\frac{1}{4}( ||x+y||^2-||x-y||^2 )
        \end{equation*}
        如果能证明$\agl{\cdot}{\cdot}$是一个内积,那么其诱导度量$||\cdot||$.
        \begin{enumerate}[$(1)$]
            \item $\agl{x}{x}=||x||\geqslant 0$,等号成立当且仅当$x=0$.
            \item $\agl{x}{y}=\agl{y}{x}$.
            \item 考虑
            \begin{align*}
                    \agl{x}{z}+\agl{y}{z}&=\frac{1}{4}( ||x+z||^2-||x-z||^2 )+\frac{1}{4}( ||y+z||^2-||y-z||^2 )\\
                    &=\frac{1}{8}( 2||x+z||^2+2||y+z||^2 )-\frac{1}{8}( 2||x-z||^2+2||y-z||^2 )\\
                    &=\frac{1}{8}( ||x+y+2z||^2+||x-y||^2 )-\frac{1}{8}( ||x+y-2z||^2+||x-y||^2 )\\
                    &=\frac{1}{2}\agl{x+y}{2z}
            \end{align*}
            特别地,当$y=0$,
            \begin{equation*}
                    \agl{0}{z}=\frac{1}{4}( ||0+z||^2-||0-z||^2 )=0
            \end{equation*}
            \begin{equation*}
                    \Rightarrow \agl{x}{z}=\agl{x}{z}+\agl{0}{z}
                    =\frac{1}{2}\agl{x}{2z}
            \end{equation*}
            上式中$x$替换为$x+y$,
            \begin{equation*}
                    \agl{x+y}{z}=\frac{1}{2}\agl{x+y}{2z}
            \end{equation*}
            于是
            \begin{equation*}
                    \agl{x}{z}+\agl{y}{z}=\frac{1}{2}\agl{x+y}{2z}=\agl{x+y}{z}
            \end{equation*}
            \item 由$(3)$,
            \begin{align*}
                    &\agl{x}{y}+\agl{x}{-y}=\agl{x-x}{y}=\agl{0}{y}=0\\
                    \Rightarrow &\agl{-x}{y}=-\agl{x}{y}
            \end{align*}
            \begin{equation*}
                    n\agl{x}{y}=\agl{x}{y}+\cdots+\agl{x}{y}=\agl{nx}{y}
            \end{equation*}
            \begin{align*}
                    &n\agl{\frac{m}{n}x}{y}=\agl{mx}{y}=m\agl{x}{y}\\
                    \Rightarrow &
                    \frac{m}{n}\agl{x}{y}=\agl{  \frac{m}{n}x}{y},\ n,m\in\mathbb{N}_+
            \end{align*}
            因此当$\lambda\in\Q$时,$\agl{\lambda x}{y}=\lambda\agl{x}{y}$.对于任意实数$a$,
            取有理数列$a_n\rightarrow a$,则有
            \begin{equation*}
                    \agl{ax}{y}=\fun{lim}{n\rightarrow\infty}\agl{a_nx}{y}
                    =\fun{lim}{n\rightarrow\infty}a_n\agl{x}{y}
                    =a\agl{x}{y}
            \end{equation*}
        \end{enumerate}
        综上可知$\agl{\cdot}{\cdot}$是一个内积。对于$\K=\C$的情况,取
        \begin{equation*}
            \agl{x}{y}=\frac{1}{4}\sum_{k=0}^3 {\rm i}^k||x+{\rm i}^ky||^2
        \end{equation*}
        其余过程同理。
        \end{proof}
        \begin{definition}
            如果$\left<x,y\right>=0$,则称$x$与$y$正交,记为$x\perp y$.
            对于$M\subset X$,如果$\forall y\in M$都有$x\perp y$,则记$x\perp M$.
            \begin{equation*}
                M^\perp \mathop{=}\limits^{\rm def} 
                \{x\in X:x\perp M\}
            \end{equation*}
            称为$M$的正交补。
        \end{definition}
        \begin{proposition}[勾股定理]
            $x\perp y\Rightarrow ||x+y||^2=||x||^2+||y||^2$
        \end{proposition}
        \begin{proposition}
            $\overline{M}=X$,$x\perp M$,则$x=0$.
        \end{proposition}
        \begin{proof}
            $x\in X$,存在$y_n\in M$使得$y_n\rightarrow x$,于是
            \begin{equation*}
                0=\left<x,y_n\right>\rightarrow \left<x,x\right>
            \end{equation*}
            所以$x=0$.

            实际上证明的是$x\perp M\Rightarrow x\perp \overline{M}$.
        \end{proof}
        \begin{proposition}
            $x\perp M\Rightarrow x\perp {\rm span\ }M$.
        \end{proposition}
        \begin{proposition}
            $M^\perp$是闭子空间。
        \end{proposition}
        \begin{proof}
            显然$M^\perp$是向量子空间,设$M^\perp \ni x_n\rightarrow x$,
            \begin{align*}
                &\forall y\in M,0=\left< x_n,y \right>\rightarrow \left< x,y \right>\\
                \Rightarrow & \left< x,y \right>=0\\
                \Rightarrow & x\in M^\perp
            \end{align*}
        \end{proof}
        \subsection{正交与正交基}
        \begin{definition}
            如果$S=\{e_\alpha\}_{\alpha\in\Lambda}\subset X$满足
            \begin{equation*}
                e_\alpha\perp e_\beta,\forall \alpha,\beta\in\Lambda,\alpha\neq\beta
            \end{equation*}
            则称$S$是$X$中的一个正交集,如果$S$还满足$||e_\alpha||=1,\forall \alpha\in\Lambda$,则称之为规范正交集,
            简记为O.N.S.

            若一个正交集$S$满足$S^\perp=\{0\}$,则称$S$完备。
        \end{definition}
        \begin{theorem}
            非平凡内积空间中一定有完备正交集。
        \end{theorem}
        \begin{definition}
            非空集合$X$上的一个偏序“$\leqslant $”是指满足以下条件的一个关系:
            \begin{enumerate}
                \item 传递性:$x\leqslant y,y\leqslant z\Rightarrow x\leqslant z$.
                \item 反身性:$x\leqslant x$.
                \item $x\leqslant y,y\leqslant x\Rightarrow x=y$.
            \end{enumerate}
            $(X,\leqslant )$称为一个偏序集。

            如果$\forall x,y\in X$,$x\leqslant y$或者$y\leqslant x$必有其一,则称
            “$\leqslant $”是$X$上的一个全序。

            对于$Y\subset X$,如果存在$p\in X$使得$\forall y\in Y$有$y\leqslant p$,称$p$是$Y$的一个上界。

            如果存在$m\in X$使得$m\leqslant x\Rightarrow x=m$,则称$m$是$X$上的一个极大元。
        \end{definition}
        \begin{lemma}[Zorn]
            $(X,\leqslant)$是一个偏序集,$X$的每个全序子集都有上界,则$X$必有极大元。
        \end{lemma}
        \begin{proof}
            令$\mathcal{F}$为$X$中正交集全体,$\subset $是集合的包含关系,则
            $(\mathcal{F},\subset )$是偏序集,设$M$是$\mathcal{F}$中的任一全序子集,令
            \begin{equation*}
                P\mathop{=}\limits^{\rm def} \bigcup_{C\in M}C
            \end{equation*}
            于是$P$是$M$的一个上界,这是因为任取$C\in X$都有$C\subset P$.
            由Zorn引理,
            $\mathcal{F}$有极大元$S$,则$S$完备,否则
            $\exists x_0\neq 0{\rm\ s.t.\ }x_0\perp S$,
            $S\cup \{x_0\}\in\mathcal{F}$,与$S$的极大性矛盾。
        \end{proof}
        \begin{definition}
            $(X,\left< \cdot,\cdot \right>)$是内积空间,$S=\{e_\alpha\}_{\alpha\in\Lambda}$是O.N.S.
            如果$\forall x\in X$均可表示为\footnote{其实对于不可数个量相加没有很合适的定义,这里的形式和需要要求
            $\{\left<x,e_\alpha\right>\}_{\alpha\in\Lambda}$只有至多可数个非零。}
            \begin{equation*}
                x=\sum_{\alpha\in \Lambda}\left<x,e_\alpha\right>\cdot e_\alpha
            \end{equation*}
            则称$S$是$X$的一个规范正交基,简称O.N.B.其中的
            $\{ \left<x,e_\alpha\right> \}$称为$x$的Fourier系数。
        \end{definition}
        \begin{theorem}[Bessel不等式]
            $\{e_\alpha\}_{\alpha\in\Lambda}$是O.N.S.则
            \begin{equation*}
                \forall x\in X,\sum_{\alpha\in\Lambda} |\left<x,e_\alpha\right>|^2\leqslant ||x||^2
            \end{equation*}
        \end{theorem}
        \begin{proof}
            \textbf{Step1}:
            \begin{equation*}
                \forall \{\alpha_1,\cdots,\alpha_N\}\subset \Lambda,
                \sum_{k=1}^N |\left< x,e_{\alpha_k} \right>|^2\leqslant ||x||^2
            \end{equation*}
            \begin{align*}
                0\leqslant &\left< x-\sum_{i=1}^N\left< x,e_{\alpha_i} \right>e_{\alpha_i},
                x-\sum_{i=1}^N\left< x,e_{\alpha_k} \right>e_{\alpha_k}\right>\\
                =&||x||^2-\sum_{i=1}^N \left< x,e_{\alpha_i} \right>
                \left< e_{\alpha_i},x \right>
                -\sum_{i=1}^N \overline{\left< x,e_{\alpha_k} \right>}
                \left< e_{\alpha_k},x \right>
                +\sum_{i=1}^N \sum_{k=1}^N 
                \left< e_{\alpha_i},x \right>\overline{\left< x,e_{\alpha_k} \right>}
                \mathop{\left< e_{\alpha_i},e_{\alpha_k} \right>}\limits_{\mbox{这一项}=\delta_{ik}}\\
                =& ||x||^2-\sum_{k=1}^N |\left< x,e_{\alpha_k} \right>|^2
            \end{align*}

            \textbf{Step2}:
            \begin{equation*}
                \Lambda'\mathop{=}\limits^{\rm def}
                \{ \alpha\in \Lambda:\left< x,e_{\alpha} \right>\neq 0\}\mbox{至多可数}
            \end{equation*}
            令
            \begin{equation*}
                \lambda_n\mathop{=}\limits^{\rm def}
                \{ \alpha\in\Lambda:|\left< x,e_{\alpha} \right>|>\frac{1}{n} \},n=1,2,\cdots
            \end{equation*}
            所有的$\Lambda_n$是有限集,否则存在$n_0$使得$\Lambda_{n_0}$是无限集,
            取$N$充分大使得
            \begin{equation*}
                \frac{N}{n_0^2}>||x||^2
            \end{equation*}
            任取$\alpha_1,\cdots,\alpha_N\in\Lambda_{n_0}$,
            \begin{equation*}
                \sum_{k=1}^N |\left< x,e_{\alpha_k} \right>|^2>\frac{N}{n_0^2}>||x||^2
            \end{equation*}
            这与Step1矛盾,进而$\Lambda'=\bigcup_{n=1}^\infty \Lambda_n$至多可数。

            \textbf{Step3}:
            给$\Lambda'$一个排列,即设$\Lambda'=\{\alpha_k\}_{k=1}^\infty$,由
            Step1,$\forall N$,
            \begin{align*}
                & \sum_{k=1}^N |\left< x,e_{\alpha_k} \right>|^2\leqslant ||x||^2\\
                \Rightarrow & \sum_{k=1}^\infty |\left< x,e_{\alpha_k} \right>|^2\leqslant ||x||^2\\
                \Rightarrow & \sum_{\alpha\in\Lambda} |\left< x,e_{\alpha} \right>|^2
                =\sum_{\alpha\in\Lambda'} |\left< x,e_{\alpha} \right>|^2\leqslant ||x||^2
            \end{align*}
        \end{proof}

            引入最佳逼近元的概念:用一组函数的线性组合去逼近一个给定的函数等价于给定$x\in X$,
            $e_1,\cdots,e_n\in X$,求$\lambda_1,\cdots,\lambda_n\in\K $使得
            \begin{equation*}
                ||x-\sum_{k=1}^n \lambda_k e_k||=\fun{inf}{\alpha\in\K^n}||x-\sum_{k=1}^n \alpha_ke_k||
            \end{equation*}
            等价于:令$M\eq{def} {\rm span}\{e_1,\cdots,e_n\}$,求$y_0\in M$使得${\rm dist}(x,y_0)={\rm dist}(x,M)$.
        \begin{theorem}
            赋范空间$(X,||\cdot ||)$,$e_1,\cdots,e_n\in X$,$\forall x\in X$,存在$\lambda_1,\cdots,\lambda_n\in\K$使得
            \begin{equation*}
                ||x-\sum_{k=1}^n \lambda_k e_k||=\fun{inf}{\alpha\in\K^n}||x-\sum_{k=1}^n \alpha_ke_k||
            \end{equation*}
        \end{theorem}
        \begin{proof}
            不妨设$e_1,\cdots,e_n$线性无关,对$x\in X$,定义
            \begin{equation*}
                F:\K^n\rightarrow \R,\alpha\mapsto ||x-\sum_{k=1}^n \alpha_k e_k||
            \end{equation*}
            则:
            \begin{enumerate}
                \item $F$连续;
                \item \begin{equation*}
                    F(\alpha)\geqslant ||\sum_{k=1}^n \alpha_k e_k||-||x||
                \end{equation*}
            \end{enumerate}
            令
                \begin{equation*}
                    |||\alpha|||\defeq ||\sum_{k=1}^n \alpha_k e_k||
                \end{equation*}
                于是$|||\cdot|||$是$\K^n$上的范数,由于有限维空间上范数等价,
                \begin{equation*}
                    \exists C>0{\rm\ s.t.\ }|||\alpha|||\geqslant C|\alpha|,\ \forall \alpha\in\K^n
                \end{equation*}
                \begin{equation*}
                    \mathop{\Rightarrow}\limits^{2} F(\alpha)\geqslant C|\alpha|-||x||\rightarrow +\infty{\rm\ as\ }|\alpha|\rightarrow\infty
                \end{equation*}
                因此$F$在$\K^n$上可以取到最小值。
        \end{proof}
        \begin{lemma}[变分引理]
            $H$是Hilbert空间,$M$是非空闭凸集,则
            \begin{equation*}
                \forall x\in X,\exists ! y\in M{\rm\ s.t.\ }||x-y||={\rm dist}(x,M)
            \end{equation*}
        \end{lemma}
        \begin{proof}
            设
            \begin{equation*}
                d\mathop{=}\limits^{\rm def}{\rm dist}(x,M)=
                \mathop{\rm inf}\limits_{\zeta\in M}||x-\zeta||
            \end{equation*}
            则存在$y_n\in M,n=1,2,\cdots{\rm\ s.t.\ }d\leqslant ||y_n-x||\leqslant d+\frac{1}{n}$.

            下面证明$\{y_n\}_{n=1}^\infty$是基本列。由P.L.
            \begin{align*}
                &||(y_n-x)+(y_m-x)||^2+||(y_n-x)-(y_m-x)||^2
                = 2( ||y_n-x||^2+||y_m-x||^2 )\\
                \Rightarrow& 
                ||y_n-y_m||^2=
                2( ||y_n-x||^2+||y_m-x||^2 )-4||\frac{y_n+y_m}{2}-x||^2\\
                &\leqslant 2( (d+\frac{1}{n})^2+(d+\frac{1}{m})^2 )-4d^2\rightarrow 0{\rm\ as\ }n,m\rightarrow\infty
            \end{align*}
            $H$完备,设$y_n\rightarrow y$,因为$M$是闭集,故$y\in M$,因此:
            \begin{equation*}
                d\leqslant||y-x||\leqslant ||y-y_n||+||y_n-x||\leqslant ||y-y_n||+d+\frac{1}{n}
            \end{equation*}
            $n$充分大时,右式$\rightarrow d^+$,于是$||y-x||=d$.

            唯一性:假设还有$y'\in M{\rm\ s.t.\ }||y'-x||=d$,
            \begin{equation*}
                ||y'-y||^2=2( ||y'-x||^2+||y-x||^2 )-4||\frac{y'+y}{2}-x||^2\leqslant 4d^2-4d^2=0
            \end{equation*}
            所以$y'=y$.
        \end{proof}
        \begin{theorem}[正交分解]
            $H$是Hilbert空间,$M$是闭子空间,则$H=M\oplus M^\perp$,即
            $\forall x\in H$,存在唯一的$y\in M$和$\zeta\in M^{\perp}$使得$x=y+\zeta$.
        \end{theorem}
        \begin{proof}
            $\forall x\in H$,由变分引理,存在唯一$y\in M$使得
            \begin{equation*}
                ||x-y||={\rm dist}(x,M)
            \end{equation*}
            Claim:$x-y\in M^\perp$,$\forall 0\neq w\in M,\forall \lambda\in\mathbb{K}$,
            \begin{align*}
                \Rightarrow & y+\lambda w\in M\\
                \Rightarrow & d^2\leqslant ||x-(y+\lambda w)||^2
                =||x-y||^2-2{\rm Re}(\overline{\lambda}\left< x-y,w \right>)+|\lambda|^2||w||^2
            \end{align*}
            取$\lambda=\frac{ \left<x-y,w\right> }{||w||^2}$,
            \begin{align*}
                &d^2\leqslant ||x-y||^2-2\frac{ |\left<x-y,w\right>|^2 }{||w||^2}+\frac{ |\left<x-y,w\right>|^2 }{||w||^2}
                =d^2-\frac{ |\left<x-y,w\right>|^2 }{||w||^2}\\
                \Rightarrow & \left<x-y,w\right>=0\\
                \Rightarrow & x-y\in M^\perp
            \end{align*}
        \end{proof}
        \begin{definition}
            映射$P_M:H\rightarrow M,x\mapsto y$,这里$y$是满足变分引理的$y$,称之为$x$在$M$中的最佳逼近元,此映射被称为
            $H$到$M$的正交投影。
        \end{definition}
        \begin{corollary}
            \begin{enumerate}
                \item $P_M(x)\in M$,$x-P_M(x)\in M^\perp$.
                \item ${\rm Im}(P_M)=M$,${\rm Ker}(P_M)=M^\perp$.
                \item $||x-P_M(x)||={\rm dist}(x,M)$.
                \item $P^2_M=P_M$.
                \item $||P_M(x)||\leqslant ||x||$.
                \item $I-P_M=P_{M^\perp}$.
            \end{enumerate}
        \end{corollary}

            在Bessel不等式的证明中,
            $\sum_{\alpha\in\Lambda'}|\left< x,e_{\alpha} \right>|^2$和$\Lambda'$的排列无关,那
            $\sum_{\alpha\in\Lambda'}\left< x,e_{\alpha} \right>e_\alpha$呢?
        \begin{lemma}
            $H$是Hilbert空间,$\{e_k\}_{k=1}^\infty$是O.N.S.则
            \begin{equation*}
                \forall x\in H,\sum_{k=1}^\infty \left<x,e_k\right>e_k\in H 
            \end{equation*}
            令$M=\overline{ {\rm span}\{e_k\}_{k=1}^\infty }$,则
            \begin{equation*}
                \sum_{k=1}^\infty \left<x,e_k\right>e_k=P_M(x)
            \end{equation*}
        \end{lemma}
        \begin{proof}
            由Bessel,
            \begin{align*}
                &\sum_{k=1}^\infty |\left< x,e_k \right>|^2\leqslant ||x||^2\\
                \Rightarrow & ||\sum_{k=n}^m \left<x,e_k\right>e_k||^2=\sum_[k=n]^\infty |\left< x,e_k \right>|^2\rightarrow 0{\rm\ as\ }n,m\rightarrow\infty\\
                \Rightarrow & \left\{ \sum_{k=1}^n \left<x,e_k\right>e_k\right\}_{n=1}^\infty\mbox{是$H$中的基本列}\\
                \Rightarrow & \sum_{k=1}^\infty \left<x,e_k\right>e_k
                \mathop{=}\limits^{\rm def} \mathop{\rm lim}\limits^{n\rightarrow\infty}\sum_{k=1}^n 
                \left<x,e_k\right>e_k\in H
            \end{align*}
            进而,由于
            \begin{align*}
                &\left< x-\sum_{k=1}^\infty \left<x,e_k\right>e_k,e_m \right>=0,\forall m\\
                \Rightarrow & x-\sum_{k=1}^\infty \left<x,e_k\right>e_k\in M^\perp\\
                \Rightarrow & \sum_{k=1}^\infty \left<x,e_k\right>e_k=P_M(x)
            \end{align*}
        \end{proof}
        \begin{corollary}
            对$\mathbb{N}$上任一置换$\sigma:\mathbb{N}\rightarrow \mathbb{N}$,
            \begin{equation*}
                \sum_{k=1}^\infty \left< x,e_{\sigma(k)} \right>e_{\sigma(k)}=\sum_{k=1}^\infty 
                \left< x,e_k \right>e_k
            \end{equation*}
        \end{corollary}
        \begin{proof}
            令$M\mathop{=}\limits^{\rm def}\overline{ {\rm span}\{e_k\}_{k=1}^\infty }$,
            $\tilde{M}\mathop{=}\limits^{\rm def}\overline{ {\rm span}\{e_{\sigma(k)}\}_{k=1}^\infty }$,
            于是$M=\tilde{M}$,
            \begin{equation*}
                \sum_{k=1}^\infty 
                \left< x,e_{\sigma(k)} \right>e_{\sigma(k)}
                =P_{\tilde{M}}(x)=P_M(x)=\sum_{k=1}^\infty \left< x,e_{k} \right>
            \end{equation*}
        \end{proof}
        \begin{corollary}
            $H$是Hilbert空间,$\{e_\alpha\}_{\alpha\in\Lambda}$是O.N.S.则
            $\forall x\in H$,
            \begin{equation*}
                \sum_{\alpha\in\Lambda}\left< x,e_{\alpha} \right>e_{\alpha}\in H
            \end{equation*}
            且
            \begin{equation*}
                ||x-\sum_{\alpha\in\Lambda}\left<x,e_\alpha\right>e_{\alpha}||^2
                =||x||^2-\sum_{\alpha\in\Lambda}|\left<x,e_\alpha\right>|^2
            \end{equation*}
        \end{corollary}
        \begin{proof}
            设
            \begin{equation*}
                \Lambda'\mathop{=}\limits^{\rm def}
                \{ \alpha\in\Lambda:\left< x,e_{\alpha} \right>\neq 0\}
                =\{\alpha_k\}_{k=1}^\infty \mbox{(任一顺序)}
            \end{equation*}
            则
            \begin{equation*}
                \sum_{\alpha\in\Lambda}\left<x,e_\alpha\right>e_{\alpha}
                =\sum_{k=1}^\infty \left<x,e_{\alpha_k}\right>e_{\alpha_k}\in H
            \end{equation*}
            且
            \begin{align*}
                &||x-\sum_{k=1}^N \left< x,e_{\alpha_k} \right>e_{\alpha_k}||^2
                =||x||^2-\sum_{k=1}^N |\left< x,e_{\alpha_k} \right>|^2\\
                \mathop{\Rightarrow}\limits^{N\rightarrow\infty}_{\mbox{范数连续}}
                &||x-\sum_{k=1}^N \left< x,e_{\alpha_k} \right>e_{\alpha_k}||^2
                =||x||^2-\sum_{k=1}^\infty |\left< x,e_{\alpha_k} \right>|^2
            \end{align*}
        \end{proof}
        \begin{theorem}
            $H$是Hilbert空间,$S=\{e_\alpha\}_{\alpha\in\Lambda}$是O.N.S.则
            下列命题等价:
            \begin{enumerate}
                \item $S^{\perp}=\{0\}$.
                \item $S$是O.N.B.
                \item $S$满足:
                \begin{equation*}
                    \forall x\in H,||x||^2=\sum_{\alpha\in\Lambda}|\left<x,e_\alpha\right>|^2
                \end{equation*}
                这被称为Parseval恒等式,简称P.I.
            \end{enumerate}
        \end{theorem}
        \begin{proof}
            \textbf{Step1}:$S^\perp=\{0\}\Rightarrow S$是O.N.B.假设不然,则
            \begin{align*}
                &\exists x_0\in H{\rm\ s.t.\ }\sum_{\alpha\in\Lambda}\left<x_0,e_\alpha\right>e_{\alpha}
                \neq x_0\\
                &\forall \beta\in\Lambda,
                \left< x_0-\sum_{\alpha\in\Lambda}\left<x_0,e_\alpha\right>e_{\alpha},e_\beta \right>
                =\left<x_0,e_\beta\right>-\left<x_0,e_\beta\right>=0\\
                \Rightarrow & 0\neq x_0-\sum_{\alpha\in\Lambda}\left<x_0,e_\alpha\right>e_{\alpha}\perp S\\
                \Rightarrow & S^\perp\neq \{0\}
            \end{align*}
            矛盾。

            \textbf{Step2}:由推论1.5.3,$S$是O.N.B.$\Rightarrow $满足Parseval。

            \textbf{Step3}:满足Parseval$\Rightarrow S^\perp=\{0\}$,否则
            \begin{align*}
                &\exists x_0\neq 0{\rm\ s.t.\ }x_0\perp S\\
                \Rightarrow & \left<x_0,e_\alpha\right>=0,\forall \alpha\in\Lambda\\
                \Rightarrow & \sum_{\alpha\in\Lambda}|\left<x_0,e_\alpha\right>|^2=0\\
                \mathop{\Rightarrow}\limits^{\rm Parseval}
                ||x_0||^2=0
            \end{align*}
            矛盾。
        \end{proof}
        \begin{example}
            数列空间$\ell^2$,$e_n$的第$n$个分量为$1$,其余分量为$0$,
            则$\{e_n\}_{n=1}^\infty$是$\ell^2$的一个O.N.B.

            但它不是$\ell^2$的Hamel基。\footnote{向量空间的Hamel基是指任何一个向量可以写成有限个基中向量的线性组合,此处不满足“有限”的条件。}
        \end{example}
        \begin{corollary}
            非平凡Hilbert空间一定有O.N.B.
        \end{corollary}
        \subsection{正交化与同构}
        \begin{theorem}[Gram-Schmidt正交化]
            $(X,\left< \cdot,\cdot \right>)$是内积空间,
            $\{x_n\}_{n=1}^\infty$线性无关,则存在一列$\{e_n\}_{n=1}^\infty$相互正交,且
            \begin{equation*}
                \forall n,{\rm span}\{e_k\}_{k=1}^n={\rm span}\{x_k\}_{k=1}^n
            \end{equation*}
        \end{theorem}
        \begin{proof}
            只需按照以下步骤构造即可:
            \begin{equation*}
                \begin{array}{ll}
                    y_1=x_1& e_1=\frac{y_1}{||y_1||}\\
                    y_2=x_2-\left<x_2,e_1\right>e_1& e_2=\frac{y_2}{||y_2||}\\
                    \vdots&\\
                    y_n=x_n-\sum_{k=1}^{n-1}\left<x_n,e_k\right>e_k& e_n=\frac{y_n}{||y_n||}
                \end{array}
            \end{equation*}
            $\forall m\neq n$,设$m<n$,
            \begin{align*}
                \left< e_m,e_n \right>&=\frac{1}{||y_n||}\left< e_m,x-\sum_{k=1}^{n-1}\left<x_n,e_k\right>e_k \right>\\
                &=\frac{1}{||y_n||}\left( \left< e_m,x-\sum_{k=1}^{n-1}\left<e_m,x\right>e_k \right>-\overline{\left< x_n,e_m \right>} \right)=0
            \end{align*}
            且
            \begin{equation*}
                y_k=x_k-\sum_{i=1}^{k-1}\left<x_k,e_i\right>e_i=x_k+\sum_{j=1}^{k-1}\alpha_{kj}x_j
            \end{equation*}
            得到
            \begin{equation*}
                \begin{pmatrix}
                    y_1\\y_2\\\vdots\\y_n
                \end{pmatrix}=
                \begin{pmatrix}
                    1&&&\\
                    \alpha_{21}&1&&\\
                    \vdots&&\ddots&\\
                    \alpha_{n1}&\alpha_{n2}&&1
                \end{pmatrix}
                \begin{pmatrix}
                    x_1\\x_2\\\vdots\\x_n
                \end{pmatrix}
            \end{equation*}
            所以${\rm span}\{ y_k \}_{k=1}^n={\rm span}\{ x_k \}_{k=1}^n$,进而
            ${\rm span}\{ e_k \}_{k=1}^n={\rm span}\{ x_k \}_{k=1}^n$.
        \end{proof}
        \begin{definition}
            $(X_1,\left< \cdot,\cdot \right>_1)$和$(X_2,\left< \cdot,\cdot \right>_2)$是内积空间,
            如果存在线性同构$T:X_1\rightarrow X_2$使得
            \begin{equation*}
                \left< T(x),T(y) \right>_2=\left< x,y \right>_1,\forall x,y\in X_1
            \end{equation*}
            则称$X_1$和$X_2$作为内积空间同构,记为$X_1\cong X_2$.
        \end{definition}
        \begin{theorem}
            $H$是Hilbert空间,$H$可分$\Leftrightarrow H$有可数的O.N.B. 
        \end{theorem}
        \begin{proof}
            必要性:如果${\rm dim\ }H\neq \infty$则显然,下面假设${\rm dim\ }H=\infty$,
            \begin{align*}
                \mbox{可分}\Rightarrow &
                \exists \{x_n\}_{n=1}^\infty \subset H{\rm\ s.t.\ }\overline{\{x_n\}_{n=1}^\infty}=H\\
                \Rightarrow &\exists \{y_n\}_{n=1}^\infty \subset\{x_n\}_{n=1}^\infty \mbox{线性无关}\\
                \Rightarrow & \exists \{e_n\}_{n=1}^\infty \mbox{正交且}
                {\rm span}\{ e_n \}_{n=1}^\infty={\rm span}\{ x_n \}_{k=n}^\infty\\
                \Rightarrow &
                \overline{ {\rm span}\{ e_n \}_{n=1}^\infty }
                =\overline{ {\rm span}\{ x_n \}_{n=1}^\infty }=H\\
                \Rightarrow & ( \{ e_n \}_{n=1}^\infty )^\perp=\{0\}\\
                \Rightarrow & \{ e_n \}_{n=1}^\infty\mbox{是O.N.B.}
            \end{align*}

            充分性:令
            \begin{equation*}
                {\rm span}^{\mathbb{Q}}(\{ e_n \}_{n=1}^\infty)\mathop{=}\limits^{\rm def}
                \{ e_n \}_{n=1}^\infty\mbox{中向量以$\mathbb{Q}+{\rm i}\mathbb{Q}$中元素为系数的线性组合全体}
            \end{equation*}
            于是${\rm span}^{\mathbb{Q}}(\{ e_n \}_{n=1}^\infty)$可数,下证
            $\overline{{\rm span}^{\mathbb{Q}}(\{ e_n \}_{n=1}^\infty)}=H$.
            \begin{align*}
                &\forall \varepsilon>0,\forall n,\exists \alpha_n\in\mathbb{Q}+{\rm i}\mathbb{Q}{\rm\ s.t.\ }
                |\alpha_n-\left<x,e_n\right>|<\frac{\varepsilon}{2^{n+1}}\\
                \Rightarrow &
                || \sum_{n=1}^N \left<x,e_n\right>e_n-\sum_{n=1}^N \alpha_n e_n ||^2
                =\sum_{n=1}^N |\left<x,e_n\right>-\alpha_n|^2<\frac{\varepsilon^2}{4},\forall N
            \end{align*}
            而当$N$充分大时,
            \begin{equation*}
                ||\sum_{n=1}^N \left<x,e_n\right>e_n-x||<\frac{\varepsilon}{2}
            \end{equation*}
            于是
            \begin{equation*}
                ||\sum_{n=1}^N \alpha_n e_n-x||<\varepsilon
            \end{equation*}
        \end{proof}
        \begin{theorem}
            \begin{enumerate}
                \item $n$维Hilbert空间$\cong \mathbb{K}^n$.
                \item 无穷维可分Hilbert空间$\cong \ell^2$.
            \end{enumerate}
        \end{theorem}
        \begin{proof}
            1以前已经证明过,只说明一下2:
            设$\{ e_n \}_{n=1}^\infty$是$H$的可数O.N.B.定义
            \begin{equation*}
                T:H\rightarrow \ell^2,x\mapsto \{ \left< x,e_n \right> \}_{n=1}^\infty
            \end{equation*}
            \begin{enumerate}[$1^\circ$]
                \item 线性:显然。
                \item 等距:由Parseval,\begin{equation*}
                    \sum_{n=1}^\infty |\left< x,e_n \right>|^2=||x||^2
                \end{equation*}
                \item 单射:由等距可得。
                \item 满射:\begin{align*}
                    &\forall a\in \ell^2,
                    ||\sum_{k=n}^m a_k e_k||^2=\sum_{k=n}^m |a_k|^2\rightarrow 0{\rm\ as\ }n,m\rightarrow\infty\\
                    \Rightarrow &
                    \sum_{k=1}^n a_ke_n\rightarrow x\in H{\rm\ and\ }\left< x,e_k \right>=a_k\\
                    \Rightarrow &
                    T(x)=a
                \end{align*}
                \item 保内积:\begin{align*}
                    \left< x,y \right>&= 
                    \left< \sum_{n=1}^\infty \left< x,e_n \right>e_n,\sum_{m=1}^\infty \left< y,e_m \right>e_m \right>\\
                    &= \sum_{n,m}\left< x,e_n \right>\overline{\left< y,e_m \right>}\left< e_n,e_m \right>\\
                    &= \sum_{n=1}^\infty \left< x,e_n \right>\overline{\left< y,e_n \right>}\\
                    &= \left< T(x),T(y) \right>_{\ell^2}
                \end{align*}
            \end{enumerate}
        \end{proof}
    
\section{应用:Fourier级数}
    \begin{definition}
        设
        \begin{equation*}
            \Pi\mathop{=}\limits^{\rm def}\{ z\in\mathbb{C}:|z|=1 \}
        \end{equation*}
        对于$\Pi$上的函数$F$,令
        \begin{equation*}
            f(t)\mathop{=}\limits^{\rm def}F( {\rm e}^{2\pi{\rm i}t} ),t\in\mathbb{R}
        \end{equation*}
        于是$f$是$\mathbb{R}$上周期为$1$的周期函数。令
        \begin{equation*}
            e_k(t)\mathop{=}\limits^{\rm def}{\rm e}^{2\pi {\rm i}kt},t\in [-\frac{1}{2},\frac{1}{2}),k=0,\pm 1,\pm 2,\cdots
        \end{equation*}
        于是$\{e_k\}_{k=-\infty}^\infty$是$L^2(\Pi)$中的O.N.S.称为三角函数系。
        \begin{equation*}
            \hat{f}(k)\mathop{=}\limits^{\rm def}\int_{-\frac{1}{2}}^{\frac{1}{2}}
            f(t){\rm e}^{-2\pi {\rm i}kt}{\rm d}t=\left< f,e_k \right>
        \end{equation*}
        \begin{equation*}
            f(x)\sim \sum_{k=-\infty}^\infty \hat{f}(k){\rm e}^{2\pi{\rm i}kx}
            =\sum_{-\infty}^\infty \left<f,e_k\right>e_k
        \end{equation*}
    \end{definition}
    \begin{theorem}
        $\forall f\in L^2(\Pi)$,
        \begin{equation*}
            ||S_N(f)-f||_2\rightarrow 0{\rm\ as\ }N\rightarrow\infty
        \end{equation*}
        其中
        \begin{equation*}
            S_N(f)(x)\mathop{=}\limits^{\rm def}\sum_{k=-N}^N \hat{f}(k){\rm e}^{2\pi{\rm i}kx}
        \end{equation*}
    \end{theorem}
    \begin{proof}
        Thm$\Leftrightarrow \{e_k\}_{k=-\infty}^\infty$是$L^2(\Pi)$中的O.N.B.
        $\Leftrightarrow (\{e_k\}_{k=-\infty}^\infty)^\perp=\{0\} $
        $\Leftrightarrow \overline{{\rm span}\{e_k\}_{k=-\infty}^\infty}=L^2(\Pi)$.
        从而约化为证明:$\forall f\in L^2(\Pi)$可由三角多项式以$L^2$范数逼近。
        \begin{equation*}
            S_N(f)(x)=(f* D_N)(x){\rm\ with\ }
            D_N(t)=\sum_{k=-N}^N {\rm e}^{2\pi {\rm i}kt}=\frac{ {\rm sin}[ (2N+1)\pi t ] }{{\rm sin}(\pi t)}
        \end{equation*}
        令
        \begin{align*}
            \sigma_N\mathop{=}\limits^{\rm def}& \frac{1}{N+1}\sum_{k=0}^N S_k(f)\\
            =& f *\left( \frac{1}{N+1}\sum_{k=1}^N D_k \right)=f* F_N
        \end{align*}
        这里$F_N(t)$是Fejér核:
        \begin{equation*}
            F_N(t)=\frac{1}{N+1}\sum_{k=1}^N D_k(t)=\frac{1}{N+1} \frac{ {\rm sin}^2 [(N+1)\pi t] }{{\rm sin}^2 (\pi t)}
        \end{equation*}
        \begin{lemma}
            \begin{enumerate}
                \item \begin{equation*}
                    \int_{-\frac{1}{2}}^{\frac{1}{2}}F_N(t){\rm d}t=1
                \end{equation*}
                \item $\forall \delta>0$,
                    \begin{equation*}
                        \mathop{\rm lim}\limits_{N\rightarrow\infty}
                        \int_{\delta<|t|<\frac{1}{2}}F_N(t){\rm d}t=0
                    \end{equation*}
            \end{enumerate}
            \begin{proof}
                1直接计算验证,只说明一下2:当$\delta<|t|<\frac{1}{2}$时,
                \begin{equation*}
                    0\leqslant F_N(t)\leqslant \frac{1}{N+1}\frac{1}{{\rm sin}^2(\pi\delta)}\rightarrow 0{\rm\ as\ }N\rightarrow\infty
                \end{equation*}
            \end{proof}
        \end{lemma}
        \begin{lemma}
            $\forall f\in L^2(\Pi)$,
            \begin{equation*}
                ||\sigma_N(f)-f||_2\rightarrow 0{\rm\ as\ }N\rightarrow \infty
            \end{equation*}
            \begin{proof}
                \begin{align*}
                    ||\sigma_N(f)-f||_2=& \sqrt{ \int_{-\frac{1}{2}}^{\frac{1}{2}} 
                    \left| \int_{-\frac{1}{2}}^{\frac{1}{2}}[ f(x-t)-f(x) ]F_N(t){\rm d}t \right|^2{\rm d}x}\\
                    \mathop{\leqslant}\limits^{\rm Minkowski}&
                    \int_{-\frac{1}{2}}^{\frac{1}{2}}|| f(\cdot-t)-f(\cdot) ||_2F_N(t){\rm d}t\\
                    =& \int_{ |t|\leqslant \delta }\cdots+\int_{\delta<|t|<\frac{1}{2}}\cdots\\
                    \leqslant &
                    \mathop{\int_{ |t|\leqslant \delta }\cdots}\limits_{\mbox{积分的绝对连续性}}+
                    \mathop{2||f||_2\int_{\delta<|t|<\frac{1}{2}}F_N(t){\rm d}t}\limits_{\mbox{引理1.6.1}}\rightarrow 0{\rm\ as\ }N\rightarrow \infty
                \end{align*}
            \end{proof}
        \end{lemma}
        注意
        \begin{equation*}
            \sigma_N(f)(x)=\sum_{k=-N}^N \left( 1-\frac{|k|}{N} \right)\hat{f}(k){\rm e}^{2\pi{\rm i}kx}
        \end{equation*}
        上述引理说明三角多项式全体在$L^2(\Pi)$中稠密。
    \end{proof}

	%第二章
	\chapterimage{empty.jpg}
	\chapter{线性算子与线性泛函}
%\begin{center}
%    线性算子=线性映射
%\end{center}
%\rightline{2023.10.27}
%\vspace{-5pt}
%\begin{center}
%    \pgfornament[width=0.36\linewidth,color=lsp]{88}
%\end{center}

\section{线性算子}
\subsection{有界线性算子}
    \begin{definition}
        $X$和$Y$是向量空间,如果映射$T:X\rightarrow Y$满足
        \begin{equation*}
            T(\alpha x+\beta y)=\alpha Tx+\beta Ty,\forall x,y\in X,\forall \alpha,\beta\in\mathbb{K}
        \end{equation*}
        则称$T$是线性算子,简称L.O.
        
        特别地,如果$Y=\mathbb{K}$,则称$T$是线性泛函。
    \end{definition}
    \begin{example}
        微分算子:开集$\Omega\subset \mathbb{R}^n$,$X=Y=C^\infty (\Omega)$,定义
        \begin{equation*}
            T\mathop{=}\limits^{\rm def}
            \sum_{|\alpha|\leqslant m}a_\alpha \partial^\alpha
        \end{equation*}
    \end{example}
    \begin{example}
        积分算子:$X=L^p(\Omega)$,$Y=L(\Omega)$,即全体可测函数。
        $K(\cdot,\cdot)$是$\Omega\times\Omega$上的可测函数,称之为积分核。
        \begin{equation*}
            T:u(x)\mapsto \int_\Omega K(x,y)u(y){\rm d}y
        \end{equation*}
        如:Poisson积分:
        \begin{equation*}
            P[u](\zeta)\mathop{=}\limits^{\rm def} \frac{1}{2\pi}\int_0^{2\pi}
            \frac{1-|\zeta|^2}{|1-\zeta {\rm e}^{-{\rm i}\theta}|^2}u( {\rm e}^{{\rm i}\theta}){\rm d}\theta
        \end{equation*}
        Fourier变换:
        \begin{equation*}
            (\mathcal{F}u)(x)\mathop{=}\limits^{\rm def}
            \int_{\mathbb{R}^n} {\rm e}^{-2\pi{\rm i}x\cdot y}u(y){\rm d}y
        \end{equation*}
    \end{example}
    \begin{example}
        非线性的例子:
        \begin{equation*}
            f(u)\mathop{=}\limits^{\rm def} \int_{\Omega}u^2(x){\rm d}x
        \end{equation*}
    \end{example}
    \begin{definition}
        $(X,||\cdot||_X)$和$(Y,||\cdot||_Y)$是赋范空间,
        $T:X\rightarrow Y$是L.O.如果存在$C>0$使得
        \begin{equation*}
            ||Tx||_Y\leqslant C||x||_X,\forall x\in X
        \end{equation*}
        则称$T$有界。
    \end{definition}

    \begin{theorem}
        赋范空间$(X,\ms{\cdot}_{X}),(Y,\ms{\cdot}_{Y})$,$T:X\rightarrow Y$是线性映射,
        $T$有界$\Leftrightarrow T$连续$\Leftrightarrow T$在$0$处连续。
    \end{theorem}
    \begin{proof}
        有界$\Rightarrow $连续:
        \begin{equation*}
            ||x_n-x||_X\rightarrow 0\Rightarrow ||Tx_a-Tx||\leqslant C||x_n-x||\rightarrow 0
        \end{equation*}
        
        在$0$处连续$\Rightarrow $有界:假设$T$在$0$连续,但无界,则
        \begin{equation*}
            \forall n,\exists x_n\in X{\rm\ s.t.\ }||Tx_n||_Y>n||x_n||_X
        \end{equation*}
        令$y_n=\frac{1}{n}\frac{x_n}{||x_n||_X}$,则$y_n\rightarrow 0$但
        $|Ty_n|_Y\geqslant 1,\forall n$,这与$T$在$0$处连续矛盾。
    \end{proof}
    \begin{theorem}
        有限维赋范空间之间的线性算子一定有界。
    \end{theorem}
    \begin{proof}
        先假设$X=\mathbb{K}^n$,$Y=\mathbb{K}^m$,则
        \begin{equation*}
            Tx=Ax{\rm\ for\ some\ }A=(a_{ij})_{1\leqslant i\leqslant m,1\leqslant j\leqslant n}
        \end{equation*}
        \begin{align*}
            \Rightarrow ||Tx||_{\mathbb{K}^m}
            =&\left(
                \sum_{i=1}^m 
                \left|
                    \sum_{j=1}^n a_{ij}x_j
                \right|^2
            \right)^{\frac{1}{2}}\\
            \mathop{\leqslant}\limits^{\rm C-S}&
            \left[
                \sum_{i=1}^m
                \left(
                    \sum_{j=1}^n |a_{ij}|^2
                \right)
                \left(
                    \sum_{j=1}^n |x_{j}|^2
                \right)
            \right]^{\frac{1}{2}}\\
            =&\left(
                \sum_{i=1}^m\sum_{j=1}^n |a_{ij}|^2
            \right)^{\frac{1}{2}}||x||_{\mathbb{K}^n}
        \end{align*}

        一般情形:把$X$和$Y$同胚到一个有限维线性空间即可:
        \begin{equation*}
            \begin{matrix}
                X&\mathop{\rightarrow}\limits^{T}&Y\\
                \varphi\downarrow&&\downarrow \psi\\
                \K^n&\mathop{\rightarrow}\limits^{\tilde{T}}&\K^m
            \end{matrix}
        \end{equation*}
        $T=\psi^{-1}\circ \tilde{T}\circ \varphi$.
    \end{proof}

    \begin{proposition}
        证明:${\rm dim\ }X<\infty$,$T:X\rightarrow Y$是线性算子,则$T$有界。(作业)
    \end{proposition}
    \begin{proof}
        设$\{e_1,\cdots,e_n\}$是$X$上的一组{\rm Hamel}基,
        记$M=\fun{max}{}\{ ||Te_1||_Y,\cdots,||Te_n||_Y \}$,
        那么$\forall x\in X$,$x=\sum_{i=1}^n x_ie_i,x_i\in\K$,
        \begin{align*}
            ||Tx||_Y&=||x_1T(e_1)+\cdots+x_nT(x_n)||_Y\\
            &\leqslant |x_1|\cdot||T(e_1)||_Y+\cdots+|x_n|\cdot ||T(e_n)||_Y\\
            &\leqslant M(|x_1|+\cdots+|x_n|)
        \end{align*}
        由于有限维线性空间上范数都等价,不妨$||x||_X\defeq |x_1|+\cdots+|x_n|$,则
        $||Tx||_Y\leqslant M||x||_X$,$T$有界。
    \end{proof}

    \begin{example}[无界算子的例子]
        $X=C^1[0,1]$,$Y=C[0,1]$,赋以一致范数,
        $T=\frac{{\rm d}}{{\rm d}t}$,设
        \begin{equation*}
            u_n(t)=t^n,t\in[0,1],n=1,2,\cdots
        \end{equation*}
        \begin{align*}
            &\Rightarrow ||u_n||=1,||Tu_n||=n\\
            &\Rightarrow \frac{||Tu_n||}{||u_n||}\rightarrow \infty\\
            &\Rightarrow T\mbox{无界}
        \end{align*}
    \end{example}
\subsection{算子范数}
    \begin{definition}
        $X$到$Y$的有界线性算子全体记作$\mathcal{L}(X,Y)$,对于
        $T\in \mathcal{L}(X,Y)$,
        \begin{equation*}
            ||T||_{X\rightarrow Y}
            \mathop{=}\limits^{\rm def}
            \mathop{\mathop{\rm sup}\limits_{x\in X}}\limits_{x\neq 0}
            \frac{||Tx||_Y}{||x||_X}=
            \mathop{\mathop{\rm sup}\limits_{x\in X}}\limits_{||x||=1}||Tx||_Y
        \end{equation*}
        称为$T$的算子范数。
    \end{definition}
    \begin{example}
        Hilbert空间$X$的一个闭子空间为$M$,正交投影$P_M:X\rightarrow M$
        有$||P_M||_{X\rightarrow M}=1$.
    \end{example}
    \begin{theorem}
        $(X,||\cdot||_X)$,$(Y,||\cdot||_Y)$,
        $ (\mathcal{L}(X,Y),||\cdot||_{X\rightarrow Y}) $是赋范空间。
        进而如果$Y$是Banach空间,则$\mathcal{L}(X,Y)$是Banach空间。特别地,
        $X^*\defeq \{X\mbox{上全体有界线性泛函}\}$是Banach空间。
    \end{theorem}
    \begin{proof}
        取$\mathcal{L}(X,Y)$上的柯西列$\{f_n\}$,
        \begin{align*}
            ||f_n-f_m||\leqslant \varepsilon\Rightarrow &
            \fun{sup}{||x||=1} ||f_n(x)-f_m(x)|| \leqslant \varepsilon\\
            \Rightarrow &\forall ||x||=1,\{f_n(x)\}\mbox{是$Y$上的柯西列}
        \end{align*}
        于是定义:
        \begin{equation*}
            f:x\mapsto \fun{lim}{n\rightarrow\infty} f_n(x)
        \end{equation*}
        $f$显然是线性映射,
        \begin{equation*}
            ||f_n-f||=\fun{sup}{||x||=1}||f_n(x)-f(x)||\rightarrow 0
        \end{equation*}        
        现只需证明$||f||$有界:$\{f_n\}$作为柯西列是有界的,不妨$||f_n||\leqslant M$,则
        \begin{equation*}
            ||f||=\fun{sup}{||x||=1}||f(x)||
            =\fun{sup}{||x||=1}\fun{lim}{n\rightarrow \infty}||f_n(x)||
            \leqslant 
            \fun{sup}{||x||=1}\fun{liminf}{n\rightarrow \infty}||f_n||\leqslant M<+\infty
        \end{equation*}
        所以$f$有界。
    \end{proof}

\section{Riesz表示定理}
    设$H$是Hilbert空间,给定$y\in H$,定义$f_y:H\rightarrow\mathbb{K},x\mapsto \left<x,y\right>$,
    则由C-S不等式:$|f_y(x)|\leqslant ||y||||x||$,
    于是$f_y\in H^*$且$||f_y\||\leqslant ||y||$.

    问:是否$\forall f\in H^*$,$f(x)=\left<x,y\right>{\rm\ for\ some\ }y\in H$?
    \begin{theorem}[Riesz表示定理]
        $H$是Hilbert空间,$\forall f\in H^*$,存在唯一$y_f\in H$使得
        \begin{equation*}
            f(x)=\left<x,y_f\right>,x\in H
        \end{equation*}
        且$||y_f||=||f||$.
    \end{theorem}
    \begin{proof}
        分析:$\mathbb{R}^n$上的线性代数$l(x)=\sum_{i=1}^n \alpha_ix_i=\left<x,a\right>$,
        如果$f(x)=\left<x,y\right>,f(x)=0\Rightarrow y\perp x\Rightarrow y\perp {\rm Ker}(f)$.

        存在性:如果$f=0\Rightarrow y_f=0$,下设$f\neq 0$,于是
        ${\rm Ker}(f)\neq H$,且是闭子空间,因为$f\in H^*$.于是
        存在$y_0\in {\rm Ker}(f)^\perp{\rm\ with\ }||y_0||=1$.
        \begin{align*}
            \forall x\in H,f( x-\frac{f(x)}{f(y_0)}y_0 )=0&\Rightarrow
            x-\frac{f(x)}{f(y_0)}y_0\in {\rm Ker}(f)\\
            &\Rightarrow \left<x-\frac{f(x)}{f(y_0)}y_0,y_0\right>=0\\
            &\Rightarrow \left<x,y_0\right>-\frac{f(x)}{f(y_0)}||y_0||^2=0\\
            &\Rightarrow f(x)=\left< x,\overline{f(y_0)}y_0 \right> 
        \end{align*}
        设$y_f=\overline{f(y_0)}y_0$即可。

        唯一性:设$y,\zeta\in H$使得
        \begin{equation*}
            f(x)=\left<x,y\right>=\left<x,\zeta\right>,x\in H
        \end{equation*}
        于是$\forall x\in H$,$\left<x,y-\zeta\right>=0\Rightarrow y-\zeta=0$.

        一方面,
        \begin{equation*}
            ||f(x)||=||\ag{x,y_f}||\leqslant ||x||\cdot ||y_f||\Rightarrow ||f||\leqslant ||y_f||
        \end{equation*}
        另一方面由$||y_0||=1$,
        \begin{equation*}
            ||y_f||=||f(y_0)||\leqslant ||f||
        \end{equation*}
        所以$||f||=||y_f||$.
    \end{proof}

    \begin{theorem}
        $H$是Hilbert空间,$a(\cdot,\cdot)$是$H$上的共轭双线性函数,如果存在$C>0$使得
        \begin{equation*}
            |a(x,y)|\leqslant C||x||||y||,\forall x,y\in H
        \end{equation*}
        则存在$A\in \mathcal{L}(H)$,使得
        \begin{equation*}
            a(x,y)=\left<x,Ay\right>,x\in H
        \end{equation*}
        且
        \begin{equation*}
            ||A||=\mathop{\rm sup}\limits_{0\neq x,y\in H}
            \frac{|a(x,y)|}{||x||||y||}
        \end{equation*}
    \end{theorem}
    \begin{proof}
        $y\in H$,定义
        \begin{equation*}
            f_y(x)=a(x,y),x\in H
        \end{equation*}
        于是$f_y\in H^*$且$||f_y||\leqslant C||y||$,由Riesz,
        存在唯一$\zeta\in H$使得$f_y(x)=\left<x,\zeta\right>,x\in H$,定义
        $A:H\rightarrow H,y\mapsto \zeta$,则
        \begin{equation*}
            a(x,y)=f_y(x)=\left<x,\zeta\right>=\left<x,Ay\right>
        \end{equation*}
        \begin{enumerate}
            \item $A$是线性映射;
            \item $\forall y\in H$,$||Ay||=||\zeta||=||f_y||\leqslant C||y||$,于是$A\in \mathcal{L}(H)$且$||A||\leqslant C$,
                由$C$的任意性可得
                \begin{equation*}
                    ||A||\leqslant \mathop{\rm sup}\limits_{0\neq x,y\in H}
                    \frac{|a(x,y)|}{||x||||y||}
                \end{equation*}
                另一方面,
                \begin{equation*}
                    |a(x,y)|=|\left<x,Ay\right>|\leqslant ||x||||Ay||\leqslant ||A||||x||||y||,\forall x,y\in H
                \end{equation*}
                于是
                \begin{equation*}
                    \mathop{\rm sup}\limits_{0\neq x,y\in H}
                    \frac{|a(x,y)|}{||x||||y||}\geqslant ||A||
                \end{equation*}
        \end{enumerate}
    \end{proof}

\section{Baire纲定理}
    \subsection*{定理内容}
    \begin{definition}
        $(X,d)$为度量空间,$E\subset X$,如果$\overline{E}$没有内点,则称$E$疏(朗)或无处稠密。
    \end{definition}
    \begin{example}
        $\mathbb{R}$中,Cantor三分集是无处稠密集。
    \end{example}
    \begin{definition}
        第一纲集,又称贫集、瘦集:可数个无处稠密集之并;

        第二纲集:不是第一纲集的集合;

        剩余集:第一纲集的余集。
    \end{definition}
    \begin{example}
        可数集是第一纲集。
    \end{example}
    \begin{lemma}[闭球套]
        设$(X,d)$完备,$\{B_n\}_{n=1}^\infty$是一列闭球,使得
        \begin{enumerate}
            \item $B_1\supset B_2\supset \cdots$
            \item ${\rm diam}B_n\rightarrow 0$
        \end{enumerate}
        则存在唯一$x\in X$使得
        \begin{equation*}
            \bigcap_{n=1}^\infty B_n=\{x\}
        \end{equation*}
    \end{lemma}
    \begin{proof}
        存在性:设$B_n=\overline{ B(x_n,r_n) }$,
        \begin{align*}
            \forall n\geqslant m,x_n\in B_n\subset B_m
            \Rightarrow & d(x_n,x_m)\leqslant r_m\rightarrow 0{\rm\ as\ }n,m\rightarrow\infty\\
            \Rightarrow & \{x_n\}_{n=1}^\infty \mbox{是基本列}\\
            \mathop{\Rightarrow}\limits^{X\mbox{完备}}& \exists x\in X{\rm\ s.t.\ }d(x_n,x)\rightarrow 0{\rm\ as\ }n\rightarrow\infty\\
            \mathop{\Rightarrow}\limits^{B_n\mbox{闭}}& x\in B_n,n=1,2,\cdots\\
            \Rightarrow & x\in\bigcap_{n=1}^\infty B_n
        \end{align*}
        
        唯一性:如果$y\in \bigcap_{n=1}^\infty B_n$,则
        \begin{equation*}
            d(x,y)\leqslant d(x,x_n)+d(x_n,y)\leqslant 2r_n\rightarrow 0{\rm\ as\ }n\rightarrow\infty
        \end{equation*}
        于是$y=x$.
    \end{proof}
    \begin{theorem}[Baire Category Theorem 1,BCT1]
        完备度量空间是第二纲集。
    \end{theorem}
    \begin{proof}
        假设$(X,d)$完备,是第一纲集,则$X$可以被表示成
        可数个无处稠密集之并,设为
        \begin{equation*}
            X=\bigcup_{n=1}^\infty E_n
        \end{equation*}
        任取$B(x_0,r_0)$,
        \begin{align*}
            E_1\mbox{疏}\Rightarrow &\overline{E_1}\mbox{无内点}\\
            \Rightarrow &\exists x_1\in B(x_0,r_0)\backslash \overline{E_1}\\
            \Rightarrow &{\rm dist}(x_1,\overline{E_1})>0\\
            \Rightarrow &
            \exists B(x_1,r_1)\subset B(x_0,r_0),r_1<1{\rm\ s.t.\ }
            \overline{ B(x_1,r_1) }\cap \overline{E_1}=\varnothing
        \end{align*}
        对$E_2$做同样的操作,
        \begin{equation*}
            \exists B(x_2,r_2)\subset B(x_1,r_1),r_2<\frac{1}{2}{\rm\ s.t.\ }
            \overline{ B(x_2,r_2) }\cap \overline{E_2}=\varnothing
        \end{equation*}
        以此类推,
        \begin{equation*}
            \exists B(x_n,r_n)\subset B(x_{n-1},r_{n-1}),r_n<\frac{1}{n}{\rm\ s.t.\ }
            \overline{ B(x_n,r_n) }\cap \overline{E_n}=\varnothing
        \end{equation*}
        由闭球套引理,
        \begin{equation*}
            \bigcap_{n=1}^\infty \overline{ B(x_n,r_n) }=\{x\}
        \end{equation*}
        另一方面,由于$\forall n,\overline{ B(x_n,r_n) }\cap \overline{E_n}=\varnothing$,故
        $\forall n,x\notin \overline{E_n}$,进而
        \begin{equation*}
            x\notin \bigcup_{n=1}^\infty \overline{E_n}=X
        \end{equation*}
        矛盾。
    \end{proof}
    \begin{theorem}[Baire Category Theorem 2,BCT2]
        $(X,d)$是完备度量空间,$\{U_n\}$是一列开集,且满足
        $\overline{U_n}=X$,则
        \begin{equation*}
            \overline{\left(\bigcap_{n=1}^\infty U_n\right)}=X
        \end{equation*}
    \end{theorem}
    \begin{proof}
        设$B_0$是$X$上的任一非空开集,
        $U_1$稠密$\Rightarrow \exists x_0\in U_1\cap B_{0}$,
        因为$U_1\cap B_{0}$是开集,
        所以存在$x_0$的邻域$B_1$满足
        $\overline{B_1}\subset U_1\cap B_{0}$.
        又因为$U_2$稠密,类似地存在$B_2$满足$\overline{B_2}\subset U_2\cap B_1$,
        以此类推可得一列开集$\{B_n\}_{n=1}^\infty$,满足
        \begin{equation*}
            \overline{B_n}\subset U_n\cap B_{n-1}
        \end{equation*}
        不妨令$B_n$为半径小于$\frac{1}{n}$的开球,根据闭球套定理可知
        \begin{equation*}
            K=\bigcap_{n=1}^\infty \overline{B_n}\neq \varnothing
        \end{equation*}
        根据构造过程可得$K\subset B_0$,
        $K\subset U_n$,从而$B_0$与$\cap_{n=1}^\infty U_n$相交,
        所以$\cap_{n=1}^\infty U_n$也是稠密的。
    \end{proof}
\subsection*{应用和推论}
    \begin{example}
        $l^2$的Hamel基是不可数集。(把$l^2$换成任一无穷维Banach空间也可以。)
    \end{example}
    \begin{proof}
        假设$l^2$的Hamel基$B$可数,设$B=\{x_n\}_{k=1}^\infty$,设
        $A_n={\rm span}\{x_1,\cdots,x_n\}$,则$A_n$闭且$l^2=\bigcup_{n=1}^\infty A_n$.由BCT,存在$n_0$使得$A_{n_0}$有内点,那么
        \begin{align*}
            &\exists B(x_0,r)\subset A_{n_0}\\
            \Rightarrow& B(0,r)=B(x_0,r)-x_0\subset A_{n_0}\\
            \Rightarrow& \forall 0\neq x\in l^2,\frac{r}{2}\cdot \frac{x}{||x||}\in B(0,r)\subset A_{n_0}\\
            \Rightarrow& x\in A_{n_0}
        \end{align*}
    \end{proof}
    \begin{theorem}[Banach, 1931]
        $C[0,1]$中处处不可微函数全体是一个剩余集,从而是第二纲集。
    \end{theorem}
    \begin{proof}
        记$X=C[0,1]$,$E=\{ f\in C[0,1]:f\mbox{处处不可微} \}$,于是
        $X\backslash E=\{ f\in C[0,1]:f\mbox{至少在某一点可微} \}$.令
        \begin{equation*}
            A_n=\left\{
                f\in X:  \exists t\in (0,1) {\rm\ s.t.\ }
                \mathop{\rm sup}\limits_{ h\in (-\frac{1}{n},\frac{1}{a}),t+h\in [0,1] }
                \left| \frac{f(t+h)-f(t)}{h} \right|\leqslant n
            \right\}
        \end{equation*}
        于是$\forall f\in X\backslash E$,存在$n$使得$f\in A_n$,所以
        \begin{equation*}
            X\backslash E=\bigcup_{n=1}^\infty A_n
        \end{equation*}
        约化为证明每个$A_n$疏。
        
        \textbf{Step1:}每个$A_n$都是闭集。设一列$f_k\in A_n$,
        $f_k$一致收敛到$f$,对于每个$k$存在$t_k\in (0,1)$使得
        \begin{equation*}
            |f_k(t_k+h)-f_k(t_k)|\leqslant n|h|,\forall h\in (-\frac{1}{n},\frac{1}{n}){\rm\ with\ }t_k+h\in [0,1]
        \end{equation*}
        $\{t_k\}_{k=1}^\infty$有收敛子列,不妨设$t_k\rightarrow t_0$,
        由$f_k\rightrightarrows f$,
        \begin{equation*}
            |f(t_0+h)-f(t_0)|\leqslant n|h|,\forall h\in (-\frac{1}{n},\frac{1}{n}){\rm\ with\ }t_k+h\in [0,1]
        \end{equation*}
        于是$f\in A_n$.

        \textbf{Step2:}$A_n$无内点,即:$\forall f\in A_n$,$\forall \varepsilon>0$,$B(f,\varepsilon)\nsubseteq A_n$.
        首先,by Weierstrass,
        \begin{equation*}
            \exists p\in P[0,1]{\rm\ s.t.\ }||f-p||<\frac{\varepsilon}{2}
        \end{equation*}
        令$M=\mathop{\rm max}\limits_{t\in [0,1]}|p'(t)|$,则$M<\infty$且
        \begin{equation*}
            |p(t+h)-p(t)|\leqslant M|h|,\forall t\in (0,1),\forall h{\rm\ with\ }t+h\in (0,1)
        \end{equation*}
        取分段仿射的连续函数$g$,使得
        \begin{enumerate}
            \item $||g||<\frac{\varepsilon}{2}$
            \item 各段斜率的绝对值$>M+n$
        \end{enumerate}
        则$p+g\in B(f,\varepsilon)$,但$p+g\notin A_n$,
        因为
        \begin{equation*}
            |(p+g)'(t)|\geqslant |g'(t)|-|p'(t)|>n
        \end{equation*}
    \end{proof}


\section{共鸣定理}
    \subsection*{定理内容}
    \begin{theorem}[一致有界原理,共鸣定理,UBP]
        $X$是Banach空间,$Y$是赋范空间,$\mathcal{F}\subset \mathcal{L}(X,Y)$,
        \begin{equation*}
            \forall x\in X,\mathop{\rm sup}\limits_{T\in \mathcal{F}}||Tx||<\infty \Rightarrow
            \mathop{\rm sup}\limits_{T\in \mathcal{F}}||T||<\infty
        \end{equation*}
        等价地,
        \begin{equation*}
            \mathop{\rm sup}\limits_{T\in \mathcal{F}}||T||=+\infty \Rightarrow 
            \exists x_0\in X{\rm\ s.t.\ }
            \mathop{\rm sup}\limits_{T\in \mathcal{F}}||Tx_0||=+\infty
        \end{equation*}
    \end{theorem}
    \begin{proof}
        令
        \begin{equation*}
            F_n=\{ x\in X: 
            \mathop{\rm sup}\limits_{T\in \mathcal{F}}||Tx||\leqslant n
            \}=\bigcap_{T\in \mathcal{F}}
            \{ x\in X:||Tx||\leqslant n \}
        \end{equation*}
        因为$T$连续,$\{ x\in X:||Tx||\leqslant n \}$是闭集,进而$F_n$是闭集,
        \begin{align*}
            \forall x\in X,\mathop{\rm sup}\limits_{T\in \mathcal{F}}||Tx||<\infty
            \Rightarrow & X=\bigcup_{n=1}^\infty F_n\\
            \mathop{\Rightarrow}\limits^{\rm BCT1}
            & \exists n_0{\rm\ s.t.\ }F_{n_0}\mbox{有内点}\\
            \Rightarrow & \exists B(x_0,r)\subset F_{n_0}\\
            \Rightarrow & ||T(x_0+rx)||\leqslant n_0,\forall x\in B(0,1),\forall T\in\mathcal{F}\\
            \Rightarrow & ||T(rx)||\leqslant n_0+||Tx_0||\leqslant 2n_0\\
            \Rightarrow & ||Tx||\leqslant \frac{2n_0}{r},\forall x\in B(0,1),\forall T\in\mathcal{F}\\
            \Rightarrow & \mathop{\rm sup}\limits_{T\in\mathcal{F}}
            \mathop{\rm sup}\limits_{x\in X,||x||\leqslant 1}
            ||Tx||\leqslant \frac{2n_0}{r} 
        \end{align*}
    \end{proof}
\subsection*{应用和推论}
    \begin{theorem}[Banach-Steinhaus]
        $X$是Banach空间,$Y$是赋范空间,$\overline{M}=X$,
        $T,T_n\in\mathcal{L}(X,Y),n=1,2,\cdots$
        \begin{equation*}
            T_nx\rightarrow T_x,\forall x\in X\Leftrightarrow
            \mathop{\rm sup}\limits_n ||T_n||<\infty{\rm\ and\ }
            T_n x\rightarrow Tx,\forall x\in M
        \end{equation*}
    \end{theorem}
    \begin{proof}
        必要性:逐点收敛$\Rightarrow $逐点有界$\mathop{\Rightarrow }\limits^{\rm UBP}$一致有界。

        充分性:记$C=\mathop{\rm sup}\limits_n ||T_n||$,
        \begin{align*}
            \overline{M}=X\Rightarrow & \forall x\in X,\forall \varepsilon>0,\exists y\in M
            {\rm\ s.t.\ }||x-y||<\frac{\varepsilon}{4( ||T||+C )}\\
            \Rightarrow & 
            ||T_nx-Tx||\leqslant 
            \mathop{||T_nx-T_ny||}\limits_{\leqslant C||x-y||}+
            \mathop{||T_ny-Ty||}\limits_{<\frac{\varepsilon}{2},n\mbox{充分大}}+
            \mathop{||T_y-Tx||}\limits_{\leqslant ||T||||x-y||}
        \end{align*}
    \end{proof}
    \begin{theorem}
        $X,Y$是Banach空间,$T_n\in \mathcal{L}(X,Y),n=1,2,\cdots$,如果
        $\forall x\in X$,$\mathop{\rm lim}\limits_{n\rightarrow\infty}T_nx$存在,定义
        $T:X\rightarrow Y,X\mapsto\mathop{\rm lim}\limits_{n\rightarrow\infty}T_nx$,
        则$T\in \mathcal{L}(X,Y)$,且
        \begin{equation*}
            ||T||\leqslant \mathop{\rm liminf}\limits_{n\rightarrow\infty}||T_n||
        \end{equation*}
    \end{theorem}
    \begin{proof}
        见习题2.3.7.
    \end{proof}
    我们设
    \begin{equation*}
        S_N(f)(x)=\sum_{k=-N}^N \hat{f}(k){\rm e}^{2\pi{\rm i}kx}
    \end{equation*}
    问:是否$\forall f\in C(\Pi)$,$S_N(f)(x)\rightarrow f(x),\forall x\in [-\frac{1}{2},\frac{1}{2})$?
    \begin{theorem}[Du Bois-Reymond,1876]
        $\exists f\in C(\Pi){\rm\ s.t.\ }S_N(f)(0)$发散。
    \end{theorem}
    \begin{proof}
        \begin{equation*}
            S_N(f)(x)=(f* D_N)(x)
        \end{equation*}
        其中
        \begin{equation*}
            D_N(t)\defeq \frac{\sin{ [(2N+1)\pi t] }}{\sin{ (\pi t) }}
        \end{equation*}
        定义$T_N:C(\Pi)\rightarrow \R,f\mapsto S_N(f)(0)$,
        \begin{align*}
            \Rightarrow & |T_N(f)|=\left| \int_{-\frac{1}{2}}^{\frac{1}{2}} f(t)D_N(-t)\d t \right|\leqslant ||D_N||_1 \cdot ||f||\\
            \Rightarrow & T_N\in C(\Pi)^*,||T_N||\leqslant ||D_N||_1
        \end{align*}
        Claim:$||T_N||=||D_N||_1$.因为$D_N$在$[-\frac{1}{2},\frac{1}{2}]$上只有有限多个零点,故
        ${\rm sgn\ }D_N$只有有限个间断点。$\forall \varepsilon>0$,存在$f_\varepsilon\in C(\Pi)$,分段仿射使得
        \begin{enumerate}
            \item $||f_\varepsilon||=1$.
            \item $f_\varepsilon={\rm sgn\ }D_N {\rm\ on\ }[-\frac{1}{2},\frac{1}{2}]\backslash I_\varepsilon{\rm\ with\ }|I_\varepsilon|<\frac{\varepsilon}{4N+3}$.
        \end{enumerate}
        \begin{align*}
            \Rightarrow |T_N(f_\varepsilon)|=&\left| \int_{-\frac{1}{2}}^{\frac{1}{2}}f_\varepsilon(t)D_N(t)\d t \right|\\
            \geqslant & \int_{[-\frac{1}{2},\frac{1}{2}]\backslash I_\varepsilon} |D_N(t)|\d t-\int_{I_\varepsilon}|D_N(t)|\d t\\
            \geqslant & ||D_N||_1-2\int_{I_\varepsilon}|D_N(t)|\d t\\
            >& ||D_N||_1-\varepsilon\\
            \Rightarrow ||T_N||\geqslant & \frac{|T_N(f_\varepsilon)|}{||f_\varepsilon||}>||D_N||_1-\varepsilon
        \end{align*}
        令$\varepsilon\rightarrow0$则Claim得证,而
        \begin{align*}
            ||D_N||_1=& 2\int_0^\frac{1}{2}\left| \frac{\sin [(2N+1)\pi t]}{\sin (\pi t)} \right|\d t\\
            \geqslant& 2\int_0^\frac{1}{2}\left| \frac{\sin [(2N+1)\pi t]}{\sin (\pi t)} \right|\d t\\
            \eq{x=(2N+1)\pi t}&\frac{2}{\pi}\int_0^{\frac{\pi}{2}(2N+1)}\left| \frac{\sin{x}}{x} \right|\d x\rightarrow \infty{\rm\ as\ }N\rightarrow\infty
        \end{align*}
        于是
        \begin{align*}
            {\rm Claim}\Rightarrow& \fun{sup}{N}||T_N||=+\infty\\
            \Ra{\rm UBP}& \exists f\in C(\Pi){\rm\ s.t.\ }\fun{sup}{N}|T_N(f)|=+\infty\\
            \Rightarrow& \fun{limsup}{N\rightarrow\infty}|S_N(f)(0)|=+\infty\\
            \Rightarrow& \{ S_N(f)(0) \}_{N=1}^\infty\mbox{发散}
        \end{align*}
    \end{proof}


\section{开映射定理}
    方程$Tx=y$,当$y$变化很小时$x$是不是变化很小(解的稳定性)?这就要考虑$T^{-1}$的连续性。
\subsection*{定理内容}
    \begin{definition}
        将任意开集都映为开集的映射,称为开映射。
    \end{definition}
    
    \begin{theorem}[开映射定理,OMT]
        $X,Y$是Banach空间,$T\in \mathcal{L}(X,Y)$,则
        $T$是满射$\Rightarrow T$是开映射。
    \end{theorem}
    \begin{lemma}
        $X,Y$是Banach空间,$T\in \mathcal{L}(X,Y)$,如果$T$满射,则$\exists \delta>0{\rm\ s.t.\ }\delta B_Y\subset T(B_X)$.
        \begin{proof}
            \textbf{Step1}:$\exists r>0{\rm\ s.t.\ }rB_Y\subset\overline{T(B_X)}$,
            \begin{align*}
                X=\bigcup_{n=1}^\infty nB_X\Ra{T\mbox{满}}&Y=T(X)=\bigcup_{n=1}^\infty T(nB_X)\\
                \Rightarrow& Y=\bigcup_{n=1}^\infty \overline{T(nB_X)}\\
                \Ra{\rm BCT}& \exists n_0{\rm\ s.t.\ }\overline{T(n_0B_X)}\mbox{有内点}\\
                \Rightarrow& \exists B_Y(y_0,t)\subset \overline{T(n_0B_X)}
            \end{align*}
            令$r\defeq t/n_0$,Claim:$rB_y\subset \overline{T(B_X)}$.
            \begin{align*}
                \forall y\in rB_Y,y_0\pm n_0y\in B_Y(y_0,t)\subset \overline{T(n_0 B_X)}\Rightarrow &
                \exists \{x_n\}_{n=1}^\infty,\{x_n'\}_{n=1}^\infty \subset n_0B_X{\rm\ s.t.\ }\\
                &Tx_n\rightarrow y_0+n_0y,Tx_n'\rightarrow y_0-n_0y\\
                \Rightarrow& T\left( \frac{x_n-x_n'}{2n_0} \right)\rightarrow y\\
                \Rightarrow& y\in \overline{T(B_X)} 
            \end{align*}

            \textbf{Step2}:令$\delta\defeq r/3$,则$\delta B_Y\subset T(B_X)$,即
            \begin{equation*}
                \forall y\in\delta B_Y,\exists x\in B_X{\rm\ s.t.\ }Tx=y
            \end{equation*}
            \begin{align*}
                {\rm Step1}\Rightarrow &\forall y\in \delta B_Y,3y\in rB_Y\subset\overline{T(B_X)}\\
                \Rightarrow &\exists \tilde{x}_1\in B_X{\rm\ s.t.\ }||3y-T\tilde{x}_1||_Y<\delta
            \end{align*}
            令$x_1\defeq \tilde{x}_1/3$,则$||y-Tx_1||_Y<\delta/3$.
            令$y_1\defeq y-Tx_1$,则
            \begin{align*}
                y_1\in \frac{\delta}{3}B_Y\Rightarrow& 9y_1\in rB_Y\subset \overline{T(B_X)}\\
                \Rightarrow& \exists x_2\in \frac{1}{3^2}B_X{\rm\ s.t.\ }||y_1-Tx_2||_Y<\frac{\delta}{3^2}\\
                &\vdots\\
                &y_n\defeq y_{n-1}-Tx_n\in\frac{\delta}{3^n}B_Y\\
                &\exists x_{n+1}\in \frac{1}{3^{n+1}}B_X{\rm\ s.t.\ }||y_n-Tx_{n+1}||_Y<\frac{\delta}{3^{n+1}}\\
                \Rightarrow& \ms{ \sum_{k=n+1}^{n+p}x_k }_{X}<\frac{\frac{1}{3^{n+1}}}{1-\frac{1}{3}}<\frac{1}{2^n}\\
                \Rightarrow& \left\{ \sum_{k=1}^n x_k \right\}_{n=1}^\infty \mbox{是}X\mbox{中的基本列}\\
                \Ra{X\ Banach}&\exists x\in X{\rm\ s.t.\ }\sum_{k=1}^n x_k\rightarrow x{\rm\ and\ }||x||_X\leqslant \ms{ x-\sum_{k=1}^N x_k}_{X}+\ms{ \sum_{k=1}^N x_k}_{X}<1\\
                \Rightarrow& x\in B_X{\rm\ and\ }\frac{\delta}{3^n}>||y_n||_Y=||y_{n-1}-Tx_n||_Y=\cdots=||y-T(x_1+\cdots+x_n)||_Y\\
                \Rightarrow& T\left( \sum_{k=1}^n x_k \right)\rightarrow y\\
                \Rightarrow& Tx=y
            \end{align*}
        \end{proof}
    \end{lemma}
    \begin{proof}
        设$U$是$X$上的开集,$\forall y\in T(U)$,$\exists x\in U{\rm\ s.t.\ }Tx=y$,令$V\defeq U-x$,
        \begin{align*}
            0\in V\mathop{\subset}\limits^{\rm open}X
            \Rightarrow & \exists t>0{\rm\ s.t.\ }tB_X\subset V\\
            \Ra{\rm Lemma} & \exists \delta>0{\rm\ s.t.\ }\delta B_Y\subset T(B_X)\subset \frac{1}{t}T(V)\\
            \Rightarrow & 0\mbox{是$T(V)$的内点}\\
            T(U)=T(V)+Tx \Rightarrow& y=Tx\mbox{是$T(U)$的内点}
        \end{align*}
    \end{proof}
    
    \begin{theorem}[逆算子定理,IMT]
        $X,Y$是Banach空间,$T\in \mathcal{L}(X,Y)$,则
        $T$是双射$\Rightarrow T^{-1}\in\mathcal{L}(Y,X)$.
    \end{theorem}
    \begin{proof}
        $f:X\rightarrow Y$连续$\Leftrightarrow \forall $开集
        $U\subset Y,f^{-1}(U)$是$X$上的开集。

        $T^{-1}:Y\rightarrow X$连续$\Leftrightarrow \forall $开集
        $U\subset X$,由OMT,$(T^{-1})^{-1}(U)=T(U)$是$Y$上的开集。
    \end{proof}
\subsection*{应用和推论}
    \begin{theorem}[Lax-Milgram]
        $H$是Hilbert空间,如果共轭双线性函数$a(\cdot,\cdot)$满足
        \begin{enumerate}
            \item 连续:$\exists C>0$使得$|a(x,y)|\leqslant C||x||||y||,\forall x,y\in H$.
            \item 强制(coersive):$\exists\delta>0$使得$\delta ||x||^2\leqslant a(x,x),\forall x\in H$.
        \end{enumerate}
        则存在唯一$A\in\mathcal{L}(H)$使得
        \begin{enumerate}[$1^\circ$]
            \item $a(x,y)=\ag{x,Ay},x,y\in H$.
            \item $A^{-1}$存在、有界且$||A^{-1}||\leqslant \frac{1}{\delta}$.
        \end{enumerate}
    \end{theorem}
    \begin{proof}
        Claim1:$A$是双射:
        \begin{enumerate}[(1).]
            \item $A$是单射:对于线性映射而言,单射$\Leftrightarrow {\rm Ker}(A)=A^{-1}(\{0\})=\{0\}$.
                \begin{align*}
                    Ay=0\Rightarrow& a(x,y)=\ag{x,Ay}=0,\forall x\in H\\
                    \Rightarrow& 0=a(y,y)\geqslant \delta||y||^2\\
                    \Rightarrow& y=0
                \end{align*}
            \item $A$是满射:先证明${\rm Ran}(A)$闭,设${\rm Ran}(A)\ni Ax_n\rightarrow y$,
                \begin{align*}
                    &\delta||x_n-x_m||^2\leqslant a(x_n-x_m,x_n-x_m)
                    =\ag{x_n-x_m,Ax_n-Ax_m}\leqslant ||x_n-x_m||\cdot ||Ax_n-Ax_m||\\
                    \Rightarrow&
                    ||x_n-x_m||\leqslant \frac{1}{\delta}||Ax_n-Ax_m||\rightarrow 0{\rm\ as\ }n,m\rightarrow\infty\\
                    \Rightarrow&\{x_n\}_{n=1}^\infty\mbox{是基本列}
                \end{align*}
                设$x_n\rightarrow x\in H$,
                $A$连续所以$Ax_n\rightarrow Ax$,而$Ax_n\rightarrow y$,
                所以$y=Ax\in{\rm Ran}(A)$,那么$H={\rm Ran}(A)\oplus {\rm Ran}(A)^\perp$.
                为了证明$A$满射,只需证明${\rm Ran}(A)^\perp=\{0\}$,
                \begin{equation*}
                    y\in {\rm Ran}(A)^\perp\Rightarrow 
                    \ag{y,Ax}=a(y,x)=0,\forall x\in H
                \end{equation*}
                特别地,考虑$x=y$,则$0=a(y,y)\geqslant \delta||y||^2\Rightarrow y=0$.
        \end{enumerate}
        那么由IMT,$A^{-1}\in\mathcal{L}(H)$,
        \begin{align*}
            \delta||x||^2\leqslant a(x,x)=\ag{x,Ax}\leqslant ||x||\cdot ||Ax||
            \Rightarrow& ||x||\leqslant \frac{1}{\delta}||Ax||,\forall x\in H\\
            \Rightarrow& ||A^{-1}y||\leqslant \frac{1}{\delta}||y||,\forall y\in H\\
            \Rightarrow& ||A^{-1}||\leqslant \frac{1}{\delta}
        \end{align*}
    \end{proof}

    \begin{theorem}[等价范数定理]
        $(X,\ms{\cdot}_{1}),(X,\ms{\cdot}_{2})$都是Banach空间,
        则$\ms{\cdot}_{2}\lesssim \ms{\cdot}_{1}\Rightarrow 
        \ms{\cdot}_{1}\cong \ms{\cdot}_{2}$.
    \end{theorem}
    \begin{proof}
        考虑恒等映射$I:(X,\ms{\cdot}_{1})\rightarrow (X,\ms{\cdot}_{2}),x\mapsto x$.
        \begin{align*}
            \exists C>0{\rm\ s.t.\ }\ms{x}_{2}\leqslant C\ms{x}_{1},\forall x\in X
            \Rightarrow& \frac{||I(x)||_2}{||x||_1}\leqslant C \\
            \Rightarrow& I\in\mathcal{L}( (X,\ms{\cdot}_{1}),(X,\ms{\cdot}_{2}) )\\
            \Ra{\rm IMT}&I^{-1}\in\mathcal{L}( (X,\ms{\cdot}_{1}),(X,\ms{\cdot}_{2}) )\\
            \Rightarrow&\exists C'>0{\rm\ s.t.\ }\ms{x}_{1}\leqslant C'\ms{x}_{2},\forall x\in X\\
            \Rightarrow&\frac{1}{C'}\ms{x}_{1}\leqslant \ms{x}_{2}\leqslant C\ms{x}_{1},\forall x\in X
        \end{align*}
    \end{proof}


\section{闭图像定理}
    \begin{definition}[乘积度量空间]
        $(X,\ms{\cdot}_{X}),(Y,\ms{\cdot}_{Y})$是两个度量空间,定义:
        \begin{equation*}
            \ms{ (x,y) }_{X\times Y}\defeq \ms{x}_{X}+\ms{y}_{Y}
        \end{equation*}
        可以证明,$( X\times Y,\ms{\cdot}_{X\times Y} )$构成一个新的度量空间,称为乘积空间。
    \end{definition}
    \begin{corollary}
        $X,Y$都是Banach空间,则其乘积空间$X\times Y$也是Banach空间。
    \end{corollary}
    \begin{definition}[闭算子]
        $T:X\rightarrow Y$是线性映射,
        \begin{equation*}
            {\rm Gr}(T)\defeq \{ (x,Tx):x\in {\rm Dom}(T) \}
        \end{equation*}
        称为$T$的图像,其中${\rm Dom}(T)\subset X$是指$T$的定义域,为$X$的子集.

        如果${\rm Gr}(T)$是$X\times Y$的闭子空间,则称$T$是闭算子。
    \end{definition}
    \begin{lemma}
        $T$是闭算子当且仅当
        \begin{equation*}
            {\rm Dom(T)}\ni x_n\rightarrow x,Tx_n\rightarrow y
        \end{equation*}
        蕴含(imply)
        \begin{equation*}
            x\in {\rm Dom}(T),y=Tx
        \end{equation*}
        即,如果$T$将${\rm Dom}(T)$上的收敛列$\{x_n\}$映为收敛列$\{y_n=Tx_n\}$,
        则收敛列$\{x_n\}$的极限$x\in {\rm Dom}(T)$且收敛列$\{y_n=Tx_n\}$的极限为$y=Tx$.
    \end{lemma}
    \begin{proof}
        ${\rm Gr}(T)$闭当且仅当:
        \begin{equation*}
            (x_n,Tx_n)\rightarrow (x,y)\Rightarrow (x,y)\in G_r(T)
        \end{equation*}
    \end{proof}

    \begin{remark}
        ${\rm Dom}(T)$不一定是闭集。
    \end{remark}

    \begin{example}[无界的闭算子]
        \begin{equation*}
            T=\frac{\d}{\d t}:C[0,1]\rightarrow C[0,1],\ {\rm Dom}(T)=C^1[0,1]
        \end{equation*}
    \end{example}

    \begin{proposition}
        $A$有界,$D={\rm Dom}(A)$闭,则$A$闭。
    \end{proposition}
    \begin{proof}
        设$\{x_n\}\subset D$收敛到$x$,$Ax_n\rightarrow y$,则
        $D$闭$\Rightarrow x\in D$,$A$连续$\Rightarrow Ax_n\rightarrow Ax$,所以$Ax=y$,$A$是闭算子。
    \end{proof}

\subsection*{定理内容}
    \begin{theorem}[B.L.T.]
        $X$是赋范空间,$Y$是Banach空间,
        任一$T\in \mathcal{ L }( {\rm Dom}(T),Y )$可唯一地、保范数地延拓为
        $\tilde{T}\in \mathcal{L}( \overline{{\rm Dom}(T)},Y )$.即
        \begin{equation*}
            \tilde{T}|_{ {\rm Dom}(T) }=T{\rm\ and\ }||\tilde{T}||=||T||.
        \end{equation*}
    \end{theorem}
    \begin{proof}
        \begin{align*}
            &\forall x\in 
            \overline{{\rm Dom}(T)},\exists x_n\in {\rm Dom}(T),n=1,2,\cdots{\rm\ s.t.\ }x_n\rightarrow x\\
            \Ra{ T\mbox{有界}}&||Tx_n-Tx_m ||\leqslant ||T|| ||x_n-x_m||\rightarrow 0{\rm\ as\ }n,m\rightarrow\infty\\
            \Rightarrow& \{Tx_n\}_{n=1}^\infty \mbox{是$Y$中的基本列}\\
            \Ra{Y\rm\ Banach}& \exists y\in Y{\rm\ s.t.\ }Tx_n\rightarrow y
        \end{align*}
        定义映射
        \begin{equation*}
            \tilde{T}:\overline{{\rm Dom}(T)}\rightarrow Y,x\mapsto y
        \end{equation*}
        容易验证良定,且$\tilde{T}$是线性映射。下面证明$\tilde{T}$有界:
        \begin{align*}
            &\forall x\in \overline{ {\rm Dom}(T) }, 
            ||\tilde{T}x||=||y||=\fun{lim}{n\rightarrow \infty}||Tx_n||
            \leqslant \fun{lim}{n\rightarrow\infty} ||T||||x_n||=||T||||x||\\
            \Rightarrow& \tilde{T}\in\mathcal{ L }(\overline{ {\rm Dom}(T) },Y){\rm\ and\ }||\tilde{T}||\leqslant ||T||
        \end{align*}
        另一方面,平凡地,$||\tilde{T}||\geqslant ||T||$.
    \end{proof}

    \begin{example}[Fourier变换]
        \begin{equation*}
            \hat{f}(\xi)\defeq \int_{\R^n}f(x){\rm e}^{-2\pi {\rm i}x\cdot \xi}\d x
        \end{equation*}
        其中$f\in L^1$,如何在$L^2$上定义?

        $L^1\cap L^2$在$L^2$上稠密,$||\hat{f}||_2=||f||_2,\forall f\in L^1\cap L^2$(Planchered),
        由B.L.T.可得$\mathcal{F}:f\mapsto \hat{f}$可唯一地、保范数地延拓到$L^2$上。
    \end{example}
    
    \begin{remark}
        由B.L.T和命题2.6.1,有界算子可以将其定义域延拓为闭集,从而成为闭算子。
    \end{remark}

    \begin{theorem}[闭图像定理,CGT]
        $X,Y$是Banach空间,$T:X\rightarrow Y$是闭线性算子,如果
        ${\rm Dom}(T)$是闭集,则$T$有界。
    \end{theorem}
    \begin{proof}
        ${\rm Gr}(T)$是$X\times Y$的闭子空间,所以$({\rm Gr}(T),||\cdot||_{X\times Y})$是Banach空间,定义:
        \begin{equation*}
            \Pi_1:{\rm Gr}(T)\rightarrow {\rm Dom}(T),\ (x,Tx)\mapsto x
        \end{equation*}
        \begin{equation*}
            \Pi_2:{\rm Gr}(T)\rightarrow Y,\ (x,Tx)\mapsto Tx
        \end{equation*}
        并设$T=\Pi_2\prod \Pi_1^{-1}$,
        \begin{equation*}
            \Pi_1\mbox{是双射}\mathop{\Rightarrow}\limits_{ {\rm Dom}(T)\mbox{闭用在这里} }^{ \rm IMT }
            \Pi^{-1}\mbox{有界}\Rightarrow T=\Pi_1\circ\Pi_1^{-1}\mbox{有界}
        \end{equation*}
    \end{proof}
    \begin{proof}
        $( {\rm Dom}(T),||\cdot||_X )$是Banach空间,令
        \begin{equation*}
            ||x||_G\defeq ||x||_X+||Tx||_Y,\ x\in {\rm Dom}(T)
        \end{equation*}
        {\rm Claim}:$( {\rm Dom}(T),||\cdot||_G )$也是Banach空间。实际上,
        \begin{equation*}
            ||x_n-x_m||_G=||x_n-x_m||_X+||Tx_n-Tx_m||_Y\rightarrow 0{\rm\ as\ }n,m\rightarrow\infty
        \end{equation*}
        $X,Y$是Banach,所以存在$x_n\rightarrow x,Tx_n\rightarrow y$,因为$T$闭所以
        $x\in{\rm Dom}(T),y=Tx$.于是
        \begin{equation*}
            ||x_n-x||_G=||x_n-x||_X+||Tx_n-Tx||_Y\rightarrow 0
        \end{equation*}
        那么$||\cdot||_X$弱于$||\cdot||_G$,因此$||\cdot||_X$等价于$||\cdot||_G$,
        所以存在$C>0$使得
        \begin{equation*}
            ||Tx||_Y\leqslant ||x||_G\leqslant C||x||_X,\forall x\in {\rm Dom}(T)
        \end{equation*}
    \end{proof}
\subsection*{应用和推论}
    \begin{example}[Hellinger-Toeplitz]
        $H$是Hilbert空间,如果$T:H\rightarrow H$满足
        $\ag{Tx,y}=\ag{x,Ty},\forall x,y\in H$,则$T$有界。
    \end{example}
    \begin{proof}
        先证明$T$闭:设$x_n\rightarrow x$,$Tx_n\rightarrow y$,存在$\delta\in H$使得
        \begin{equation*}
            \ag{\delta,Tx}=\ag{T\delta,x}
            =\fun{lim}{n\rightarrow\infty}\ag{T\delta,x_n}
            =\fun{lim}{n\rightarrow\infty}\ag{\delta,Tx_n}
            =\ag{\delta,y}
        \end{equation*}
        所以$Tx=y$.于是由CGT可知$T$有界。
    \end{proof}


\section{Hahn-Banach定理}
\subsection{代数形式——线性泛函的延拓}
    \begin{definition}
        $X$是向量空间,如果函数$p:X\rightarrow \R$使得
        \begin{enumerate}
            \item 正齐性:$p(tx)=tp(x),\forall x\in X,t>0$.
            \item 次可加性:$p(x+y)\leqslant p(x)+p(y),\forall x,y\in X$.
        \end{enumerate}
        则称$p$是$X$上的一个次线性泛函。

        如果$p$还满足齐次性,即
        \begin{equation*}
            p(\lambda x)=|\lambda |p(x),\forall x\in X,\forall \lambda\in \K
        \end{equation*}
        则称$p$是一个半范数。
    \end{definition}
    \begin{remark}
        \begin{enumerate}
            \item 半范数非负:$\forall x\in X,2p(x)=p(x)+p(-x)\geqslant p(0)=0$.
            \item 如果半范数$p$满足$p(x)=0\Rightarrow x=0$,则$p$是一个范数。
        \end{enumerate}
    \end{remark}
    
    \begin{theorem}[HBT for real version]
        设$X$为实向量空间,$p$是$X$上次线性泛函,$M$是$X$的子空间,
        $f$是$M$上的线性泛函,并满足$f(x)\leqslant p(x),\forall x\in M$.
        则存在$X$上的线性泛函$F$满足
        \begin{enumerate}
            \item $F|_M=f$.
            \item $F(x)\leqslant p(x),\forall x\in X$.
        \end{enumerate}
    \end{theorem}
    \begin{lemma}
        在定理条件下,设$x_0\in X\backslash M$,定义
        \begin{equation*}
            \tilde{M}\defeq M\oplus {\rm span}{x_0}
        \end{equation*}
        则存在线性映射$\tilde{f}:\tilde{M}\rightarrow\R$,满足
        \begin{enumerate}
            \item $\tilde{f}|_M=f$.
            \item $\tilde{f}(x)\leqslant p(x),\forall x\in \tilde{M}$.
        \end{enumerate}
        \begin{proof}
            \begin{align*}
                &\forall x,y\in M,f(x)+f(y)=f(x+y)\leqslant p(x+y)\leqslant p(x-x_0)+p(y+x_0)\\
                \Rightarrow&
                f(x)-p(x-x_0)\leqslant p(y+x_0)-f(y)\\
                \Rightarrow& \fun{sup}{x\in M}[ f(x)-p(x-x_0) ]
                \leqslant \fun{inf}{y\in M}[ p(y+x_0)-f(y) ]\\
                \Rightarrow& \exists \beta\in\R{\rm\ s.t.\ }
                f(x)-p(x-x_0)\leqslant \beta\leqslant p(y+x_0)-f(y),\forall x,y\in M\tag{*}
            \end{align*}
            令
            \begin{equation*}
                \tilde{f}:\tilde{M}\rightarrow \R,x+\lambda x_0\mapsto f(x)+\lambda\beta
            \end{equation*}
            于是$\tilde{f}$是线性映射,且$\tilde{f}|_M=f$.下面证明:
            \begin{equation*}
                \tilde{f}(x+\lambda x_0)\leqslant p(x+\lambda x_0),\forall x\in M,\forall \lambda\in\R
            \end{equation*}
            $\lambda=0$显然,$\lambda\neq 0$时,
            不妨设$\lambda>0$,(*)式中$x,y$均代以$\frac{x}{\lambda}$可得
            \begin{align*}
                & f(\frac{x}{\lambda})-p(\frac{x}{\lambda}-x_0)\leqslant \beta\leqslant p(\frac{x}{\lambda}+x_0)-f(\frac{x}{\lambda})\\
                \Rightarrow &
                f(x)-p(x-\lambda x_0)\leqslant \lambda\beta\leqslant p(x+\lambda x_0)-f(x)\\
                \Rightarrow &
                \left\{ \begin{array}{l}
                    f(x)-\lambda \beta=\tilde{f}(x-\lambda x_0)\leqslant p(x-\lambda x_0)\\
                    f(x)+\lambda \beta=\tilde{f}(x+\lambda x_0)\leqslant p(x+\lambda x_0)
                \end{array} \right.
            \end{align*}
        \end{proof}
    \end{lemma}
    \begin{proof}
        对两个线性泛函$g,h$,如果${\rm Dom}(g)$是
        ${\rm Dom}(h)$的闭子空间,且
        $h_{ {\rm Dom}(g) }=g$,则称$h$是$g$的一个延拓。令
        \begin{equation*}
            \mathcal{F}\defeq \{ g:g\mbox{是}f\mbox{的延拓,}g(x)\leqslant p(x),\forall x\in {\rm Dom}(g) \}
        \end{equation*}
        引入偏序:
        \begin{equation*}
            g\leqslant h\Leftrightarrow h\mbox{是}g\mbox{的延拓}
        \end{equation*}
        设$\mathcal{T}$是$\mathcal{F}$的任一全序子集,令
        \begin{equation*}
            Y\defeq \bigcup_{g\in \mathcal{T}}{\rm Dom}(g)
        \end{equation*}
        于是$Y$是$X$的闭子空间,令
        \begin{equation*}
            G:Y\rightarrow\R,x\mapsto g(x){\rm\ if\ }x\in {\rm Dom}(g)
        \end{equation*}
        $\mathcal{T}$全序$\Rightarrow G$良定且是$\mathcal{T}$的一个上界,
        由Zorn引理可得$\mathcal{F}$有极大元$F$,下面证明${\rm Dom}(F)=X$,从而$F$即为所求。

        假设不然,即存在$x_0\in X\backslash {\rm Dom}(F)$,那么由引理可得
        存在${\rm Dom}(F)\oplus {\rm span}\{x_0\}$上的线性泛函$\tilde{F}$,满足
        \begin{enumerate}
            \item $\tilde{F}|_{{\rm Dom}(F)}=F$.
            \item $\tilde{F}(x)\leqslant p(x),\forall x\in {\rm Dom}(F)\oplus {\rm span}\{x_0\}$.
        \end{enumerate}
        于是$\tilde{F}\in \mathcal{F}$且$F\leqslant \tilde{F}$,这与$F$的极大性矛盾。
    \end{proof}

    \begin{theorem}[HBT for complex version]
        设$X$为复向量空间,$p$是$X$上次线性泛函,$M$是$X$的子空间,
        $f:M\rightarrow \C$是$M$上的线性泛函,
        并满足$|f(x)|\leqslant p(x),\forall x\in M$.
        则存在$X$上的线性泛函$F:X\rightarrow\C$满足
        \begin{enumerate}
            \item $F|_M=f$.
            \item $|F(x)|\leqslant p(x),\forall x\in X$.
        \end{enumerate}
    \end{theorem}
    \begin{proof}
        \textbf{Step1}:先把$X$看作实向量空间,令
        $g\defeq {\rm Re\ }f$,则$g$是$M$上的实线性泛函,且满足
        \begin{equation*}
            g(x)\leqslant |f(x)|\leqslant p(x),\forall x\in M
        \end{equation*}
        那么由实HBT,存在线性映射$G:X\rightarrow \R$,满足
        \begin{enumerate}
            \item $G|_M=g$.
            \item $G(x)\leqslant p(x),\forall x\in X$.
        \end{enumerate}        

        \textbf{Step2}:复化。令
        \begin{equation*}
            F(x)\defeq G(x)-\i G( \i x )
        \end{equation*}
        显然有
        \begin{align*}
            &F(x+y)=F(x)+F(y)\\
            &F(\alpha x)=\alpha F(x),\forall x\in X,\forall \alpha\in \R
        \end{align*}
        所以
        \begin{equation*}
            F( (\alpha_1+\i \alpha_2)x )
            =F(\alpha_1 x)+F(\i \alpha_2 x)
            =\alpha_1 F(x)+\alpha_2 F(\i x),\forall x\in X,\forall \alpha_1,\alpha_2\in\R
        \end{equation*}
        故只需证明$F(\i x)=\i F(x)$,
        \begin{equation*}
            F(\i x)=G(\i x)-\i G(-x)
            =G(\i x)+\i G(x)
            =\i (-\i G(\i x)+G(x))
            =\i F(x)
        \end{equation*}
        得证。

        \textbf{Step3}:证明$F|_M=f$.
        \begin{align*}
            \forall x\in M,F(x)&=G(x)-\i G(\i x)\\
            &=g(x)-\i g(\i x)\\
            &={\rm Re\ }f(x)-\i\cdot {\rm Re\ }f(\i x)\\
            &={\rm Re\ }f(x)-\i\cdot {\rm Re\ }\{\i f(x)\}\\
            &={\rm Re\ }f(x)+\i\cdot {\rm Im\ }f(x)=f(x)
        \end{align*}

        \textbf{Step4}:证明$|F(x)|\leqslant p(x),\forall x\in X$.
        $F(x)=0$时显然,设$F(x)\neq 0$,存在$\theta\in\R$使得$|F(x)|=\e^{-\i \theta}F(x)$,于是
        \begin{equation*}
            |F(x)|=F(\e^{-\i \theta}x)
            =G(\e^{-\i \theta}x)-\i G( \i\e^{-\i \theta}x )
            =G(\e^{-\i \theta}x)\leqslant p(\e^{-\i \theta}x)
            =p(x)
        \end{equation*}
    \end{proof}

    \begin{theorem}[HBT]
        $X$为度量空间,$M$是其子空间,则
        \begin{equation*}
            \forall f\in M^*,\exists F\in X^*{\rm\ s.t.\ }
            F|_M=f{\rm\ and\ }||F||=||f||
        \end{equation*}
        这称为保范延拓。
    \end{theorem}
    \begin{proof}
        令
        \begin{equation*}
            p(x)\defeq ||f||\cdot ||x||,x\in X
        \end{equation*}
        则$|f(x)|\leqslant ||f||\cdot ||x||$,由复HBT可得
        存在$X$上线性泛函$F:X\rightarrow \C$满足
        $F|_M=f$且$|F(x)|\leqslant p(x)$,进而
        $|F(x)|\leqslant ||F||\cdot ||x||$,因此
        $F$是$X$上有界线性泛函,且$||F||\leqslant ||f||$.
        同时显然有$||F||\geqslant ||f||$.
    \end{proof}

    \begin{example}[HBT中延拓不唯一]
        $X=(\R^2,||\cdot||_1)$,其中$||(x_1,x_2)||_1\defeq |x_1|+|x_2|$,
        取$M=\R\times \{0\}$,$f:M\rightarrow\R,(x,0)\mapsto x$,那么$f$
        是$M$上有界线性泛函,且$||f||=1$.

        令
        \begin{equation*}
            F_t:\R^2\rightarrow\R,(x_1,x_2)\mapsto x_1+tx_2
        \end{equation*}
        那么$F_t|_M=f$,而且对于$\forall t\in (-1,1)$,
        \begin{equation*}
            |F_t(x_1,x_2)|=|x_1+tx_2|\leqslant |x_1|+|t||x_2|\leqslant ||(x_1,x_2)||_1
            \Rightarrow ||F_t||\leqslant 1
        \end{equation*}
    \end{example}
\subsection*{应用和推论}
    \begin{corollary}
        $\forall x_0\in X$,$\exists f\in X^*$满足$||f||=1$且$f(x_0)=||x_0||$.
    \end{corollary}
    \begin{proof}
        令$M\defeq {\rm span}\{x_0\}$,
        \begin{equation*}
            f_0:M\rightarrow\K,x=\lambda x_0\mapsto \lambda||x_0||
        \end{equation*}
        于是$|f_0(x)|\leqslant |\lambda| \cdot ||x_0||=||x||$,
        进而$f_0\in M^*$且$||f_0||=1$,由HBT可得存在$f\in X^*$满足
        $f|_M=f_0$,即$f(x_0)=f_0(x_0)=||x_0||$,同时$||f||=||f_0||=1$.
    \end{proof}

    \begin{corollary}
        $X\neq \{0\}\Rightarrow X^* \neq \{0\}$.
    \end{corollary}
    \begin{proof}
        取$0\neq x\in X$,由推论2.7.1可得存在$f\in X^*$满足$||f||=1$且
        $f(x)=||x||\neq 0$,此即为$0\neq f\in X^*$.
    \end{proof}

    \begin{corollary}
        $x,y\in X,x\neq y\Rightarrow \exists f\in X^*{\rm\ s.t.\ }f(x)\neq f(y)$.
    \end{corollary}
    \begin{proof}
        取$x_0=x-y\neq 0$,由推论2.7.1可得存在$f\in X^*$使得
        $f(x-y)=||x-y||\neq 0\Rightarrow f(x)\neq f(y)$.
    \end{proof}

    \begin{corollary}
        $f(x)=0,\forall f\in X^*\Rightarrow x=0$
    \end{corollary}
    \begin{proof}
        假设$x\neq 0$,由推论2.7.3可得存在$f\in X^*$使得
        $f(x)\neq f(0)=0$,矛盾。
    \end{proof}

    \begin{corollary}
        $||x||=\fun{sup}{f\in X^*,||f||=1}  |f(x)|$.
    \end{corollary}
    \begin{proof}
        $\forall f\in X^*$满足$||f||=1$,
        \begin{equation*}
            |f(x)|\leqslant ||f||\cdot ||x||=||x||
        \end{equation*}
        于是
        \begin{equation*}
            \fun{sup}{f\in X^*,||f||=1}  |f(x)|\leqslant ||x||
        \end{equation*}
        另一方面,存在$f\in X^*$满足$||f||=1$且$f(x)=||x||$,得证。
    \end{proof}

    \begin{theorem}
        $X$是度量空间,$M$是其子空间,$x_0\in X$满足
        $d={\rm dist}(x_0,M)>0$,则存在
        $f\in X^*$满足$||f||=1$且
        \begin{equation*}
            f(M)=\{0\},f(x_0)=d
        \end{equation*}
    \end{theorem}
    \begin{proof}
        令$\tilde{M}\defeq M\oplus {\rm span}\{x_0\}$,定义
        \begin{equation*}
            f_0:\tilde{M}\rightarrow \K,x=y+\lambda x_0\mapsto \lambda d
        \end{equation*}
        于是$f_0(M)=\{0\}$且$f(x_0)=d$,而且对于
        $\forall x=y+\lambda x_0$,其中$y\in M$,
        \begin{enumerate}[$1^\circ$]
            \item 如果$\lambda=0$,则$f_0(x)=0$.
            \item 如果$\lambda\neq 0$,
                \begin{equation*}
                    |f_0(x)|=|\lambda d|=|\lambda|\cdot d
                    \leqslant |\lambda |\cdot \ms{x_0+\frac{y}{\lambda}}
                    =\ms{y+\lambda x_0}=\ms{x}.
                \end{equation*}
                于是$f_0\in \tilde{M}^*$且$\ms{f_0}=1$,由HBT可得存在$f\in X^*$满足
                $f|_{\tilde{M}^*}=f_0$且$||f||=||f_0||\leqslant 1$,则
                $f(M)=\{0\},f(x_0)=d$.
        \end{enumerate}
        只需证明$||f||\geqslant 1$.
        \begin{align*}
            d=\fun{inf}{y\in M}\ms{x_0-y}\Rightarrow&
            \forall n,\exists y_n\in M{\rm\ s.t.\ }\ms{x_0-y_n}<d+\frac{1}{n}\\
            \Rightarrow& 
            \frac{ |f(x_0-y_n)| }{\ms{x_0-y_n}}
            =\frac{|f(x_0)|}{\ms{x_0-y_n}}>
            \frac{d}{d+\frac{1}{n}}\rightarrow 1{\rm\ as\ }n\rightarrow \infty\\
            \Rightarrow& \fun{sup}{n}\frac{ |f(x_0-y_n)| }{\ms{x_0-y_n}}\geqslant 1\\
            \Rightarrow& ||f||\geqslant 1
        \end{align*}
    \end{proof}

    \begin{theorem}
        $X$是度量空间,$M$是其子空间,$0\neq x_0\in X$,那么
        \begin{equation*}
            x_0\in \overline{ {\rm span\ }M }
            \Leftrightarrow 
            f(x_0)=0,\forall f\in X^*{\rm\ with\ }f(M)=\{0\}
        \end{equation*}
    \end{theorem}
    \begin{proof}
        必要性:设$x_0\in \overline{ {\rm span\ }M }$,对于$\forall f\in X^*$满足$f(M)=\{0\}$,有:
        \begin{equation*}
            f({\rm span\ }M)=f(\overline{ {\rm span\ }M })=\{0\}
        \end{equation*}
        从而$f(x_0)=0$.

        充分性:假设$x_0\notin \overline{ {\rm span\ }M }$,
        则
        \begin{equation*}
            d\defeq {\rm dist}(x_0,\overline{ {\rm span\ }M })>0
        \end{equation*}
        由定理2.7.4,存在$f\in X^*$满足$||f||=1$且
        \begin{equation*}
            f( \overline{ {\rm span\ }M} )=\{0\},f(x_0)=d>0
        \end{equation*}
        矛盾。
    \end{proof}
\subsection{几何形式——凸集分离}
    \begin{definition}
        $X$是向量空间,$C\subset X$,
        \begin{enumerate}
            \item 如果$-C=C$,称$C$对称。
            \item 如果$\forall x,y\in C$,$\forall t\in [0,1]$,都有$tx+(1-t)y\in C$,称$C$是凸集(Convex set)。
            \item 如果$\forall x\in X$,存在$t>0$使得$\frac{x}{t}\in C$,称$C$是吸收的。
        \end{enumerate}
    \end{definition}
    \begin{proposition}
        任一族凸集之交仍然是凸集。
    \end{proposition}
    \begin{definition}
        对于集合$A$,包含$A$的所有凸集之交称为$A$的凸包,记作
        \begin{equation*}
            {\rm conv}(A)\defeq 
            \bigcap_{{\rm Convex\ }C\supset A} C
        \end{equation*}
    \end{definition}
    \begin{definition}
        对于
        \begin{equation*}
            \sum_{k=1}^n \lambda_k=1,\{ \lambda_k \}_{k=1}^n
        \end{equation*}
        称
        \begin{equation*}
            \sum_{k=1}^n \lambda_k x_k
        \end{equation*}
        为$x_1,\cdots,x_n$的一个凸组合。
    \end{definition}

    \begin{proposition}
        ${\rm conv}(A)$就是$A$中向量的凸组合全体。
    \end{proposition}

    \begin{definition}
        $X$是向量空间,$C$是包含$0$的凸集,广义实值函数$P_C:X\rightarrow [0,+\infty]$,
        \begin{equation*}
            P_C(x)\defeq {\rm inf}\{ t>0:\frac{x}{t}\in C \}
        \end{equation*}
        称为$C$的Minkowski泛函。
        \begin{equation*}
            P_C(x)=+\infty\Leftrightarrow \{ t>0:\frac{x}{t}\in C \}=\varnothing
        \end{equation*}
    \end{definition}
    \begin{proposition}
        关于$P_C$,
        \begin{enumerate}
            \item $P_C(0)=0$.
            \item 正齐次性:$P_C(tx)=tP_C(x),\forall x\in X,\forall t>0$.
            \item 次可加性:$P_C(x+y)\leqslant P_C(x)+P_C(y)$
        \end{enumerate}
        注意$P_C$可能取$+\infty$,不一定是次线性泛函。
    \end{proposition}
    \begin{proof}
        只说明3.不妨设$P_C(x),P_C(y)\in \R$,
        \begin{equation*}
            \forall \varepsilon>0,\lambda\defeq P_C(x)+\frac{\varepsilon}{2},
            \mu\defeq P_C(y)+\frac{\varepsilon}{2}
        \end{equation*}
        于是
        \begin{align*}
            \frac{x}{\lambda},\frac{y}{\mu}\in C\Rightarrow&
            \frac{x+y}{\lambda+\mu}=\frac{\lambda}{\lambda+\mu}\frac{x}{\lambda}+\frac{\mu}{\lambda+\mu}\frac{y}{\mu}\in C\\
            \Rightarrow& \lambda+\mu\geqslant P_C(x+y)\\
            \Rightarrow& P_C(x+y)\leqslant P_C(x)+P_C(y)+\varepsilon
        \end{align*}
        令$\varepsilon\rightarrow 0$即得到结论。
    \end{proof}

    \begin{definition}
        $X$是复向量空间,$C$是包含$0$的凸集,
        如果$\forall x\in C,\forall \theta\in\R$,都有$\e^{\i \theta}x\in C$,则称$C$均衡。
    \end{definition}

    \begin{proposition}
        复向量空间中每个均衡、吸收凸集都决定了一个半范数。
    \end{proposition}
    \begin{proof}
        吸收$\Rightarrow P_C$是次线性泛函,均衡$\Rightarrow $齐次性。
    \end{proof}

    \begin{definition}
        $X$是实向量空间,$M$是闭子空间,
        称$M$是$X$的极大子空间是指
        任一$X$的闭子空间$Y$,若满足$M\subsetneqq Y$,则$Y=X$,
    \end{definition}

    \begin{proposition}
        $M$是极大子空间$\Leftrightarrow \exists x_0\in X$使得
        $X=M\oplus {\rm span}\{x_0\}$,也就是${\rm codim\ }M={\rm dim}(X/M)=1$.
        (习题2.4.8.)
    \end{proposition}
    \begin{proof}
        $(\Leftarrow)$:
        ${\rm dim}(X/M)=1$,则$\forall x_0\notin M$
        \begin{equation*}
            X/M=\{\lambda [x_0]:\lambda\in \K\}
        \end{equation*}
        假设存在线性子空间$S$使得$M\subsetneqq S$,则取
        $x_0\in S\backslash M$,从而
        \begin{equation*}
            {\rm span}{x_0}={\rm span}{x_0}\oplus M\subset S
        \end{equation*}
        而实际上$[\lambda x_0]=\{ \lambda x_0+m:m\in M \}$,因此
        \begin{equation*}
            X
            =\{ \lambda x_0+m:m\in M,\lambda\in\K \}
            ={\rm span}\{x_0\}\oplus M\subset S
        \end{equation*}
        只能$X=S$,与极大性矛盾。

        $(\Rightarrow )$:取$x_0\notin M$,则
        \begin{equation*}
            M\subsetneqq \bigcup_{\lambda\in\K}(ax_0+M)
            ={\rm span}\{x_0\}\oplus M
        \end{equation*}
        从而后者$=X$,所以${\rm dim}(X/M)=1$.
    \end{proof}

    \begin{definition}
        超平面是指极大子空间的平移。

        对于$X$上线性泛函$f$和$r\in\R$,
        \begin{equation*}
            H_f^r\defeq f^{-1}(\{r\})=\{ x\in X:f(x)=r \}
        \end{equation*}
    \end{definition}

    \begin{proposition}
        $L$是超平面$\Leftrightarrow $存在某个$f$和$r$,使得$L=H_f^r$.
    \end{proposition}
    \begin{proof}
        充分性:注意到$H_f^0={\rm Ker}(f)$,Claim:$H_f^0$是极大子空间。
        取$x_0\in X\backslash H_f^0$,
        \begin{align*}
            f(x-\frac{f(x)}{f(x_0)}x_0)=0,\forall x\in X
            \Rightarrow& f-\frac{f(x)}{f(x_0)}x_0\in H_f^0,\forall x\in X\\
            \Rightarrow& X=H_f^0\oplus {\rm span}\{x_0\}
        \end{align*}
        Claim得证。令$r\defeq f(x_0)$,则
        \begin{equation*}
            x\in H_f^r\Leftrightarrow f(x-x_0)=f(x)-f(x_0)=0
            \Leftrightarrow x-x_0\in H_f^0\Leftrightarrow x\in H_f^0+x_0
        \end{equation*}
        所以$H_f^r$是极大子空间$H_f^0$的平移,为超平面。

        必要性:设$L=M+a$,$M$是极大子空间,那么存在某个$x_0$使得
        $X=M\oplus {\rm span}\{x_0\}$,令
        \begin{equation*}
            f:X\rightarrow \R,x=y+\lambda x_0\mapsto \lambda
        \end{equation*}
        于是$f(M)=\{ 0\}$且$f(x_0)=1$,进而
        $M\subset H_f^0$,由$M$极大$\Rightarrow M=H_f^0$,因此$L=H_f^r{\rm\ with\ }r=1$.
    \end{proof}

    \begin{proposition}
        $f\in X^*\Rightarrow \forall r\in \R,H_f^r$是闭超平面。
    \end{proposition}

    \begin{definition}
        设$X$是实向量空间,$A,B\subset X$,
        \begin{enumerate}
            \item  称$H_f^r$分离$A,B$是指:
                \begin{equation*}
                    \fun{sup}{x\in X}f(x)\leqslant r\leqslant \fun{inf}{y\in B}f(y)
                \end{equation*}
                或者
                \begin{equation*}
                    \fun{sup}{y\in B}f(y)\leqslant r\leqslant \fun{inf}{x\in A}f(x)
                \end{equation*}
            \item 称$H_f^r$严格分离$A,B$是指上述不等号严格成立。
        \end{enumerate}
    \end{definition}

    \begin{theorem}
        $X$是实赋范空间,$C$是有内点的凸集,$x_0\notin C\Rightarrow \exists f\in X^*,\exists r\in \R$满足
        $H_f^r$分离$x_0$和$C$.
    \end{theorem}
    \begin{proof}
        不妨设$0$是$C$的内点,$P_C$是$C$的Minkowski泛函,
        由习题1.5.1,
        $P_C$是次线性泛函,并且
        \begin{equation*}
            \overline{C}=\{ x\in X:P_C(x)\leqslant 1 \}
        \end{equation*}
        $x_0\notin C\Rightarrow P_C(x_0)\geqslant 1$,$0$是$C$的内点
        $\Rightarrow \exists \varepsilon>0{\rm\ s.t.\ }B(0,\varepsilon)\subset C$,于是
        \begin{equation*}
            \forall x\in X,x\neq 0,\varepsilon\frac{x}{||x||}\in \overline{B(0,\varepsilon)}\subset\overline{C}
        \end{equation*}
        进而
        \begin{equation*}
            P_C(\varepsilon\frac{x}{||x||})\leqslant 1,\forall 0\neq x\in X
        \end{equation*}
        也就是
        \begin{equation*}
            P_C(x)\leqslant \frac{1}{\varepsilon}||x||,\forall x\in X
        \end{equation*}
        
        令$M={\rm span}\{x_0\}$,定义
        \begin{equation*}
            f_0:M\rightarrow\R,x=\lambda x_0\mapsto \lambda P_C(x_0)
        \end{equation*}
        则$f_0(x)\leqslant P_C(x),\forall x\in M$,由实HBT可得
        存在$f:X\rightarrow \R$满足$f|_M=f_0$且$f(x)\leqslant P_C(x),\forall x\in X$,进而
        \begin{equation*}
            f(x_0)=f_0(x_0)=P_C(x_0)\geqslant 1,f(x)\leqslant P_C(x)\leqslant 1,\forall x\in C
        \end{equation*}
        因此$H_f^1$分离$x_0$和$C$.

        只剩下证明$f\in X^*$,
        \begin{align*}
            f(x)\leqslant P_C(x)\leqslant \frac{1}{\varepsilon}||x||,\forall x\in X
            \Rightarrow & -f(x)\leqslant \frac{1}{\varepsilon}||x||\\
            \Rightarrow & |f(x)|\leqslant \frac{1}{\varepsilon}||x||,\forall x\in X\\
            \Rightarrow & f\in X^*
        \end{align*}
    \end{proof}

    \begin{theorem}[Hahn-Banach 凸集分离定理]
        $X$是实赋范空间,$A$是开凸集,$B$是凸集,若$A\cap B=\varnothing$,
        则存在$H_f^r$闭,并分离$A,B$.
    \end{theorem}
    \begin{proof}
        令$C=A-B$,即
        \begin{equation*}
            C=\{ x-y:x\in A,y\in B \}
            =\bigcup_{y\in B} (A-y)
        \end{equation*}
        那么$C$是凸开集,而且$0\notin C$,
        由定理2.7.6,$H_f^0$分离$C$和$\{0\}$,
        即存在$f\in X^*$使得
        \begin{equation*}
            \fun{sup}{\delta\in C}f(\delta)\leqslant 0=f(0)
        \end{equation*}
        而
        \begin{equation*}
            \fun{sup}{\delta\in C}f(\delta)
            =\fun{sup}{x\in A,y\in B}[f(x)-f(y)]
            =\fun{sup}{x\in A}f(x)-\fun{inf}{y\in B}f(y)
        \end{equation*}
        因此
        \begin{equation*}
            \fun{sup}{x\in A}f(x)\leqslant r\leqslant \fun{inf}{y\in B}f(y)
        \end{equation*}
        这里
        \begin{equation*}
            r\defeq \frac{1}{2}[ \fun{sup}{x\in A}f(x)+\fun{inf}{y\in B}f(y) ]
        \end{equation*}
    \end{proof}

    \begin{theorem}[Hahn-Banach 凸集分离定理2]
        $X$是实赋范空间,$A$是闭凸集,$B$是紧凸集,若$A\cap B=\varnothing$,
        则存在$H_f^r$闭,并严格分离$A,B$.
    \end{theorem}
    \begin{proof}
        $A$闭$B$紧且不交,所以${\rm dist}(A,B)>0$,
        令$\varepsilon=\frac{1}{4}{\rm dist}(A,B)$,
        \begin{equation*}
            A_\varepsilon\defeq A+B(0,\varepsilon),
            B_\varepsilon\defeq B+B(0,\varepsilon)
        \end{equation*}
        这两个都是开凸集且不交,由定理2.7.7可知$\exists f\in X^*,\exists r\in \R$满足
        \begin{equation*}
            \fun{sup}{x\in A_\varepsilon}f(x)\leqslant r\leqslant 
            \fun{inf}{y\in B_\varepsilon}f(y)
        \end{equation*}
        所以
        \begin{equation*}
            f(x+\varepsilon\delta)\leqslant r\leqslant f(y+\varepsilon\delta),\forall x\in A,\forall y\in B,\forall \delta\in B(0,1)
        \end{equation*}
        \begin{align*}
            \Rightarrow & -f(\delta)\leqslant \frac{f(y)-r}{\varepsilon}\\
            \Rightarrow & ||f||=\fun{sup}{\delta\in B(0,1)}f(-\delta)\leqslant \frac{f(y)-r}{\varepsilon}\\
            \Rightarrow & r\leqslant f(y)-\varepsilon||f||,\forall y\in B\\
            \Rightarrow & r\leqslant \fun{inf}{y\in B}f(y)-\varepsilon||f||<\fun{inf}{y\in B}f(y)
        \end{align*}
        同理,
        \begin{equation*}
            \fun{sup}{x\in A}f(x)<\fun{sup}{x\in A}f(x)+\varepsilon ||f||\leqslant r
        \end{equation*}
    \end{proof}
\subsection*{应用和推论}
    \begin{corollary}[Ascoli]
        $X$是实赋范空间,$C$是闭凸集,若$x_0\notin C$,则
        \begin{equation*}
            \exists f\in X^*,\exists r\in \R{\rm\ s.t.\ }
            \fun{sup}{x\in C}f(x)<r<f(x_0)
        \end{equation*}
    \end{corollary}

    \begin{corollary}
        $X$是实赋范空间,$M$是其子空间,
        \begin{equation*}
            \overline{M}\neq X\Leftrightarrow \exists f\in X^*,f\neq 0{\rm\ s.t.\ }f(M)=\{0\}
        \end{equation*}
        等价地,
        \begin{equation*}
            \overline{M}=X\Leftrightarrow \forall f\in X^*{\rm\ with\ }f(M)=\{0\}\Rightarrow f=0
        \end{equation*}
    \end{corollary}
    \begin{proof}
        假设存在$x_0\in X\backslash \overline{M}$,由Ascoli,
        \begin{equation*}
            \exists f\in X^*,\exists r\in \R{\rm\ s.t.\ }
            \fun{sup}{x\in\overline{M}}f(x)<r<f(x_0)
        \end{equation*}
        进而$f|_{\overline{M}}=0\Rightarrow f(M)=\{0\}$,所以$0<r<f(x_0)$,$f\neq 0$,矛盾。所以$X=\overline{M}$.
    \end{proof}

    \begin{corollary}[Mazur]
        $X$是实赋范空间,$C$是开凸集,$F$是线性子流形(子空间的平移),
        若$C\cap F=\varnothing$,则存在$H_f^r$闭满足$F\subset H_f^r$且
        $\fun{sup}{x\in C}f(x)\leqslant r$.
    \end{corollary}
    \begin{proof}
        设$F=M+x_0$,$M$是子空间,由分离定理
        \begin{equation*}
            \exists f\in X^*,\exists s\in\R
            {\rm\ s.t.\ }
            \fun{sup}{x\in C}f(x)\leqslant s\leqslant \fun{inf}{y\in F}f(y)
            =\fun{inf}{\delta\in M}f(\delta)+f(x_0)
        \end{equation*}
        进而
        \begin{align*}
            \fun{inf}{\delta\in M}f(\delta)\geqslant s-f(x_0)
            \Rightarrow & f|_M=0\\
            \Rightarrow & M\subset H_f^0\\
            \Rightarrow & F\subset H_f^r{\rm\ with\ }r=f(x_0)
        \end{align*}
        同时,
        \begin{equation*}
            \fun{sup}{x\in C}f(x)\leqslant s\leqslant f(x_0)=r
        \end{equation*}
    \end{proof}

    \begin{definition}
        称超平面$L=H_f^r$是凸集$C$在$x_0$处的承托超平面是指:
        $C$完全落在$L$的一侧,且$x_0\in \overline{C}\cap L$,即
        \begin{equation*}
            \fun{sup}{x\in C}f(x)\leqslant r=f(x_0)
        \end{equation*}
        或者
        \begin{equation*}
            \fun{inf}{x\in C}f(x)\geqslant r=f(x_0)
        \end{equation*}
    \end{definition}

    \begin{theorem}
        $X$是实赋范空间,$C$是有内点的闭凸集,$\forall x_0\in \partial C$均有$C$的一个承托超平面。
    \end{theorem}
    \begin{proof}
        令$E=C^\circ$,即$C$的全体内点,$F=\{x_0\}$,由Mazur可得
        \begin{equation*}
            \exists f\in X^*,\exists r\in\R{\rm\ s.t.\ }
            \fun{sup}{x\in E}f(x)\leqslant r{\rm\ and\ }\{x_0\}\subset H_f^r
        \end{equation*}
        由连续性
        \begin{equation*}
            \fun{sup}{x\in C}f(x)\leqslant r=f(x_0)
        \end{equation*}
    \end{proof}

    \begin{example}
        $C=B(0,r)$,$\forall x_0\in \partial B(0,r)$,均有承托超平面。
    \end{example}
    \begin{proof}
        $\exists f\in X^*,||f||=1$使得$f(x_0)=||x_0||=r$,而
        \begin{equation*}
            \fun{sup}{x\in C}f(x)\leqslant ||f||\fun{sup}{x\in C}||x||=r
        \end{equation*}
    \end{proof}

    \begin{example}
        设$\sum_{k=1}^\infty x_k$是Banach空间$X$中绝对收敛级数,
        $\{y_k\}_{k=1}^\infty$是$\{x_k\}_{k=1}^\infty$
        的任一重排,则
        \begin{equation*}
            \sum_{k=1}^\infty y_k=\sum_{k=1}^\infty x_k
        \end{equation*}
    \end{example}
    \begin{proof}
        $\forall f\in X^*$,
        \begin{equation*}
            \sum_{k=1}^\infty |f(x_k)|\leqslant ||f||\sum_{k=1}^\infty ||x_k||<\infty
        \end{equation*}
        于是$\sum_{k=1}^\infty f(x_k)$是$\K$上的绝对收敛级数,
        重排不变,所以
        \begin{equation*}
            \sum_{k=1}^\infty f(y_k)=\sum_{k=1}^\infty f(x_k)
            \Rightarrow f(\sum_{k=1}^\infty y_k)
            =f(\sum_{k=1}^\infty x_k)
            ,\forall f\in X^*
            \mathop{\Rightarrow}\limits^{ \rm Cor\ 2.7.4 }
            \sum_{k=1}^\infty y_k=\sum_{k=1}^\infty x_k
        \end{equation*}
    \end{proof}


\section{对偶空间、自反空间、弱收敛}
%Lec21
\subsection{对偶空间}
    \begin{definition}
        $X$的对偶空间$X^*$是指$X$上所有线性泛函组成的空间,即
        \begin{equation*}
            X^*\defeq \mathcal{L}(X,\K)
        \end{equation*}
    \end{definition}    

    回顾:
    $(X,m,\mu)$是一可测空间,
    $\Omega$是$X$上可测函数全体,
    \begin{equation*}
        L^p=L^p(\Omega,m,\mu)\defeq \{ f\in\Omega:||f||_p=(\int_X |f|^p\d \mu)^\frac{1}{p}<\infty  \}
    \end{equation*}
    对于$1\leqslant p\leqslant \infty$,对偶空间$(L^p)^*$是什么?
    \begin{theorem}[Riesz]
        设$1\leqslant p<\infty$,则$(L^p)^*=L^{q}$,其中
        \begin{equation*}
            q=\left\{ \begin{array}{ll}
                \frac{p}{p-1}&,1<p<\infty\\
                \infty&,p=1
            \end{array} \right.
        \end{equation*}
    \end{theorem}
    \begin{proof}
        我们希望构造出一个线性等距同构$J:L^q\rightarrow (L^p)^*$,如下:
        \begin{equation*}
            \forall g\in L^q,\Lambda_g:L^p\rightarrow \K,f\mapsto \Lambda_g(f)\defeq \int fg
        \end{equation*}
        \begin{equation*}
            J:L^q\rightarrow (L^p)^*,g\mapsto \Lambda_g
        \end{equation*}
        需要证明:
        \begin{enumerate}[$1^\circ$]
            \item $\Lambda_g\in (L^p)^*$.
            \item $J$线性。
            \item $||\Lambda_g||=||g||_q$,即$J$等距。
            \item $\forall \Lambda\in (L^p)^*$,存在唯一$g\in L^{p'}$使得$\Lambda=\Lambda_g$,即$J$是双射。
        \end{enumerate}
        
        \textbf{Proof of }$1^\circ-3^\circ$:
        先考虑$1<p<\infty$,
        \begin{equation*}
            \forall f\in L^p,|\Lambda_g(f)|=|\int fg|\leqslant ||g||_q\cdot ||f||_p
        \end{equation*}
        于是$\Lambda_g\in (L^p)^*$,且$||\Lambda_g||\leqslant ||g||_q$.取
        \begin{equation*}
            \tilde{f}\defeq |g|^{q-1}{\rm sgn}(g)
        \end{equation*}
        则
        \begin{align*}
            \left\{ \begin{array}{ll}
                |\tilde{f}|^p=|g|^{(q-1)p}=|g|^q\\
                \tilde{f}\cdot g=|g|^q
        	\end{array} \right.\Rightarrow& 
            \left\{ \begin{array}{ll}
                |\tilde{f}|_p^p=||g||_q^q\\
                \Lambda_g(\tilde{f})=||g||_q^q
        	\end{array} \right.\\
            \Rightarrow&
            \frac{|\Lambda_g(\tilde{f})|}{||\tilde{f}||_p}
            =\frac{||g||^q_q}{||g||_q^{\frac{q}{p}}}=
            ||g||_q^{q(1-\frac{1}{p})}=||g||_q\\
            \Rightarrow& ||\Lambda_g||\geqslant ||g||_q
        \end{align*}
        接下来考虑$p=1$,此时需假设$\mu$是$\sigma$-有限的,不妨设$\mu$有限,
        由
        \begin{equation*}
            |\Lambda_g(f)|\leqslant ||g||_\infty ||f||_1
        \end{equation*}
        得$\Lambda_g\in (L^1)^*$且$||\Lambda_g||\leqslant ||g||_\infty$,
        令
        \begin{equation*}
            E_k\defeq \{ t\in\Omega:|g(t)|>||\Lambda_g||+\frac{1}{k} \},
            f_k\defeq \chi_{E_k}\cdot {\rm sgn}(g)
        \end{equation*}
        \begin{align*}
            \Rightarrow& ||f_k||_1=\int_{E_k} |{\rm sgn}(g)|\d \mu\leqslant \mu(E_k)\\
            \Rightarrow& ||\Lambda_g||\mu(E_k)\geqslant 
            ||\Lambda g||\geqslant |\Lambda_g(f_k)|=\int \chi_{E_k}{\rm sgn}(g)g\d \mu=\int_{E_k}|g|\d \mu\geqslant
            ( ||\Lambda_g||+\frac{1}{k} )\mu(E_k)\\
            \Rightarrow& \mu(E_k)=0,\forall k\\
            \Rightarrow& \{ t\in \Omega:|g(t)|>||\Lambda_g|| \}=\bigcup_{k=1}^\infty E_k\mbox{零测}\\
            \Rightarrow& ||g||_\infty\leqslant ||\Lambda_g||
        \end{align*}

        \textbf{Proof of }$4^\circ$:
        以下假设$\Omega=[0,1]$,$\mu=m$,
        \begin{lemma}
            设$g\in L^1$,如果存在$C>0$满足
            \begin{equation*}
                |\int fg|\leqslant C||f||_p,\forall f\in L^\infty
            \end{equation*}
            则$g\in L^q$且$||g||_q\leqslant C$.
            \begin{proof}
                
            \end{proof}
        \end{lemma}
        后面也太复杂了,不抄了。
    \end{proof}
%Lec22
    那么,$(L^\infty)^*$是$L^1$吗?答案是否定的。
    \begin{theorem}
        $L^1\subsetneqq (L^\infty)^*$.
    \end{theorem}
    \begin{proof}
        对于$\forall g\in L^1$,
        \begin{equation*}
            |\Lambda_g(f)|=|\int fg|\leqslant ||g||_1||f||_\infty\Rightarrow\Lambda_g\in (L^\infty)^*
            \Rightarrow L^1\subset (L^\infty)^*
        \end{equation*}
        注意到$C[0,1]$是$L^\infty$的闭子空间,取$f_0\in L^\infty\backslash C[0,1]$,
        于是$d\defeq {\rm dist}(f_0,C[0,1])>0$,由HBT可知
        存在$\Lambda\in (L^\infty)^*,||\Lambda||=1$且
        \begin{equation*}
            \Lambda(C[0,1])=\{0\},\Lambda(f_0)=d
        \end{equation*}
        假设存在$g\in L^1$满足$\Lambda=\Lambda_g$,即
        \begin{equation*}
            \Lambda(f)=\int fg,\forall f\in L^\infty
        \end{equation*}
        那么对于$f\in C[0,1]$,$\Lambda(f)=\int fg=0$,
        取$\{ f_n \}_{n=1}^\infty \subset C[0,1]$,
        满足
        \begin{equation*}
            ||f_n-{\rm sgn}(g)||_1\rightarrow 0{\rm\ as\ }n\rightarrow\infty
        \end{equation*}
        于是有子列$f_{n_k}\mathop{\rightarrow}\limits^{\rm a.e.} 
        {\rm sgn}(g)$,由MCT得
        \begin{equation*}
            \int |g|=\fun{lim}{k\rightarrow\infty}\int f_{n_k}g=0\Rightarrow g\mathop{=}\limits^{\rm a.e.} 0
            \Rightarrow \Lambda=\Lambda_g=0
        \end{equation*}
        这与$\Lambda(f_0)=d>0$矛盾。
    \end{proof}

    接下来讨论$C[a,b]$的对偶空间。
    \begin{definition}
        对$f:[a,b]\rightarrow \mathbb{C}$和$[a,b]$的划分$P$,定义
        \begin{equation*}V(f,P)\mathop{=}\limits^{\rm def}\sum_{k=1}^N |f(t_k)-f(t_{k-1})|\end{equation*}
        如果$\mathop{\rm sup}\limits_{P} V(f,P)<\infty$,则称$f$是有界变差的。
        \begin{equation*}V_a^b(f)\mathop{=}\limits^{\rm def} \mathop{\rm sup}\limits_{P} V(f,P)\end{equation*}
        称为$f$在$[a,b]$上的全变差。
        \begin{equation*}BV[a,b]\mathop{=}\limits^{\rm def}\{ [a,b]\mbox{上的有界变差函数全体} \}\end{equation*}

        我们进一步给出$BV[a,b]$上的范数:
        \begin{equation*}
            ||f||_{BV}\defeq |f(a)|+V_a^b(f)
        \end{equation*}
        那么$(BV[a,b],||\cdot||_{BV})$是Banach空间,定义:
        \begin{equation*}
            BV_0[a,b]\defeq \{ f\in BV[a,b]:f\mbox{在$(a,b)$上右连续,}f(a)=0 \}
        \end{equation*}
        那么$BV_0[a,b]$是$BV[a,b]$的闭子空间,也是Banach空间。
    \end{definition}

    \begin{definition}[Riemann-Stieltjes积分]
        设$f,g$是$[a,b]$上实值函数,$I\in\R$,
        对$[a,b]$进行划分:
        \begin{equation*}
            \sigma(\Delta,\xi)\defeq \sum_{k=1}^n f(\xi_k)[ g(t_k)-g(t_{k-1}) ]
        \end{equation*}
        其中
        \begin{equation*}
            \xi=\{\xi_k\}_{k=1}^\infty,\xi_k\in [t_{k-1},t_k]
        \end{equation*}
        如果$\sigma(\Delta,\xi)\rightarrow I{\rm\ as\ }||\Delta||\rightarrow 0$,
        则记
        \begin{equation*}
            I=\int_a^b f\d g
        \end{equation*}
        称为$f$关于$g$的Riemann-Stieltjes积分,简记为R-S积分。
    \end{definition}

    \begin{theorem}[Riesz]
        $C[a,b]^*=BV_0[a,b]$.

        思路:$\forall g\in BV_0[a,b]$,取
        \begin{equation*}
            \Lambda_g(f)\defeq \int_a^b f\d g,f\in C[a,b]
        \end{equation*}
        证明$g\mapsto \Lambda_g$是线性等距同构。
    \end{theorem}
\subsection{自反空间}
    \begin{definition}
        $X^{**}=\mathcal{L}(X^*,\K)$,
        称为$X$的二次对偶或第二共轭空间。
        
        设$x\in X$,定义映射:
        \begin{equation*}
            x^{**}:X^*\rightarrow \K,f\mapsto f(x)
        \end{equation*}
        则
        \begin{equation*}
            |x^{**}(f)|=|f(x)|\leqslant ||x||\cdot ||f||,\forall f\in X^*
        \end{equation*}
        因此$x^{**}\in X^{**}$且$||x^{**}||\leqslant ||x||$.
        另一方面由推论2.7.1,存在$f_0\in X^*{\rm\ with\ }||f_0||=1{\rm\ s.t.\ }f_0(x)=||x||$,即
        \begin{equation*}
            x^{**}(f_0)=f_0(x)=||x||
        \end{equation*}
        于是$||x^{**}||\geqslant |x^{**}(f_0)|/||f_0||=||x||$,
        于是映射
        \begin{equation*}
            i:X\rightarrow X^{**},x\mapsto x^{**}
        \end{equation*}
        是线性等距嵌入,
        称为$X$到$X^{**}$的自然映射或自然嵌入(canonical map).
    \end{definition}

    \begin{definition}
        如果自然映射$i$是满射,从而是线性等距同构,则称$X$自反,记作$X^{**}=X$.
    \end{definition}

    \begin{example}
        不完备的空间一定不自反、
        有限维赋范空间一定自反(习题2.5.4)。
    \end{example}

    \begin{example}
        Hilbert空间自反。(作业)
    \end{example}
    \begin{proof}
        由 Riesz 表示定理,有等距同构 $\varphi:H\to H^*,x\mapsto\langle\cdot,x\rangle.\varphi$ 诱导了$H^*$ 上的内积 $\langle f,g\rangle=\overline{\langle\varphi^{-1}(f),\varphi^{-1}(g)\rangle}$。对偶空间自然完备,因此$H^*$ 也 
        是 Hilbert 空间,由 Riesz 表示定理,又有等距同构 $\Phi:H^*\to H^{**},f\mapsto$ $\overline{{\langle\varphi^{-1}(\cdot),\varphi^{-1}(f)\rangle}}.\:计算得$
        \begin{equation*}
            \Phi\circ\varphi(x)(f)=\overline{\langle\varphi^{-1}(f),x\rangle}=\langle x,\varphi^{-1}(f)\rangle=f(x),\forall x\in H,f\in H^{*}
        \end{equation*}
        故$\Phi\circ\varphi$ 就是 $H\to H^{**}$ 的自然映射 $\phi$, 因此 $\phi$ 也是等距同构,从而$H$自反。
    \end{proof}

    \begin{theorem}
        当$1<p<\infty$时,$L^p$自反。
    \end{theorem}
    \begin{proof}
        即证明:$\forall \Lambda\in (L^p)^{**}$,存在$u\in L^p$使得
        \begin{equation*}
            \Lambda(f)=f(u),\forall f\in (L^p)^*
        \end{equation*}
        这是因为
        \begin{equation*}
            i:L^p\rightarrow (L^p)^{**} \mbox{满}
            \Leftrightarrow \forall \Lambda\in (L^p)^{**},\exists u\in L^p{\rm\ s.t.\ }
            u^{**}=\Lambda ( \Lambda(f)=u^{**}(f)=f(u) )
        \end{equation*}
        回顾定理2.8.1,
        \begin{equation*}
            J:L^q\rightarrow (L^p)^*,v\mapsto f_v
        \end{equation*}
        是线性等距同构,这里$f_v(u)=\int uv$.取
        \begin{equation*}
            \varphi\defeq \Lambda\circ J
        \end{equation*}
        则$\varphi\in (L^q)^*=L^p$,
        存在唯一$u\in L^p$满足$\varphi(v)=\int uv,\forall v\in L^q$.
        那么对于$\forall f\in (L^p)^*$,令
        \begin{equation*}
            v_f\defeq J^{-1}(f)
        \end{equation*}
        那么
        \begin{equation*}
            \Lambda(f)=\Lambda( J(v_f) )=\varphi(v_f)=\int v_fu=f(u)
        \end{equation*}
    \end{proof}
    
    \begin{theorem}
        $C[a,b]$不自反。
    \end{theorem}
    \begin{proof}
        假设自反,则
        $\forall \Lambda\in C[a,b]^{**}$,存在$u\in C[a,b]$满足
        \begin{equation*}
            \Lambda(f)=f(u),\forall f\in C[a,b]^*\tag{*}
        \end{equation*}
        根据$C[a,b]^*=BV_0[a,b]$,
        \begin{equation*}
            \exists ! v_f\in BV_0[a,b]{\rm\ s.t.\ }
            f(u)=\int_a^b u\d v_f,\forall u\in C[a,b]{\rm\ and\ }
            ||v_f||_{BV}=||f||
        \end{equation*}
        令$c=\frac{a+b}{2}$,定义
        \begin{equation*}
            F_c:C[a,b]^*\rightarrow\R,f\mapsto v_f(c+0)-v_f(c-0)
        \end{equation*}
        那么
        \begin{equation*}
            |F_c(f)|\leqslant V_a^b (v_f)=||v_f||_{BV}=||f||\Rightarrow F_c\in C[a,b]^{**}
        \end{equation*}
        根据(*),存在$u_c\in C[a,b]$满足
        \begin{equation*}
            F_c(f)=f(u_c)=\int_a^b u_c\d v_f,\forall f\in C[a,b]^*
        \end{equation*}
        令
        \begin{equation*}
            v(f)\defeq \int_a^t u_c(s)\d s
        \end{equation*}
        那么$v\in BV_0[a,b]$,令$f_v\defeq J(v)$,这里
        \begin{equation*}
            J:BV_0[a,b]\rightarrow C[a,b]^*,v\mapsto f_v,f_v(u)=\int_a^b u \d v_f
        \end{equation*}
        $f_v$对应的$v_{f_v}=v$连续,于是$F_c(f_v)=0$,所以
        \begin{equation*}
            0=F_c(f_v)=\int_a^b u_c\d v=\int_a^b u_c^2\d t
        \end{equation*}
        所以$u_c=0$,进而$F_c=0$,矛盾。
    \end{proof}
%Lec23
    \begin{theorem}[Banach]
        $X^*$可分$\Rightarrow X$可分。逆命题不成立,例如$L^1$可分但$L^\infty$不可分。
    \end{theorem}
    \begin{proof}
        \textbf{Step1}:证明$X^*$中单位球面$S_1^*$可分。
        $X^*$可分,取$\{f_n\}_{n=1}^\infty$为其稠密子集,
        不妨$f_n\neq 0$,令
        \begin{equation*}
            g_n\defeq \frac{f_n}{||f_n||}\in S_1^* 
        \end{equation*}
        那么对于$\forall g\in S_1^*$,存在$f_{n_k}\rightarrow g$.
        \begin{align*}
            \Rightarrow ||g-g_{n_k}||
            \leqslant & ||g-f_{n_k}||+||f_{n_k}-g_{n_k}||\\
            =& ||g-f_{n_k}||+||( ||f_{n_k}||-1 )\frac{f_{n_k}}{||f_{n_k}||}||\\
            =& ||g-f_{n_k}||+\left| ||f_{n_k}||+1 \right|\rightarrow 0{\rm\ as\ }k\rightarrow\infty
        \end{align*}

        \textbf{Step2}:证明存在$\{x_n\}_{n=1}^\infty\subset X$,
        其中$\forall ||x_n||=1$,且
        \begin{equation*}
            \overline{ {\rm span}\{x_n\}_{n=1}^\infty }=X
        \end{equation*}
        注意到
        \begin{equation*}
            ||g_n||=\fun{sup}{x\in X,||x||=1}|g_n(x)|=1
        \end{equation*}
        所以存在$\{x_n\}_{n=1}^\infty\subset X$,
        其中$\forall ||x_n||=1$,且$|g_n(x_n)|>\frac{1}{2}$.令$M\defeq {\rm span}\{x_n\}_{n=1}^\infty$,
        假设$\overline{M}\neq X$,取$x_0\in X\backslash \overline{M}$,由HBT可得
        \begin{equation*}
            \exists f\in X^*,||f||=1{\rm\ s.t.\ }f(\overline{M})=\{0\}{\rm\ and\ }
            f(x_0)={\rm dist}(x_0,\overline{M})>0
        \end{equation*}
        于是
        \begin{equation*}
            ||g_n-f||= \fun{sup}{x\in X,||x||=1}|g_n(x)-f(x)|>|g_n(x_n)-f(x_n)|=|g_n(x_n)|>\frac{1}{2}
        \end{equation*}
        这与Step1矛盾。

        \textbf{Step3}:证明$\overline{ {\rm span}^{\mathbb{Q}}\{x_n\}_{n=1}^\infty }=X$.
    \end{proof}

    \begin{theorem}
        当$1\leqslant p<\infty$,$L^p[0,1]$可分。
    \end{theorem}
    \begin{proof}
        \begin{equation*}
            \left\{ \sum_{k=0}^{2^n-1} r_k\chi_{ [ \frac{k}{2^n},\frac{k+1}{2^n} ) }:r_k\in\Q,n\in\N_0 \right\}
        \end{equation*}
        是$L^p[0,1]$的可数稠密子集。
    \end{proof}

    \begin{theorem}
        $L^\infty$不可分。
    \end{theorem}
    \begin{proof}
        假设存在稠密子集$\{f_n\}_{n=1}^\infty$,
        \begin{equation*}
            \forall t\in (0,1),\exists f_{n_t}\in B(\chi_{[0,t]},\frac{1}{3})
        \end{equation*}
        而$t\neq s$时,${\rm dist}(\chi_{[0,t]},\chi_{[0,s]})=1$,
        因此不同的$B(\chi_{[0,t]},\frac{1}{3})$不相交,所以$\varphi:(0,1)\rightarrow\N,t\mapsto n_t$是单射,
        于是$(0,1)$可数,矛盾。
    \end{proof}

    \begin{theorem}
        $L^1$不自反。
    \end{theorem}
    \begin{proof}
        $(L^1)^*=L^\infty$,假设$L^1$自反,
        则$(L^\infty)^*\cong L^1$,$L^1$可分
        $\Rightarrow L^\infty $可分,矛盾。
    \end{proof}

    \begin{theorem}[共轭算子]
        $X,Y$是赋范空间,$T\in \L(X,Y)\Rightarrow \exists T^*\in \L( Y^*,X^* )$使得
        \begin{equation*}
            (T^* f)(x)=f(Tx),\forall f\in Y^*,\forall x\in X
        \end{equation*}
        $T^*$称为$T$的共轭算子。进而映射$*:\L(X,Y)\rightarrow \L(Y^*,X^*),T\mapsto T^*$
        是一个线性等距嵌入。
    \end{theorem}
    \begin{proof}
        设$f\in Y^*$,定义映射:
        \begin{equation*}
            \Lambda_f:X\rightarrow\K,x\mapsto f(Tx)
        \end{equation*}
        则
        \begin{equation*}
            |\Lambda_f(x)|=|f(Tx)|\leqslant ||f||\cdot ||Tx||\leqslant ||f||\cdot ||T||\cdot ||x||,\forall x\in X
        \end{equation*}
        于是$\Lambda_f\in X^*$,且$||\Lambda_f||\leqslant ||f||\cdot||T||$,定义映射
        \begin{equation*}
            T^*:Y^*\rightarrow X^*,f\mapsto \Lambda_f
        \end{equation*}
        于是$T^*$线性而且$||T^*f||=||\Lambda_f||\leqslant ||T||\cdot ||f||$,
        从而$T^*$有界且$||T^*||\leqslant ||T||$.

        对于$\forall x\in X$,不妨$x\neq 0$,由HBT,
        \begin{equation*}
            \exists f\in Y^*,||f||=1,f(Tx)=||Tx||
        \end{equation*}
        于是
        \begin{equation*}
            ||Tx||=|f(Tx)|=|(T^*f)(x)|
            \leqslant ||T^*f||\cdot ||x||\leqslant 
            ||T^*||\cdot ||f||\cdot ||x||=||T^*||\cdot ||x||
        \end{equation*}
        所以$||T||\leqslant ||T^*||$.
    \end{proof}

    \begin{example}
        $T:\K^n \rightarrow \K^m,x\mapsto Ax{\rm\ with\ }A=(a_{ij})_{m\times n}$,
        $T^*:\K^m \rightarrow \K^n,y\mapsto \overline{A^T}y$.
    \end{example}

    \begin{theorem}[pettis]
        自反空间的闭子空间自反。
    \end{theorem}
    \begin{proof}
        设$X$自反,$Y$是其闭子空间,只需证明:
        \begin{equation*}
            \forall a\in Y^{**},\exists y\in Y{\rm\ s.t.\ }
            a(f)=f(y),\forall f\in Y^*
        \end{equation*}
        定义映射
        \begin{equation*}
            T:X^*\rightarrow Y^*,f\mapsto f|_Y
        \end{equation*}
        则$T$是有界线性映射,于是取$T^*\in\mathcal{L}(Y^{**},X^{**})$满足
        \begin{equation*}
            (T^*a)(f)=a(Tf),\forall f\in X^*
        \end{equation*}
        $X$自反,所以自然映射$i_X$是满射,$T^*a\in X^{**}\Rightarrow \exists y\in X{\rm\ s.t.\ }T^*a=y^{**}$,
        所以
        \begin{equation*}
            (T^*a)(f)=y^{**}(f)=f(y),\forall f\in X^*
        \end{equation*}

        下面证明$y\in Y$,假设不然,则存在$\tilde{f}\in X^*$使得
        $\tilde{f}(Y)=0$,
        \begin{equation*}
            T(\tilde{f})=\tilde{f}|_Y=0\Rightarrow \tilde{f}(y)
            =(T^*a)(\tilde{f})=a(T(\tilde{f}))=0
        \end{equation*}
        这与$\tilde{f}(y)={\rm dist}(y,Y)>0$矛盾。

        最后说明:$a(f)=f(y),\forall f\in Y^*$.
        对于$\forall f\in Y^*$,由HBT,
        存在$F\in X^*$使得$f=TF$,
        所以
        \begin{equation*}
            a(f)=a(TF)=( T^*a )(F)=F(y)=f(y)
        \end{equation*}
    \end{proof}
%Lec24
\subsection{弱收敛}
    \begin{definition}
        $X$是赋范空间,称$\{ x_n \}_{n=1}^\infty \subset X$弱收敛于$x_0\in X$是指:
        \begin{equation*}
            f(x_n)\rightarrow f(x_0),\forall f\in X^*
        \end{equation*}
        记为$x_n\mathop{\rightarrow}\limits^{w} x_0$或者$x_n\rightarrow x_0$,称
        $x_0$为$\{ x_n \}_{n=1}^\infty $的弱极限。

        依范数拓扑收敛即为强收敛。
    \end{definition}
    \begin{proposition}
        强收敛$\Rightarrow $弱收敛。
    \end{proposition}
    \begin{proof}
        \begin{equation*}
            ||x_n-x_0||\rightarrow 0\Rightarrow |f(x_n)-f(x_0)|=| f(x_n-x_0) |
            \leqslant ||f||\cdot ||x_n-x_0||\rightarrow 0,\forall f\in X^*
        \end{equation*}
    \end{proof}

    \begin{example}
        $X=L^2(\Pi)$, $e_k(t)\defeq \e^{ -2\pi \i kt },k\in \Z$,则
        $e_k\mathop{\rightarrow}\limits^{w} 0{\rm\ as\ }|k|\rightarrow\infty$.即:
        \begin{equation*}
            \forall f\in X^*,\exists v\in L^2(\Pi){\rm\ s.t.\ }
            f(u)=\int_{\Pi} uv,u\in L^2(\Pi)
        \end{equation*}
        因此
        \begin{equation*}
            f(e_k)=\int_0^1 v(t) \e^{-2\pi \i kt}\d t=\hat{v}(k)\rightarrow 0{\rm\ as\ }|k|\rightarrow\infty
        \end{equation*}
    \end{example}

    \begin{theorem}
        ${\rm dim}X<\infty\Rightarrow$弱收敛与强收敛等价。
    \end{theorem}
    \begin{proof}
        设${\rm dim}(X)=m$,设$\{e_1,\cdots,e_m\}$是$X$的一组基,由HBT(习题2.4.7)知存在
        对偶基$f_1,\cdots,f_m\in X^*$满足
        \begin{equation*}
            f_k(e_j)=\delta_{kj},1\leqslant k,j\leqslant m
        \end{equation*}
        设$x_n\wto x_0$,即
        \begin{equation*}
            \sum_{j=1}^m \alpha_j^{(n)}e_j\wto \sum_{j=1}^m \alpha_j^{(0)}e_j
        \end{equation*}
        于是
        \begin{equation*}
            f_j(x_n)\rightarrow f_k(x_0),k=1,2,\cdots,m
        \end{equation*}
        因此$||x_n-x_0||_{\infty}\rightarrow 0{\rm\ with\ }||x||_{\infty}=\fun{max}{1\leqslant k\leqslant m}|\alpha_i|$,
        由有限维空间范数等价可得$||x_n-x_0||\rightarrow 0$.
    \end{proof}

    \begin{theorem}[Mazur]
        $x_n\mathop{\rightarrow}\limits^{w} x_0\Rightarrow x_0\in \overline{ {\rm conv}( \{ x_n \}_{n=1}^\infty ) }$
    \end{theorem}
    \begin{proof}
        令$C=\overline{ {\rm conv}( \{ x_n \}_{n=1}^\infty ) }$,假设$x_0\notin C$,则
        由Ascoli,存在$f\in X^*,\exists \alpha\in\R$使得
        \begin{equation*}
            \fun{sup}{x\in C}f(x)<\alpha<f(x_0)
        \end{equation*}
        $\Rightarrow f(x_n)<\alpha <f(x_0),n=1,2,\cdots$,
        与$f(x_n)\rightarrow f(x_0)$矛盾。
    \end{proof}

    \begin{definition}
        称$\{f_n\}_{n=1}^\infty \subset X^*$弱*收敛于$f\in X^*$是指
        \begin{equation*}
            f_n(x)\rightarrow f(x),\forall x\in X
        \end{equation*}
        记作$f_n\mathop{\rightarrow}\limits^{w^*}f $.
    \end{definition}
    $X^*$中,强收敛$\Rightarrow $弱收敛$\Rightarrow $弱*收敛。
    \begin{align*}
        f_n\mathop{\rightarrow}\limits^{w} f
        \mathop{\Leftrightarrow}\limits^{\rm def}& \Lambda(f_n)\rightarrow \Lambda(f),\forall \Lambda\in X^{**}\\
        \Rightarrow& x^{**}(f_n)\rightarrow x^{**}(f),\forall x\in X\\
        \Leftrightarrow& f_n(x)\rightarrow f(x),\forall x\in X\\
        \mathop{\Leftrightarrow}\limits^{\rm def}& f_n\mathop{\rightarrow}\limits^{w^*}f
    \end{align*}

    \begin{proposition}
        $X$自反$\Rightarrow X^*$中弱*收敛与弱收敛等价。
    \end{proposition}

    \begin{theorem}
        $X$是度量空间,
        \begin{equation*}
            x_n\mathop{\rightarrow}\limits^{w} x_n
            \Leftrightarrow
            \left\{ \begin{array}{l}
                \fun{sup}{n}||x_n||<\infty\\
                \exists \mathcal{F}\mathop{\subset }^{\rm dense} X^*{\rm\ s.t.\ }
                f(x_n)\rightarrow f(x_0),\forall f\in\mathcal{F}
            \end{array} \right.
        \end{equation*}
    \end{theorem}
    \begin{proof}
        \begin{align*}
            x_n\wto x_0\Leftrightarrow& f(x_n)\rightarrow f(x_0),\forall f\in X^*\\
            \Leftrightarrow& x_n^{**}(f)\rightarrow x_0^{**}(f),\forall f\in X^*\\
            \mathop{\Rightarrow}\limits^{\rm B-S}&
            \left\{ \begin{array}{ll}
                \fun{sup}{n}||x_n^{**}||<\infty\\
                \exists \mathcal{F}\mathop{\subset}\limits^{\rm dense}X^*{\rm\ s.t.\ }
                x_n^{**}(f)\rightarrow x_0^{**}(f),\forall f\in\mathcal{F}
            \end{array} \right.
        \end{align*}
    \end{proof}

    \begin{theorem}
        $X$是Banach空间,则
        \begin{equation*}
            f_k\mathop{\rightarrow}\limits^{w^*} f
            \Leftrightarrow
            \left\{ \begin{array}{l}
                \fun{sup}{n}||f_n||<\infty\\
                \exists M\mathop{\subset }^{\rm dense} X{\rm\ s.t.\ }
                f_n(x)\rightarrow f(x),\forall x\in M
            \end{array} \right.
        \end{equation*}
    \end{theorem}

    \begin{definition}
        称$M\subset X$弱列紧是指$M$中任一序列均有弱收敛子列;
        称$\mathcal{F}\subset X^*$弱*列紧是指$\mathcal{F}$中任一序列均有弱*收敛子列。
    \end{definition}

    \begin{theorem}[可分Banach-Alaoglu]
        $X$可分$\Rightarrow X^*$中有界集弱*列紧。
    \end{theorem}
    \begin{proof}
        设$\{f_n\}_{n=1}^\infty\subset X^*$有界,记$C=\fun{sup}{n}||f_n||$,
        $X$可分$\Rightarrow \exists \{x_n\}_{n=1}^\infty\mathop{\subset}\limits^{\rm dense}X$.

        对于$\forall m$,$\{f_n(x_m)\}_{n=1}^\infty$是有界数列,有收敛子列,
        由对角线法,$\{f_n\}_{n=1}^\infty$有子列
        $\{f_{n_k}\}_{k=1}^\infty$使得
        $\{f_{n_k}(x_m)\}_{k=1}^\infty$收敛,Claim:
        \begin{equation*}
            \exists f\in X^*{\rm\ s.t.\ }f_{n_k}\w*to f
        \end{equation*}
        对于$\forall x\in X,\forall \varepsilon>0$,存在$x_m\in \{x_n\}_{n=1}^\infty{\rm\ s.t.\ }||x-x_m||<\frac{\varepsilon}{3C}$,于是
        \begin{equation*}
            |f_{n_{k+p}}(x)-f_{n_k}(x)|
            \leqslant 
            |f_{n_{k+p}}(x)-f_{n_{k+p}}(x_m)|
            +|f_{n_{k+p}}(x_m)-f_{n_{k}}(x_m)|
            +|f_{n_{k}}(x_m)-f_{n_{k}}(x)|<\varepsilon,k\mbox{充分大,}\forall p
        \end{equation*}
        第一项$\leqslant C||x-x_m||<\frac{\varepsilon}{3}$,
        第二项在$k$充分大时$<\frac{\varepsilon}{3}$,
        第三项$\leqslant C||x_m-x||<\frac{\varepsilon}{3}$.

        所以$f(x)\defeq \fun{lim}{k\rightarrow\infty}f_{n_k}(x)$存在,
        且
        \begin{equation*}
            |f(x)|\leqslant \fun{sup}{n}|f_n(x)|\leqslant \fun{sup}{n}||f_n||\cdot ||x||
        \end{equation*}
        所以$f\in X^*$且$f_{n_k}\w*to f$.
    \end{proof}

    \begin{theorem}[Alaoglu]
        $X$是赋范空间,$X^*$中单位闭球是弱*紧的。
    \end{theorem}

    \begin{theorem}[Eberlein-Smulian]
        $X$是自反空间,则
        \begin{enumerate}
            \item $X$中有界集弱列紧;
            \item $X$中闭单位球弱自列紧。
        \end{enumerate}
    \end{theorem}
    \begin{proof}
        对于1,只需证明:$\forall R,\overline{B(0,R)}$弱列紧。
        令$Y\defeq \overline{ {\rm span}\{x_n\}_{n=1}^\infty }$,为闭子空间,
        由定理2.8.11(Pettis)知$Y$自反,同时因为$Y$可分,所以$Y^{**}=Y$可分,
        由定理2.8.6(Banach)知$Y^*$可分,再由定理2.8.16(可分B-A)知
        $Y^{**}$中有界集弱*列紧,所以
        \begin{align*}
            ||x_n^{**}||\leqslant R\Rightarrow& \{x_n^{**}\}_{n=1}^\infty \mbox{有子列}x_{n_k}^{**}\w*to x_0^{**}\in Y^{**}\\
            \Rightarrow& \forall f\in Y^*,f(x_{n_k})=x_{n_k}^{**}(f)\rightarrow x_0^{**}(f)=f(x_0)\\
            \Rightarrow& \forall F\in X^*,F(x_{n_k})
            =(F|_Y)(x_{n_k})\rightarrow
            (F|_Y)(x_0)=F(x_0)\\
            \Rightarrow& x_{n_k}\wto x_0
        \end{align*}

        对于2,
        \begin{equation*}
            x_{n_k}\wto \mathop{\Rightarrow }\limits^{\rm Ex\ 2.5.4}
            ||x_0||\leqslant \fun{liminf}{k\rightarrow\infty}||x_{n_k}||\leqslant R
        \end{equation*}
    \end{proof}






	%第三章
	\chapterimage{empty.jpg}
	\chapter{谱理论}
%\begin{center}
%    线性算子=线性映射
%\end{center}
%\rightline{2023.10.27}
%\vspace{-5pt}
%\begin{center}
%    \pgfornament[width=0.36\linewidth,color=lsp]{88}
%\end{center}

%Lec25

\section{谱}

\subsection{谱的定义与例子}
    \begin{definition}
        $X$是复Banach空间,在$\L(X)$上引入乘法:
        \begin{equation*}
            (AB)x\defeq A(Bx)
        \end{equation*}
        则满足
        \begin{enumerate}
            \item 结合律:$(AB)C=A(BC)$.
            \item 分配律:$(A+B)C=AB+AC$,$A(B+C)=AB+AC$.
            \item $\lambda(AB)=(\lambda A)B=A(\lambda B)$.
            \item $AI=IA=A$.
            \item $||AB||\leqslant ||A||\cdot ||B||$
        \end{enumerate}
        可得$\L(X)$是一个Banach代数。
    \end{definition}

    \begin{definition}
        称$A\in\L(X)$可逆是指:存在$B\in \L(X)$使得
        \begin{equation*}
            AB=BA=I
        \end{equation*}
    \end{definition}

    \begin{definition}
        \begin{equation*}
            \sigma(A)\defeq \{ \lambda\in\C:\lambda I-A\mbox{不可逆} \}
        \end{equation*}
        称为A的谱(spectrum),$\sigma(A)$中的元素称为谱点。
        \begin{equation*}
            \rho(A)\defeq \{ \lambda\in\C:\lambda I-A\mbox{可逆} \}=\C \backslash \sigma(A)
        \end{equation*}
        称为$A$的预解集(resolvent set),$\rho(A)$中元素称为正则值。
    \end{definition}

    \begin{definition}
        如果$\lambda\in C$使得${\rm Ker}(\lambda I-A)\neq \{0\}$,
        即
        \begin{equation*}
            \exists 0\neq x\in X{\rm\ s.t.\ }Ax=\lambda x
        \end{equation*}
        则称$\lambda$为$A$的特征值,
        \begin{equation*}
            \sigma_p(A)\defeq \{ A\mbox{的特征值} \}
        \end{equation*}
        称为$A$的点谱。
    \end{definition}

    \begin{example}
        有限维线性空间上的线性映射$A\in \L(\C^n)\Rightarrow \sigma(A)=\sigma_p(A)\neq \varphi$.
    \end{example}

    \begin{example}
        设
        \begin{equation*}
            A:C[0,1]\rightarrow C[0,1],u(t)\mapsto t\cdot u(t)
        \end{equation*}
        $A$有特征值吗?
    \end{example}
    \begin{solve}
        \begin{align*}
            (\lambda I-A)u=0&\Leftrightarrow \lambda u(t)-tu(t)=0,\forall t\in [0,1]\\
            &\Leftrightarrow u(t)=0,\forall t\in [0,1]
        \end{align*}
        无特征值。
    \end{solve}

    \begin{definition}
        对于$\lambda\in \C$,
        满足${\rm Ker}(\lambda I-A)\neq\{0\}$,则有以下分类:
        \begin{enumerate}
            \item ${\rm Ran}(\lambda I-A)\neq X$,${\rm Ran}(\lambda I-A)\mathop{\subset}\limits^{\rm dense}X$,则称$\lambda $为$A$的连续谱点,其全体记为$\sigma_c(A)$,称为$A$的连续谱。
            \item $\overline{{\rm Ran}(\lambda I-A)}\neq X$,则称$\lambda $为$A$的剩余谱点,其全体$\sigma_r(A)$称为$A$的剩余谱。
            \item ${\rm Ran}(\lambda I-A)=X$,此时$\lambda\in\rho(A)$.
        \end{enumerate}
    \end{definition}

    \begin{example}
        设
        \begin{equation*}
            A:C[0,1]\rightarrow C[0,1],u(t)\mapsto t\cdot u(t)
        \end{equation*}
        则$\sigma(A)=\sigma_r(A)=[0,1]$.
    \end{example}
    \begin{proof}
        先证明:$\C\backslash [0,1]\subset \rho(A)$.
        设$\lambda\in \C\backslash [0,1]$,令
        \begin{equation*}
            T:C[0,1]\rightarrow C[0,1],u(t)\mapsto \frac{1}{\lambda-t}u(t)
        \end{equation*}
        于是$(\lambda I-A)T=I=T(\lambda I-A)$,且
        \begin{equation*}
            ||Tu||_\infty \leqslant \left[ \fun{max}{t\in [0,1]}\frac{1}{|\lambda-t|} \right]||u||_\infty
        \end{equation*}
        所以$(\lambda I-A)^{-1}=T\in\L (X)\Rightarrow \lambda\in\rho(A)$.

        再证明$[0,1]\subset \sigma_r(A)$.设$\lambda\in[0,1]$,对于
        $\forall v\in{\rm Ran}(\lambda I-A)$,存在$u\in C[0,1]$满足
        \begin{equation*}
            (\lambda-t)u(t)=v(t),t\in [0,1]
        \end{equation*}
        于是$v(\lambda)=0\Rightarrow 1\notin \overline{ {\rm Ran}(\lambda I-A) }\Rightarrow \overline{ {\rm Ran}(\lambda I-A) }\neq X$.

        最后,$[0,1]\subset\sigma_r(A)\subset\sigma(A)\subset [0,1]\Rightarrow \sigma(A)=\sigma_r(A)=[0,1]$.
    \end{proof}

    \begin{example}
        设
        \begin{equation*}
            A:L^2[0,1]\rightarrow L^2[0,1],u(t)\mapsto t\cdot u(t)
        \end{equation*}
        则$\sigma(A)=\sigma_c(A)=[0,1]$.
    \end{example}
    \begin{proof}
        与上例同理可证:$\C\backslash [0,1]\subset \rho(A)$.
        
        再证明:对于$\forall \lambda\in [0,1]$,${\rm Ran}(\lambda I-A)\neq X$.
        {\rm Claim}:$1\notin (\lambda I-A)$,否则存在$u\in L^2{\rm\ s.t.\ }1=(\lambda-t)u(t),t\in [0,1]$,
        从而
        \begin{equation*}
            \frac{1}{\lambda-t}\in u(t)\in L^2[0,1]
        \end{equation*}
        矛盾。

        最后证明:$\forall \lambda\in [0,1],{\rm Ran}(\lambda I-A)\mathop{\subset}\limits^{\rm dense}X$,
        对于$\forall v\in L^2,\forall \varepsilon>0$,定义
        \begin{equation*}
            u_\varepsilon (t)\defeq \frac{1}{\lambda-t}v(t)\cdot 
            \chi_{ [0,1]\backslash (\lambda-\varepsilon,\lambda+\varepsilon) }(t)
        \end{equation*}
        于是$u_\varepsilon\in L^2$且
        \begin{equation*}
            (\lambda I-A)u_\varepsilon=\chi_{ [0,1]\backslash(\lambda-\varepsilon,\lambda+\varepsilon) }v
            \mathop{\rightarrow}\limits^{L^2} v{\rm\ as\ }\varepsilon\rightarrow0^+
        \end{equation*}
        由积分的绝对连续性,$v\in \overline{ {\rm Ran}(\lambda I-A) }$.
    \end{proof}

\subsection{谱的基本性质}
    \begin{definition}
        算子值函数:
        \begin{equation*}
            R_\lambda(A):\rho(A)\rightarrow\L(X),\lambda\mapsto (\lambda I-A)^{-1}
        \end{equation*}
        称为$A$的预解式(resolvent).
    \end{definition}

    \begin{lemma}
        设$T\in \L(X)$,$||T||\leqslant 1$,则
        \begin{enumerate}
            \item $(I-T)^{-1}\in\L(X)$.
            \item $(I-T)^{-1}=\sum_{n=1}^\infty T^n$.(Von Neumann级数)
            \item $||(I-T)^{-1}||\leqslant (1-||T||)^{-1}$.
        \end{enumerate}
    \end{lemma}
    \begin{proof}
        \begin{enumerate}
            \item 令
                \begin{equation*}
                    S_n\defeq \sum_{k=1}^n T^k
                \end{equation*}
                则
                \begin{equation*}
                    ||S_{n+p}-S_n||=\ms{ \sum_{k=n+1}^{n+p}T^k }
                    \leqslant \sum_{k=n+1}^{n+p}||T||^k<\frac{||T||^{k+1}}{1-||T||}
                \end{equation*}
                $\L(X)$完备$\Rightarrow \exists S\in\L(X)$使得$||S_n-S||\rightarrow 0$.
                Claim:
                \begin{equation*}
                    S(I-T)=(I-T)S=I
                \end{equation*}
                注意到
                \begin{equation*}
                    ||S_n(I-T)-I||=
                    ||I-T^{n+1}-I||\leqslant ||T||^{n+1}\rightarrow 0{\rm\ as\ }n\rightarrow\infty
                \end{equation*}
                \begin{align*}
                    \Rightarrow ||S(I-T)-I||
                    \leqslant& ||S(I-T)-S_n(I-T)||+||S_n(I-T)-I||\\
                    \leqslant& ||S-S_n||\cdot ||I-T||+||S_n(I-T)-I||\rightarrow 0
                \end{align*}
                从而$S(I-T)=I$,同理$(I-T)S=I$.
            \item $||S_n-S||\rightarrow 0\Rightarrow (I-T)^{-1}=\sum_{k=0}^\infty T^n$.
            \item $||S||\leqslant \fun{sup}{n}||S_n||$.
            \end{enumerate}
    \end{proof}
    
    \begin{theorem}
        $\rho(A)$是开集,进而$\sigma(A)$是闭集。
    \end{theorem}
    \begin{proof}
        设$\lambda_0\in \rho(A)$,
        \begin{equation*}
            \lambda I-A=\lambda_0 I-A+(\lambda-\lambda_0)I
            =(\lambda_0 I-A)[ I+(\lambda-\lambda_0)(\lambda_0I-A)^{-1} ]
        \end{equation*}
        由引理3.1.1,当$|\lambda-\lambda_0|<||(\lambda_0I-A)^{-1}||^{-1}$时,
        \begin{equation*}
            B\defeq [ I+(\lambda-\lambda_0)(\lambda_0I-A)^{-1} ]^{-1}\in\L(X)
        \end{equation*}
        于是
        \begin{equation*}
            (\lambda I-A)^{-1}
            =B\cdot R_{\lambda_0}(A)\in\L(X)\Rightarrow \lambda\in\rho(A)
            \Rightarrow \mathbb{D}(\lambda_0,\frac{1}{||(\lambda_0I-A)^{-1}||})\subset \rho(A)
        \end{equation*}
    \end{proof}

    \begin{proposition}
        $A\in \L(X)\Rightarrow \sigma(A)\subset \overline{ \mathbb{D}(0,||A||) }$.
    \end{proposition}
    \begin{proof}
        即证明:$\forall \lambda\in\C{\rm\ with\ }|\lambda|>||A||,(\lambda I-A)^{-1}\in\L(X)$,
        \begin{align*}
            |\lambda|>||A||\Rightarrow& \ms{\frac{A}{\lambda}}<1\\
            \mathop{\Rightarrow}\limits^{\rm Lem3.1.1}&\left(I-\frac{A}{\lambda}\right)^{-1}\in\L(X)\\
            \Leftrightarrow& (\lambda I-A)^{-1}\in\L(X)
        \end{align*}
    \end{proof}

    \begin{corollary}
        $\sigma(A)$是$\C$中紧集。
    \end{corollary}

    \begin{definition}
        $X$是复Banach空间,$\Omega$是$\C$上的开集,称算子值函数
        \begin{equation*}
            T:\Omega\rightarrow\L(X),\lambda\mapsto T_\lambda
        \end{equation*}
        在$\lambda_0\in\Omega$全纯是指:
        存在$\lambda_0$的邻域$U$使得
        \begin{equation*}
            \forall \lambda\in U,\exists S_\lambda\in\L(X){\rm\ s.t.\ }
            \ms{ \frac{T_{\lambda+\delta}-T_\lambda}{\delta}-S_\lambda }\rightarrow 0{\rm\ as\ }\delta\rightarrow 0
        \end{equation*}
    \end{definition}
%Lec26
    \begin{theorem}
        $\lambda\mapsto R_\lambda(A)$是$\rho(A)$上的算子值全纯函数。
    \end{theorem}
    \begin{proof}
        \begin{lemma}[Resolvent Identity]
            \begin{equation*}
                R_\lambda(A)-R_\mu(A)=(\mu-\lambda)R_\lambda(A)R_\mu(A),\forall \lambda,\mu\in\rho(A)
            \end{equation*}
            \begin{proof}
                \begin{align*}
                    R_\lambda(A)&=(\lambda I-A)^{-1}(\mu I-A)(\mu I-A)^{-1}\\
                    &=(\lambda I-A)^{-1} [ \lambda I-A+(\mu-\lambda)I ](\mu I-A)^{-1}\\
                    &=R_\mu(A)+(\mu-\lambda)R_\lambda(A)R_\mu(A)
                \end{align*}        
            \end{proof}
        \end{lemma}

        \textbf{Step1}:连续性。
            对于$\forall \lambda_0\in\rho(A)$,
            \begin{equation*}
                \lambda I-A=(\lambda_0I-A)[ I+(\lambda-\lambda_0)(\lambda_0I-A)^{-1} ]
            \end{equation*}
            当$|\lambda-\lambda_0|<||(\lambda_0I-A)^{-1}||^{-1}$时,
            \begin{equation*}
                R_\lambda(A)=[ I+(\lambda-\lambda_0)R_{\lambda_0}(A) ]^{-1}R_{\lambda_0}(A)
            \end{equation*}
            于是当$|\lambda-\lambda_0|<(2||R_{\lambda_0}(A)||)^{-1}$时,
            \begin{equation*}
                ||R_{\lambda}(A)||\leqslant 
                ||[I+(\lambda-\lambda_0)R_{\lambda_0}(A)]^{-1}||\cdot ||R_{\lambda_0}(A)||
                \leqslant \frac{1}{1-\frac{1}{2}}||R_{\lambda_0}(A)||
                =2||R_{\lambda_0}(A)||
            \end{equation*}
            由引理3.1.2可知,
            \begin{equation*}
                ||R_{\lambda}(A)-R_{\lambda_0}(A)||\leqslant |\lambda-\lambda_0|\cdot
                ||R_{\lambda}(A)||\cdot ||R_{\lambda_0}(A)||
                \leqslant 2||R_{\lambda_0}(A)||^2\cdot |\lambda-\lambda_0|
            \end{equation*}
        \textbf{Step2}:全纯性。
        \begin{align*}
            \ms{ \frac{R_{\lambda}(A)-R_{\lambda_0}(A)}{\lambda-\lambda_0}+R_{\lambda_0}(A)^2}
            \mathop{=}\limits^{\rm R.I.}&
            \ms{ -R_{\lambda}(A)R_{\lambda_0}(A)+R_{\lambda_0}(A)^2 }\\
            \leqslant&||R_{\lambda_0}(A)||\cdot ||R_{\lambda}(A)-R_{\lambda_0}(A)||\rightarrow 0{\rm\ as\ }\lambda\rightarrow\lambda_0
        \end{align*}
    \end{proof}

    \begin{theorem}[Gelfand,谱不空定理]
        $0\neq A\in\L(X)\Rightarrow \sigma(A)\neq \varnothing$.
    \end{theorem}
    \begin{proof}
        假设$\sigma(A)=\varnothing$,则
        $\rho(A)=\C$,说明$\lambda\mapsto R_{\lambda}(A)$时算子值整函数,于是
        \begin{equation*}
            \forall f\in\L(X)^*,
            u_f(\lambda)\defeq f(R_{\lambda}(A)),\lambda\in\C
        \end{equation*}
        是整函数,因为
        \begin{equation*}
            \left| \frac{u_f(\lambda)-u_f(\lambda_0)}{\lambda-\lambda_0}+f(R_{\lambda_0}(A)^2) \right|
            \leqslant ||f||\cdot \ms{\frac{R_{\lambda}(A)-R_{\lambda_0}(A)}{\lambda-\lambda_0}+R_{\lambda_0}(A)^2}
            \rightarrow 0{\rm\ as\ }\lambda\rightarrow\lambda_0
        \end{equation*}
        另一方面,当$|\lambda|>2||A||$时,
        \begin{equation*}
            ||R_{\lambda}(A)||\leqslant \frac{1}{|\lambda|}\frac{1}{1-\ms{\frac{A}{\lambda}}}
            =\frac{1}{|\lambda|-||A||}\leqslant \frac{1}{||A||}
        \end{equation*}
        而$\lambda\mapsto R_\lambda(A)$连续,在$\overline{ \mathbb{D}(0,2||A||) }$上有界,
        于是存在$C>0$使得$||R_{\lambda}(A)||\leqslant C,\forall \lambda\in \C$,
        \begin{equation*}
            |u_f(\lambda)|\leqslant ||f||\cdot ||R_{\lambda}(A)||\leqslant C||f||,\forall \lambda\in\C
        \end{equation*}
        由Liouvillle定理,$u_f$是常函数。从而
        \begin{equation*}
            f(R_{\lambda}(A))=f(R_{\mu}(A)),\forall \lambda,\mu\in\C,\forall f\in\L(X)^*
        \end{equation*}
        由HBT,$R_{\lambda}(A)=R_{\mu}(A)$,这与R.I.矛盾。
    \end{proof}

    \begin{definition}
        对于$A\in\L(X)$,
        \begin{equation*}
            r_\sigma(A)\defeq {\rm sup}\{ |\lambda|:\lambda\in\sigma(A) \}
        \end{equation*}
        称为$A$的谱半径。
    \end{definition}
    \begin{theorem}[Gelfand,谱半径公式]
        \begin{equation*}
            r_\sigma(A)=\fun{lim}{n\rightarrow\infty}||A^n||^{\frac{1}{n}}
        \end{equation*}
    \end{theorem}
    \begin{proof}
        \textbf{Step1}:先证明右式极限存在,令$r\defeq \fun{inf}{n}||A^n||^{\frac{1}{n}}$,
        则
        \begin{equation*}
            \fun{liminf}{n\rightarrow\infty}||A^n||^{\frac{1}{n}}\geqslant r
        \end{equation*}
        另一方面,$\forall \varepsilon>0$,存在$m$使得
        \begin{equation*}
            ||A^m||^{\frac{1}{m}}<r+\varepsilon
        \end{equation*}
        所以对于$\forall n\in \N$,有唯一分解$n=p_nm+q_n{\rm\ with\ }0\leqslant q_n<m$,
        所以
        \begin{equation*}
            ||A^n||^{\frac{1}{n}}\leqslant ||A^{p_nm}||^{\frac{1}{n}}
            ||A^{q_n}||^{\frac{1}{n}}
            \leqslant ||A^m||^{\frac{q_n}{n}}||A||^{\frac{q_n}{n}}
            <(r+\varepsilon)^{\frac{p_nm}{n}}||A||^{\frac{q_n}{n}}
        \end{equation*}
        当$n\rightarrow\infty$时,$\frac{q_n}{n}\rightarrow 0,\frac{p_nm}{n}\rightarrow 1$,所以
        \begin{equation*}
            \fun{limsup}{n\rightarrow\infty}||A^n||^{\frac{1}{n}}\leqslant r+\varepsilon
        \end{equation*}
        令$\varepsilon\rightarrow0 $则
        \begin{equation*}
            \fun{limsup}{n\rightarrow\infty}||A^n||^{\frac{1}{n}}\leqslant r
        \end{equation*}

        \textbf{Step2}:证明$r_\sigma(A)\leqslant \fun{lim}{n\rightarrow\infty}||A^n||^{\frac{1}{n}}$.我们知道
        幂级数
        \begin{equation*}
            \sum_{n=0}^\infty ||A^n||z^n
        \end{equation*}
        的收敛半径为
        \begin{equation*}
            r=\frac{1}{\fun{lim}{n\rightarrow\infty}||A^n||^{\frac{1}{n}}}
        \end{equation*}
        令$z=\frac{1}{\lambda}$,可知当$|\lambda|>\fun{lim}{n\rightarrow\infty}||A^n||^{\frac{1}{n}}$
        时(收敛圆内绝对收敛)
        \begin{equation*}
            \sum_{n=0}^\infty \ms{\frac{A^n}{\lambda^{n+1}}}<\infty
        \end{equation*}
        $\L(X)$完备,根据引理1.4.2,级数
        \begin{equation*}
            \sum_{n=0}^\infty \ms{\frac{A^n}{\lambda^{n+1}}}
        \end{equation*}
        也收敛。另一方面,
        \begin{equation*}
            \ms{ \left( \sum_{n=1}^N \frac{A^n}{\lambda^{n+1}} \right)(\lambda I-A)-I }
            =\ms{ I-\frac{A^{N+1}}{\lambda^{N+1}-I} }\rightarrow 0
        \end{equation*}
        所以
        \begin{equation*}
            \sum_{n=1}^\infty \frac{A^n}{\lambda^{n+1}}=(\lambda I-A)^{-1}=R_{\lambda}(A)
        \end{equation*}
        从而$R_\lambda(A)\in\L(X)\Rightarrow \lambda\in\rho(A)\rightarrow r_\sigma(A)\leqslant \fun{lim}{n\rightarrow\infty}||A^n||^{\frac{1}{n}}$.

        \textbf{Step3}:证明$r_\sigma(A)\geqslant \fun{lim}{n\rightarrow\infty}||A^n||^{\frac{1}{n}}$.
        设$|\lambda|>r_\sigma(A)$,则$\lambda\in\rho(A)\Rightarrow \forall f\in\L(X)^*,f(R_\lambda(A))$在$\lambda$全纯,
        从而$f(R_\lambda(A))$在圆环$|\lambda|>r_\sigma(A)$内全纯,故可展为收敛的Laurent级数。
        另一方面,由Step2,当$|\lambda|>\fun{lim}{n\rightarrow\infty}||A^n||^{\frac{1}{n}}$时,
        \begin{equation*}
            R_\lambda(A)=\sum_{n=0}^\infty \frac{A^n}{\lambda^{n+1}}
            \Rightarrow f(R_\lambda(A))=\sum_{n=0}^\infty \frac{f(A^n)}{\lambda^{n+1}}
        \end{equation*}
        Laurent展式唯一,所以这一展式在$|\lambda|>r_\sigma(A)$上也成立。
        在内部绝对收敛,所以
        \begin{equation*}
            \forall \varepsilon>0,\sum_{n=1}^\infty \frac{ |f(A^n)| }{(r_\sigma(A)+\varepsilon)^{n+1}}<\infty
        \end{equation*}        
        记
        \begin{equation*}
            T_n\defeq \frac{A^n}{(r_\sigma(A)+\varepsilon)^{n+1}}
        \end{equation*}
        收敛级数通项有界,所以
        \begin{equation*}
            \fun{sup}{n}|f(T_n)|<\infty,\forall f\in\L(X)^*
        \end{equation*}
        由UBP,$C\defeq \fun{sup}{n}||T_n||<\infty$,从而
        \begin{equation*}
            ||A^n||\leqslant C( r_\sigma(A)+\varepsilon )^{n+1}
            \Rightarrow 
            \fun{lim}{n\rightarrow\infty}||A^n||^{\frac{1}{n}}
            \leqslant r_\sigma(A)+\varepsilon
        \end{equation*}
        令$\varepsilon\rightarrow0$得证。
    \end{proof}

%Lec27
    $\sigma(A)=\sigma_p(A)\cup\sigma_c(A)\cup\sigma_r(A)$.

    \begin{example}
        右移位算子:
        \begin{equation*}
            A:l^2\rightarrow l^2,(x_1,x_2,\cdots)\mapsto (0,x_1,x_2,\cdots)
        \end{equation*}
        则$\sigma_p(A)=\varnothing$,$\sigma_c(A)=\partial \mathbb{D}$,
        $\sigma_r(A)=\mathbb{D}$.
    \end{example}
    \begin{proof}
        $||A||=1\Rightarrow \sigma(A)=\overline{\mathbb{D}}$,
        先证明:$\sigma_p(A)=\varnothing$,否则$\exists\lambda\in\C,\exists 0\neq x\in \ell^2$使得
        \begin{equation*}
            (0,x_1,x_2,\cdots)=Ax=\lambda x=(\lambda x_1,\lambda x_2,\cdots)
        \end{equation*}
        $\lambda=0\Rightarrow x=0$,$\lambda \neq 0\Rightarrow x_1=0\Rightarrow x_2=0\Rightarrow \cdots\Rightarrow x=0$.矛盾。

        再证明:$\mathbb{D}\subset \sigma_r(A)$,
        设$\lambda\in\mathbb{D}$,Claim:
        $\overline{ {\rm Ran}(\lambda I-A) }\neq \ell^2$.
        这等价于${\rm Ran}(\lambda I-A)^\perp\neq \{0\}$.令
        $z=(1,\overline{\lambda},\overline{\lambda}^2,\cdots)$,则
        \begin{align*}
            \ag{(\lambda I-A)x,z}&=\ag{ (\lambda x_1,\lambda x_2-x_2,\lambda x_3-x_2,\cdots),(1,\overline{\lambda},\overline{\lambda}^2,\cdots) }\\
            &=\lambda x_1+\lambda^2 x_2-\lambda x_1+\lambda^3 x_3-\lambda^2 x_2+\cdots=0\\
            \Rightarrow 0\neq z&\in {\rm Ran}(\lambda I-A)^\perp
        \end{align*}

        然后证明:$\partial \mathbb{D}\subset\sigma_c(A)$.
        设$\lambda\in\partial\mathbb{D}$,
        \begin{enumerate}[$1^\circ$]
            \item 证明${\rm Ran}(\lambda I-A)\neq \ell^2$.
                \begin{align*}
                    {\rm Ran}(\lambda I-A)\ni y=(\lambda I-A)x
                    \Rightarrow& y_1=\lambda x_1,y_k=\lambda x_k-x_{k-1},k\geqslant 2\\
                    \Rightarrow& y_1=\lambda x_1,\lambda^{k-1}y_k=\lambda^k x_k-
                    \lambda^{k-1}x_{k-1},k\geqslant 2\\
                    \Rightarrow& \sum_{k=1}^n \lambda^{k-1}y_j=\lambda^n x_n
                \end{align*}
                假设${\rm Ran}(\lambda I-A)=\ell^2$,令$y=e_1$,
                \begin{align*}
                    \exists x\in\ell^2{\rm\ s.t.\ }e_1=(\lambda I-A)x
                    \Rightarrow& \lambda^n x_n=1,n=1,2,\cdots\\
                    \Rightarrow& x=( \frac{1}{\lambda},\frac{1}{\lambda^2},\cdots )
                \end{align*}
                $|\lambda|=1$,不收敛,这与$x\in \ell^2$矛盾。
            \item 证明$\overline{ {\rm Ran}(\lambda I-A) }=\ell^2$.只需证明
                ${\rm Ran}(\lambda I-A)^\perp=\{0\}$.对于$\forall x\in {\rm Ran}(\lambda I-A)^\perp$,
                \begin{equation*}
                    0=\ag{z,(\lambda I-A)e_n}=\overline{\lambda}z_n-z_{n+1},\forall n
                \end{equation*}
                所以$z_{n+1}=\overline{\lambda}z_n\Rightarrow |z_{n+1}|=|z_n|\Rightarrow z=0$.
        \end{enumerate}

        最后:$\overline{\mathbb{D}}\subset \sigma_c(A)\cup \sigma_r(A)\subset \sigma(A)\subset \overline{\mathbb{D}}$,
        由前面结论可得$\sigma_c(A)=\partial \mathbb{D}$,$\sigma_r(A)=\mathbb{D}$.
    \end{proof}


\section{紧算子的谱}
    
\subsection{紧算子}
\begin{definition}
    $X,Y$是Banach空间,$A\in \L(X,Y)$,
    \begin{enumerate}
        \item 如果$A$把每个有界集映为列紧集,称$A$紧,记作$A\in \mathcal{T}(X,Y)$.
        \item 如果$A$把$X$中每个弱收敛序列映为$Y$中强收敛序列,称$A$全连续。
        \item 如果${\rm dim}( {\rm Ran}(A) )<\infty$,则称$A$是有限秩算子,记作$A\in\mathcal{F}(X,Y)$.
    \end{enumerate}
\end{definition}

\begin{proposition}
    $\mathcal{F}(X,Y)\subset \mathcal{T}(X,Y)$,且是闭子空间。
\end{proposition}
\begin{proof}
    有限维线性空间里的有界集列紧。
\end{proof}

\begin{example}
    $I\in \mathcal{T}(X)\Leftrightarrow {\rm dim}(X)<\infty$.
\end{example}

\begin{example}
    设$K(\cdot,\cdot)$在$[a,b]^2$上连续,
    \begin{equation*}
        (Tu)(s)\defeq \int_a^b K(s,t)u(t)\d t
    \end{equation*}
    则$T:C[a,b]\rightarrow C[a,b]$紧。
\end{example}
\begin{proof}
    设$\mathcal{F}\subset C[a,b]$有界,
    记$M=\fun{sup}{u\in\mathcal{F}}||u||$,则
    \begin{equation*}
        ||Tu||\leqslant ||T||\cdot M,\forall u\in \mathcal{F}
    \end{equation*}
    所以$T(\mathcal{F})$一致有界。

    同时,$\forall \varepsilon>0,\forall u\in\mathcal{F}$,
    $K(\cdot,\cdot)$一致连续$\Rightarrow \exists \delta>0{\rm\ s.t.}$
    \begin{equation*}
        |K(s',t)-K(s'',t)|<\frac{\varepsilon}{M(b-a)},\forall s',s''\in [a,b]{\rm\ with\ }|s'-s''|<\delta,\forall t\in [a,b]
    \end{equation*}
    所以
    \begin{equation*}
        |(Tu)(s')-(Tu)(s'')|\leqslant \int_a^b 
        | K(s',t)-K(s'',t) ||u(t)|\d t
        <\varepsilon,\forall s',s''{\rm\ with\ }|s'-s''|<\delta,\forall u\in\mathcal{F}
    \end{equation*}
    所以$T(\mathcal{F})$等度连续,进而列紧。
\end{proof}

\begin{proposition}
    $\mathcal{T}(X,Y)\subset \mathcal{L}(X,Y)$,且是闭子空间。
\end{proposition}
\begin{proof}
    设$A_n\in\mathcal{T}(X,Y)$使得$||A_n-A||\rightarrow 0$,下证$A$紧。
    设$M\subset X$有界,$C\defeq \fun{sup}{n}||x||<\infty$,Claim:$A(M)$列紧。
    对于$\forall \varepsilon$,取$N$充分大使得
    \begin{equation*}
        ||A_N-A||<\frac{\varepsilon}{3C}
    \end{equation*}
    $A_N(M)$列紧,
    \begin{equation*}
        \exists x_1,\cdots,x_m\in M{\rm\ s.t.\ }
        A_N(M)\subset \bigcup_{k=1}^m B(A_N x_k,\frac{\varepsilon}{3})
    \end{equation*}
    所以$\forall x\in M,\exists k\in \{1,2,\cdots,m\}{\rm\ s.t.\ }$
    \begin{equation*}
        ||A_Nx-A_Nx_{k}||<\frac{\varepsilon}{3}
    \end{equation*}
    从而
    \begin{equation*}
        ||Ax-Ax_k||\leqslant 
        ||Ax-A_Nx||+||A_Nx-A_Nx_k||+||A_Nx_k-Ax_k||<\varepsilon
    \end{equation*}
    于是$\{ Ax_1,\cdots,Ax_m \}$是$A(M)$的有穷$\varepsilon$网。
\end{proof}

\begin{proposition}
    紧算子的值域可分。
\end{proposition}
\begin{proof}
    \begin{equation*}
        {\rm Ran}(A)=\bigcup_{k=1}^\infty A(B(0,n))
    \end{equation*}
    列紧$\Rightarrow $可分,设$M_n$是$A(B(0,n))$的可数稠密子集,取$M_n$的并即为
    ${\rm Ran}(A)$的可数稠密子集。
\end{proof}

\begin{proposition}
    紧算子与有界算子的(两种)复合是紧算子。
\end{proposition}
\begin{proof}
    设$T$有界,$A$紧,
    若$\{x_n\}_{n=1}^\infty\subset X$有界,$A$紧所以
    $\{ Ax_n \}_{n=1}^\infty$列紧,有子列$\{Ax_{n_k}\}_{k=1}^\infty$收敛,
    $T$有界所以
    $\{ TAx_{n_k} \}_{k=1}^\infty$收敛,因此$TA$是紧算子。

    若$M$有界,则$T(M)$有界,$A$紧所以$AT(M)$列紧,$AT$是紧算子。
\end{proof}

\begin{theorem}
    对于$A\in\L(X,Y)$,
    \begin{enumerate}
        \item 紧$\Rightarrow$全连续;
        \item 如果$X$自反,则$A$紧$\Leftrightarrow A$全连续。
    \end{enumerate}
\end{theorem}
\begin{proof}
    对于1,假设$A$紧而不全连续,即存在$x_n\wto x_0$但$||Ax_n-Ax_0||\nrightarrow 0$:
    存在$\varepsilon_0>0$,存在子列$\{x_{n_k}\}_{k=1}^\infty$使得$||Ax_{n_k}-Ax_0||\geqslant \varepsilon_0$.

    那么$x_{n_k}\wto x_0$,由UBP知$\{ x_{n_k} \}_{k=1}^\infty$有界,
    $A$紧所以$\{ Ax_{n_k} \}_{k=1}^\infty$有收敛子列,
    不妨设$Ax_{n_k}\rightarrow y$.

    另一方面,$\forall f\in Y^*$,
    \begin{equation*}
        f(Ax_{n_k}-Ax_0)=(A^*f)(x_{n_k}-x_0)\rightarrow 0
    \end{equation*}
    则$Ax_{n_k}\wto Ax_0\Rightarrow Ax_0=y\Rightarrow ||Ax_{n_k}-Ax_0||\rightarrow 0$,矛盾。

    对于2,设$\{x_n\}$有界,由定理2.8.18(Eberlein‑Smulian),
    $X$自反$\Rightarrow $有子列$x_{n_k}\wto x_0$,
    $A$全连续$\Rightarrow ||Ax_{n_k}-Ax_0||\rightarrow 0$.
\end{proof}
%Lec28

\subsection{Riesz-Fredholm定理}

\begin{definition}
    对于$\mathcal{F}\subset X^*$,
    \begin{equation*}
        \mathcal{F}^\perp\defeq \{ x\in X:f(x)=0,\forall f\in \mathcal{F} \}
    \end{equation*}
    称之为$\mathcal{F}$在$X$中的零化子。
\end{definition}

\begin{theorem}[Riesz-Fredholm]
    设$A\in\mathcal{T}(X)$,$T=I-A$,则
    \begin{enumerate}
        \item ${\rm dim}({\rm Ker}(T))<\infty$.
        \item ${\rm Ran}(T)$闭。
        \item Fredholm Alternative,二择一律:$T$单$\Leftrightarrow T$满。
        \item ${\rm Ran}(T)={\rm Ker}(T^*)$.
        \item ${\rm dim}({\rm Ker}(T))={\rm dim}({\rm Ker}(T^*))$.
    \end{enumerate}
\end{theorem}
\begin{proof}
    1.记$M={\rm Ker}(T)$,$S_M$为$M$中的单位球面,
            $S_X$为$X$中的单位球面,于是
            \begin{equation*}
                x\in S_M\Leftrightarrow x\in S_X,(I-A)x\in 0
                \Leftrightarrow x\in S_X,x=Ax\in A(S_X)
            \end{equation*}
            所以$S_M\subset A(S_X)$,后者列紧,
            从而$S_M$列紧,因此${\rm dim}(M)<\infty$.
\end{proof}
\begin{proof}
    2.设${\rm Ran}(T)\ni y_n\rightarrow y$,
    其中$y_n=Tx_n=x_n-Ax_n$.

    \textbf{Case 1}:$\{x_n\}_{n=1}^\infty$有界,
    $A$紧$\Rightarrow \{Ax_n\}_{n=1}^\infty$有收敛子列
    $\{Ax_{n_k}\}_{k=1}^\infty$,设$Ax_{n_k}\rightarrow u$,
    \begin{align*}
        x_{n_k}=y_{n_k}+Ax_{n_k}\rightarrow y+u\Rightarrow& 
        y_{n_k}=Tx_{n_k}\rightarrow T(y+u)\\
        \Rightarrow&y=T(y+u)\in {\rm Ran}(T)
    \end{align*}

    \textbf{Case 2}:$\{x_n\}_{n=1}^\infty$无界,
    令$d_n\defeq {\rm dist}(x_n,{\rm Ker}(T))$,
    存在$z_n\in {\rm Ker}(T)$满足$||x_n-z_n||=d_n$,Claim:
    $\{x_n-z_n\}_{n=1}^\infty$有界。假设不然,
    不妨$d_n\rightarrow +\infty$,令
    \begin{equation*}
        v_n\defeq \frac{x_n-z_n}{||x_n-z_n||}
    \end{equation*}
    于是
    \begin{equation*}
        Tv_n=\frac{Tx_n-Tz_n}{d_n}=\frac{y_n}{d_n}\rightarrow 0
    \end{equation*}
    由于$||v_n||=1$,所以$\{Av_n\}_{n=1}^\infty$有收敛子列,
    设$Ax_{n_k}\rightarrow w$,则$v_{n_k}=Av_{n_k}+Tv_{n_k}\rightarrow w
    $,而$Tv_{n_k}\rightarrow 0$,所以$Tw=0$,$w\in {\rm Ker}(T)$,
    \begin{equation*}
        ||v_n-z||=\frac{1}{d_n}||x_n-(z_n+d_n z)||\geqslant \frac{d_n}{d_n}=1
    \end{equation*}
    这与$v_{n_k}\rightarrow w\in {\rm Ker}(T)$矛盾。
    从而$\{x_n-z_n\}$有界且$T(x_n-z_n)=Tx_n=y_n$,约化为Case1.
\end{proof}
\begin{proof}
    3.
    \begin{lemma}
        \begin{enumerate}[(1)]
            \item ${\rm Ker}(T)\subset {\rm Ker}(T^2)\subset \cdots$
            \item $\exists n{\rm\ s.t.\ }{\rm Ker}(T^n)={\rm Ker}(T^{n+1})$.
        \end{enumerate}
        \begin{proof}
            (1)显然,只说明(2):
            假设不成立,即$\forall n,{\rm Ker}(T^{n})\subsetneqq {\rm Ker}(T^{n+1})$,
            由Riesz阴历,存在$x_n\in {\rm Ker}(T^{n+1}),||x_n||=1{\rm\ s.t.\ }$
            ${\rm dist}(x_n,{\rm Ker}(T^n))>\frac{1}{2}$.

            对于$\forall n,m$,不妨设$n>m$,
            \begin{equation*}
                T^n(Tx_n+Ax_m)=T^{n+1}x_n+T^nAx_m=A(T^nx_m)=0
            \end{equation*}
            所以$Tx_n+Ax_m\in Tx_n+Ax_m(T^n)$,从而
            \begin{equation*}
                ||Ax_n-Ax_m||=||x_n-(Tx_n+Ax_m)||>\frac{1}{2}
            \end{equation*}
            这说明$\{Ax_n\}$无收敛子列,与$A$紧矛盾。
        \end{proof}
    \end{lemma}

    假设$T$满但不单,也就是${\rm Ker}(T)\neq \{0\}$,
    取$0\neq x_0\in {\rm Ker}(T)$,因为$T$是满射,存在$Tx_1=x_0$、
    $Tx_2=x_1\cdots$
    \begin{equation*}
        0\neq x_0=Tx_1=T^2x_2=\cdots
        \Rightarrow T^n x_n\neq 0,T^{n+1}x_n=0
    \end{equation*}
    从而$x_n\in {\rm Ker}(T^{n+1})\backslash {\rm Ker}(T^n)$.这与引理矛盾。

    假设$T$单而不满,令$X_1=T(X)={\rm Ran}(T)$,
    $X_1$是$X$的闭真子空间,取$X_2=T(X_1)$,则$X_2$为$X_1$的
    闭真子空间,
    否则$T(X_1)=X_1$,取$x_0\in X\backslash X_1$,
    \begin{equation*}
        Tx_0\in T(X)=X_1=T(X_1)\Rightarrow Tx_0'=Tx_0
    \end{equation*}
    与$T$单射矛盾,因此可以取出一系列$X_n=T^n(X)$满足
    $X_{n+1}$是$X_n$的真闭子空间,由Riesz,
    \begin{equation*}
        \exists x_n\in X_n,||x_n||=1{\rm\ s.t.\ }
        {\rm dist}(x_n,X_{n+1})>\frac{1}{2}
    \end{equation*}
    那么对于$\forall n,m$,不妨$n>m$,
    \begin{equation*}
        Ax_m-Ax_n=-(x_m-Ax_m)+(x_n-Ax_n)+x_m-x_n
        =x_m-(x_n+Tx_n-Tx_n)\in X_{m+1}
    \end{equation*}
    于是
    \begin{equation*}
        ||Ax_m-Ax_n||\geqslant {\rm dist}(x_m,X_{m+1})>\frac{1}{2}
    \end{equation*}
    从而$\{Ax_n\}$无收敛子列,与$A$紧矛盾。
\end{proof}
\subsection{Riesz-Schauder定理}
\begin{theorem}[Riesz-Schauder]
    设$A\in\mathcal{T}(X)$,则
    \begin{enumerate}
        \item 如果${\rm dim}(X)=\infty$,则$0\in\sigma(A)$.
        \item $\sigma(A)\backslash \{0\}=\sigma_p(A)\backslash \{0\}$.
        \item 非零特征值的特征子空间一定是有限维的。
        \item 不同特征值的特征向量线性无关。
        \item $0$是$\sigma(A)$的唯一可能的极限点。
    \end{enumerate}
\end{theorem}
\begin{proof}
    1.假设$0\in\rho(A)$,则$A^{-1}\in\L(X)$,$I=A^{-1}\circ A$是有界算子和紧算子的复合,也是紧算子,从而${\rm dim}(X)<\infty$.
\end{proof}
\begin{proof}
    2.只需证明:
    \begin{equation*}
        \forall \lambda\notin \sigma_p(A),\lambda\neq 0\Rightarrow \lambda\in\rho(A)
    \end{equation*}
    实际上,$\lambda\notin \sigma_p(A)\Rightarrow \lambda I-A$单
    ,由F.A.$\Rightarrow \lambda I-A$是双射,
    由IMT$\Rightarrow (\lambda I-A)^{-1}\in\L(X)$.
\end{proof}
\begin{proof}
    3.对于$\forall 0\neq \lambda \in\sigma_p(A)$,
    \begin{equation*}
        {\rm Ker}(\lambda I-A)={\rm Ker}(I-\frac{1}{\lambda}A)
        \mathop{\Rightarrow }\limits^{\rm Riesz-Fredholm} {\rm dim}(\lambda I-A)<\infty
    \end{equation*}
\end{proof}
\begin{proof}
    5.假设$\sigma(A)$有极限点$\lambda_0\neq 0$,则
    存在$\lambda_n\in\sigma(A)$使得$\lambda_n\rightarrow\lambda_0$.
    不妨设$\{\lambda_n\}$互不相同,$\lambda_0\Rightarrow n$充分大时
    $\lambda_n\neq 0$,故不妨设所有$\lambda_n\neq 0$.
    \begin{equation*}
        \frac{1}{\lambda_n}\rightarrow \frac{1}{\lambda_0}\Rightarrow
        \fun{sup}{n}\left|\frac{1}{\lambda_n}\right|<\infty
    \end{equation*}
    取$x_n\in{\rm Ker}(\lambda_n I-A)$,由4.可得
    $\{x_n\}_{n=1}^\infty$线性无关,令$X_n\defeq {\rm span}\{x_1,\cdots,x_n\}$,则
    $X_n$是$X_{n+1}$的真闭子空间,由Riesz引理$\Rightarrow \exists y_n\in X_n,||y_n||=1$使得
    \begin{equation*}
        {\rm dist}(y_n,X_{n-1})>\frac{1}{2}
    \end{equation*}
    而
    \begin{equation*}
        y_n=\sum_{k=1}^n \alpha_k x_k\Rightarrow
        (\lambda_n I-A)y_n=\sum_{k=1}^n \alpha_k(\lambda_n-\lambda_k) x_k
    \end{equation*}
    对于$\forall n,m$,不妨$n>m$,
    \begin{equation*}
        \ms{A\left(\frac{y_n}{\lambda_n}\right)-A\left(\frac{y_m}{\lambda_m}\right)}
        =\ms{ y_n-\mathop{\left[ y_n-A\left(\frac{y_n}{\lambda_n}\right)\right]}\limits_{\in X_{n-1}}-
        \mathop{\left[A\left(\frac{y_m}{\lambda_m}\right)\right]}\limits_{\in X_{m}\subset X_{n-1}} }
        \geqslant {\rm dist}(y_n,X_{n-1})>\frac{1}{2}
    \end{equation*}
    另一方面,$\{ 
        \frac{y_n}{\lambda_n}
     \}$是有界集,从而$\{A\left(\frac{y_n}{\lambda_n}\right)\}$有收敛子列,矛盾。
\end{proof}

\begin{corollary}
    $A\in\mathcal{T}(X)\Rightarrow \sigma(A)$至多可数。
\end{corollary}
\begin{proof}
    令
    \begin{equation*}
        E_k\defeq \sigma_p(A)\cap \{ \lambda\in\C:|\lambda|>\frac{1}{k} \}
    \end{equation*}
    则
    \begin{equation*}
        \sigma(A)\backslash
        \{0\}=\sigma_p(A)\backslash\{0\}=\bigcup_{k=1}^\infty E_k
    \end{equation*}
    只需证明$E_k$元素个数有限。假设不然,由B-W$\Rightarrow E_k$有极限点,
    而且${\rm dist}(0,E_k)\geqslant \frac{1}{k}$,从而$\lambda_0\neq 0$,
    但$\sigma(A)$只可能以$0$作为极限点,矛盾。
\end{proof}

\begin{corollary}
    如果$X$无穷维,$A\in\mathcal{T}(X)$,则只有三种情形:
    \begin{enumerate}[(1).]
        \item $\sigma(A)=\{0\}$,例如取$A=0$.
        \item $\sigma(A)=\{0,\lambda_1,\cdots,\lambda_n\}$.
        \item $\sigma(A)=\{0,\lambda_1,\lambda_2,\cdots\}$且$\lambda_n\rightarrow 0$.
    \end{enumerate}
    这里面的$\lambda_i\in \sigma_p(A)$.
\end{corollary}
\begin{proof}
    令
    \begin{align*}
        F_0\defeq& \sigma(A)\cap\{\lambda\in\C:|\lambda|\geqslant 1\}\\
        F_k\defeq& \sigma(A)\cap\{\lambda\in\C:\frac{1}{k+1}\leqslant |\lambda|<\frac{1}{k}\},k=1,2,\cdots
    \end{align*}
    则$\sigma(A)\subset \bigcup_{k=1}^\infty F_k$,且$F_k$中元素个数有限,
    则按照$F_k$顺次排列$\lambda_1,\lambda_2$即可。
\end{proof}

\begin{example}
    给定$\lambda_1,\cdots,\lambda_n\in \C$,令
    \begin{equation*}
        A_n:\ell^2\rightarrow\ell^2,(x_1,x_2,\cdots)\mapsto 
        (\lambda_1x_1,\lambda_2x_2,\cdots,\lambda_nx_n,0,\cdots)
    \end{equation*}
    则$A_n$是有界有限秩算子,从而是紧算子,且
    \begin{equation*}
        \{ 0,\lambda_1,\cdots,\lambda_n \}\subset \sigma_p(A)
    \end{equation*}
    而$\forall \lambda\in\C\backslash \{ 0,\lambda_1,\cdots,\lambda_n \}$有:
    \begin{align*}
        (\lambda I-A_n)x=0\Leftrightarrow& 
        ( (\lambda-\lambda_1)x_1,\cdots,(\lambda-\lambda_n)x_n,\lambda x_{n+1},\cdots )=0\\
        \Leftrightarrow& x=0\\
        \Rightarrow& \lambda I-A_n\mbox{单}\\
        \Rightarrow& \lambda I-A_n\mbox{是双射}\\
        \Rightarrow& \lambda\in\rho(A_n)\Rightarrow \sigma(A_n)=\sigma_p(A_n)=\{0,\lambda_1,\cdots,\lambda_n\}
    \end{align*}
\end{example}

\begin{example}
    设$\{\lambda_n\}_{n=1}^\infty \subset\C\backslash \{0\}$,$\lambda_n\rightarrow 0$,
    \begin{equation*}
        A:\ell^2\rightarrow \ell^2,(x_1,x_2,\cdots)\mapsto 
        (\lambda_1x_1,\lambda_2x_2,\cdots)
    \end{equation*}
    则根据收敛列有界,$A$是有界算子。同时
    \begin{equation*}
        ||A-A_n||={\rm sup}{||x||_2=1}
        \left( \sum_{k=n+1}^\infty |\lambda_k|^2|x_k|^2 \right)^\frac{1}{2}\rightarrow 0
    \end{equation*}
    $A_n$是紧算子$\Rightarrow A$是紧算子(经典方法)。

    考虑$Ae_k=\lambda_k e_k$,$\{ \lambda_k \}\subset \sigma_p(A)$,
    而且$0\notin\sigma_p(A)$.
    \begin{equation*}
        \forall \lambda\in\backslash \{0,\lambda_1,\lambda_2,\cdots\}\Rightarrow \fun{inf}{k}|\lambda-\lambda_k|>0
    \end{equation*}
    令
    \begin{equation*}
        T:\ell^2\rightarrow\ell^2,(x_1,x_2,\cdots)\mapsto 
        ( \frac{x_1}{\lambda-\lambda_1},\frac{x_2}{\lambda-\lambda_2},\cdots )
    \end{equation*}
    则$T=(\lambda I-A)^{-1}$,且
    \begin{equation*}
        ||Tx||_2\leqslant ||x||_2\cdot \fun{sup}{k}\frac{1}{|\lambda-\lambda_k|}
    \end{equation*}
    于是$T\in\L(X)$,从而$\lambda\in\rho(A)$,这说明
    $\sigma(A)=\{0,\lambda_1,\lambda_2,\cdots\}$.
\end{example}

	%第四章
	%\chapterimage{chap4.jpg}
	%\input{chapter/chapter4.tex}

	%第五章
	%\chapterimage{chap5.jpg}
	%\input{chapter/chapter5.tex}
	
	%第六章
	%\chapterimage{chap6.png}
	%\input{chapter/chapter6.tex}

	%作业
	\chapterimage{empty.jpg}
	\input{chapter/hw.tex}
\end{document}
