
\frontmatter
\thispagestyle{empty}
\newpage
\begin{center}
	\textbf{\LARGE 前言}
\end{center}

    本文档为中国科学技术大学
	刘聪文老师2023年秋季学期泛函分析课程\footnote{教材:林源渠、张恭庆.泛函分析讲义(上)[M],第二版,北京大学出版社,2021}
	的笔记,主要基于讲义、课堂板书和助教习题课。涵盖范围大致为课本3.3节之前。
	由于课程内容、顺序、侧重点并不和课本完全一致,所以章节和小节标题是按照我个人喜好划分的,
	比如我将有界算子的谱相关内容挪到了第三章,因为和第二章其他内容没什么联系。课本上的习题都会标注课本原题号,
	所有习题答案来自个人解答、网上资料,不能保证完全正确,笔记里也可能会有typo,还望读者斧正。

	泛函分析和实分析是我目前学的最酣畅淋漓的两门数学课,这两门带给我的感受,既不是
	复分析、微分几何的直观之美(以及与考试计算大赛对比带来的的割裂感),也不是统计课对于实际问题的巧妙求解。
	套用李小龙的一句话:“Don't think, feel.”很多定理、题目的证明,都可以靠一个这样的过程:
	要从$A$证明$D$,我先感受条件和结论的联系,从直观上找到可能位于中间的两个$B$和$C$,
	猜想这道题的证明方法就是$A\Rightarrow B\Rightarrow C\Rightarrow D$,
	然后完善证明细节。例如证明$(\ell^p)^*=\ell^q$时,
	难点无非就是怎么构造等距同构,回忆证明$(L^p)^*=L^q$的过程,那时构造的等距同构是函数相乘再积分,
	这和$L^2$上的内积非常像,姑且就叫它“内积”好了。所以猜测$(\ell^p)^*=\ell^q$的等距同构也是通过数列的内积构造的。
	然后再去完善其它细节。实分析很多也有类似的题目,印象比较深刻的比如
	关于紧支函数的那一部分,我就把紧支函数想象成一个“小函数”,
	它在整个实数空间里只占据了有限大的空间,然后用这些“小函数”去逼近全体函数。
	无论是泛函还是实分析,我都没有刻意去掌握做题技巧之类的东西,
	而是整理好整门课程的思路和脉络,最终不仅获得还算不错的成绩,
	也让自己切实感受到有所收获。
	不过,也有可能是我根本没学明白复分析和微分几何(笑),这些都是我个人观点。
	我相信讨厌泛函和实分析、喜欢其他课程的同学不在少数,所以上述讨论仅图一乐,请大家务必不要上纲上线。

	最后是一些格式上的说明:
	非文本类型的字母均为斜体,{\rm i}、{\rm e}等特殊量为正体;
	某些引理的证明会写在引理的框架内部,目的是与定理的证明相区分,单独引理的证明还是会写在框架外面;
	对于步骤很多的证明,暂时没有找到好用的排版方式,目前大部分采用分段然后标注粗体Step;
	${\rm Ran}(A)$、${\rm R}(A)$和${\rm Im}(A)$都代表算子$A$的值域,
	${\rm D}(A)$和${\rm Dom}(A)$代表算子$A$的定义域。
	\begin{flushright}
		最后更新:\today
	\end{flushright}
\frontmatter