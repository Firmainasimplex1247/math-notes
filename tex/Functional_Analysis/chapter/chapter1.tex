\chapter{度量空间}
%\begin{center}
%    “泛函分析心泛寒”,说明泛函是一门很难的课,但是泛函其实并没有比数分、线代、实分析更难。
%    这是一门很抽象的课,由于前面的课没学好,很多同学学到这里学崩了。泛函是压死骆驼的最后一根稻草。
%\end{center}
%\rightline{——刘聪文老师,2023.09.04}
%\vspace{-5pt}
%\begin{center}
%    \pgfornament[width=0.36\linewidth,color=lsp]{88}
%\end{center}
\section{压缩映射原理}
        \begin{definition}
            $X$是一个非空集合,映射$d:X\times X\rightarrow \mathbb{R}$满足:
            \begin{enumerate}[(1)]
                \item 唯一性:$d(x,y)=0\Leftrightarrow x=y$.
                \item 非负性:$\forall x,y\in X,d(x,y)\geqslant 0$.
                \item 对称性:$\forall x,y\in X,d(x,y)=d(y,x)$.
                \item 三角不等式:$\forall x,y,z\in X,d(x,z)\leqslant d(x,y)+d(y,z)$.
            \end{enumerate}
            则称$d$是$X$上的一个距离函数(度量),$(X,d)$称为一个度量空间,度量空间里的元素称为“点”。

            对于度量空间$X$,$Y\subset X$,
            限制在$Y$上的$d$,记作$d|_Y$,是$Y$上的度量,$(Y,d|_Y)$称为$(X,d)$的子度量空间。
        \end{definition}
        \begin{remark}
            实际上$(2)(3)(4)\Rightarrow (1)$:
            \begin{equation*}
                2d(x,y)=d(x,y)+d(y,x)\geqslant d(x,x)=0
            \end{equation*}
        \end{remark}
        \begin{example}
            $\mathbb{R}^n$上定义:$x=(x_1,\cdots,x_n),y=(y_1,\cdots,y_n)$,对于$1\leqslant p<\infty$,
            \begin{equation*}
                d_p(x,y)\mathop{=}\limits^{\rm def}\sqrt[p]{\sum_{k=1}^n |x_k-y_k|^p}
            \end{equation*}
            是距离函数。

            $p=\infty$,取$d_p(x,y) \mathop{=}\limits^{\rm def} \mathop{\rm max}\limits_{1\leqslant k\leqslant n}|x_k-y_k|$,也是距离函数。
        \end{example}
        \begin{example}
            数组空间
            \begin{equation*}
                \ell^p(\mathbb{F})\mathop{=}\limits^{\rm def}\{ (x_k):x_k\in\mathbb{F},k=1,2,\cdots,\sum_{k=1}^\infty |x_k|^p<\infty \}
            \end{equation*}
            一般$\mathbb{F}$取$\mathbb{R},\mathbb{C}$,对于$1\leqslant p<\infty$,
            \begin{equation*}
                d_p(x,y)\mathop{=}\limits^{\rm def}\sqrt[p]{\sum_{k=1}^\infty |x_k-y_k|^p}
            \end{equation*}
            是距离函数。

            $p=\infty$,
            \begin{equation*}
                \ell^\infty(\mathbb{F})=\{ (x_k):\mathop{\rm sup}_k |x_k|<\infty \}
            \end{equation*}
            \begin{equation*}
                d_\infty(x,y)=\mathop{\rm sup}_k|x_k-y_k|
            \end{equation*}
        \end{example}
        \begin{example}
            离散度量:
            \begin{equation*}
                d(x,y)=\left\{ \begin{array}{ll}0&,x=y\\1&,x\neq y\end{array} \right.
            \end{equation*}
        \end{example}
        \begin{example}
            积度量空间:对于度量空间$(X,d)$,$(Y,\rho)$,
            \begin{equation*}
                X\times Y=\{ (x,y):x\in X,y\in Y \}
            \end{equation*}
            取
            \begin{equation*}
                d_{X\times Y}( (a,b),(c,d) )\mathop{=}\limits^{\rm def}d(a,c)+\rho(b,d)
            \end{equation*}
            是距离函数。
        \end{example}
            约定一些记号:度量空间$(X,d)$上,对于$x_0\in X,r>0$,
            \begin{equation*}
                B(x_0,r)\mathop{=}\limits^{\rm def}
                \{ x\in X:d(x,x_0)<r \}
            \end{equation*}
            称为$x_0$的一个$r$邻域(球形邻域)。此外,记
            \begin{equation*}
                \overline{B}(x_0,r)\mathop{=}\limits^{\rm def}\overline{B(x_0,r)}=\{ x\in X:d(x,x_0)\leqslant r \}
            \end{equation*}
            \begin{equation*}
                S(x_0,r)\mathop{=}\limits^{\rm def}\{ x\in X:d(x,x_0) = r \}
            \end{equation*}
            对于$A\subset X$,
            \begin{equation*}
                {\rm diam}(A)\mathop{=}\limits^{\rm def}\mathop{\rm sup}\limits_{x,y\in A}d(x,y)
            \end{equation*}
            那么如果${\rm diam}(A)$有限,称$A$有界,等价条件为$\exists B(x_0,R)\supset A$.
        \begin{definition}
            度量空间$(X,d)$上,称$\{x_n\}_{n=1}^\infty$收敛,是指存在
            $x_0\in X$使得
            \begin{equation*}
                d(x_n,x_0)\rightarrow 0{\rm\ as\ }n\rightarrow\infty
            \end{equation*}
        \end{definition}
        \begin{corollary}
            度量空间$(X,d)$上的收敛列的极限唯一,且收敛列有界。
        \end{corollary}
        \begin{proof}
            设$\{x_n\}_{n=1}^\infty$是$X$上的收敛列,

            极限唯一:假设$\{x_n\}_{n=1}^\infty$有两个极限$a,b$且$a\neq b$,即
            \begin{equation*}
                d(x_n,a)\rightarrow 0,d(x_n,b)\rightarrow 0
            \end{equation*}
            取$\varepsilon=\frac{1}{2}d(a,b)>0$,存在$N$使得$n>N$时$d(x_n,a)<\varepsilon$,于是
            \begin{align*}
                d(a,b)&\leqslant d(x_n,a)+d(x_n,b)\\
                \Rightarrow d(x_n,b)&\geqslant d(a,b)-d(x_n,a)>\varepsilon-\frac{1}{2}\varepsilon=\frac{1}{2}\varepsilon
            \end{align*}
            这与$d(x_n,b)\rightarrow 0$矛盾。
        
            收敛列有界:设$\{x_n\}_{n=1}^\infty$收敛到$x_0$,$\varepsilon>0$,存在$N$使得$n>N$时
            $d(x_n,x_0)<\varepsilon$,令
            \begin{equation*}
                R_0=\mathop{\rm max}\limits_{1\leqslant n\leqslant N}d(x_n,x_0),R={\rm max}\{ R_0,\varepsilon \}
            \end{equation*}
            则$\{ x_n \}_{n=1}^\infty\subset B(x_0,R)$.            
        \end{proof}
        \begin{example}
            $C[0,1]$为$[0,1]$上连续函数全体,定义度量:
            \begin{equation*}
                d(f,g)\mathop{=}\limits^{\rm def} \mathop{\rm max}\limits_{x\in[0,1]}|f(x)-g(x)|
            \end{equation*}
            于是
            \begin{equation*}
                d(f_n,f)\rightarrow 0\Leftrightarrow f_n\rightrightarrows f
            \end{equation*}
            若定义:
            \begin{equation*}
                \rho_1(f,g)\mathop{=}\limits^{\rm def}
                \int_0^1|f(t)-g(t)|{\rm d}t
            \end{equation*}
            则$\rho_1$也是$C[0,1]$上的一个度量,设
            \begin{equation*}
                f_k(t)=\left\{ \begin{array}{ll}
                    -k^3(t-\frac{1}{k^2})&,t\in [0,\frac{1}{k^2}]\\
                    0&,t\in [\frac{1}{k^2},1]
                \end{array} \right.
            \end{equation*}
            那么$\rho_1(f_k,0)=\frac{1}{2}k\cdot \frac{1}{k^2}\rightarrow 0{\rm\ as\ }k\rightarrow \infty$,
            但是$d(f_k,0)=k\nrightarrow 0$.

            这个例子说明不同度量下的点列的收敛情况可能不同。
        \end{example}
        \begin{definition}
            度量空间$(X,d)$,称$X$中的集合$E$是开集是指:
            $\forall x\in E$,$\exists r>0$使得$B(x,r)\subset E$,即$\forall x\in E$是$E$的内点。

            开集的余集称为闭集。\footnote{实际上这是一般拓扑度量空间上闭集的最原始定义,如果该拓扑是由度量诱导的,
            即所有点都是内点的集合作为开集构成拓扑(见定理1.1.1),则闭集的定义有其他等价表述(见推论1.1.2)}
        \end{definition}
        \begin{theorem}
            记$X$上所有的开集为$\tau$,$\tau$满足拓扑的定义:
            \begin{enumerate}
                \item $\varnothing\in\tau$,$X\in \tau$.
                \item $\tau$对于任意并封闭。
                \item $\tau$对于有限交封闭。
            \end{enumerate}
        \end{theorem}
        \begin{definition}
            度量空间$(X,d)$,$E\subset X$满足:存在$x_0\in X$使得
            \begin{enumerate}
                \item $\forall \varepsilon>0$,$B(x_0,\varepsilon)\cap E\neq \varnothing$,则$x_0$为$E$的接触点;
                \item $\forall \varepsilon>0$,$B(x_0,\varepsilon)\cap (E\backslash \{x_0\})\neq \varnothing$,则$x_0$为$E$的聚点。
            \end{enumerate}
            $E$的接触点全体称为$E$的闭包,记作$\overline{E}$.
        \end{definition}
        \begin{corollary}
            度量空间$(X,d)$,$E\subset X$,下列命题等价:
            \begin{enumerate}
                \item $E$是闭集;
                \item $E=\overline{E}$;
                \item $\forall \{x_n\}_{n=1}^\infty\subset E$,如果$x_n\rightarrow x_0$,则$x_0\in E$.
            \end{enumerate}
        \end{corollary}
        \begin{proof}
            证明:
            $(1)\Rightarrow (2)$:$E$闭$\Rightarrow E^c$开,如果存在
            $x_0\in \overline{E}$且$x_0\notin E$,则$x_0\in E^c$,存在邻域$B=B(x_0,\varepsilon_0)\subset E^c\Rightarrow B\cap E=\varnothing$,
            这与$x_0$是接触点矛盾,故$\overline{E}\subset E$,进而$\overline{E}=E$.
        
            $(2)\Rightarrow (1)$:$\overline{E}=E\Rightarrow \forall y\in E^c,y\notin \overline{E}\Rightarrow $存在邻域$B(y,\varepsilon)\cap E=\varnothing\Rightarrow
            B(y,\varepsilon)\subset E^c\Rightarrow E^c$开$\Rightarrow E$闭。
        
            $(2)\Rightarrow (3)$:任取收敛列$\{x_n\}_{n=1}^\infty\subset E$,$x_n\rightarrow x_0$,则$\forall \varepsilon>0$,存在
            $N$使得$n>N$时$x_n\in B(x_0,\varepsilon)$,于是
            $x_{N+1}\in B(x_0,\varepsilon)\cap E\neq \varnothing\Rightarrow x_0\in \overline{E}$.
        
            $(3)\Rightarrow (2)$:设$x_0\in \overline{E}$,取$\varepsilon_1>0$,存在$x_1\in B(x_0,\varepsilon_1)\cap E$;
            取$0<\varepsilon_2<\varepsilon_1$使得
            存在$x_2\in B(x_0,\varepsilon_2)\cap E$且$x_2\neq x_1$;以此类推,并可以要求$\varepsilon_n\rightarrow 0$,否则
            $B(x_0,\mathop{\rm inf}\limits_{n}\{\varepsilon_n\} )\cap E=\varnothing$,这与$x_0\in\overline{E}$矛盾。最后得到点列$\{x_n\}$,且
            $x_n\rightarrow x_0$,由$(3)$可知$x_0\in E$,所以$\overline{E}\subset E$,进而$\overline{E}=E$.
        \end{proof}

        \begin{definition}
            度量空间$(X,d)$,$E\subset X$,$x_0\in E$,如果
            $\forall \varepsilon>0$,$ B(x_0,\varepsilon)\backslash\{x_0\}\cap E=\varnothing $,
            等价于存在点列$\{x_n\}\subset E$使得$x_n\rightarrow x_0$,则称
            $x_0$是$E$的聚点或者极限点。

            记$E'$为$E$的聚点全体,称为$E$的导集。

            记$\overline{E}=E\cup E'$,称为$E$的闭包。如果$\overline{E}=X$,称$E$在$X$中稠密,记作$E\mathop{\subset}\limits^{\rm dense} X$或者$E\subset\subset X$.
            如果$X$有一个可数稠密子集,称$X$可分。
        \end{definition}

        \begin{example}
            $\mathbb{Q}\mathop{\subset}\limits^{\rm dense}\mathbb{R}$,
            多项式全体$P[a,b]\mathop{\subset}\limits^{\rm dense} C[a,b]$.
        \end{example}

        \begin{example}
            $C[a,b]$可分。
        \end{example}
        \begin{proof}
            记$Q[a,b]$为$[a,b]$上全体有理系数多项式,这是一个可数集;$P[a,b]$是$[a,b]$上全体实系数多项式。设
            \begin{equation*}
                r(x)=r_nx^n+\cdots+r_0\in P[a,b],\forall r_i\in \mathbb{R}
            \end{equation*}
            因为$\mathbb{Q}\mathop{\subset}\limits^{\rm dense}\mathbb{R}$,$\forall \varepsilon>0$,对于每个$r_i$,存在
            $q_i\in \mathbb{Q}{\rm\ s.t.\ }|r_i-q_i|<\varepsilon$,于是令
            \begin{equation*}
                q(x)=q_nx^n+\cdots+q_0\in Q[a,b]
            \end{equation*}
            \begin{align*}
                \Rightarrow d(r,q)&=\mathop{\rm sup}\limits_{x\in [a,b]}
                | (r_n-q_n)x^n+\cdots+(r_0-q_0) |\\
                &\leqslant \mathop{\rm sup}\limits_{x\in [a,b]}\varepsilon(|x^n|+\cdots+|x|+1)\\
                &=C\varepsilon\rightarrow 0{\rm\ as\ }\varepsilon\rightarrow 0
            \end{align*}
            因此$Q[a,b]\mathop{\subset}\limits^{\rm dense}P[a,b]$,而$P[a,b]\mathop{\subset}\limits^{\rm dense}C[a,b]$,$\forall f\in C[a,b]$,设
            $d(r,f)<\frac{1}{2}\varepsilon$,$d(q,r)<\frac{1}{2}\varepsilon$,
            \begin{equation*}
                d(q,f)\leqslant d(r,f)+d(q,r)<\varepsilon
            \end{equation*}
            所以$C[a,b]$有可数稠密子集$Q[a,b]$.
        \end{proof}

        \begin{definition}
            $(X,d)$,$(Y,\rho)$是两个度量空间,设映射:
            \begin{equation*}
                T:X\rightarrow Y
            \end{equation*}
            在$x_0\in X$处连续是指:$\forall \varepsilon>0$,存在$\delta>0$使得
            \begin{equation*}
                d(x,x_0)<\delta\Rightarrow \rho(T(x),T(x_0))<\varepsilon
            \end{equation*}
            如果$T$在$X$中的每一点都连续,则称$T$是连续映射。
        \end{definition}

        \begin{theorem}
            $(X,d)$,$(Y,\rho)$是两个度量空间,
            映射$ T:X\rightarrow Y $连续$\Leftrightarrow $任取$Y$上的开集$U$,
            $T^{-1}(U)$是$X$上的开集。
        \end{theorem}
        \begin{proof}
            记$B_X(x_0,r)=\{ x\in X|d(x,x_0)<r \}$,$B_Y(y_0,r)=\{ y\in Y|\rho(y,y_0)<r \}$.

            $(\Rightarrow )$:$\forall y_0\in U$,$B_Y(y_0,\varepsilon)\subset U$,设$x_0\in T^{-1}(U)$,
            因为$T$连续,存在$\delta>0$使得$x\in B_X(x_0,\delta)\Rightarrow T(x)\in B_Y(y_0,\varepsilon)\subset U$,所以
            $B_X(x_0,\delta)\subset T^{-1}(U)$,由于$y_0$的任意性,$x_0$能取遍整个$T^{-1}(U)$,故$T^{-1}(U)$为开集。
        
            $(\Leftarrow)$:对于$x_0\in X$,设$y_0=T(x_0)$,$U=B_Y(y_0,\varepsilon)$是$Y$上的开集,
            且$x_0\in T^{-1}(U)$,所以存在$B_X(x_0,\delta)\subset T^{-1}(U)$,则
            $x\in B_X(x_0,\delta)\Rightarrow T(x)\in U=B_Y(y_0,\varepsilon)$.由$\varepsilon$的任意性,$T$是连续映射。
        \end{proof}

        \begin{theorem}[Heine]
            $T$在$x_0$处连续$\Leftrightarrow$
            任取$X$上收敛到$x_0$的点列$\{x_n\}$,都有$T(x_n)\rightarrow T(x_0)$.
        \end{theorem}
        \begin{proof}
            $(\Rightarrow )$:$\forall \varepsilon>0,\exists \delta>0{\rm\ s.t.\ }x\in B_X(x_0,\delta)\Rightarrow T(x)\in B_Y(y_0,\varepsilon)$,
            存在$N$使得$n>N$时$x_n\in B_X(x_0,\delta)\Rightarrow T(x_n)\in B_Y(T(x_0),\varepsilon)\Rightarrow T(x_n)\rightarrow T(x_0)$.
        
            $(\Leftarrow)$:假设$T$在$x_0$处不连续,即存在$\varepsilon>0$,$\forall \delta>0$,存在$x\in B_X(x_0,\delta){\rm\ s.t.\ }T(x)\notin B_Y(y_0,\varepsilon)$,
            取$x_n\rightarrow x_0$,每个$T(x_n)\notin B_Y(y_0,\varepsilon)$,则$T(x_n)\nrightarrow T(x_0)$,矛盾。
        \end{proof}

        \begin{definition}
            度量空间$(X,d)$,称$\{x_n\}$是一个基本列(或者叫Cauthy列)是指:
            $\forall \varepsilon>0$,存在$N$使得:
            \begin{equation*}
                d(x_m,x_n)<\varepsilon,\forall m,n\geqslant N
            \end{equation*}
            如果$(X,d)$中任意基本列都收敛,则称$(X,d)$完备。完备的度量空间称为Banach空间,
        \end{definition}

        \begin{example}
            $(\mathbb{R},d)$完备,$(\mathbb{Q},d)$不完备;$L^p[0,1]$完备。
        \end{example}
        \begin{example}
            例1.1.5中的$(C[0,1],d)$完备。
        \end{example}
        \begin{proof}
            设$\{f_n\}_{n=1}^\infty$是$C[0,1]$中任一基本列,于是
            \begin{equation*}
                \forall \varepsilon,\exists N{\rm\ s.t.\ }
                \mathop{\rm max}\limits_{t\in[0,1]}|f_m(t)-f_n(t)|<\varepsilon,\forall m,n\geqslant N
            \end{equation*}
            于是对于每个固定的$t\in [0,1]$,$|f_m(t)-f_n(t)|<\varepsilon$,从而可知
            $\{f_n(t)\}_{n=1}^\infty$是$\mathbb{R}$中的基本列,于是存在极限$f(t)=\mathop{\rm lim}\limits_{n\rightarrow\infty}f_n(t)$.
            在(1.1.1)式中令$m\rightarrow\infty$,于是
            \begin{align*}
                &\mathop{\rm max}\limits_{t\in[0,1]}|f_m(t)-f(t)|\leqslant \varepsilon,\forall n\geqslant N\\
                \Rightarrow &f_n\rightrightarrows f\\
                \Rightarrow &f\in C[0,1],d(f_n,f)\rightarrow 0
            \end{align*}
        \end{proof}

        \begin{example}
            例1.1.5中的$(C[0,1],\rho_1)$不完备。
        \end{example}
        \begin{proof}
            令:
            \begin{equation*}
                f_n(t)\mathop{=}\limits^{\rm def}
                \left\{ \begin{array}{ll}
                    0&,t\in[0,\frac{1}{2}-\frac{1}{n}]\\
                    nt-\frac{n}{2}+1&,t\in [\frac{1}{2}-\frac{1}{n},\frac{1}{2}]\\
                    1&,t\in[\frac{1}{2},1]
                \end{array} \right.
            \end{equation*}
            于是
            \begin{equation*}
                \rho_1(f_n,f_m)=\frac{1}{2}\left| \frac{1}{n}-\frac{1}{m} \right|\rightarrow 0{\rm\ as\ }n,m\rightarrow\infty
            \end{equation*}
            因此$\{f_n\}$是基本列。下面证明它不收敛,假设存在$f\in C[0,1]$使得
            $\rho_1(f_n,f)\rightarrow 0$,
            \begin{align*}
                \rho_1(f_n,f)&=\int_0^1 |f_n(t)-f(t)|{\rm d}t\\
                &=\int_0^{\frac{1}{2}-\frac{1}{n}}|f(t)|{\rm d}t+
                \int_{\frac{1}{2}-\frac{1}{n}}^{\frac{1}{2}}
                |f_n(t)-f(t)|{\rm d}t+\int_{\frac{1}{2}}^1 |1-f(t)|{\rm d}t
            \end{align*}
            令$n\rightarrow\infty$可得
            \begin{equation*}
                \int_0^{\frac{1}{2}}|f(t)|{\rm d}t+\int_{\frac{1}{2}}^1 |1-f(t)|{\rm d}t=0
            \end{equation*}
            所以
            \begin{equation*}
                f(t)=\left\{ \begin{array}{ll}
                    0&,t\in (0,\frac{1}{2})\\
                    1&,t\in (\frac{1}{2},1)
                \end{array} \right.
            \end{equation*}
            和$f$连续矛盾。
        \end{proof}

        \begin{example}
            离散度量空间完备。
        \end{example}
        \begin{proof}
            任取$X$上的柯西列$\{x_n\}_{n=1}^\infty$,取$\varepsilon=\frac{1}{2}$,
            存在$N$使得$\forall n,m>N$,$d(x_n,x_m)<\frac{1}{2}\Rightarrow d(x_n,x_m)=0$,即
            $\forall n>N,x_n=x_{N+1}$,所以$x_n\rightarrow x_{N+1}\in X$.故离散度量空间完备。
        \end{proof}

        \begin{definition}
            度量空间$(X,d)$,映射$T:X\rightarrow X$,如果存在
            $x^*\in X$使得$T(x^*)=x^*$,则称$x^*$是$T$的一个不动点。

            如果存在$\alpha\in (0,1)$,使得
            \begin{equation*}
                d(T(x),T(y))\leqslant \alpha d(x,y),\forall x,y\in X
            \end{equation*}
            则称$T$是一个压缩映射。
        \end{definition}
        \begin{theorem}[Banach不动点定理、压缩映射原理]
            完备度量空间到自身的压缩映射有唯一的不动点。
        \end{theorem}
        \begin{proof}
            存在性:这个证明方法叫做Picard迭代,任取$x_0\in X$,定义迭代序列:
            \begin{equation*}
                x_{n+1}=T(x_n),n=0,1,\cdots
            \end{equation*}
            \begin{align*}
                \Rightarrow d(x_{n+1},x_n)&=d( T(x_n),T(x_{n-1}) )\\
                &\leqslant \alpha d(x_n,x_{n-1})=\alpha d( T(x_{n-1}),T(x_{n-2}) )\\
                &\leqslant \alpha^2 d(x_{n-1},x_{n-2})\\
                &\leqslant\cdots\leqslant \alpha^n d(x_1,x_0)
            \end{align*}
            利用三角不等式,
            \begin{align*}
                \Rightarrow d( x_{n+p},x_n )&\leqslant \sum_{k=1}^p d(x_{n+k},x_{n+k-1})\\
                &\leqslant \sum_{k=1}^p \alpha^{n+k-1}d(x_1,x_0)\\
                &\leqslant \frac{\alpha^n}{1-\alpha}d(x_1,x_0)<\varepsilon,n\mbox{充分大时},\forall p
            \end{align*}
            \begin{align*}
                \Rightarrow& \{x_n\}_{n=1}^\infty \mbox{是$X$中的基本列}\\
                X\mbox{完备}\Rightarrow& \exists x^*\in X{\rm\ s.t.\ }d(x_n,x^*)\rightarrow 0{\rm\ as\ }n\rightarrow\infty\\
                \Rightarrow& d( T(x^*),x^* )\leqslant d( T(x^*),T(x_n) )+d( T(x_n),x_n )+
                d(x_n,x^*)\\
                &\leqslant \alpha d(x^*,x_n)+d(x_{n+1},x_n)+d(x_n,x^*)\rightarrow 0{\rm\ as\ }n\rightarrow\infty\\
                \Rightarrow& T(x^*)=x^*
            \end{align*}

            唯一性:设$y^*$是另一个不动点,则
            \begin{equation*}
                d(x^*,y^*)=d(T(x^*),T(y^*))\leqslant \alpha d(x^*,y^*)
            \end{equation*}
            因此只能$d(x^*,y^*)=0\Rightarrow x^*=y^*$.
        \end{proof}
        \begin{example}
            完备的条件不可去,例如$X=(0,1)$不完备,
            度量$d(x,y)=|x-y|$,压缩映射$T(x)=\frac{1}{2}x$,无不动点。
        \end{example}
    
\section{完备化}
        \begin{definition}
            $(X_1,d_1)$和$(X_2,d_2)$是两个度量空间,如果映射$T:X_1\rightarrow X_2$保持距离不变,即
            \begin{equation*}
                d_1(x,y)=d_2(T(x),T(y)),\forall x,t\in X_1
            \end{equation*}
            则称$T$是等距映射。如果存在从$X_1$到$X_2$的既单又满的等距映射,则称
            $(X_1,d_1)$和$(X_2,d_2)$是等距同构的。
        \end{definition}
        \begin{definition}
            如果$(X_1,d_1)$和$(X_2,d_2)$的某个子空间等距同构,则称
            $(X_1,d_1)$可等距嵌入$(X_2,d_2)$,记作
            \begin{equation*}
                (X_1,d_1) \hookrightarrow (X_2,d_2)
            \end{equation*}
            在此意义下,称$X_1$是$X_2$的子空间。
        \end{definition}
        \begin{definition}
            对于度量空间$(X,d)$,如果存在完备的度量空间
            $(\tilde{X},\tilde{d})$,它的某个稠密子空间$X_0$和$X$等距同构,则称
            $(\tilde{X},\tilde{d})$是$(X,d)$的一个完备化。
        \end{definition}
        \begin{example}
            \begin{enumerate}
                \item $\mathbb{R}$是$\mathbb{Q}$的完备化。
                \item $L^1[a,b]$是$(C[a,b],\rho_1)$的完备化。
                \item $C[a,b]$是$( P[a,b],d )$的完备化。
            \end{enumerate}
        \end{example}
        \begin{theorem}
            任何度量空间都有完备化,且完备化在等距同构意义下唯一,
            即如果$(\tilde{X},\tilde{d})$和$(X',d')$都是$(X,d)$的完备化,则二者等距同构。
        \end{theorem}
        我们的证明思路是:
        \begin{enumerate}
            \item 构造$(\tilde{X},\tilde{d})$.
            \item 构造稠密子空间$(X_0,\tilde{d})$和等距同构。
            \item 证明$(\tilde{X},\tilde{d})$完备。
            \item 证明等距同构意义下的唯一性。
        \end{enumerate}
        \begin{proof}
            \begin{enumerate}
                \item $\mathcal{F}$定义为$(X,d)$上基本列全体,记$\xi=\{x_n\}_{n=1}^\infty$,
                $\eta=\{y_n\}_{n=1}^\infty$,
                在$\mathcal{F}$上引入等价关系:
                \begin{equation*}
                    \xi\sim\eta\mathop{\Leftrightarrow}\limits^{\rm def}
                    \mathop{\rm lim}\limits_{n\rightarrow\infty}d(x_n,y_n)=0
                \end{equation*}
                $\tilde{X}\mathop{=}\limits^{\rm def}\mathcal{F}/_\sim$,定义
                $\tilde{X}$上的度量:
                \begin{equation*}
                    \tilde{d}( [\xi],[\eta] )
                    \mathop{=}\limits^{\rm def}
                    \mathop{\rm lim}\limits_{n\rightarrow\infty}d(x_n,y_n)
                \end{equation*}
                这里$\{x_n\}_{n=1}^\infty$是$[\xi]$中任一代表元,
                $\{y_n\}_{n=1}^\infty$是$[\eta]$中任一代表元,
                    \begin{enumerate}
                        \item $\tilde{d}$良定:\begin{enumerate}
                            \item $\mathop{\rm lim}\limits_{n\rightarrow\infty}d(x_n,y_n)$存在:由三角不等式:
                            \begin{align*}
                                | d(x_n,y_n)-d(x_m,y_m) |=&| d(x_n,y_n)-d(y_n,x_m)+d(x_m,y_n)-d(x_m,y_m) |\\
                                \leqslant & |d(x_n,y_n)-d(y_n,x_m)|+|d(x_m,y_n)-d(x_m,y_m)|\\
                                \leqslant & d(x_n,x_m)+d(y_n,y_m) \rightarrow 0{\rm\ as\ }n,m\rightarrow\infty
                            \end{align*}
                            所以$\{ d(x_n,y_n) \}_{n=1}^\infty$是$\mathbb{R}$中基本列,再由$\mathbb{R}$的完备性可知极限存在。
                        \item $\tilde{d}([\xi],[\eta])$不依赖于$[\xi],[\eta]$的代表元选取:
                            设$\xi^{(1)}=\{ x_n^{(1)} \}_{n=1}^\infty,\xi^{(2)}=\{ x_n^{(2)} \}_{n=1}^\infty\in[\xi]$,
                            根据定义有$\mathop{\rm lim}\limits_{n\rightarrow\infty}d( x_n^{(1)}, x_n^{(2)} )=0$,
                            \begin{align*}
                                &|d(x_n^{(1)},y_n)-d( x_n^{(2)},y_n )|\leqslant d(x_n^{(1)},x_n^{(2)})\rightarrow 0{\rm\ as\ }n\rightarrow\infty\\
                                \Rightarrow &
                                \mathop{\rm lim}\limits_{n\rightarrow\infty}d(x_n^{(1)},y_n)=\mathop{\rm lim}\limits_{n\rightarrow\infty}d(x_n^{(2)},y_n)
                            \end{align*}
                            $[\eta]$同理。
                        \end{enumerate}
                        \item $\tilde{d}$是度量:平凡。
                    \end{enumerate}
                \item 对$x\in X$,记$\xi_x\mathop{=}\limits^{\rm def}(x,x,\cdots)$,称为常驻点列,当然也是一个基本列。设
                    \begin{equation*}
                        X_0\mathop{=}\limits^{\rm def}
                        \{ [\xi_x]:x\in X \}\subset\tilde{X}
                    \end{equation*}
                    设$T:X\rightarrow X_0,x\mapsto [\xi_x]$,则$T$是
                    $X$到$X_0$的等距同构。任取$[\xi]\in \tilde{X}$,任取代表元
                    $\{x_n\}_{n=1}^\infty$,根据$\tilde{d}$的定义有(常驻点列$\xi_{x_n}$的第$k$项是$x_n$,点列$\xi$的第$k$项为$x_k$)
                    \begin{equation*}
                        \mathop{\rm lim}\limits_{n\rightarrow\infty}
                        \tilde{d}( [\xi_{x_n}],[\xi] )
                        =\mathop{\rm lim}\limits_{n\rightarrow\infty}
                        \mathop{\rm lim}\limits_{k\rightarrow\infty}
                        d(x_n,x_k)
                    \end{equation*}
                    而$\xi=\{x_n\}_{n=1}^\infty$是基本列,所以上式等于$0$,即
                    柯西列$[\xi_{x_n}]\rightarrow [\xi]$,可得$\overline{X_0}=\tilde{X}$.
                \item $(\tilde{X},\tilde{d})$完备:设$ \{ [\xi^{(k)}] \}_{k=1}^\infty $是
                    $(\tilde{X},\tilde{d})$中任一基本列,这里$\xi^{(k)}=
                    \{x_n^{(k)}\}_{n=1}^\infty$,
                    由上一步的结论,取常驻点列$\xi_{x_{n}^{(k)}}=(x_{n}^{(k)},\cdots)$,
                    于是对每个$k$,$[\xi_{x_{n}^{(k)}}]\rightarrow [\xi^{(k)}]$
                    取充分大的$n_k$使得:
                    \begin{equation*}
                        \tilde{d}( [\xi^{(k)}],[\xi_{x_{n_k}^{(k)}}] )<\frac{1}{k}
                    \end{equation*}
                    可得
                    \begin{align*}
                        \tilde{d}( [\xi_{x_{n_k}^{(k)}}],[\xi_{x_{n_j}^{(j)}}] )
                        &\leqslant \tilde{d}( [\xi_{x_{n_k}^{(k)}}],[\xi^{(k)}] )
                        +\tilde{d}( [\xi^{(k)}],[\xi^{(j)}] )+
                        \tilde{d}( [\xi^{(j)}],[\xi_{x_{n_j}^{(j)}}] )\\
                        &< \frac{1}{k}+\tilde{d}( [\xi^{(k)}],[\xi^{(j)}] )
                        +\frac{1}{j}\rightarrow 0{\rm\ as\ }j,k\rightarrow\infty
                    \end{align*}
                    令$\xi'\mathop{=}\limits^{\rm def}
                    \{ x_{n_k}^{(k)} \}_{k=1}^\infty \in \mathcal{F}$,于是$[\xi']\in\tilde{X}$,
                    根据$\tilde{d}$的定义有(常驻点列$\xi_{x_{n_k}^{(k)}}$的第$j$项是$\xi_{x_{n_k}^{(k)}}$,
                    点列$\xi'$的第$j$项为$x_{n_j}^{(j)}$)
                    \begin{equation*}
                        \mathop{\rm lim}\limits_{k\rightarrow\infty}
                        \tilde{d}( [\xi_{x_{n_k}^{(k)}}],[\xi'] )
                        =\mathop{\rm lim}\limits_{k\rightarrow\infty}
                        \mathop{\rm lim}\limits_{j\rightarrow\infty}
                        d(x_{n_k}^{(k)},x_{n_j}^{(j)})=0
                    \end{equation*}
                    最后
                    \begin{equation*}
                        \tilde{d}( [\xi^{(k)}],[\xi'] )
                        \leqslant \mathop{\tilde{d}( [\xi^{(k)}],[\xi_{x_{n_k}^{(k)}}] )}\limits_{<\frac{1}{k}\rightarrow 0}
                        +\mathop{\tilde{d}( [\xi'],[\xi_{x_{n_k}^{(k)}}] )}\limits_{\rightarrow 0}
                    \end{equation*}
                    所以$[\xi^{(k)}]\rightarrow [\xi']$,完备性得证。
                \item 唯一性:
                设$(X',d')$也是$(X,d)$的完备化,即$(X,d)$等距同构于
                $(X',d')$的一个稠密子空间$(X_0',d')$,设
                $T':X\rightarrow X_0'$是等距同构,则
                $\varphi=T'\circ T^{-1}$是$(X_0,\tilde{d})$
                到$(X_0',d')$的等距同构,下面把$\varphi$延拓为
                $(\tilde{X},\tilde{d})$到$(X',d')$的等距同构。
                \begin{equation*}
                    \forall [\xi]\in \tilde{X}
                    ,\exists [\xi^n]\in X_0,n=1,2,\cdots{\rm\ s.t.\ }
                    \tilde{d}( [\xi^n],[\xi] )\rightarrow 0\mbox{(由稠密性)}
                \end{equation*}
                因为$\varphi$是等距映射,所以$\varphi( [\xi^{(n)}] )$是$(X',d')$中的基本列。$X'$完备,所以
                存在$y\in X'$使得$d'(\varphi[\xi^{(n)}],y)\rightarrow 0$,定义映射:
                $\Phi:\tilde{X}\rightarrow X',[\xi]\mapsto y$,接下来验证$\Phi$是等距同构(作业):

                任取$[\xi^{(1)}],[\xi^{(2)}]\in\tilde{X}$,设$\Phi([\xi^{(1)}])=y_1,\Phi([\xi^{(2)}])=y_2$,
                \begin{align*}
                    &d'(T'(x_n^{(1)}),T'(x_n^{(2)}))-d'(y_1,y_2)\\
                    &\leqslant ( d'(T'(x_n^{(1)}),y_2)+d'(y_2,T'(x_n^{(2)})) )-( d(T'(x_n^{(1)}),y_2)-d'(T'(x_n^{(1)}),y_1) )\\
                    &\leqslant d'(T'(x_n^{(1)}),y_1)+d'(T'(x_n^{(2)}),y_2)\rightarrow 0
                \end{align*}
                而$d'(T'(x_n^{(1)}),T'(x_n^{(2)}))=\tilde{d}( [\xi_{x_n^{(1)}}],[\xi_{x_n^{(2)}}] )$,同理
                \begin{align*}
                    \tilde{d}( [\xi_{x_n^{(1)}}],[\xi_{x_n^{(2)}}] )-\tilde{d}( [\xi^{(1)}],[\xi^{(2)}] )
                    \leqslant \tilde{d}( [\xi_{x_n^{(1)}}],[\xi^{(1)}] )+\tilde{d}( [\xi_{x_n^{(2)}}],[\xi^{(2)}] )\rightarrow 0
                \end{align*}
                因此
                \begin{equation*}
                    \tilde{d}( [\xi^{(1)}],[\xi^{(2)}] )=\mathop{\rm lim}\limits_{n\rightarrow\infty}\tilde{d}( [\xi_{x_n^{(1)}}],[\xi_{x_n^{(2)}}] )
                    =\mathop{\rm lim}\limits_{n\rightarrow\infty}d'(T'(x_n^{(1)}),T'(x_n^{(2)}))
                    =d'(y_1,y_2)
                \end{equation*}
                这就证明了$\Phi$是等距同构。
            \end{enumerate}
        \end{proof}
    
\section{紧性推理}
        \begin{definition}
            度量空间$(X,d)$,$A\subset X$,
            \begin{enumerate}
                \item 如果一族开集$\{G_\alpha\}_{\alpha\in \Lambda}$使得
                    \begin{equation*}
                        \bigcup_{\alpha\in\Lambda}G_\alpha\supset A
                    \end{equation*}
                    则称$\{G_\alpha\}_{\alpha\in \Lambda}$是$A$的一个开覆盖。
                \item 如果$A$的任一开覆盖$\{G_\alpha\}_{\alpha\in \Lambda}$都有有限子覆盖,即
                    存在$\alpha_1,\alpha_2,\cdots,\alpha_N\in \Lambda$使得
                    \begin{equation*}
                        \bigcup_{k=1}^N G_{\alpha_k}\supset A
                    \end{equation*}
                    则称$A$紧。
                \item 如果$A$中任一点列都有在$X$中收敛的子列,则称$A$列紧。
                \item 如果$A$中任一点列都有在$A$中收敛的子列,则称$A$自列紧。
                \item 如果空间$X$自身列紧,则称$X$为列紧空间。
            \end{enumerate}
        \end{definition}
        \begin{corollary}
            度量空间上,自列紧集等价于列紧闭集。
        \end{corollary}
        \begin{example}
            $\mathbb{R}^n$中,列紧集等价于有界集,自列紧集等价于有界闭集等价于紧集。
        \end{example}
        \begin{example}
            一般度量空间中,有界集不一定列紧,如无穷维线性空间和欧式度量构成的度量空间,设
            $e_n$为第$n$个分量为$1$,其余为$0$的向量,无穷点列$\{e_n\}_{n=1}^\infty$是有界的,
            但是$d(e_n,e_m)=\sqrt{2},\forall n\neq m$,无收敛子列。
        \end{example}
        \begin{proposition}
            列紧空间中任一集合都列紧,任一闭集都自列紧。
        \end{proposition}
        \begin{proposition}
            列紧空间一定完备。
        \end{proposition}
        \begin{proof}
            设$\{x_n\}_{n=1}^\infty$是$(X,d)$中基本列,
            $(X,d)$列紧$\Rightarrow $有子列$x_{n_k}\rightarrow x_0\in X$,
            \begin{equation*}
                \Rightarrow d(x_n,x_0)\leqslant d(x_n,x_{n_k})+d(x_{n_k},x_0)\rightarrow 0{\rm\ as\ }n,k\rightarrow\infty
            \end{equation*}
        \end{proof}
        \begin{definition}
            度量空间$(X,d)$,$A\subset X$,$\varepsilon>0$,
            \begin{enumerate}
                \item 称$N_\varepsilon\subset A$是$A$的一个$\varepsilon$网是指\begin{equation*}
                    \forall x\in A,\exists y\in N_\varepsilon{\rm\ s.t.\ }d(x,y)<\varepsilon
                \end{equation*}
                等价于
                \begin{equation*}
                    A\subset \bigcup_{y\in N_\varepsilon}B(y,\varepsilon)
                \end{equation*}
                即由半径为$\varepsilon$的开球组成的$A$的开覆盖。
                \item 如果$\forall \varepsilon>0$,都有$A$的一个元素有限的$\varepsilon$网,则称$A$完全有界。即可以选取有限个半径为$\varepsilon$的开球作为$A$的开覆盖。
            \end{enumerate}
        \end{definition}
        \begin{proposition}
            完全有界$\Rightarrow $有界
        \end{proposition}
        \begin{proof}
            有$A$的有限$1$网$N_1:=\{ y_1,\cdots,y_m \}$,于是令
            $R=\sum_{k=2}^m d(y_k,y_1)+1$,则
            \begin{equation*}
                A\subset \bigcup_{k=1}^m B(y_k,1)\subset B(y_1,R)
            \end{equation*}
        \end{proof}
        \begin{example}
            有界并不一定完全有界,考虑例$1.3.2$,
            $A=\{e_n\}_{n=1}^\infty$没有有限的$\frac{1}{2}$网,因为每个$\frac{1}{2}$球只能覆盖球心。
        \end{example}
        \begin{theorem}[Hausdorff]
            \begin{enumerate}
                \item 列紧$\Rightarrow$完全有界
                \item 完备度量空间中,列紧$\Leftrightarrow $完全有界。
            \end{enumerate}
        \end{theorem}
        \begin{proof}
            \begin{enumerate}
                \item 假设$A$列紧不完全有界,即存在$\varepsilon_0>0$使得有限个
                    半径为$\varepsilon_0$的球不能覆盖$A$,按照以下方式选取出一列点:
                    \begin{align*}
                        x_1&\in A\\
                        x_2&\in A\backslash B(x_1,\varepsilon_0)\\
                        x_3&\in A\backslash \bigcup_{k=1}^2 B( x_k,\varepsilon_0 )\\
                        &\vdots\\
                        x_n&\in A\backslash \bigcup_{k=1}^n B( x_k,\varepsilon_0 )\\
                        &\vdots
                    \end{align*}
                    那么序列$\{x_n\}_{n=1}^\infty \subset A$使得
                    \begin{equation*}
                        x_n\notin \bigcup_{i=1}^{n-1}B(x_i,\varepsilon_0)
                    \end{equation*}
                    因此$d(x_n,x_m)\geqslant \varepsilon_0,\forall n\neq m$,说明$A$并不列紧,矛盾。
                \item 只需证明:$X$完备并且$A$完全有界$\Rightarrow A$列紧。设$\{x_n\}_{n=1}^\infty\subset A$,
                    对于$\varepsilon=1$,有$A$的有限$1$网$N_1=\{ y_1^{(1)},\cdots,y_{m_1}^{(1)} \}$,于是
                    \begin{equation*}
                        \{ x_n \}_{n=1}^\infty\subset A\subset \bigcup_{k=1}^{m_1} B( y_k^{(1)},1 )
                    \end{equation*}
                    因此,存在某个$k$使得$B(y_k^{(1)},1)$包含$\{ x_n \}_{n=1}^\infty$中的无穷多项,记作$\{ x_n^{(1)} \}_{n=1}^\infty$,
                    并记$y_k^{(1)}=y^{(1)}$.
                    同理,$\exists y^{(2)}\in N_{\frac{1}{2}}$使得有$\{ x_n^{(1)} \}_{n=1}^\infty$的子列
                    $\{x_n^{(2)}\}_{n=1}^\infty\subset B(y^{(2)},\frac{1}{2})$,以此类推:
                    \begin{equation*}
                        \begin{matrix}
                            x_1^{(1)}&,x_2^{(1)}&,x_3^{(1)}&,\cdots&\in B(y^{(1)},1)\\
                            x_1^{(2)}&,x_2^{(2)}&,x_3^{(2)}&,\cdots&\in B(y^{(2)},\frac{1}{2})\\
                            \vdots&\vdots&\vdots&&\\
                            x_1^{(n)}&,x_2^{(n)}&,x_3^{(n)}&,\cdots&\in B(y^{(n)},\frac{1}{n})\\
                            \vdots&\vdots&\vdots&&
                        \end{matrix}
                    \end{equation*}
                    而且
                    \begin{equation*}
                        \{ x_n^{(1)} \}_{n=1}^\infty\supset \{ x_n^{(2)} \}_{n=1}^\infty\supset\cdots\supset\{ x_n^{(n)} \}_{n=1}^\infty\supset \cdots
                    \end{equation*}
                    取对角线子列:
                    \begin{equation*}
                        x_n^{(n)}\in \bigcap_{k=1}^n B(y^{(k)},\frac{1}{k}),n=1,2,\cdots
                    \end{equation*}
                    所以$\forall n,p$,由于$x_{n+p}^{(n+p)},x_n^{(n)}\in B(y^{(n)},\frac{1}{n})$,因此
                    \begin{equation*}
                        d( x_{n+p}^{(n+p)},x_n^{(n)} )\leqslant \frac{2}{n}\rightarrow 0
                    \end{equation*}
                    所以$\{x_n^{(n)}\}$是基本列,由$X$完备可知$\{x_n^{(n)}\}$收敛,这就是$\{x_n\}_{n=1}^\infty$的收敛子列,于是$A$列紧。
            \end{enumerate}
        \end{proof}
        \begin{theorem}
            度量空间中,紧$\Leftrightarrow $自列紧。
        \end{theorem}
        \begin{proof}
            必要性:\begin{enumerate}
                \item 紧集是闭集。设$A$是紧集,希望证明$X\backslash A$是开集,任取
                    $x\in X\backslash A$,取开覆盖
                    \begin{equation*}
                        \bigcup_{y\in A} B( y,\frac{1}{3}d(x,y) )\supset A
                    \end{equation*}
                    存在子覆盖
                    \begin{equation*}
                        \bigcup_{k=1}^m B(y_k,\frac{1}{3}d(x,y_k))\supset A
                    \end{equation*}
                    令$\delta=\mathop{\rm min}\limits_{1\leqslant k\leqslant m}\frac{1}{3}d(x,y_k)$,
                    \begin{align*}
                        &\Rightarrow B(x,\delta)\cap \bigcup_{k=1}^m B( y_k,\frac{1}{3}d(x,y_k) )=\varnothing\\
                        &\Rightarrow B(x,\delta)\subset X\backslash A
                    \end{align*}
                    所以$x$是内点,进而$X\backslash A$是开集。
                \item 紧集是列紧集。假设$A$紧而不列紧,则存在$\{x_n\}_{n=1}^\infty\subset A$没有收敛子列,不妨假设$x_n$互不相同,令
                    \begin{equation*}
                        S_n=\{x_k\}_{k=1}^\infty \backslash {x_n}
                    \end{equation*}
                    则$S_n$是闭集\footnote{$S_n$没有聚点,也符合闭集定义。},
                    $X\backslash S_n$是开集。而
                    \begin{equation*}
                        \bigcup_{n=1}^\infty (X\backslash S_n)=X\backslash \left( \bigcap_{n=1}^\infty S_n \right)=X\supset A
                    \end{equation*}
                    这就是$A$的一个开覆盖,存在$N$使得
                    \begin{equation*}
                        \bigcup_{n=1}^N (X\backslash S_n)\supset A
                    \end{equation*}
                    但是
                    \begin{equation*}
                        \bigcup_{n=1}^N (X\backslash S_n)=
                        X\backslash \left( \bigcap_{n=1}^N S_n \right)
                        =X\backslash \{ x_n \}_{n=N+1}^\infty
                    \end{equation*}
                    不能是$A$的覆盖,矛盾。
            \end{enumerate}

            充分性:假设$A$自列紧但不紧,即存在$A$的一个开覆盖
            $\{G_\alpha\}_{\alpha\in\Lambda}$使得任取有限个$G_\alpha$都不能覆盖$A$,
            $A$自列紧,所以完全有界,对于$\forall n$,取有限$\frac{1}{n}$网:
            \begin{equation*}
                N_{\frac{1}{n}}=\{ y_1^{(n)},\cdots,y_{m_n}^{(n)} \},\ A\subset \bigcup_{k=1}^{m_n}B(y_k,\frac{1}{n})
            \end{equation*}
            那么,对于每个$n$,存在$y_k^{(n)}\in N_{\frac{1}{n}}$使得$B(y_k^{(n)},\frac{1}{n})$
            不能被有限个$G_\alpha$覆盖\footnote{否则,每个$B(y_k^{(n)},\frac{1}{n})$都能被有限覆盖,取这些有限覆盖的并就是$A$的有限覆盖,矛盾。},记$y_k^{(n)}=y^{(n)}$,
            得到点列$\{y^{(n)}\}_{n=1}^\infty$,
            因为$A$自列紧,所以有收敛子列$\{y^{(n_k)}\}_{k=1}^\infty$,并设其收敛到$y_0\in A$.
            
            设$y_0\in G_{\alpha_0}$,则存在$\delta>0$使得
            $B(y_0,\delta)\subset G_{\alpha_0}$,而
            $y^{(n_k)}\rightarrow y_0$,当$k$充分大时,
            $n_k>\frac{2}{\delta}$且$d( y^{(n_k)},y_0 )<\frac{\delta}{2}$,则$\forall y\in B(y^{(n_k)},\frac{1}{n_k})$,
            \begin{equation*}
                d( y,y_0 )\leqslant d(y_0,y^{(n_k)})+d( y,y^{(n_k)} )\leqslant \frac{2}{\delta}+\frac{1}{n_k}\leqslant \delta 
            \end{equation*}
            于是
            \begin{equation*}
                B(y_{n_k},\frac{1}{n_k})\subset B(y_0,\delta)\subset G_{\alpha_0}
            \end{equation*}
            和$B(y_{n_k},\frac{1}{n_k})$不能被有限个$G_\alpha$覆盖矛盾。
        \end{proof}
%        \begin{example}
%            Hilbert长方体:
%            \begin{equation*}
%                \{ x=(x_1,x_2,\cdots,x_n,\cdots)\in \ell^2:|x_n|\leqslant \frac{1}{2^n},n=1,2,\cdots \}
%            \end{equation*}
%            是$\ell^2$上的列紧集。
%
%            这里$\ell^p(1\leqslant p<\infty)$是数列空间,每个元素都是一个数列,度量为:
%            \begin{equation*}
%                d( \{x_n\},\{y_n\} )=\left(\sum_{k=1}^\infty |x_n-y_n|^p\right)^{\frac{1}{p}}
%            \end{equation*}
%        \end{example}
%        \begin{proof}
%            作业。
%        \end{proof}
        \begin{property}
            \begin{equation*}
                \mbox{有界闭}\mathop{\Leftrightarrow}\limits_{X=\mathbb{R}^n}^{} 
                \mbox{紧}\mathop{\Leftrightarrow}\limits_{}^{} 
                \mbox{自列紧}\mathop{\leftrightarrows}\limits_{}^{\mbox{闭}} 
                \mbox{列紧}\mathop{\leftrightarrows}\limits_{}^{X\mbox{完备}} 
                \mbox{完全有界}\mathop{\leftrightarrows}\limits_{}^{X=\mathbb{R}^n} 
                \mbox{有界}
            \end{equation*}
        \end{property}
        \begin{theorem}
            列紧空间可分。
        \end{theorem}
        \begin{proof}
            列紧$\Rightarrow $完全有界$\Rightarrow \forall n$,存在有限的
            $\frac{1}{n}$网$N_{\frac{1}{n}}$,然后对所有的$n$取并得到一个可数集:
            \begin{equation*}
                \bigcup_{n=1}^\infty N_{\frac{1}{n}}\mathop{\subset }\limits^{\rm dense} X
            \end{equation*}
            这是因为$\forall x\in X$,$\forall n$,存在$x_n\in N_{\frac{1}{n}}$使得$d(x_n,x)<\frac{1}{n}$,从而$x_n\rightarrow x$.
        \end{proof}
        \begin{proposition}
            $(M,\rho)$是紧度量空间,$C(M)$为$M$上的连续函数全体,定义
            \begin{equation*}
                d(f,g)\mathop{=}\limits^{\rm def}
                \mathop{\rm sup}\limits_{x\in M}|f(x)-g(x)|
            \end{equation*}
            则$d$是$C(M)$上的度量,且$(C(M),d)$完备。(作业)
        \end{proposition}
        \begin{proof}
            先证明$d$是$C(M)$上的度量:
    \begin{enumerate}[$(1)$]
        \item 唯一性:$d(f,g)=\fun{sup}{x\in M}|f(x)-g(x)|=0\Leftrightarrow f(x)=g(x){\rm\ on\ }M$.
        \item 非负性:绝对值非负,故$d(f,g)$非负。
        \item 对称性:$|f-g|=|g-f|$.
        \item 三角不等式:
            \begin{align*}
                d(f,g)+d(g,h)&=
                \fun{sup}{x\in M} |f(x)-g(x)|+\fun{sup}{x\in M} |g(x)-h(x)|\\
                &\geqslant \fun{sup}{x\in M} (|f(x)-g(x)|+|g(x)-h(x)|)\\
                &\geqslant \fun{sup}{x\in M} |f(x)-h(x)|=d(f,h)
            \end{align*}
    \end{enumerate}

    再证明$(C(M),d)$完备:任取$C(M)$上的柯西列$\{ f_n \}_{n=1}^\infty$,即
    \begin{equation*}
        \forall \varepsilon>0,\exists N{\rm\ s.t.\ }\forall m,n\geqslant N,d(f_n,f_m)=\fun{sup}{x\in M}|f_n(x)-f_m(x)|<\varepsilon
    \end{equation*}
    则固定$x\in M$,$\{f_n(x)\}_{n=1}^\infty$是$\R$上的柯西列,
    进而收敛,设其收敛到$f_0(x)$,于是
    \begin{equation*}
        \forall x\in M,\forall \varepsilon>0,\exists N>0{\rm\ s.t.\ }
        n\geqslant N\Rightarrow \fun{sup}{x\in M}|f_n(x)-f_0(x)|<\varepsilon\tag*{(1)}
    \end{equation*}
    $f_n$连续,所以对于$\varepsilon$,
    \begin{equation*}
        \exists \delta{\rm\ s.t.\ }\forall y\in B(x,\delta),|f_n(x)-f_n(y)|<\varepsilon\tag*{(2)}
    \end{equation*}
    则
    \begin{align*}
        |f_0(x)-f_0(y)|&\leqslant \mathop{|f_0(x)-f_n(x)|}\limits_{(1)}+
        \mathop{|f_n(x)-f_n(y)|}\limits_{(2)}+
        \mathop{|f_n(y)-f_0(x)|}\limits_{(1)}\\
        &\leqslant \varepsilon+\varepsilon+\varepsilon=3\varepsilon\rightarrow 0
    \end{align*}
    所以$f_0$连续,即$f_0\in C(M)$,于是$(C(M),d)$完备。
        \end{proof}
        \begin{definition}
            $(M,\rho)$是紧度量空间,$C(M)$为$M$上的连续函数全体,称$C(M)$中的一族函数$\mathcal{F}$等度连续是指:$\forall \varepsilon>0$,存在$\delta>0$使得
            \begin{equation*}
                |\varphi(x')-\varphi(x'')|<\varepsilon,\forall x',x''\in M{\rm\ with\ }\rho(x',x'')<\delta,\forall \varphi\in\mathcal{F}
            \end{equation*}
        \end{definition}
        \begin{theorem}[Argela-Ascoli]
            $\mathcal{F}$列紧当且仅当$\mathcal{F}$作为函数族一致有界、等度连续。
        \end{theorem}
        \begin{proof}
            必要性:$\mathcal{F}$列紧$\Rightarrow $完全有界$\Rightarrow $有界,
            \begin{equation*}
                d(f,0)\leqslant R,\forall f\in \mathcal{F}\Rightarrow 
                \mathop{\rm sup}\limits_{x\in M}|f(x)|\leqslant R,\forall f\in\mathcal{F}
            \end{equation*}
            $\Rightarrow $一致有界。下证等度连续:$\forall \varepsilon>0$,存在
            $N_{\frac{\varepsilon}{3}}=\{ \varphi_{1},\cdots,\varphi{m} \}$使得
            \begin{equation*}
                \bigcup_{k=1}^m B(\varphi_k,\frac{\varepsilon}{3})\supset \mathcal{F}\tag*{(1)}
            \end{equation*}
            因为$\varphi_k$一致连续,对每个$k$,存在$\delta_k>0$使得
            \begin{equation*}
                |\varphi_k(x')-\varphi_k(x'')|<\frac{\varepsilon}{3},\forall x',x''\in M{\rm\ with\ }\rho(x',x'')<\delta_k
            \end{equation*}
            令$\delta={\rm min}\{\delta_1,\cdots,\delta_m\}$,则
            \begin{equation*}
                |\varphi_k(x')-\varphi_k(x'')|<\frac{\varepsilon}{3},\forall x',x''\in M{\rm\ with\ }\rho(x',x'')<\delta,\forall k\tag*{(2)}
            \end{equation*}
            由(1),
            \begin{equation*}
                \forall \varphi\in\mathcal{F},\exists k{\rm\ s.t.\ }d(\varphi,\varphi_k)<\frac{\varepsilon}{3}
            \end{equation*}
            于是$\forall x,|\varphi(x)-\varphi_k(x)|\leqslant
            \mathop{\rm sup}\limits_{x\in M}|\varphi(x)-\varphi_k(x)|=d(\varphi,\varphi_k)\leqslant \frac{\varepsilon}{3}$.
            而当$\rho(x',x'')<\delta$时,由(2),$|\varphi_k(x')-\varphi_k(x'')|<\frac{\varepsilon}{3}$,所以
            \begin{equation*} 
                |\varphi(x')-\varphi(x'')|\leqslant 
                |\varphi(x')-\varphi_k(x')|+|\varphi_k(x')-\varphi_k(x'')|+|\varphi_k(x'')-\varphi(x'')|<\varepsilon
            \end{equation*}

            充分性:$\mathcal{F}$等度连续,$\forall \varepsilon>0$,存在$\delta>0$,使得
            \begin{equation*}
                |\varphi(x')-\varphi(x'')|<\varepsilon,
                \forall x',x''\in M{\rm\ with\ }\rho(x',x'')<\delta,\forall \varphi\in\mathcal{F}
            \end{equation*}
            $M$紧,所以有有限$\delta$网$N_\delta=\{x_1,\cdots,x_n\}$,定义映射:
            \begin{equation*}
                T:\mathcal{F}\rightarrow\mathbb{R}^n,\varphi\mapsto ( \varphi(x_1),\cdots,\varphi(x_n) )
            \end{equation*}
            $\mathcal{F}$一致有界,所以可令
            \begin{equation*}
                R\mathop{=}\limits^{\rm def}
                \mathop{\rm sup}\limits_{\varphi\in\mathcal{F}}
                \mathop{\rm sup}\limits_{x\in M}|\varphi(x)|<\infty
            \end{equation*}
            则
            \begin{equation*}
                \left[ \sum_{i=1}^n |\varphi(x_i)|^2 \right]^{\frac{1}{2}}\leqslant \sqrt{n}R,\forall \varphi\in\mathcal{F}
            \end{equation*}
            所以$T(\mathcal{F})$是$\mathbb{R}^n$中有界集,故列紧,设
            $T(\mathcal{F})$的有限$\frac{\varepsilon}{3}$网为
            \begin{equation*}
                M_{\frac{\varepsilon}{3}}=\{ T(\varphi_1),\cdots,T(\varphi_m) \}
            \end{equation*}
            Claim:$\{\varphi_1,\cdots,\varphi_m\}$是$\mathcal{F}$的$\varepsilon$网。
            \begin{equation*}
                \forall \varphi\in\mathcal{F},\exists k{\rm\ s.t.\ }d_{\mathbb{R}^n}(T(\varphi),T(\varphi_m))<\frac{\varepsilon}{3}
            \end{equation*}
            于是$|\varphi(x_i)-\varphi_k(x_i)|\leqslant d_{\mathbb{R}^n}(T(\varphi),T(\varphi_m))<\frac{\varepsilon}{3}$.
            同时,$\forall x\in M,\exists x_i\in N_\delta{\rm\ s.t.\ }\rho(x_i,x)<\delta$,由(1.3.3),
            \begin{equation*}
                |\varphi(x)-\varphi(x_i)|,|\varphi_k(x_i)-\varphi_k(x)|<\frac{\varepsilon}{3}
            \end{equation*}
            所以
            \begin{equation*}
                |\varphi(x)-\varphi_k(x)|\leqslant |\varphi(x)-\varphi(x_i)|+|\varphi(x_i)-\varphi_k(x_i)|+|\varphi_k(x_i)-\varphi_k(x)|<\varepsilon
            \end{equation*}
            所以$d(\varphi,\varphi_k)\leqslant\varepsilon$.
        \end{proof}
            
            $L^p$空间中列紧集是怎样的?
        \begin{theorem}[Riesz-Frechet-kolmogorov]
            设$1\leqslant p<\infty$,$\mathcal{F}\subset L^p(\mathbb{R}^n)$列紧当且仅当:
            \begin{enumerate}
                \item $\mathcal{F}$有界,即$\mathop{\rm sup}\limits_{f\in\mathcal{F}}||f||_p<\infty$.
                \item $\forall \varepsilon>0$,$\exists R>0{\rm\ s.t.}$
                    \begin{equation*}
                        \int_{|x|>R}|f(x)|^p{\rm d}x<\varepsilon^p,\forall f\in\mathcal{F}
                    \end{equation*}
                \item $\forall \varepsilon$,$\exists\delta>0{\rm\ s.t.}$
                    \begin{equation*}
                        ||\tau_h f-f||_p<\varepsilon,\forall h\in\mathbb{R}^n{\rm\ with\ }|h|<\delta,\forall f\in\mathcal{F}
                    \end{equation*}
                    其中$(\tau_h f)(x)=f(x+h)$.
            \end{enumerate}
        \end{theorem}
        \begin{example}
            $A$是$\ell^2$上的列紧集$\Leftrightarrow $
            \begin{enumerate}
                \item $A$有界。
                \item $\forall \varepsilon>0$,$\exists N$使得\begin{equation*}
                    \sum_{k=N+1}^\infty |x_k|^2<\varepsilon,\forall x=(x_1,x_2,\cdots)\in A
                \end{equation*}
            \end{enumerate}
            (作业)
        \end{example}
        \begin{proof}
            $(\Rightarrow)$:假设$A$无界,则能取发散点列$\{a_n\}_{n=1}^\infty$满足
    $d(a_n,0)\rightarrow\infty $,与$A$列紧矛盾;
    $A$列紧则完全有界,$\forall\varepsilon>0$,
    取$A$的有限$\varepsilon/2$网$\{a^{(i)}\}_{i=1}^n$,
    其中每个$a^{(i)}=(a^{(i)}_1,\cdots,a^{(i)}_j,\cdots)$,
    根据定义有:
    \begin{equation*}
        \sum_{j=1}^\infty |a_j^{(i)}|^2<+\infty,\ i=1,2,\cdots,n
    \end{equation*}
    所以
    \begin{equation*}
        \exists N_i{\rm\ s.t.\ } 
        \sum_{j=N_i+1}^\infty |a_j^{(i)}|^2<\varepsilon/2,
        \ i=1,2,\cdots,n
    \end{equation*}
    取$N=\fun{max}{i=1,2,\cdots,n}N_i$,因为$\{a^{(i)}\}_{i=1}^n$是$\varepsilon$网,
    \begin{equation*}
        \forall x\in A,\exists a^{(i)}{\rm\ s.t.\ }d(x,a^{(i)})=\sum_{j=1}^\infty | a_j^{i}-x_i |^2<\varepsilon/2
    \end{equation*}
    所以
    \begin{equation*}
        \sum_{i=N+1}^\infty |x_i|^2\leqslant \sum_{i=N+1}^\infty 
        ( |a_i^{(k)}|^2+|x_i-a_i^{(k)}|^2 )\leqslant \varepsilon
    \end{equation*}

    $(\Leftarrow)$:$\forall \varepsilon>0$,存在$N$使得
    \begin{equation*}
        \sum_{i=N+1}^\infty |x_i|^2\leqslant \varepsilon,\forall x\in A
    \end{equation*}
    取连续映射$\varphi:A\rightarrow\R^N,x\mapsto (x_1,\cdots,x_N)$,
    $A$有界$\Rightarrow \varphi(A)$有界$\Rightarrow \varphi(A)$完全有界,
    对于$\delta=\frac{\varepsilon^2}{2}$,存在
    $\varphi(A)$上的有限$\delta$网$\{ \varphi(x^{(i)}) \}_{i=1}^n$,
    记$X=\{ x^{(i)} \}_{i=1}^n\subset A$.
    \begin{equation*}
        \forall y\in A,\exists x^{(i)}\in X{\rm\ s.t.\ }
        d_{\R^N}(\varphi(y),\varphi(x^{(i)}))<\frac{\varepsilon^2}{2}
    \end{equation*}
    \begin{align*}
        \Rightarrow d(y,x^{(i)})^2&=
        \sum_{j=1}^N |y_j-x_j^{(i)}|^2+\sum_{j=N+1}^\infty |y_j-x_j^{(i)}|^2\\
        &<d_{\R^N}(\varphi(y),\varphi(x^{(k)}))
        +2\sum_{j=N+1}^\infty ( |y_j|^2+|x_j^{(i)}|^2 )<\varepsilon^2
    \end{align*}
    于是$X$为$A$的有限$\varepsilon$网,$A$完全有界$\Rightarrow A$列紧。
        \end{proof}
    
\section{赋范线性空间}
        \subsection{Banach空间}
        \begin{definition}
            $X$是非空集合,$\mathbb{K}$表示$\mathbb{C}$或者$\mathbb{R}$,如果能在$X$上定义两种封闭的运算:
            \begin{enumerate}
                \item 加法:$X\times X\rightarrow X,(x,y)\mapsto x+y$.满足:
                    \begin{enumerate}[(i)]
                        \item 结合律
                        \item 交换律
                        \item 零元
                        \item 负元
                    \end{enumerate}
                \item 乘法:$\mathbb{K}\times X\rightarrow X,(\lambda,x)\mapsto \lambda x$.满足:
                    \begin{enumerate}
                        \item[(v)] $1x=x$.
                        \item[(vi)] $\alpha(\beta x)=(\alpha\beta)x$.
                        \item[(vii)] $(\alpha+\beta)x=\alpha x+\beta y$.
                        \item[(viii)] $\alpha(x+y)=\alpha x+\alpha y$.
                    \end{enumerate}
            \end{enumerate}
            则称$X$是$\mathbb{K}$上的向量空间,线性空间。$X$中的元素称为向量。

            如果向量空间$X$的子集$Y$,如果对同一数域$\mathbb{K}$上的加法和乘法
            构成向量空间,则称之为$X$的向量子空间,也等价于$Y$关于加法和乘法封闭。
        \end{definition}

        约定一些记号:
            \begin{align*}
                x+E&:=\{ x+y:y\in E \}\\
                \lambda E&:= \{ \lambda y:y\in E \}\\
                E+F&:=\{ x+y:x\in E,y\in F \}
            \end{align*}
            \begin{equation*}
                {\rm span}(E):= \left\{
                    \sum_{k=1}^n \lambda_k x_k,x_k\in E,\lambda_k\in\mathbb{K},n\in \mathbb{N}
                \right\}
            \end{equation*}
            称为$E$张成的子空间。

            如果$E$线性无关且${\rm span}(E)=X$,则称$E$是$X$的Hamel基(代数基,线性基)。
        
        \begin{theorem}
            任一向量空间一定有Hamel基。
        \end{theorem}

        如果Hamel基是有限集,则定义${\rm dim\ }X=\# E$,否则记${\rm dim\ }X=\infty$.
        \begin{definition}
            $\mathbb{K}$是$\mathbb{C}$或者$\mathbb{R}$,$X$是$\mathbb{K}$上的向量空间,
            如果函数$||\cdot ||:X\rightarrow\mathbb{R}$满足:
            \begin{enumerate}
                \item 正定性
                \item 齐次性
                \item 三角不等式
            \end{enumerate}
            则称之为$X$上的一个范数。$(X,||\cdot||)$称为一个赋范空间。定义
            \begin{equation*}
                d(x,y)\mathop{=}\limits^{\rm def}||x-y||
            \end{equation*}
            称为范数诱导的度量,也叫典则度量。

            如果$(X,||\cdot ||)$在此度量下完备,则称之为Banach空间。
        \end{definition}
        \begin{example}
            Banach空间的一些例子:
            
            函数空间$L^p,L^\infty,C(M)$;
            数列空间$\ell^p,\ell^\infty$(有界数列空间,$||x||_\infty \mathop{=}\limits^{\rm def} \mathop{\rm sup}\limits_{k\geqslant 1}|x_k|$),
            $C$(收敛数列空间),$C_0$(收敛到零的数列全体).
        \end{example}
        \begin{example}
            $\Omega$是$\mathbb{R}^n$中的有界域,
            $C^k(\overline{\Omega})$是$\overline{\Omega}$上$k$次连续可微的函数全体,定义
            \begin{equation*}
                ||u||_{k,p}\mathop{=}\limits^{\rm def}
                \left( \sum_{|\alpha|\leqslant k}\int_\Omega |\partial^\alpha u|^p \right)^{\frac{1}{p}}
            \end{equation*}
            这是$C^k(\overline{\Omega})$上的一个范数,
            \begin{equation*}
                S\mathop{=}\limits^{\rm def}
                \left\{ u\in C^k(\overline{\Omega}):||u||_{k,p}<\infty\right\}
            \end{equation*}
            的完备化称为Sobolev空间,记作$H^{k,p}(\Omega)$.
        \end{example}
        \subsection{范数等价}
        \begin{definition}
            $X$是向量空间,$||\cdot||_1$和$||\cdot||_2$是$X$上的两个范数,称
            $||\cdot||_2$强于$||\cdot||_1$是指:$\forall \{x_n\}_{n=1}^\infty \subset X$,
            \begin{equation*}
                ||x_n||_2\rightarrow 0\Rightarrow ||x_n||_1\rightarrow 0
            \end{equation*}
            记作$||\cdot||_1\lesssim ||\cdot||_2$.如果$||\cdot||_2$强于$||\cdot||_1$,同时$||\cdot||_1$强于$||\cdot||_2$,则称二者是等价范数。
        \end{definition}
        \begin{proposition}
            $||\cdot||_2$强于$||\cdot||_1\Leftrightarrow \exists C>0$
            使得$||x||_1\leqslant C||x||_2,\forall x\in X$.
        \end{proposition}
        \begin{proof}
            充分性显然,下证必要性:假设不然,则$\forall n$,存在$x_n\in X$使得
            $||x_n||_1\geqslant m||x_n||_2$,令$y_n=\frac{x_n}{||x_n||_1}$,则
            \begin{equation*}
                ||y_n||_2\leqslant \frac{1}{n}\rightarrow 0{\rm\ as\ }n\rightarrow\infty
            \end{equation*}
            $||\cdot||_2$强于$||\cdot||_1$,所以$||y_n||_1\rightarrow 0$,
            但是$||y_n||_1$恒等于$1$,矛盾。
        \end{proof}
        \begin{corollary}
            $||\cdot||_1$和$||\cdot||_2$等价,当且仅当存在
            $C_1,C_2>0$,使得 
            \begin{equation*}
                C_1||x||_1\leqslant ||x||_2\leqslant C_2||x||_1,\forall x\in X
            \end{equation*}
        \end{corollary}
        \begin{example}
            $\mathbb{R}^n$上$||\cdot ||_p(1\leqslant p\leqslant \infty)$彼此等价,因为
            \begin{equation*}
                ||x||_\infty\leqslant ||x||_p \leqslant n^{\frac{1}{p}}||x||_\infty
            \end{equation*}
        \end{example}
        \begin{theorem}
            有限维空间上所有范数都等价。
        \end{theorem}
        \begin{proof}
            设${\rm dim\ }X=n$,$\{e_1,\cdots,e_n\}$是一组基,
            $\forall x\in X$有唯一表示
            \begin{equation*}
                x=\sum_{i=1}^n \xi_i e_i,\xi_i\in\mathbb{K}
            \end{equation*}
            定义映射:$T\rightarrow \mathbb{K}^n,x\mapsto (\xi_1,\xi_2,\cdots,\xi_n)$,
            则$T$是$X$到$\mathbb{K}$的代数同构,设
            \begin{equation*}
                |\xi|=\left( \sum_{i=1}^n |\xi_i|^2 \right)^{\frac{1}{2}},\xi\in\mathbb{K}^n
            \end{equation*}
            令$||x||_T=|T(x)|$,则$||\cdot||_T$是$X$上的范数。任取一个$X$上的范数$||\cdot||$,下面证明
            $||\cdot||_T$和$||\cdot||$等价。

            定义函数
            \begin{equation*}
                p:\mathbb{K}^n\rightarrow \mathbb{R},\xi\mapsto || \sum_{i=1}^n \xi_i e_i ||
            \end{equation*}
            \begin{enumerate}
                \item $p(\xi)=|\xi|\cdot p(\frac{\xi}{|\xi|}),\forall \xi\neq 0$.
                \item $p$在$\mathbb{K}$上连续:\begin{align*}
                    |p(a)-p(b)|&=\left| ||\sum_{i=1}^n a_i e_i||-||\sum_{i=1}^n b_i e_i|| \right|\\
                    &\leqslant || \sum_{i=1}^n (a_i-b_i) e_i ||\\
                    &\leqslant \sum_{i=1}^n |a_i-b_i| e_i\\
                    &\leqslant \left( \sum_{i=1}^n ||e_i||^2 \right)^{\frac{1}{2}}|a-b|
                \end{align*}
                最后一步是Caurhy-Schwarz不等式。
            \end{enumerate}
            令$S_1=\{ \xi\in\mathbb{K}^n:|\xi|=1 \}$,则$S_1$是紧集,故$p$在$S_1$上存在最小值和最大值,分别记作
            $C_1$和$C_2$,从而
            \begin{align*}
                &\Rightarrow C_1\leqslant p(\frac{\xi}{|\xi|})\leqslant C_2,\forall \xi\neq 0\\
                &\Rightarrow C_1|\xi|\leqslant p(\xi)\leqslant C_2|\xi|,\forall \xi\\
                &\Rightarrow C_1|T(x)|\leqslant p(T(x))\leqslant C_2|T(x)|,\forall x\in X\\
                &\Leftrightarrow C_1||x||_T\leqslant ||x||\leqslant C_2||x||_T
            \end{align*}
            只需证明$C_1>0$,假设$C_1=0$,则存在$\xi^*\in S_1$使得
            \begin{equation*}
                ||\sum_{i=1}^n \xi_i^* e_i||=p(\xi^*)=0 \Rightarrow \sum_{i=1}^n \xi_i^* e_i=0\Rightarrow \xi^*=0
            \end{equation*}
            这与$\xi^*\in S_1$矛盾。
        \end{proof}
        \begin{corollary}
            同维数的两个有限维赋范空间(作为向量空间)代数同构且
            (作为拓扑空间)同胚。
        \end{corollary}
        \begin{proof}
            \begin{equation*}
                T:X\rightarrow \mathbb{K}^n,x=\sum_{i=1}^n \xi_i e_i\mapsto \xi
            \end{equation*}
            是一个代数同构,满足
            \begin{equation*}
                C_1|T(x)|\leqslant ||x||\leqslant C_2|T(x)|,\forall x\in X
            \end{equation*}
            第一个不等号得到$T$连续,第二个不等号得到$T^{-1}$连续,故$T$也是同胚。
        \end{proof}
        \begin{corollary}
            有限维赋范空间一定是Banach空间。
        \end{corollary}
        \begin{proof}
            $C_1|T(x)|\leqslant ||x||\leqslant C_2|T(x)|$,设$\{x_n\}_{n=1}^\infty$是$X$中基本列,
            则$\{ T(x_n) \}_{n=1}^\infty$是$\mathbb{K}^n$中基本列,设$T(x_k)\rightarrow \xi$,
            \begin{equation*}
                ||x_n-T^{-1}(\xi)||\leqslant C_2|T(x_n)-\xi|\rightarrow 0
            \end{equation*}
        \end{proof}
        \begin{corollary}
            任何赋范空间的有限维子空间一定是闭子空间。(作业)
        \end{corollary}
        \begin{proof}
            任何赋范空间的有限维子空间是有限维赋范空间,所以是Banach空间,
            所有的收敛列都是柯西列,所有的柯西列都收敛到子空间内某点,所以是闭子空间。
        \end{proof}
        \begin{theorem}
            赋范空间$(X,||\cdot||)$,$X$中单位球面列紧
            $\Leftrightarrow {\rm dim\ }X<\infty$.
        \end{theorem}
        \begin{lemma}[Riesz]
            赋范空间$(X,||\cdot||)$,$(Y,||\cdot||)$是$X$的闭子空间,$Y\neq X$,则
            $\forall \varepsilon>0$,$\exists e\in X{\rm\ with\ }||e||=1$,使得
            \begin{equation*}
                {\rm dist}(e,Y):=\fun{inf}{y\in Y} ||e-y|| \geqslant 1-\varepsilon
            \end{equation*}
            \begin{proof}
                取$x\in X\backslash Y$,令
                \begin{equation*}
                    d_i={\rm dist}(x,y)=\mathop{\rm inf}\limits_{y\in Y}||x-y||
                \end{equation*}
                $Y$是闭集,$d>0$,且存在$y_0\in Y$使得
                \begin{equation*}
                    d\leqslant ||x-y_0||\leqslant \frac{d}{1-\varepsilon}
                \end{equation*}
                令
                \begin{equation*}
                    e\mathop{=}\limits^{\rm def}
                    \frac{x-y_0}{||x-y_0||}\Rightarrow ||e||=1,e\notin Y
                \end{equation*}
                $\forall \zeta\in Y$,
                \begin{align*}
                    ||e-\zeta||&=|| \frac{x-y_0}{||x-y_0||}-\zeta ||\\
                    &=\frac{1}{||x-y_0||}\cdot ||x-\mathop{( y_0+||x-y_0||\zeta )}\limits_{\mbox{这一堆}\in Y}||\\
                    &\geqslant \frac{1-\varepsilon}{d}\cdot d=1-\varepsilon
                \end{align*}
                所以${\rm dist}(e,Y)\geqslant 1-\varepsilon$.
            \end{proof}
        \end{lemma}
        \begin{proof}
            充分性:
            \begin{equation*}
                T:X\rightarrow \mathbb{K}^n,x=\sum \xi_i e_i\mapsto \xi
            \end{equation*}
            满足
            \begin{equation*}
                C_1|T(x)|\leqslant ||x||\leqslant C_2|T(x)|
            \end{equation*}
            于是$T(S_1)$有界,故列紧,
            \begin{align*}
                \Rightarrow& \forall \{x_n\}_{n=1}^\infty\subset S_1,\exists T(x_{n_k})\rightarrow y\in\mathbb{K}^n\\
                \Rightarrow& x_{n_k}\rightarrow T^{-1}(y)\in X
            \end{align*}
            必要性:假设${\rm dim\ }X=\infty$,则存在一列向量$\{e_n\}_{n=1}^\infty $线性无关,令子空间
            \begin{equation*}
                X_n\mathop{=}\limits^{\rm def} \left< e_1,e_2,\cdots,e_n \right>,n=1,2,\cdots
            \end{equation*}
            于是$X_n\subsetneqq X_{n+1}$且是闭子空间,由引理,取$\varepsilon=\frac{1}{2}$,
            $\forall n\in\mathbb{N}$,存在$x_n\in X_n{\rm\ with\ }||x_n||=1$,使得
            \begin{equation*}
                {\rm dist}(x_n,X_{n-1})\geqslant \frac{1}{2}
            \end{equation*}
            \begin{align*}
                \Rightarrow & ||x_n-x_m||\geqslant \frac{1}{2},\forall n\neq m\\
                \Rightarrow & \{ x_n \}_{n=1}^\infty\mbox{没有收敛子列}
            \end{align*}
            这与$S_1$列紧矛盾。
        \end{proof}
        \subsection{商空间}
        \begin{definition}[商空间]
            赋范空间$(X,||\cdot||)$,$(X_0,||\cdot||)$是$X$的闭子空间。
            在$X$中定义
            \begin{equation*}
                x\sim y\mathop{\Leftrightarrow}\limits^{\rm def} x-y\in X_0
            \end{equation*}
            $[x]\mathop{=}\limits^{\rm def} x$所在的等价类,
            \begin{equation*}
                X/X_0\mathop{=}\limits^{\rm def}\{ [x]:x\in X \}
            \end{equation*}
            定义运算:
            \begin{equation*}
                [x]+[y]=[x+y],\lambda [x]=[\lambda x]
            \end{equation*}
            则$X/X_0$成为向量空间。并定义:
            \begin{equation*}
                ||[x]||_*\mathop{=}\limits^{\rm def} \mathop{\rm inf}\limits_{y\in [x]}||y||
            \end{equation*}
        \end{definition}
        \begin{remark}
            注意这里必须要求$X_0$是闭子空间。
        \end{remark}
        \begin{theorem}
            $||\cdot ||_*$是$X/X_0$上的范数。
        \end{theorem}
        \begin{proof}
            \begin{enumerate}
                \item 正定性:$\forall y\in X,||y||\geqslant 0\Rightarrow ||[x]||_*\geqslant 0$,
                    \begin{align*}
                        ||[x]||_*=0 &\Rightarrow \exists \{x_n\}_{n=1}^\infty \subset [x]{\rm\ s.t.\ }||x_n||\rightarrow 0\\
                        &\Rightarrow x_n\rightarrow 0
                    \end{align*}
                    $[x]=x+X_0$是闭集,所以$0\in [x]$,进而可得$[x]=[0]$.
                \item 齐次性:显然。
                \item 三角不等式:由下确界的定义,$\forall \varepsilon$,
                    \begin{align*}
                        \exists x'\in [x]{\rm\ s.t.\ }||x'||<||[x]||_*+\frac{\varepsilon}{2}\\
                        \exists y'\in [y]{\rm\ s.t.\ }||y'||<||[y]||_*+\frac{\varepsilon}{2}
                    \end{align*}
                    \begin{align*}
                        \Rightarrow & ||x'+y'||\leqslant ||x'||+||y'||\leqslant 
                        ||[x]||_*+||[y]||_*+\varepsilon\\
                        \mathop{\Rightarrow}\limits^{x'+y'\in [x+y]}&
                        ||[x+y]||_*\leqslant ||[x]||_*+||[y]||_*+\varepsilon \\
                        \mathop{\Rightarrow}\limits^{\varepsilon\rightarrow 0}& 
                        ||[x+y]||_*\leqslant ||[x]||_*+||[y]||_*
                    \end{align*}
            \end{enumerate}
        \end{proof}
        \begin{theorem}
            $(X,||\cdot||)$完备,则$(X/X_0,||\cdot||_*)$也完备。
        \end{theorem}
        \begin{lemma}
            $X$完备$\Leftrightarrow \forall \{x_n\}_{n=1}^\infty \subset X$,
            \begin{equation*}
                \sum_{n=1}^\infty ||x_n||<\infty \Rightarrow \sum_{n=1}^\infty x_n\mbox{收敛}
            \end{equation*}
            \begin{proof}
                习题1.4.7.
            \end{proof}
        \end{lemma}
        \begin{proof}
            由引理,只需证明$X/X_0$中任一绝对收敛级数都收敛。设
            \begin{equation*}
                \sum_{n=1}^\infty ||[x_n]||_*<\infty
            \end{equation*}
            对每个$n$,
            \begin{align*}
                &\exists y_n\in X_0{\rm\ s.t.\ }||x_n+y_n||\leqslant ||[x_n]||_*+\frac{1}{2^n}\\
                \Rightarrow &
                \sum_{n=1}^\infty ||x_n+y_n||\leqslant \sum_{n=1}^\infty ||[x_n]||_*+1<\infty\\
                \mathop{\Leftrightarrow}\limits^{X\mbox{完备}}&
                \exists x\in X{\rm\ s.t.\ }
                ||\sum_{k=1}^n (x_k+y_k)-x||\rightarrow 0{\rm\ as\ }n\rightarrow\infty\\
                \Rightarrow &
                ||\sum_{k=1}^n [x_j]-[x]||_*
                \mathop{=}\limits^{y_k\in X_0}
                ||[ \sum_{k=1}^n (x_k+y_k)-x ]||_*
                \leqslant ||\sum_{k=1}^n (x_k+y_k)-x||\rightarrow 0{\rm\ as\ }n\rightarrow\infty
            \end{align*}
        \end{proof}
    
\section{内积空间}
        \subsection{Hilbert空间}
        \begin{definition}
            $X$是$\mathbb{K}$上的向量空间,如果函数
            \begin{equation*}
                \left< \cdot,\cdot \right>:X\times X\rightarrow \mathbb{K}
            \end{equation*}
            满足:
            \begin{enumerate}
                \item 对第一变元线性:$\left< \alpha x_1+\beta x_2,y \right>=\alpha
                \left<x_1,y\right>+\beta\left<x_2,y\right>$.
                \item 对第二变元共轭线性:
                $\left< x,\alpha y_1+\beta y_2 \right>=\overline{\alpha}
                \left<x,y_1\right>+\overline{\beta}\left<x,y_2\right>$.
                \item 共轭对称:$\overline{\left<x,y\right>}=\left<y,x\right>$.
                \item $\left<x,x\right>\geqslant 0,\forall x\in X$.等号成立当且仅当$x=0$.
            \end{enumerate}
            则称$\left< \cdot,\cdot \right>$是$X$上的一个内积,
            $(X,\left< \cdot,\cdot \right>)$称为内积空间。
        \end{definition}
        \begin{lemma}[Cauthy-Schwarz]
            $(X,\left< \cdot,\cdot \right>)$是内积空间,令
            \begin{equation*}
                ||x||\mathop{=}\limits^{\rm def}\sqrt{ \left<x,x\right> },x\in X
            \end{equation*}
            则$|\left<x,y\right>|\leqslant ||x||||y||,\forall x,y\in X$,等号成立当且仅当
            存在$\lambda\in\mathbb{K}$,使得$x=\lambda y$.
            \begin{proof}
                不妨设$y\neq 0$,则$\forall \lambda\in\mathbb{K}$,
                \begin{align*}
                    0&\leqslant \left< x+\lambda y,x+\lambda y \right>\\
                    &=\left<x,x\right>+\lambda \left<y,x\right>+\overline{\lambda}\left<x,y\right>+
                    |\lambda|^2\left<y,y\right>\\
                    &=||x||^2+2{\rm Re}\{ \overline{\lambda}\left<x,y\right> \}+|\lambda|^2 ||y||^2
                \end{align*}
                这里取$\lambda=-\frac{\left<x,y\right>}{||y||^2}$,
                \begin{equation*}
                    0\leqslant ||x||^2-2{\rm Re}\{ \frac{ |\left<x,y\right>|^2 }{||y||^2} \}+
                    \frac{ |\left<x,y\right>|^2 }{||y||^4}\cdot ||y||^2
                    =||x||^2-\frac{ |\left<x,y\right>|^2 }{||y||^2}
                \end{equation*}
                于是得证。
            \end{proof}
        \end{lemma}
        \begin{proposition}
            Cauthy-Schwarz引理中的$||x||$是一个范数。
        \end{proposition}
        \begin{proof}
            只需验证三角不等式:
            \begin{align*}
                ||x+y||^2&=||x||^2+2{\rm Re}\left<x,y\right>+||y||^2\\
                &\leqslant ||x||^2+2||x||||y||+||y||^2
            \end{align*}
        \end{proof}
        \begin{definition}
            如果一个内积空间在其内积诱导范数下是一个Banach空间,则称之为Hilbert空间。
        \end{definition}
        \begin{example}
            $\ell^2$是一个Hilbert空间。(作业)
        \end{example}
        \begin{proof}
            只需证明$\ell^2$完备。任取$\ell^2$上基本列$\{x^{(n)}\}_{n=1}^\infty$,
            \begin{equation*}
                \forall \varepsilon>0,\exists N{\rm\ s.t.\ }
                ||x^{(n)}-x^{(m)}||_2<\varepsilon,\forall n,m\geqslant N
            \end{equation*}
            即
            \begin{align*}
                &\sum_{k=1}^\infty |x_k^{(n)}-x_k^{(m)}|^2<\varepsilon,\forall n,m\geqslant N\tag*{(1)}\\
                \Rightarrow& \forall \mbox{固定}k,|x_k^{(n)}-x_k^{(m)}|<\varepsilon,\forall n,m\geqslant N\\
                \Rightarrow& \{x_k^{n}\}_{n=1}^\infty \mbox{是$\R$中基本列}\\
                \Rightarrow& \exists x_k\in\R{\rm\ s.t.\ }x_k^{(n)}\rightarrow x_k{\rm\ as\ }n\rightarrow\infty
            \end{align*}
            令$x\defeq \{x_k\}_{k=1}^\infty$,Claim:$x\in \ell^2$且$||x^{(n)}\rightarrow x||_2\rightarrow 0$.
            由(1),$\forall p\in \N$
            \begin{align*}
                &\sum_{k=1}^p |x_k^{(n)}-x_k^{(m)}|^2<\varepsilon^2,\forall n,m\geqslant N\\
                \mathop{\Rightarrow}\limits^{m\rightarrow\infty}&\forall p,
                \sum_{k=1}^p |x_k^{(n)}-x_k|^2\leqslant \varepsilon^2,\forall n\geqslant N\\
                \mathop{\Rightarrow}\limits^{p\rightarrow\infty}&
                \sum_{k=1}^\infty |x_k^{(n)}-x_k|^2\leqslant \varepsilon^2,\forall n\geqslant N\tag*{(2)}\\
                \Rightarrow& x-x^{(N)}\in \ell^2\\
                \Rightarrow& x=x-x^{(N)}+x^{(N)}\in \ell^2
            \end{align*}
            而且(2)就是
            \begin{equation*}
                ||x^{(n)}-x||_2\leqslant \varepsilon,\forall n\geqslant N
                \Rightarrow ||x^{(n)}-x||_2\rightarrow 0{\rm\ as\ }n\rightarrow\infty
            \end{equation*}
        \end{proof}

        \begin{proposition}[极化恒等式]
            $(X,\left< \cdot,\cdot \right>)$是内积空间,内积诱导范数$||x||$,
            \begin{enumerate}
                \item $\mathbb{K}=\mathbb{R}$,则\begin{equation*}
                    \left<x,y\right>=\frac{1}{2}( ||x+y||^2-||x||^2-||y||^2 )
                \end{equation*}
                \item $\mathbb{K}=\mathbb{C}$,则\begin{equation*}
                    \left<x,y\right>=\frac{1}{4}\sum_{k=0}^3 {\rm i}^3 ||x+{\rm i}^ky||^2
                \end{equation*}
            \end{enumerate}
            (作业)
        \end{proposition}
        \begin{proof}
            $\K=\R$,
        \begin{align*}
            \frac{1}{2}( ||x+y||^2-||x||^2-||y||^2 )
            &=\frac{1}{2}( \agl{x+y}{x+y}-\agl{x}{x}-\agl{y}{y} )\\
            &=\frac{1}{2}( \agl{x+y}{x}+\agl{x+y}{y}-\agl{x}{x}-\agl{y}{y} )\\
            &=\frac{1}{2}( \agl{y}{x}+\agl{x}{y} )\\
            &=\agl{x}{y}
        \end{align*}
        $\K=\C$,
        \begin{align*}
            \frac{1}{4}\sum_{k=0}^3 {\rm i}^k||x+{\rm i}^ky||^2
            &=\frac{1}{4}( 
            \agl{x+y}{x+y}
            +{\rm i}\agl{x+{\rm i}y}{x+{\rm i}y}
            -\agl{x-y}{x-y}
            -{\rm i}\agl{x-{\rm i}y}{x-{\rm i}y} )\\
            &=\frac{1}{2}(
            \agl{x}{y}+\agl{y}{x}+{\rm i}\agl{x}{{\rm i}y}+{\rm i}\agl{{\rm i}y}{x})\\
            &=\frac{1}{2}(
            \agl{x}{y}+\agl{y}{x}+\agl{x}{y}-\agl{y}{x})\\
            &=\agl{x}{y}
        \end{align*}
        \end{proof}
        \begin{proposition}[平行四边形法则,P.L.]
            $(X,\left< \cdot,\cdot \right>)$是内积空间,内积诱导范数$||\cdot ||$,
            \begin{equation*}
                ||x+y||^2+||x-y||^2=2( ||x||^2+||y||^2 )
            \end{equation*}
            (作业)
        \end{proposition}
        \begin{proof}
        \begin{align*}
            ||x+y||^2+||x-y||^2&=\agl{x+y}{x+y}+\agl{x-y}{x-y}\\
            &=2\agl{x}{x}+2\agl{y}{y}
            +\agl{x}{y}+\agl{y}{x}+
            +\agl{x}{-y}+\agl{-y}{x}\\
            &=2\agl{x}{x}+2\agl{y}{y}
            +\agl{x}{y}+\agl{y}{x}
            -\agl{x}{y}-\agl{y}{x}\\
            &=2(||x||^2+||y||^2)
        \end{align*}
        \end{proof}
        \begin{theorem}[Frechet-von Neumann]
            $(X,||\cdot ||)$是赋范空间,$||\cdot ||$可由某个内积诱导出
            $\Leftrightarrow ||\cdot ||$满足P.L.(作业)
        \end{theorem}
        \begin{proof}
            必要性显然,只说明充分性:
        先考虑$\K=\R$,令
        \begin{equation*}
            \agl{x}{y}=\frac{1}{4}( ||x+y||^2-||x-y||^2 )
        \end{equation*}
        如果能证明$\agl{\cdot}{\cdot}$是一个内积,那么其诱导度量$||\cdot||$.
        \begin{enumerate}[$(1)$]
            \item $\agl{x}{x}=||x||\geqslant 0$,等号成立当且仅当$x=0$.
            \item $\agl{x}{y}=\agl{y}{x}$.
            \item 考虑
            \begin{align*}
                    \agl{x}{z}+\agl{y}{z}&=\frac{1}{4}( ||x+z||^2-||x-z||^2 )+\frac{1}{4}( ||y+z||^2-||y-z||^2 )\\
                    &=\frac{1}{8}( 2||x+z||^2+2||y+z||^2 )-\frac{1}{8}( 2||x-z||^2+2||y-z||^2 )\\
                    &=\frac{1}{8}( ||x+y+2z||^2+||x-y||^2 )-\frac{1}{8}( ||x+y-2z||^2+||x-y||^2 )\\
                    &=\frac{1}{2}\agl{x+y}{2z}
            \end{align*}
            特别地,当$y=0$,
            \begin{equation*}
                    \agl{0}{z}=\frac{1}{4}( ||0+z||^2-||0-z||^2 )=0
            \end{equation*}
            \begin{equation*}
                    \Rightarrow \agl{x}{z}=\agl{x}{z}+\agl{0}{z}
                    =\frac{1}{2}\agl{x}{2z}
            \end{equation*}
            上式中$x$替换为$x+y$,
            \begin{equation*}
                    \agl{x+y}{z}=\frac{1}{2}\agl{x+y}{2z}
            \end{equation*}
            于是
            \begin{equation*}
                    \agl{x}{z}+\agl{y}{z}=\frac{1}{2}\agl{x+y}{2z}=\agl{x+y}{z}
            \end{equation*}
            \item 由$(3)$,
            \begin{align*}
                    &\agl{x}{y}+\agl{x}{-y}=\agl{x-x}{y}=\agl{0}{y}=0\\
                    \Rightarrow &\agl{-x}{y}=-\agl{x}{y}
            \end{align*}
            \begin{equation*}
                    n\agl{x}{y}=\agl{x}{y}+\cdots+\agl{x}{y}=\agl{nx}{y}
            \end{equation*}
            \begin{align*}
                    &n\agl{\frac{m}{n}x}{y}=\agl{mx}{y}=m\agl{x}{y}\\
                    \Rightarrow &
                    \frac{m}{n}\agl{x}{y}=\agl{  \frac{m}{n}x}{y},\ n,m\in\mathbb{N}_+
            \end{align*}
            因此当$\lambda\in\Q$时,$\agl{\lambda x}{y}=\lambda\agl{x}{y}$.对于任意实数$a$,
            取有理数列$a_n\rightarrow a$,则有
            \begin{equation*}
                    \agl{ax}{y}=\fun{lim}{n\rightarrow\infty}\agl{a_nx}{y}
                    =\fun{lim}{n\rightarrow\infty}a_n\agl{x}{y}
                    =a\agl{x}{y}
            \end{equation*}
        \end{enumerate}
        综上可知$\agl{\cdot}{\cdot}$是一个内积。对于$\K=\C$的情况,取
        \begin{equation*}
            \agl{x}{y}=\frac{1}{4}\sum_{k=0}^3 {\rm i}^k||x+{\rm i}^ky||^2
        \end{equation*}
        其余过程同理。
        \end{proof}
        \begin{definition}
            如果$\left<x,y\right>=0$,则称$x$与$y$正交,记为$x\perp y$.
            对于$M\subset X$,如果$\forall y\in M$都有$x\perp y$,则记$x\perp M$.
            \begin{equation*}
                M^\perp \mathop{=}\limits^{\rm def} 
                \{x\in X:x\perp M\}
            \end{equation*}
            称为$M$的正交补。
        \end{definition}
        \begin{proposition}[勾股定理]
            $x\perp y\Rightarrow ||x+y||^2=||x||^2+||y||^2$
        \end{proposition}
        \begin{proposition}
            $\overline{M}=X$,$x\perp M$,则$x=0$.
        \end{proposition}
        \begin{proof}
            $x\in X$,存在$y_n\in M$使得$y_n\rightarrow x$,于是
            \begin{equation*}
                0=\left<x,y_n\right>\rightarrow \left<x,x\right>
            \end{equation*}
            所以$x=0$.

            实际上证明的是$x\perp M\Rightarrow x\perp \overline{M}$.
        \end{proof}
        \begin{proposition}
            $x\perp M\Rightarrow x\perp {\rm span\ }M$.
        \end{proposition}
        \begin{proposition}
            $M^\perp$是闭子空间。
        \end{proposition}
        \begin{proof}
            显然$M^\perp$是向量子空间,设$M^\perp \ni x_n\rightarrow x$,
            \begin{align*}
                &\forall y\in M,0=\left< x_n,y \right>\rightarrow \left< x,y \right>\\
                \Rightarrow & \left< x,y \right>=0\\
                \Rightarrow & x\in M^\perp
            \end{align*}
        \end{proof}
        \subsection{正交与正交基}
        \begin{definition}
            如果$S=\{e_\alpha\}_{\alpha\in\Lambda}\subset X$满足
            \begin{equation*}
                e_\alpha\perp e_\beta,\forall \alpha,\beta\in\Lambda,\alpha\neq\beta
            \end{equation*}
            则称$S$是$X$中的一个正交集,如果$S$还满足$||e_\alpha||=1,\forall \alpha\in\Lambda$,则称之为规范正交集,
            简记为O.N.S.

            若一个正交集$S$满足$S^\perp=\{0\}$,则称$S$完备。
        \end{definition}
        \begin{theorem}
            非平凡内积空间中一定有完备正交集。
        \end{theorem}
        \begin{definition}
            非空集合$X$上的一个偏序“$\leqslant $”是指满足以下条件的一个关系:
            \begin{enumerate}
                \item 传递性:$x\leqslant y,y\leqslant z\Rightarrow x\leqslant z$.
                \item 反身性:$x\leqslant x$.
                \item $x\leqslant y,y\leqslant x\Rightarrow x=y$.
            \end{enumerate}
            $(X,\leqslant )$称为一个偏序集。

            如果$\forall x,y\in X$,$x\leqslant y$或者$y\leqslant x$必有其一,则称
            “$\leqslant $”是$X$上的一个全序。

            对于$Y\subset X$,如果存在$p\in X$使得$\forall y\in Y$有$y\leqslant p$,称$p$是$Y$的一个上界。

            如果存在$m\in X$使得$m\leqslant x\Rightarrow x=m$,则称$m$是$X$上的一个极大元。
        \end{definition}
        \begin{lemma}[Zorn]
            $(X,\leqslant)$是一个偏序集,$X$的每个全序子集都有上界,则$X$必有极大元。
        \end{lemma}
        \begin{proof}
            令$\mathcal{F}$为$X$中正交集全体,$\subset $是集合的包含关系,则
            $(\mathcal{F},\subset )$是偏序集,设$M$是$\mathcal{F}$中的任一全序子集,令
            \begin{equation*}
                P\mathop{=}\limits^{\rm def} \bigcup_{C\in M}C
            \end{equation*}
            于是$P$是$M$的一个上界,这是因为任取$C\in X$都有$C\subset P$.
            由Zorn引理,
            $\mathcal{F}$有极大元$S$,则$S$完备,否则
            $\exists x_0\neq 0{\rm\ s.t.\ }x_0\perp S$,
            $S\cup \{x_0\}\in\mathcal{F}$,与$S$的极大性矛盾。
        \end{proof}
        \begin{definition}
            $(X,\left< \cdot,\cdot \right>)$是内积空间,$S=\{e_\alpha\}_{\alpha\in\Lambda}$是O.N.S.
            如果$\forall x\in X$均可表示为\footnote{其实对于不可数个量相加没有很合适的定义,这里的形式和需要要求
            $\{\left<x,e_\alpha\right>\}_{\alpha\in\Lambda}$只有至多可数个非零。}
            \begin{equation*}
                x=\sum_{\alpha\in \Lambda}\left<x,e_\alpha\right>\cdot e_\alpha
            \end{equation*}
            则称$S$是$X$的一个规范正交基,简称O.N.B.其中的
            $\{ \left<x,e_\alpha\right> \}$称为$x$的Fourier系数。
        \end{definition}
        \begin{theorem}[Bessel不等式]
            $\{e_\alpha\}_{\alpha\in\Lambda}$是O.N.S.则
            \begin{equation*}
                \forall x\in X,\sum_{\alpha\in\Lambda} |\left<x,e_\alpha\right>|^2\leqslant ||x||^2
            \end{equation*}
        \end{theorem}
        \begin{proof}
            \textbf{Step1}:
            \begin{equation*}
                \forall \{\alpha_1,\cdots,\alpha_N\}\subset \Lambda,
                \sum_{k=1}^N |\left< x,e_{\alpha_k} \right>|^2\leqslant ||x||^2
            \end{equation*}
            \begin{align*}
                0\leqslant &\left< x-\sum_{i=1}^N\left< x,e_{\alpha_i} \right>e_{\alpha_i},
                x-\sum_{i=1}^N\left< x,e_{\alpha_k} \right>e_{\alpha_k}\right>\\
                =&||x||^2-\sum_{i=1}^N \left< x,e_{\alpha_i} \right>
                \left< e_{\alpha_i},x \right>
                -\sum_{i=1}^N \overline{\left< x,e_{\alpha_k} \right>}
                \left< e_{\alpha_k},x \right>
                +\sum_{i=1}^N \sum_{k=1}^N 
                \left< e_{\alpha_i},x \right>\overline{\left< x,e_{\alpha_k} \right>}
                \mathop{\left< e_{\alpha_i},e_{\alpha_k} \right>}\limits_{\mbox{这一项}=\delta_{ik}}\\
                =& ||x||^2-\sum_{k=1}^N |\left< x,e_{\alpha_k} \right>|^2
            \end{align*}

            \textbf{Step2}:
            \begin{equation*}
                \Lambda'\mathop{=}\limits^{\rm def}
                \{ \alpha\in \Lambda:\left< x,e_{\alpha} \right>\neq 0\}\mbox{至多可数}
            \end{equation*}
            令
            \begin{equation*}
                \lambda_n\mathop{=}\limits^{\rm def}
                \{ \alpha\in\Lambda:|\left< x,e_{\alpha} \right>|>\frac{1}{n} \},n=1,2,\cdots
            \end{equation*}
            所有的$\Lambda_n$是有限集,否则存在$n_0$使得$\Lambda_{n_0}$是无限集,
            取$N$充分大使得
            \begin{equation*}
                \frac{N}{n_0^2}>||x||^2
            \end{equation*}
            任取$\alpha_1,\cdots,\alpha_N\in\Lambda_{n_0}$,
            \begin{equation*}
                \sum_{k=1}^N |\left< x,e_{\alpha_k} \right>|^2>\frac{N}{n_0^2}>||x||^2
            \end{equation*}
            这与Step1矛盾,进而$\Lambda'=\bigcup_{n=1}^\infty \Lambda_n$至多可数。

            \textbf{Step3}:
            给$\Lambda'$一个排列,即设$\Lambda'=\{\alpha_k\}_{k=1}^\infty$,由
            Step1,$\forall N$,
            \begin{align*}
                & \sum_{k=1}^N |\left< x,e_{\alpha_k} \right>|^2\leqslant ||x||^2\\
                \Rightarrow & \sum_{k=1}^\infty |\left< x,e_{\alpha_k} \right>|^2\leqslant ||x||^2\\
                \Rightarrow & \sum_{\alpha\in\Lambda} |\left< x,e_{\alpha} \right>|^2
                =\sum_{\alpha\in\Lambda'} |\left< x,e_{\alpha} \right>|^2\leqslant ||x||^2
            \end{align*}
        \end{proof}

            引入最佳逼近元的概念:用一组函数的线性组合去逼近一个给定的函数等价于给定$x\in X$,
            $e_1,\cdots,e_n\in X$,求$\lambda_1,\cdots,\lambda_n\in\K $使得
            \begin{equation*}
                ||x-\sum_{k=1}^n \lambda_k e_k||=\fun{inf}{\alpha\in\K^n}||x-\sum_{k=1}^n \alpha_ke_k||
            \end{equation*}
            等价于:令$M\eq{def} {\rm span}\{e_1,\cdots,e_n\}$,求$y_0\in M$使得${\rm dist}(x,y_0)={\rm dist}(x,M)$.
        \begin{theorem}
            赋范空间$(X,||\cdot ||)$,$e_1,\cdots,e_n\in X$,$\forall x\in X$,存在$\lambda_1,\cdots,\lambda_n\in\K$使得
            \begin{equation*}
                ||x-\sum_{k=1}^n \lambda_k e_k||=\fun{inf}{\alpha\in\K^n}||x-\sum_{k=1}^n \alpha_ke_k||
            \end{equation*}
        \end{theorem}
        \begin{proof}
            不妨设$e_1,\cdots,e_n$线性无关,对$x\in X$,定义
            \begin{equation*}
                F:\K^n\rightarrow \R,\alpha\mapsto ||x-\sum_{k=1}^n \alpha_k e_k||
            \end{equation*}
            则:
            \begin{enumerate}
                \item $F$连续;
                \item \begin{equation*}
                    F(\alpha)\geqslant ||\sum_{k=1}^n \alpha_k e_k||-||x||
                \end{equation*}
            \end{enumerate}
            令
                \begin{equation*}
                    |||\alpha|||\defeq ||\sum_{k=1}^n \alpha_k e_k||
                \end{equation*}
                于是$|||\cdot|||$是$\K^n$上的范数,由于有限维空间上范数等价,
                \begin{equation*}
                    \exists C>0{\rm\ s.t.\ }|||\alpha|||\geqslant C|\alpha|,\ \forall \alpha\in\K^n
                \end{equation*}
                \begin{equation*}
                    \mathop{\Rightarrow}\limits^{2} F(\alpha)\geqslant C|\alpha|-||x||\rightarrow +\infty{\rm\ as\ }|\alpha|\rightarrow\infty
                \end{equation*}
                因此$F$在$\K^n$上可以取到最小值。
        \end{proof}
        \begin{lemma}[变分引理]
            $H$是Hilbert空间,$M$是非空闭凸集,则
            \begin{equation*}
                \forall x\in X,\exists ! y\in M{\rm\ s.t.\ }||x-y||={\rm dist}(x,M)
            \end{equation*}
        \end{lemma}
        \begin{proof}
            设
            \begin{equation*}
                d\mathop{=}\limits^{\rm def}{\rm dist}(x,M)=
                \mathop{\rm inf}\limits_{\zeta\in M}||x-\zeta||
            \end{equation*}
            则存在$y_n\in M,n=1,2,\cdots{\rm\ s.t.\ }d\leqslant ||y_n-x||\leqslant d+\frac{1}{n}$.

            下面证明$\{y_n\}_{n=1}^\infty$是基本列。由P.L.
            \begin{align*}
                &||(y_n-x)+(y_m-x)||^2+||(y_n-x)-(y_m-x)||^2
                = 2( ||y_n-x||^2+||y_m-x||^2 )\\
                \Rightarrow& 
                ||y_n-y_m||^2=
                2( ||y_n-x||^2+||y_m-x||^2 )-4||\frac{y_n+y_m}{2}-x||^2\\
                &\leqslant 2( (d+\frac{1}{n})^2+(d+\frac{1}{m})^2 )-4d^2\rightarrow 0{\rm\ as\ }n,m\rightarrow\infty
            \end{align*}
            $H$完备,设$y_n\rightarrow y$,因为$M$是闭集,故$y\in M$,因此:
            \begin{equation*}
                d\leqslant||y-x||\leqslant ||y-y_n||+||y_n-x||\leqslant ||y-y_n||+d+\frac{1}{n}
            \end{equation*}
            $n$充分大时,右式$\rightarrow d^+$,于是$||y-x||=d$.

            唯一性:假设还有$y'\in M{\rm\ s.t.\ }||y'-x||=d$,
            \begin{equation*}
                ||y'-y||^2=2( ||y'-x||^2+||y-x||^2 )-4||\frac{y'+y}{2}-x||^2\leqslant 4d^2-4d^2=0
            \end{equation*}
            所以$y'=y$.
        \end{proof}
        \begin{theorem}[正交分解]
            $H$是Hilbert空间,$M$是闭子空间,则$H=M\oplus M^\perp$,即
            $\forall x\in H$,存在唯一的$y\in M$和$\zeta\in M^{\perp}$使得$x=y+\zeta$.
        \end{theorem}
        \begin{proof}
            $\forall x\in H$,由变分引理,存在唯一$y\in M$使得
            \begin{equation*}
                ||x-y||={\rm dist}(x,M)
            \end{equation*}
            Claim:$x-y\in M^\perp$,$\forall 0\neq w\in M,\forall \lambda\in\mathbb{K}$,
            \begin{align*}
                \Rightarrow & y+\lambda w\in M\\
                \Rightarrow & d^2\leqslant ||x-(y+\lambda w)||^2
                =||x-y||^2-2{\rm Re}(\overline{\lambda}\left< x-y,w \right>)+|\lambda|^2||w||^2
            \end{align*}
            取$\lambda=\frac{ \left<x-y,w\right> }{||w||^2}$,
            \begin{align*}
                &d^2\leqslant ||x-y||^2-2\frac{ |\left<x-y,w\right>|^2 }{||w||^2}+\frac{ |\left<x-y,w\right>|^2 }{||w||^2}
                =d^2-\frac{ |\left<x-y,w\right>|^2 }{||w||^2}\\
                \Rightarrow & \left<x-y,w\right>=0\\
                \Rightarrow & x-y\in M^\perp
            \end{align*}
        \end{proof}
        \begin{definition}
            映射$P_M:H\rightarrow M,x\mapsto y$,这里$y$是满足变分引理的$y$,称之为$x$在$M$中的最佳逼近元,此映射被称为
            $H$到$M$的正交投影。
        \end{definition}
        \begin{corollary}
            \begin{enumerate}
                \item $P_M(x)\in M$,$x-P_M(x)\in M^\perp$.
                \item ${\rm Im}(P_M)=M$,${\rm Ker}(P_M)=M^\perp$.
                \item $||x-P_M(x)||={\rm dist}(x,M)$.
                \item $P^2_M=P_M$.
                \item $||P_M(x)||\leqslant ||x||$.
                \item $I-P_M=P_{M^\perp}$.
            \end{enumerate}
        \end{corollary}

            在Bessel不等式的证明中,
            $\sum_{\alpha\in\Lambda'}|\left< x,e_{\alpha} \right>|^2$和$\Lambda'$的排列无关,那
            $\sum_{\alpha\in\Lambda'}\left< x,e_{\alpha} \right>e_\alpha$呢?
        \begin{lemma}
            $H$是Hilbert空间,$\{e_k\}_{k=1}^\infty$是O.N.S.则
            \begin{equation*}
                \forall x\in H,\sum_{k=1}^\infty \left<x,e_k\right>e_k\in H 
            \end{equation*}
            令$M=\overline{ {\rm span}\{e_k\}_{k=1}^\infty }$,则
            \begin{equation*}
                \sum_{k=1}^\infty \left<x,e_k\right>e_k=P_M(x)
            \end{equation*}
        \end{lemma}
        \begin{proof}
            由Bessel,
            \begin{align*}
                &\sum_{k=1}^\infty |\left< x,e_k \right>|^2\leqslant ||x||^2\\
                \Rightarrow & ||\sum_{k=n}^m \left<x,e_k\right>e_k||^2=\sum_[k=n]^\infty |\left< x,e_k \right>|^2\rightarrow 0{\rm\ as\ }n,m\rightarrow\infty\\
                \Rightarrow & \left\{ \sum_{k=1}^n \left<x,e_k\right>e_k\right\}_{n=1}^\infty\mbox{是$H$中的基本列}\\
                \Rightarrow & \sum_{k=1}^\infty \left<x,e_k\right>e_k
                \mathop{=}\limits^{\rm def} \mathop{\rm lim}\limits^{n\rightarrow\infty}\sum_{k=1}^n 
                \left<x,e_k\right>e_k\in H
            \end{align*}
            进而,由于
            \begin{align*}
                &\left< x-\sum_{k=1}^\infty \left<x,e_k\right>e_k,e_m \right>=0,\forall m\\
                \Rightarrow & x-\sum_{k=1}^\infty \left<x,e_k\right>e_k\in M^\perp\\
                \Rightarrow & \sum_{k=1}^\infty \left<x,e_k\right>e_k=P_M(x)
            \end{align*}
        \end{proof}
        \begin{corollary}
            对$\mathbb{N}$上任一置换$\sigma:\mathbb{N}\rightarrow \mathbb{N}$,
            \begin{equation*}
                \sum_{k=1}^\infty \left< x,e_{\sigma(k)} \right>e_{\sigma(k)}=\sum_{k=1}^\infty 
                \left< x,e_k \right>e_k
            \end{equation*}
        \end{corollary}
        \begin{proof}
            令$M\mathop{=}\limits^{\rm def}\overline{ {\rm span}\{e_k\}_{k=1}^\infty }$,
            $\tilde{M}\mathop{=}\limits^{\rm def}\overline{ {\rm span}\{e_{\sigma(k)}\}_{k=1}^\infty }$,
            于是$M=\tilde{M}$,
            \begin{equation*}
                \sum_{k=1}^\infty 
                \left< x,e_{\sigma(k)} \right>e_{\sigma(k)}
                =P_{\tilde{M}}(x)=P_M(x)=\sum_{k=1}^\infty \left< x,e_{k} \right>
            \end{equation*}
        \end{proof}
        \begin{corollary}
            $H$是Hilbert空间,$\{e_\alpha\}_{\alpha\in\Lambda}$是O.N.S.则
            $\forall x\in H$,
            \begin{equation*}
                \sum_{\alpha\in\Lambda}\left< x,e_{\alpha} \right>e_{\alpha}\in H
            \end{equation*}
            且
            \begin{equation*}
                ||x-\sum_{\alpha\in\Lambda}\left<x,e_\alpha\right>e_{\alpha}||^2
                =||x||^2-\sum_{\alpha\in\Lambda}|\left<x,e_\alpha\right>|^2
            \end{equation*}
        \end{corollary}
        \begin{proof}
            设
            \begin{equation*}
                \Lambda'\mathop{=}\limits^{\rm def}
                \{ \alpha\in\Lambda:\left< x,e_{\alpha} \right>\neq 0\}
                =\{\alpha_k\}_{k=1}^\infty \mbox{(任一顺序)}
            \end{equation*}
            则
            \begin{equation*}
                \sum_{\alpha\in\Lambda}\left<x,e_\alpha\right>e_{\alpha}
                =\sum_{k=1}^\infty \left<x,e_{\alpha_k}\right>e_{\alpha_k}\in H
            \end{equation*}
            且
            \begin{align*}
                &||x-\sum_{k=1}^N \left< x,e_{\alpha_k} \right>e_{\alpha_k}||^2
                =||x||^2-\sum_{k=1}^N |\left< x,e_{\alpha_k} \right>|^2\\
                \mathop{\Rightarrow}\limits^{N\rightarrow\infty}_{\mbox{范数连续}}
                &||x-\sum_{k=1}^N \left< x,e_{\alpha_k} \right>e_{\alpha_k}||^2
                =||x||^2-\sum_{k=1}^\infty |\left< x,e_{\alpha_k} \right>|^2
            \end{align*}
        \end{proof}
        \begin{theorem}
            $H$是Hilbert空间,$S=\{e_\alpha\}_{\alpha\in\Lambda}$是O.N.S.则
            下列命题等价:
            \begin{enumerate}
                \item $S^{\perp}=\{0\}$.
                \item $S$是O.N.B.
                \item $S$满足:
                \begin{equation*}
                    \forall x\in H,||x||^2=\sum_{\alpha\in\Lambda}|\left<x,e_\alpha\right>|^2
                \end{equation*}
                这被称为Parseval恒等式,简称P.I.
            \end{enumerate}
        \end{theorem}
        \begin{proof}
            \textbf{Step1}:$S^\perp=\{0\}\Rightarrow S$是O.N.B.假设不然,则
            \begin{align*}
                &\exists x_0\in H{\rm\ s.t.\ }\sum_{\alpha\in\Lambda}\left<x_0,e_\alpha\right>e_{\alpha}
                \neq x_0\\
                &\forall \beta\in\Lambda,
                \left< x_0-\sum_{\alpha\in\Lambda}\left<x_0,e_\alpha\right>e_{\alpha},e_\beta \right>
                =\left<x_0,e_\beta\right>-\left<x_0,e_\beta\right>=0\\
                \Rightarrow & 0\neq x_0-\sum_{\alpha\in\Lambda}\left<x_0,e_\alpha\right>e_{\alpha}\perp S\\
                \Rightarrow & S^\perp\neq \{0\}
            \end{align*}
            矛盾。

            \textbf{Step2}:由推论1.5.3,$S$是O.N.B.$\Rightarrow $满足Parseval。

            \textbf{Step3}:满足Parseval$\Rightarrow S^\perp=\{0\}$,否则
            \begin{align*}
                &\exists x_0\neq 0{\rm\ s.t.\ }x_0\perp S\\
                \Rightarrow & \left<x_0,e_\alpha\right>=0,\forall \alpha\in\Lambda\\
                \Rightarrow & \sum_{\alpha\in\Lambda}|\left<x_0,e_\alpha\right>|^2=0\\
                \mathop{\Rightarrow}\limits^{\rm Parseval}
                ||x_0||^2=0
            \end{align*}
            矛盾。
        \end{proof}
        \begin{example}
            数列空间$\ell^2$,$e_n$的第$n$个分量为$1$,其余分量为$0$,
            则$\{e_n\}_{n=1}^\infty$是$\ell^2$的一个O.N.B.

            但它不是$\ell^2$的Hamel基。\footnote{向量空间的Hamel基是指任何一个向量可以写成有限个基中向量的线性组合,此处不满足“有限”的条件。}
        \end{example}
        \begin{corollary}
            非平凡Hilbert空间一定有O.N.B.
        \end{corollary}
        \subsection{正交化与同构}
        \begin{theorem}[Gram-Schmidt正交化]
            $(X,\left< \cdot,\cdot \right>)$是内积空间,
            $\{x_n\}_{n=1}^\infty$线性无关,则存在一列$\{e_n\}_{n=1}^\infty$相互正交,且
            \begin{equation*}
                \forall n,{\rm span}\{e_k\}_{k=1}^n={\rm span}\{x_k\}_{k=1}^n
            \end{equation*}
        \end{theorem}
        \begin{proof}
            只需按照以下步骤构造即可:
            \begin{equation*}
                \begin{array}{ll}
                    y_1=x_1& e_1=\frac{y_1}{||y_1||}\\
                    y_2=x_2-\left<x_2,e_1\right>e_1& e_2=\frac{y_2}{||y_2||}\\
                    \vdots&\\
                    y_n=x_n-\sum_{k=1}^{n-1}\left<x_n,e_k\right>e_k& e_n=\frac{y_n}{||y_n||}
                \end{array}
            \end{equation*}
            $\forall m\neq n$,设$m<n$,
            \begin{align*}
                \left< e_m,e_n \right>&=\frac{1}{||y_n||}\left< e_m,x-\sum_{k=1}^{n-1}\left<x_n,e_k\right>e_k \right>\\
                &=\frac{1}{||y_n||}\left( \left< e_m,x-\sum_{k=1}^{n-1}\left<e_m,x\right>e_k \right>-\overline{\left< x_n,e_m \right>} \right)=0
            \end{align*}
            且
            \begin{equation*}
                y_k=x_k-\sum_{i=1}^{k-1}\left<x_k,e_i\right>e_i=x_k+\sum_{j=1}^{k-1}\alpha_{kj}x_j
            \end{equation*}
            得到
            \begin{equation*}
                \begin{pmatrix}
                    y_1\\y_2\\\vdots\\y_n
                \end{pmatrix}=
                \begin{pmatrix}
                    1&&&\\
                    \alpha_{21}&1&&\\
                    \vdots&&\ddots&\\
                    \alpha_{n1}&\alpha_{n2}&&1
                \end{pmatrix}
                \begin{pmatrix}
                    x_1\\x_2\\\vdots\\x_n
                \end{pmatrix}
            \end{equation*}
            所以${\rm span}\{ y_k \}_{k=1}^n={\rm span}\{ x_k \}_{k=1}^n$,进而
            ${\rm span}\{ e_k \}_{k=1}^n={\rm span}\{ x_k \}_{k=1}^n$.
        \end{proof}
        \begin{definition}
            $(X_1,\left< \cdot,\cdot \right>_1)$和$(X_2,\left< \cdot,\cdot \right>_2)$是内积空间,
            如果存在线性同构$T:X_1\rightarrow X_2$使得
            \begin{equation*}
                \left< T(x),T(y) \right>_2=\left< x,y \right>_1,\forall x,y\in X_1
            \end{equation*}
            则称$X_1$和$X_2$作为内积空间同构,记为$X_1\cong X_2$.
        \end{definition}
        \begin{theorem}
            $H$是Hilbert空间,$H$可分$\Leftrightarrow H$有可数的O.N.B. 
        \end{theorem}
        \begin{proof}
            必要性:如果${\rm dim\ }H\neq \infty$则显然,下面假设${\rm dim\ }H=\infty$,
            \begin{align*}
                \mbox{可分}\Rightarrow &
                \exists \{x_n\}_{n=1}^\infty \subset H{\rm\ s.t.\ }\overline{\{x_n\}_{n=1}^\infty}=H\\
                \Rightarrow &\exists \{y_n\}_{n=1}^\infty \subset\{x_n\}_{n=1}^\infty \mbox{线性无关}\\
                \Rightarrow & \exists \{e_n\}_{n=1}^\infty \mbox{正交且}
                {\rm span}\{ e_n \}_{n=1}^\infty={\rm span}\{ x_n \}_{k=n}^\infty\\
                \Rightarrow &
                \overline{ {\rm span}\{ e_n \}_{n=1}^\infty }
                =\overline{ {\rm span}\{ x_n \}_{n=1}^\infty }=H\\
                \Rightarrow & ( \{ e_n \}_{n=1}^\infty )^\perp=\{0\}\\
                \Rightarrow & \{ e_n \}_{n=1}^\infty\mbox{是O.N.B.}
            \end{align*}

            充分性:令
            \begin{equation*}
                {\rm span}^{\mathbb{Q}}(\{ e_n \}_{n=1}^\infty)\mathop{=}\limits^{\rm def}
                \{ e_n \}_{n=1}^\infty\mbox{中向量以$\mathbb{Q}+{\rm i}\mathbb{Q}$中元素为系数的线性组合全体}
            \end{equation*}
            于是${\rm span}^{\mathbb{Q}}(\{ e_n \}_{n=1}^\infty)$可数,下证
            $\overline{{\rm span}^{\mathbb{Q}}(\{ e_n \}_{n=1}^\infty)}=H$.
            \begin{align*}
                &\forall \varepsilon>0,\forall n,\exists \alpha_n\in\mathbb{Q}+{\rm i}\mathbb{Q}{\rm\ s.t.\ }
                |\alpha_n-\left<x,e_n\right>|<\frac{\varepsilon}{2^{n+1}}\\
                \Rightarrow &
                || \sum_{n=1}^N \left<x,e_n\right>e_n-\sum_{n=1}^N \alpha_n e_n ||^2
                =\sum_{n=1}^N |\left<x,e_n\right>-\alpha_n|^2<\frac{\varepsilon^2}{4},\forall N
            \end{align*}
            而当$N$充分大时,
            \begin{equation*}
                ||\sum_{n=1}^N \left<x,e_n\right>e_n-x||<\frac{\varepsilon}{2}
            \end{equation*}
            于是
            \begin{equation*}
                ||\sum_{n=1}^N \alpha_n e_n-x||<\varepsilon
            \end{equation*}
        \end{proof}
        \begin{theorem}
            \begin{enumerate}
                \item $n$维Hilbert空间$\cong \mathbb{K}^n$.
                \item 无穷维可分Hilbert空间$\cong \ell^2$.
            \end{enumerate}
        \end{theorem}
        \begin{proof}
            1以前已经证明过,只说明一下2:
            设$\{ e_n \}_{n=1}^\infty$是$H$的可数O.N.B.定义
            \begin{equation*}
                T:H\rightarrow \ell^2,x\mapsto \{ \left< x,e_n \right> \}_{n=1}^\infty
            \end{equation*}
            \begin{enumerate}[$1^\circ$]
                \item 线性:显然。
                \item 等距:由Parseval,\begin{equation*}
                    \sum_{n=1}^\infty |\left< x,e_n \right>|^2=||x||^2
                \end{equation*}
                \item 单射:由等距可得。
                \item 满射:\begin{align*}
                    &\forall a\in \ell^2,
                    ||\sum_{k=n}^m a_k e_k||^2=\sum_{k=n}^m |a_k|^2\rightarrow 0{\rm\ as\ }n,m\rightarrow\infty\\
                    \Rightarrow &
                    \sum_{k=1}^n a_ke_n\rightarrow x\in H{\rm\ and\ }\left< x,e_k \right>=a_k\\
                    \Rightarrow &
                    T(x)=a
                \end{align*}
                \item 保内积:\begin{align*}
                    \left< x,y \right>&= 
                    \left< \sum_{n=1}^\infty \left< x,e_n \right>e_n,\sum_{m=1}^\infty \left< y,e_m \right>e_m \right>\\
                    &= \sum_{n,m}\left< x,e_n \right>\overline{\left< y,e_m \right>}\left< e_n,e_m \right>\\
                    &= \sum_{n=1}^\infty \left< x,e_n \right>\overline{\left< y,e_n \right>}\\
                    &= \left< T(x),T(y) \right>_{\ell^2}
                \end{align*}
            \end{enumerate}
        \end{proof}
    
\section{应用:Fourier级数}
    \begin{definition}
        设
        \begin{equation*}
            \Pi\mathop{=}\limits^{\rm def}\{ z\in\mathbb{C}:|z|=1 \}
        \end{equation*}
        对于$\Pi$上的函数$F$,令
        \begin{equation*}
            f(t)\mathop{=}\limits^{\rm def}F( {\rm e}^{2\pi{\rm i}t} ),t\in\mathbb{R}
        \end{equation*}
        于是$f$是$\mathbb{R}$上周期为$1$的周期函数。令
        \begin{equation*}
            e_k(t)\mathop{=}\limits^{\rm def}{\rm e}^{2\pi {\rm i}kt},t\in [-\frac{1}{2},\frac{1}{2}),k=0,\pm 1,\pm 2,\cdots
        \end{equation*}
        于是$\{e_k\}_{k=-\infty}^\infty$是$L^2(\Pi)$中的O.N.S.称为三角函数系。
        \begin{equation*}
            \hat{f}(k)\mathop{=}\limits^{\rm def}\int_{-\frac{1}{2}}^{\frac{1}{2}}
            f(t){\rm e}^{-2\pi {\rm i}kt}{\rm d}t=\left< f,e_k \right>
        \end{equation*}
        \begin{equation*}
            f(x)\sim \sum_{k=-\infty}^\infty \hat{f}(k){\rm e}^{2\pi{\rm i}kx}
            =\sum_{-\infty}^\infty \left<f,e_k\right>e_k
        \end{equation*}
    \end{definition}
    \begin{theorem}
        $\forall f\in L^2(\Pi)$,
        \begin{equation*}
            ||S_N(f)-f||_2\rightarrow 0{\rm\ as\ }N\rightarrow\infty
        \end{equation*}
        其中
        \begin{equation*}
            S_N(f)(x)\mathop{=}\limits^{\rm def}\sum_{k=-N}^N \hat{f}(k){\rm e}^{2\pi{\rm i}kx}
        \end{equation*}
    \end{theorem}
    \begin{proof}
        Thm$\Leftrightarrow \{e_k\}_{k=-\infty}^\infty$是$L^2(\Pi)$中的O.N.B.
        $\Leftrightarrow (\{e_k\}_{k=-\infty}^\infty)^\perp=\{0\} $
        $\Leftrightarrow \overline{{\rm span}\{e_k\}_{k=-\infty}^\infty}=L^2(\Pi)$.
        从而约化为证明:$\forall f\in L^2(\Pi)$可由三角多项式以$L^2$范数逼近。
        \begin{equation*}
            S_N(f)(x)=(f* D_N)(x){\rm\ with\ }
            D_N(t)=\sum_{k=-N}^N {\rm e}^{2\pi {\rm i}kt}=\frac{ {\rm sin}[ (2N+1)\pi t ] }{{\rm sin}(\pi t)}
        \end{equation*}
        令
        \begin{align*}
            \sigma_N\mathop{=}\limits^{\rm def}& \frac{1}{N+1}\sum_{k=0}^N S_k(f)\\
            =& f *\left( \frac{1}{N+1}\sum_{k=1}^N D_k \right)=f* F_N
        \end{align*}
        这里$F_N(t)$是Fejér核:
        \begin{equation*}
            F_N(t)=\frac{1}{N+1}\sum_{k=1}^N D_k(t)=\frac{1}{N+1} \frac{ {\rm sin}^2 [(N+1)\pi t] }{{\rm sin}^2 (\pi t)}
        \end{equation*}
        \begin{lemma}
            \begin{enumerate}
                \item \begin{equation*}
                    \int_{-\frac{1}{2}}^{\frac{1}{2}}F_N(t){\rm d}t=1
                \end{equation*}
                \item $\forall \delta>0$,
                    \begin{equation*}
                        \mathop{\rm lim}\limits_{N\rightarrow\infty}
                        \int_{\delta<|t|<\frac{1}{2}}F_N(t){\rm d}t=0
                    \end{equation*}
            \end{enumerate}
            \begin{proof}
                1直接计算验证,只说明一下2:当$\delta<|t|<\frac{1}{2}$时,
                \begin{equation*}
                    0\leqslant F_N(t)\leqslant \frac{1}{N+1}\frac{1}{{\rm sin}^2(\pi\delta)}\rightarrow 0{\rm\ as\ }N\rightarrow\infty
                \end{equation*}
            \end{proof}
        \end{lemma}
        \begin{lemma}
            $\forall f\in L^2(\Pi)$,
            \begin{equation*}
                ||\sigma_N(f)-f||_2\rightarrow 0{\rm\ as\ }N\rightarrow \infty
            \end{equation*}
            \begin{proof}
                \begin{align*}
                    ||\sigma_N(f)-f||_2=& \sqrt{ \int_{-\frac{1}{2}}^{\frac{1}{2}} 
                    \left| \int_{-\frac{1}{2}}^{\frac{1}{2}}[ f(x-t)-f(x) ]F_N(t){\rm d}t \right|^2{\rm d}x}\\
                    \mathop{\leqslant}\limits^{\rm Minkowski}&
                    \int_{-\frac{1}{2}}^{\frac{1}{2}}|| f(\cdot-t)-f(\cdot) ||_2F_N(t){\rm d}t\\
                    =& \int_{ |t|\leqslant \delta }\cdots+\int_{\delta<|t|<\frac{1}{2}}\cdots\\
                    \leqslant &
                    \mathop{\int_{ |t|\leqslant \delta }\cdots}\limits_{\mbox{积分的绝对连续性}}+
                    \mathop{2||f||_2\int_{\delta<|t|<\frac{1}{2}}F_N(t){\rm d}t}\limits_{\mbox{引理1.6.1}}\rightarrow 0{\rm\ as\ }N\rightarrow \infty
                \end{align*}
            \end{proof}
        \end{lemma}
        注意
        \begin{equation*}
            \sigma_N(f)(x)=\sum_{k=-N}^N \left( 1-\frac{|k|}{N} \right)\hat{f}(k){\rm e}^{2\pi{\rm i}kx}
        \end{equation*}
        上述引理说明三角多项式全体在$L^2(\Pi)$中稠密。
    \end{proof}