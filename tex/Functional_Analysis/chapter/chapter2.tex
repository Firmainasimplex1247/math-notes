\chapter{线性算子与线性泛函}
%\begin{center}
%    线性算子=线性映射
%\end{center}
%\rightline{2023.10.27}
%\vspace{-5pt}
%\begin{center}
%    \pgfornament[width=0.36\linewidth,color=lsp]{88}
%\end{center}

\section{线性算子}
\subsection{有界线性算子}
    \begin{definition}
        $X$和$Y$是向量空间,如果映射$T:X\rightarrow Y$满足
        \begin{equation*}
            T(\alpha x+\beta y)=\alpha Tx+\beta Ty,\forall x,y\in X,\forall \alpha,\beta\in\mathbb{K}
        \end{equation*}
        则称$T$是线性算子,简称L.O.
        
        特别地,如果$Y=\mathbb{K}$,则称$T$是线性泛函。
    \end{definition}
    \begin{example}
        微分算子:开集$\Omega\subset \mathbb{R}^n$,$X=Y=C^\infty (\Omega)$,定义
        \begin{equation*}
            T\mathop{=}\limits^{\rm def}
            \sum_{|\alpha|\leqslant m}a_\alpha \partial^\alpha
        \end{equation*}
    \end{example}
    \begin{example}
        积分算子:$X=L^p(\Omega)$,$Y=L(\Omega)$,即全体可测函数。
        $K(\cdot,\cdot)$是$\Omega\times\Omega$上的可测函数,称之为积分核。
        \begin{equation*}
            T:u(x)\mapsto \int_\Omega K(x,y)u(y){\rm d}y
        \end{equation*}
        如:Poisson积分:
        \begin{equation*}
            P[u](\zeta)\mathop{=}\limits^{\rm def} \frac{1}{2\pi}\int_0^{2\pi}
            \frac{1-|\zeta|^2}{|1-\zeta {\rm e}^{-{\rm i}\theta}|^2}u( {\rm e}^{{\rm i}\theta}){\rm d}\theta
        \end{equation*}
        Fourier变换:
        \begin{equation*}
            (\mathcal{F}u)(x)\mathop{=}\limits^{\rm def}
            \int_{\mathbb{R}^n} {\rm e}^{-2\pi{\rm i}x\cdot y}u(y){\rm d}y
        \end{equation*}
    \end{example}
    \begin{example}
        非线性的例子:
        \begin{equation*}
            f(u)\mathop{=}\limits^{\rm def} \int_{\Omega}u^2(x){\rm d}x
        \end{equation*}
    \end{example}
    \begin{definition}
        $(X,||\cdot||_X)$和$(Y,||\cdot||_Y)$是赋范空间,
        $T:X\rightarrow Y$是L.O.如果存在$C>0$使得
        \begin{equation*}
            ||Tx||_Y\leqslant C||x||_X,\forall x\in X
        \end{equation*}
        则称$T$有界。
    \end{definition}

    \begin{theorem}
        赋范空间$(X,\ms{\cdot}_{X}),(Y,\ms{\cdot}_{Y})$,$T:X\rightarrow Y$是线性映射,
        $T$有界$\Leftrightarrow T$连续$\Leftrightarrow T$在$0$处连续。
    \end{theorem}
    \begin{proof}
        有界$\Rightarrow $连续:
        \begin{equation*}
            ||x_n-x||_X\rightarrow 0\Rightarrow ||Tx_a-Tx||\leqslant C||x_n-x||\rightarrow 0
        \end{equation*}
        
        在$0$处连续$\Rightarrow $有界:假设$T$在$0$连续,但无界,则
        \begin{equation*}
            \forall n,\exists x_n\in X{\rm\ s.t.\ }||Tx_n||_Y>n||x_n||_X
        \end{equation*}
        令$y_n=\frac{1}{n}\frac{x_n}{||x_n||_X}$,则$y_n\rightarrow 0$但
        $|Ty_n|_Y\geqslant 1,\forall n$,这与$T$在$0$处连续矛盾。
    \end{proof}
    \begin{theorem}
        有限维赋范空间之间的线性算子一定有界。
    \end{theorem}
    \begin{proof}
        先假设$X=\mathbb{K}^n$,$Y=\mathbb{K}^m$,则
        \begin{equation*}
            Tx=Ax{\rm\ for\ some\ }A=(a_{ij})_{1\leqslant i\leqslant m,1\leqslant j\leqslant n}
        \end{equation*}
        \begin{align*}
            \Rightarrow ||Tx||_{\mathbb{K}^m}
            =&\left(
                \sum_{i=1}^m 
                \left|
                    \sum_{j=1}^n a_{ij}x_j
                \right|^2
            \right)^{\frac{1}{2}}\\
            \mathop{\leqslant}\limits^{\rm C-S}&
            \left[
                \sum_{i=1}^m
                \left(
                    \sum_{j=1}^n |a_{ij}|^2
                \right)
                \left(
                    \sum_{j=1}^n |x_{j}|^2
                \right)
            \right]^{\frac{1}{2}}\\
            =&\left(
                \sum_{i=1}^m\sum_{j=1}^n |a_{ij}|^2
            \right)^{\frac{1}{2}}||x||_{\mathbb{K}^n}
        \end{align*}

        一般情形:把$X$和$Y$同胚到一个有限维线性空间即可:
        \begin{equation*}
            \begin{matrix}
                X&\mathop{\rightarrow}\limits^{T}&Y\\
                \varphi\downarrow&&\downarrow \psi\\
                \K^n&\mathop{\rightarrow}\limits^{\tilde{T}}&\K^m
            \end{matrix}
        \end{equation*}
        $T=\psi^{-1}\circ \tilde{T}\circ \varphi$.
    \end{proof}

    \begin{proposition}
        证明:${\rm dim\ }X<\infty$,$T:X\rightarrow Y$是线性算子,则$T$有界。(作业)
    \end{proposition}
    \begin{proof}
        设$\{e_1,\cdots,e_n\}$是$X$上的一组{\rm Hamel}基,
        记$M=\fun{max}{}\{ ||Te_1||_Y,\cdots,||Te_n||_Y \}$,
        那么$\forall x\in X$,$x=\sum_{i=1}^n x_ie_i,x_i\in\K$,
        \begin{align*}
            ||Tx||_Y&=||x_1T(e_1)+\cdots+x_nT(x_n)||_Y\\
            &\leqslant |x_1|\cdot||T(e_1)||_Y+\cdots+|x_n|\cdot ||T(e_n)||_Y\\
            &\leqslant M(|x_1|+\cdots+|x_n|)
        \end{align*}
        由于有限维线性空间上范数都等价,不妨$||x||_X\defeq |x_1|+\cdots+|x_n|$,则
        $||Tx||_Y\leqslant M||x||_X$,$T$有界。
    \end{proof}

    \begin{example}[无界算子的例子]
        $X=C^1[0,1]$,$Y=C[0,1]$,赋以一致范数,
        $T=\frac{{\rm d}}{{\rm d}t}$,设
        \begin{equation*}
            u_n(t)=t^n,t\in[0,1],n=1,2,\cdots
        \end{equation*}
        \begin{align*}
            &\Rightarrow ||u_n||=1,||Tu_n||=n\\
            &\Rightarrow \frac{||Tu_n||}{||u_n||}\rightarrow \infty\\
            &\Rightarrow T\mbox{无界}
        \end{align*}
    \end{example}
\subsection{算子范数}
    \begin{definition}
        $X$到$Y$的有界线性算子全体记作$\mathcal{L}(X,Y)$,对于
        $T\in \mathcal{L}(X,Y)$,
        \begin{equation*}
            ||T||_{X\rightarrow Y}
            \mathop{=}\limits^{\rm def}
            \mathop{\mathop{\rm sup}\limits_{x\in X}}\limits_{x\neq 0}
            \frac{||Tx||_Y}{||x||_X}=
            \mathop{\mathop{\rm sup}\limits_{x\in X}}\limits_{||x||=1}||Tx||_Y
        \end{equation*}
        称为$T$的算子范数。
    \end{definition}
    \begin{example}
        Hilbert空间$X$的一个闭子空间为$M$,正交投影$P_M:X\rightarrow M$
        有$||P_M||_{X\rightarrow M}=1$.
    \end{example}
    \begin{theorem}
        $(X,||\cdot||_X)$,$(Y,||\cdot||_Y)$,
        $ (\mathcal{L}(X,Y),||\cdot||_{X\rightarrow Y}) $是赋范空间。
        进而如果$Y$是Banach空间,则$\mathcal{L}(X,Y)$是Banach空间。特别地,
        $X^*\defeq \{X\mbox{上全体有界线性泛函}\}$是Banach空间。
    \end{theorem}
    \begin{proof}
        取$\mathcal{L}(X,Y)$上的柯西列$\{f_n\}$,
        \begin{align*}
            ||f_n-f_m||\leqslant \varepsilon\Rightarrow &
            \fun{sup}{||x||=1} ||f_n(x)-f_m(x)|| \leqslant \varepsilon\\
            \Rightarrow &\forall ||x||=1,\{f_n(x)\}\mbox{是$Y$上的柯西列}
        \end{align*}
        于是定义:
        \begin{equation*}
            f:x\mapsto \fun{lim}{n\rightarrow\infty} f_n(x)
        \end{equation*}
        $f$显然是线性映射,
        \begin{equation*}
            ||f_n-f||=\fun{sup}{||x||=1}||f_n(x)-f(x)||\rightarrow 0
        \end{equation*}        
        现只需证明$||f||$有界:$\{f_n\}$作为柯西列是有界的,不妨$||f_n||\leqslant M$,则
        \begin{equation*}
            ||f||=\fun{sup}{||x||=1}||f(x)||
            =\fun{sup}{||x||=1}\fun{lim}{n\rightarrow \infty}||f_n(x)||
            \leqslant 
            \fun{sup}{||x||=1}\fun{liminf}{n\rightarrow \infty}||f_n||\leqslant M<+\infty
        \end{equation*}
        所以$f$有界。
    \end{proof}

\section{Riesz表示定理}
    设$H$是Hilbert空间,给定$y\in H$,定义$f_y:H\rightarrow\mathbb{K},x\mapsto \left<x,y\right>$,
    则由C-S不等式:$|f_y(x)|\leqslant ||y||||x||$,
    于是$f_y\in H^*$且$||f_y\||\leqslant ||y||$.

    问:是否$\forall f\in H^*$,$f(x)=\left<x,y\right>{\rm\ for\ some\ }y\in H$?
    \begin{theorem}[Riesz表示定理]
        $H$是Hilbert空间,$\forall f\in H^*$,存在唯一$y_f\in H$使得
        \begin{equation*}
            f(x)=\left<x,y_f\right>,x\in H
        \end{equation*}
        且$||y_f||=||f||$.
    \end{theorem}
    \begin{proof}
        分析:$\mathbb{R}^n$上的线性代数$l(x)=\sum_{i=1}^n \alpha_ix_i=\left<x,a\right>$,
        如果$f(x)=\left<x,y\right>,f(x)=0\Rightarrow y\perp x\Rightarrow y\perp {\rm Ker}(f)$.

        存在性:如果$f=0\Rightarrow y_f=0$,下设$f\neq 0$,于是
        ${\rm Ker}(f)\neq H$,且是闭子空间,因为$f\in H^*$.于是
        存在$y_0\in {\rm Ker}(f)^\perp{\rm\ with\ }||y_0||=1$.
        \begin{align*}
            \forall x\in H,f( x-\frac{f(x)}{f(y_0)}y_0 )=0&\Rightarrow
            x-\frac{f(x)}{f(y_0)}y_0\in {\rm Ker}(f)\\
            &\Rightarrow \left<x-\frac{f(x)}{f(y_0)}y_0,y_0\right>=0\\
            &\Rightarrow \left<x,y_0\right>-\frac{f(x)}{f(y_0)}||y_0||^2=0\\
            &\Rightarrow f(x)=\left< x,\overline{f(y_0)}y_0 \right> 
        \end{align*}
        设$y_f=\overline{f(y_0)}y_0$即可。

        唯一性:设$y,\zeta\in H$使得
        \begin{equation*}
            f(x)=\left<x,y\right>=\left<x,\zeta\right>,x\in H
        \end{equation*}
        于是$\forall x\in H$,$\left<x,y-\zeta\right>=0\Rightarrow y-\zeta=0$.

        一方面,
        \begin{equation*}
            ||f(x)||=||\ag{x,y_f}||\leqslant ||x||\cdot ||y_f||\Rightarrow ||f||\leqslant ||y_f||
        \end{equation*}
        另一方面由$||y_0||=1$,
        \begin{equation*}
            ||y_f||=||f(y_0)||\leqslant ||f||
        \end{equation*}
        所以$||f||=||y_f||$.
    \end{proof}

    \begin{theorem}
        $H$是Hilbert空间,$a(\cdot,\cdot)$是$H$上的共轭双线性函数,如果存在$C>0$使得
        \begin{equation*}
            |a(x,y)|\leqslant C||x||||y||,\forall x,y\in H
        \end{equation*}
        则存在$A\in \mathcal{L}(H)$,使得
        \begin{equation*}
            a(x,y)=\left<x,Ay\right>,x\in H
        \end{equation*}
        且
        \begin{equation*}
            ||A||=\mathop{\rm sup}\limits_{0\neq x,y\in H}
            \frac{|a(x,y)|}{||x||||y||}
        \end{equation*}
    \end{theorem}
    \begin{proof}
        $y\in H$,定义
        \begin{equation*}
            f_y(x)=a(x,y),x\in H
        \end{equation*}
        于是$f_y\in H^*$且$||f_y||\leqslant C||y||$,由Riesz,
        存在唯一$\zeta\in H$使得$f_y(x)=\left<x,\zeta\right>,x\in H$,定义
        $A:H\rightarrow H,y\mapsto \zeta$,则
        \begin{equation*}
            a(x,y)=f_y(x)=\left<x,\zeta\right>=\left<x,Ay\right>
        \end{equation*}
        \begin{enumerate}
            \item $A$是线性映射;
            \item $\forall y\in H$,$||Ay||=||\zeta||=||f_y||\leqslant C||y||$,于是$A\in \mathcal{L}(H)$且$||A||\leqslant C$,
                由$C$的任意性可得
                \begin{equation*}
                    ||A||\leqslant \mathop{\rm sup}\limits_{0\neq x,y\in H}
                    \frac{|a(x,y)|}{||x||||y||}
                \end{equation*}
                另一方面,
                \begin{equation*}
                    |a(x,y)|=|\left<x,Ay\right>|\leqslant ||x||||Ay||\leqslant ||A||||x||||y||,\forall x,y\in H
                \end{equation*}
                于是
                \begin{equation*}
                    \mathop{\rm sup}\limits_{0\neq x,y\in H}
                    \frac{|a(x,y)|}{||x||||y||}\geqslant ||A||
                \end{equation*}
        \end{enumerate}
    \end{proof}

\section{Baire纲定理}
    \subsection*{定理内容}
    \begin{definition}
        $(X,d)$为度量空间,$E\subset X$,如果$\overline{E}$没有内点,则称$E$疏(朗)或无处稠密。
    \end{definition}
    \begin{example}
        $\mathbb{R}$中,Cantor三分集是无处稠密集。
    \end{example}
    \begin{definition}
        第一纲集,又称贫集、瘦集:可数个无处稠密集之并;

        第二纲集:不是第一纲集的集合;

        剩余集:第一纲集的余集。
    \end{definition}
    \begin{example}
        可数集是第一纲集。
    \end{example}
    \begin{lemma}[闭球套]
        设$(X,d)$完备,$\{B_n\}_{n=1}^\infty$是一列闭球,使得
        \begin{enumerate}
            \item $B_1\supset B_2\supset \cdots$
            \item ${\rm diam}B_n\rightarrow 0$
        \end{enumerate}
        则存在唯一$x\in X$使得
        \begin{equation*}
            \bigcap_{n=1}^\infty B_n=\{x\}
        \end{equation*}
    \end{lemma}
    \begin{proof}
        存在性:设$B_n=\overline{ B(x_n,r_n) }$,
        \begin{align*}
            \forall n\geqslant m,x_n\in B_n\subset B_m
            \Rightarrow & d(x_n,x_m)\leqslant r_m\rightarrow 0{\rm\ as\ }n,m\rightarrow\infty\\
            \Rightarrow & \{x_n\}_{n=1}^\infty \mbox{是基本列}\\
            \mathop{\Rightarrow}\limits^{X\mbox{完备}}& \exists x\in X{\rm\ s.t.\ }d(x_n,x)\rightarrow 0{\rm\ as\ }n\rightarrow\infty\\
            \mathop{\Rightarrow}\limits^{B_n\mbox{闭}}& x\in B_n,n=1,2,\cdots\\
            \Rightarrow & x\in\bigcap_{n=1}^\infty B_n
        \end{align*}
        
        唯一性:如果$y\in \bigcap_{n=1}^\infty B_n$,则
        \begin{equation*}
            d(x,y)\leqslant d(x,x_n)+d(x_n,y)\leqslant 2r_n\rightarrow 0{\rm\ as\ }n\rightarrow\infty
        \end{equation*}
        于是$y=x$.
    \end{proof}
    \begin{theorem}[Baire Category Theorem 1,BCT1]
        完备度量空间是第二纲集。
    \end{theorem}
    \begin{proof}
        假设$(X,d)$完备,是第一纲集,则$X$可以被表示成
        可数个无处稠密集之并,设为
        \begin{equation*}
            X=\bigcup_{n=1}^\infty E_n
        \end{equation*}
        任取$B(x_0,r_0)$,
        \begin{align*}
            E_1\mbox{疏}\Rightarrow &\overline{E_1}\mbox{无内点}\\
            \Rightarrow &\exists x_1\in B(x_0,r_0)\backslash \overline{E_1}\\
            \Rightarrow &{\rm dist}(x_1,\overline{E_1})>0\\
            \Rightarrow &
            \exists B(x_1,r_1)\subset B(x_0,r_0),r_1<1{\rm\ s.t.\ }
            \overline{ B(x_1,r_1) }\cap \overline{E_1}=\varnothing
        \end{align*}
        对$E_2$做同样的操作,
        \begin{equation*}
            \exists B(x_2,r_2)\subset B(x_1,r_1),r_2<\frac{1}{2}{\rm\ s.t.\ }
            \overline{ B(x_2,r_2) }\cap \overline{E_2}=\varnothing
        \end{equation*}
        以此类推,
        \begin{equation*}
            \exists B(x_n,r_n)\subset B(x_{n-1},r_{n-1}),r_n<\frac{1}{n}{\rm\ s.t.\ }
            \overline{ B(x_n,r_n) }\cap \overline{E_n}=\varnothing
        \end{equation*}
        由闭球套引理,
        \begin{equation*}
            \bigcap_{n=1}^\infty \overline{ B(x_n,r_n) }=\{x\}
        \end{equation*}
        另一方面,由于$\forall n,\overline{ B(x_n,r_n) }\cap \overline{E_n}=\varnothing$,故
        $\forall n,x\notin \overline{E_n}$,进而
        \begin{equation*}
            x\notin \bigcup_{n=1}^\infty \overline{E_n}=X
        \end{equation*}
        矛盾。
    \end{proof}
    \begin{theorem}[Baire Category Theorem 2,BCT2]
        $(X,d)$是完备度量空间,$\{U_n\}$是一列开集,且满足
        $\overline{U_n}=X$,则
        \begin{equation*}
            \overline{\left(\bigcap_{n=1}^\infty U_n\right)}=X
        \end{equation*}
    \end{theorem}
    \begin{proof}
        设$B_0$是$X$上的任一非空开集,
        $U_1$稠密$\Rightarrow \exists x_0\in U_1\cap B_{0}$,
        因为$U_1\cap B_{0}$是开集,
        所以存在$x_0$的邻域$B_1$满足
        $\overline{B_1}\subset U_1\cap B_{0}$.
        又因为$U_2$稠密,类似地存在$B_2$满足$\overline{B_2}\subset U_2\cap B_1$,
        以此类推可得一列开集$\{B_n\}_{n=1}^\infty$,满足
        \begin{equation*}
            \overline{B_n}\subset U_n\cap B_{n-1}
        \end{equation*}
        不妨令$B_n$为半径小于$\frac{1}{n}$的开球,根据闭球套定理可知
        \begin{equation*}
            K=\bigcap_{n=1}^\infty \overline{B_n}\neq \varnothing
        \end{equation*}
        根据构造过程可得$K\subset B_0$,
        $K\subset U_n$,从而$B_0$与$\cap_{n=1}^\infty U_n$相交,
        所以$\cap_{n=1}^\infty U_n$也是稠密的。
    \end{proof}
\subsection*{应用和推论}
    \begin{example}
        $l^2$的Hamel基是不可数集。(把$l^2$换成任一无穷维Banach空间也可以。)
    \end{example}
    \begin{proof}
        假设$l^2$的Hamel基$B$可数,设$B=\{x_n\}_{k=1}^\infty$,设
        $A_n={\rm span}\{x_1,\cdots,x_n\}$,则$A_n$闭且$l^2=\bigcup_{n=1}^\infty A_n$.由BCT,存在$n_0$使得$A_{n_0}$有内点,那么
        \begin{align*}
            &\exists B(x_0,r)\subset A_{n_0}\\
            \Rightarrow& B(0,r)=B(x_0,r)-x_0\subset A_{n_0}\\
            \Rightarrow& \forall 0\neq x\in l^2,\frac{r}{2}\cdot \frac{x}{||x||}\in B(0,r)\subset A_{n_0}\\
            \Rightarrow& x\in A_{n_0}
        \end{align*}
    \end{proof}
    \begin{theorem}[Banach, 1931]
        $C[0,1]$中处处不可微函数全体是一个剩余集,从而是第二纲集。
    \end{theorem}
    \begin{proof}
        记$X=C[0,1]$,$E=\{ f\in C[0,1]:f\mbox{处处不可微} \}$,于是
        $X\backslash E=\{ f\in C[0,1]:f\mbox{至少在某一点可微} \}$.令
        \begin{equation*}
            A_n=\left\{
                f\in X:  \exists t\in (0,1) {\rm\ s.t.\ }
                \mathop{\rm sup}\limits_{ h\in (-\frac{1}{n},\frac{1}{a}),t+h\in [0,1] }
                \left| \frac{f(t+h)-f(t)}{h} \right|\leqslant n
            \right\}
        \end{equation*}
        于是$\forall f\in X\backslash E$,存在$n$使得$f\in A_n$,所以
        \begin{equation*}
            X\backslash E=\bigcup_{n=1}^\infty A_n
        \end{equation*}
        约化为证明每个$A_n$疏。
        
        \textbf{Step1:}每个$A_n$都是闭集。设一列$f_k\in A_n$,
        $f_k$一致收敛到$f$,对于每个$k$存在$t_k\in (0,1)$使得
        \begin{equation*}
            |f_k(t_k+h)-f_k(t_k)|\leqslant n|h|,\forall h\in (-\frac{1}{n},\frac{1}{n}){\rm\ with\ }t_k+h\in [0,1]
        \end{equation*}
        $\{t_k\}_{k=1}^\infty$有收敛子列,不妨设$t_k\rightarrow t_0$,
        由$f_k\rightrightarrows f$,
        \begin{equation*}
            |f(t_0+h)-f(t_0)|\leqslant n|h|,\forall h\in (-\frac{1}{n},\frac{1}{n}){\rm\ with\ }t_k+h\in [0,1]
        \end{equation*}
        于是$f\in A_n$.

        \textbf{Step2:}$A_n$无内点,即:$\forall f\in A_n$,$\forall \varepsilon>0$,$B(f,\varepsilon)\nsubseteq A_n$.
        首先,by Weierstrass,
        \begin{equation*}
            \exists p\in P[0,1]{\rm\ s.t.\ }||f-p||<\frac{\varepsilon}{2}
        \end{equation*}
        令$M=\mathop{\rm max}\limits_{t\in [0,1]}|p'(t)|$,则$M<\infty$且
        \begin{equation*}
            |p(t+h)-p(t)|\leqslant M|h|,\forall t\in (0,1),\forall h{\rm\ with\ }t+h\in (0,1)
        \end{equation*}
        取分段仿射的连续函数$g$,使得
        \begin{enumerate}
            \item $||g||<\frac{\varepsilon}{2}$
            \item 各段斜率的绝对值$>M+n$
        \end{enumerate}
        则$p+g\in B(f,\varepsilon)$,但$p+g\notin A_n$,
        因为
        \begin{equation*}
            |(p+g)'(t)|\geqslant |g'(t)|-|p'(t)|>n
        \end{equation*}
    \end{proof}


\section{共鸣定理}
    \subsection*{定理内容}
    \begin{theorem}[一致有界原理,共鸣定理,UBP]
        $X$是Banach空间,$Y$是赋范空间,$\mathcal{F}\subset \mathcal{L}(X,Y)$,
        \begin{equation*}
            \forall x\in X,\mathop{\rm sup}\limits_{T\in \mathcal{F}}||Tx||<\infty \Rightarrow
            \mathop{\rm sup}\limits_{T\in \mathcal{F}}||T||<\infty
        \end{equation*}
        等价地,
        \begin{equation*}
            \mathop{\rm sup}\limits_{T\in \mathcal{F}}||T||=+\infty \Rightarrow 
            \exists x_0\in X{\rm\ s.t.\ }
            \mathop{\rm sup}\limits_{T\in \mathcal{F}}||Tx_0||=+\infty
        \end{equation*}
    \end{theorem}
    \begin{proof}
        令
        \begin{equation*}
            F_n=\{ x\in X: 
            \mathop{\rm sup}\limits_{T\in \mathcal{F}}||Tx||\leqslant n
            \}=\bigcap_{T\in \mathcal{F}}
            \{ x\in X:||Tx||\leqslant n \}
        \end{equation*}
        因为$T$连续,$\{ x\in X:||Tx||\leqslant n \}$是闭集,进而$F_n$是闭集,
        \begin{align*}
            \forall x\in X,\mathop{\rm sup}\limits_{T\in \mathcal{F}}||Tx||<\infty
            \Rightarrow & X=\bigcup_{n=1}^\infty F_n\\
            \mathop{\Rightarrow}\limits^{\rm BCT1}
            & \exists n_0{\rm\ s.t.\ }F_{n_0}\mbox{有内点}\\
            \Rightarrow & \exists B(x_0,r)\subset F_{n_0}\\
            \Rightarrow & ||T(x_0+rx)||\leqslant n_0,\forall x\in B(0,1),\forall T\in\mathcal{F}\\
            \Rightarrow & ||T(rx)||\leqslant n_0+||Tx_0||\leqslant 2n_0\\
            \Rightarrow & ||Tx||\leqslant \frac{2n_0}{r},\forall x\in B(0,1),\forall T\in\mathcal{F}\\
            \Rightarrow & \mathop{\rm sup}\limits_{T\in\mathcal{F}}
            \mathop{\rm sup}\limits_{x\in X,||x||\leqslant 1}
            ||Tx||\leqslant \frac{2n_0}{r} 
        \end{align*}
    \end{proof}
\subsection*{应用和推论}
    \begin{theorem}[Banach-Steinhaus]
        $X$是Banach空间,$Y$是赋范空间,$\overline{M}=X$,
        $T,T_n\in\mathcal{L}(X,Y),n=1,2,\cdots$
        \begin{equation*}
            T_nx\rightarrow T_x,\forall x\in X\Leftrightarrow
            \mathop{\rm sup}\limits_n ||T_n||<\infty{\rm\ and\ }
            T_n x\rightarrow Tx,\forall x\in M
        \end{equation*}
    \end{theorem}
    \begin{proof}
        必要性:逐点收敛$\Rightarrow $逐点有界$\mathop{\Rightarrow }\limits^{\rm UBP}$一致有界。

        充分性:记$C=\mathop{\rm sup}\limits_n ||T_n||$,
        \begin{align*}
            \overline{M}=X\Rightarrow & \forall x\in X,\forall \varepsilon>0,\exists y\in M
            {\rm\ s.t.\ }||x-y||<\frac{\varepsilon}{4( ||T||+C )}\\
            \Rightarrow & 
            ||T_nx-Tx||\leqslant 
            \mathop{||T_nx-T_ny||}\limits_{\leqslant C||x-y||}+
            \mathop{||T_ny-Ty||}\limits_{<\frac{\varepsilon}{2},n\mbox{充分大}}+
            \mathop{||T_y-Tx||}\limits_{\leqslant ||T||||x-y||}
        \end{align*}
    \end{proof}
    \begin{theorem}
        $X,Y$是Banach空间,$T_n\in \mathcal{L}(X,Y),n=1,2,\cdots$,如果
        $\forall x\in X$,$\mathop{\rm lim}\limits_{n\rightarrow\infty}T_nx$存在,定义
        $T:X\rightarrow Y,X\mapsto\mathop{\rm lim}\limits_{n\rightarrow\infty}T_nx$,
        则$T\in \mathcal{L}(X,Y)$,且
        \begin{equation*}
            ||T||\leqslant \mathop{\rm liminf}\limits_{n\rightarrow\infty}||T_n||
        \end{equation*}
    \end{theorem}
    \begin{proof}
        见习题2.3.7.
    \end{proof}
    我们设
    \begin{equation*}
        S_N(f)(x)=\sum_{k=-N}^N \hat{f}(k){\rm e}^{2\pi{\rm i}kx}
    \end{equation*}
    问:是否$\forall f\in C(\Pi)$,$S_N(f)(x)\rightarrow f(x),\forall x\in [-\frac{1}{2},\frac{1}{2})$?
    \begin{theorem}[Du Bois-Reymond,1876]
        $\exists f\in C(\Pi){\rm\ s.t.\ }S_N(f)(0)$发散。
    \end{theorem}
    \begin{proof}
        \begin{equation*}
            S_N(f)(x)=(f* D_N)(x)
        \end{equation*}
        其中
        \begin{equation*}
            D_N(t)\defeq \frac{\sin{ [(2N+1)\pi t] }}{\sin{ (\pi t) }}
        \end{equation*}
        定义$T_N:C(\Pi)\rightarrow \R,f\mapsto S_N(f)(0)$,
        \begin{align*}
            \Rightarrow & |T_N(f)|=\left| \int_{-\frac{1}{2}}^{\frac{1}{2}} f(t)D_N(-t)\d t \right|\leqslant ||D_N||_1 \cdot ||f||\\
            \Rightarrow & T_N\in C(\Pi)^*,||T_N||\leqslant ||D_N||_1
        \end{align*}
        Claim:$||T_N||=||D_N||_1$.因为$D_N$在$[-\frac{1}{2},\frac{1}{2}]$上只有有限多个零点,故
        ${\rm sgn\ }D_N$只有有限个间断点。$\forall \varepsilon>0$,存在$f_\varepsilon\in C(\Pi)$,分段仿射使得
        \begin{enumerate}
            \item $||f_\varepsilon||=1$.
            \item $f_\varepsilon={\rm sgn\ }D_N {\rm\ on\ }[-\frac{1}{2},\frac{1}{2}]\backslash I_\varepsilon{\rm\ with\ }|I_\varepsilon|<\frac{\varepsilon}{4N+3}$.
        \end{enumerate}
        \begin{align*}
            \Rightarrow |T_N(f_\varepsilon)|=&\left| \int_{-\frac{1}{2}}^{\frac{1}{2}}f_\varepsilon(t)D_N(t)\d t \right|\\
            \geqslant & \int_{[-\frac{1}{2},\frac{1}{2}]\backslash I_\varepsilon} |D_N(t)|\d t-\int_{I_\varepsilon}|D_N(t)|\d t\\
            \geqslant & ||D_N||_1-2\int_{I_\varepsilon}|D_N(t)|\d t\\
            >& ||D_N||_1-\varepsilon\\
            \Rightarrow ||T_N||\geqslant & \frac{|T_N(f_\varepsilon)|}{||f_\varepsilon||}>||D_N||_1-\varepsilon
        \end{align*}
        令$\varepsilon\rightarrow0$则Claim得证,而
        \begin{align*}
            ||D_N||_1=& 2\int_0^\frac{1}{2}\left| \frac{\sin [(2N+1)\pi t]}{\sin (\pi t)} \right|\d t\\
            \geqslant& 2\int_0^\frac{1}{2}\left| \frac{\sin [(2N+1)\pi t]}{\sin (\pi t)} \right|\d t\\
            \eq{x=(2N+1)\pi t}&\frac{2}{\pi}\int_0^{\frac{\pi}{2}(2N+1)}\left| \frac{\sin{x}}{x} \right|\d x\rightarrow \infty{\rm\ as\ }N\rightarrow\infty
        \end{align*}
        于是
        \begin{align*}
            {\rm Claim}\Rightarrow& \fun{sup}{N}||T_N||=+\infty\\
            \Ra{\rm UBP}& \exists f\in C(\Pi){\rm\ s.t.\ }\fun{sup}{N}|T_N(f)|=+\infty\\
            \Rightarrow& \fun{limsup}{N\rightarrow\infty}|S_N(f)(0)|=+\infty\\
            \Rightarrow& \{ S_N(f)(0) \}_{N=1}^\infty\mbox{发散}
        \end{align*}
    \end{proof}


\section{开映射定理}
    方程$Tx=y$,当$y$变化很小时$x$是不是变化很小(解的稳定性)?这就要考虑$T^{-1}$的连续性。
\subsection*{定理内容}
    \begin{definition}
        将任意开集都映为开集的映射,称为开映射。
    \end{definition}
    
    \begin{theorem}[开映射定理,OMT]
        $X,Y$是Banach空间,$T\in \mathcal{L}(X,Y)$,则
        $T$是满射$\Rightarrow T$是开映射。
    \end{theorem}
    \begin{lemma}
        $X,Y$是Banach空间,$T\in \mathcal{L}(X,Y)$,如果$T$满射,则$\exists \delta>0{\rm\ s.t.\ }\delta B_Y\subset T(B_X)$.
        \begin{proof}
            \textbf{Step1}:$\exists r>0{\rm\ s.t.\ }rB_Y\subset\overline{T(B_X)}$,
            \begin{align*}
                X=\bigcup_{n=1}^\infty nB_X\Ra{T\mbox{满}}&Y=T(X)=\bigcup_{n=1}^\infty T(nB_X)\\
                \Rightarrow& Y=\bigcup_{n=1}^\infty \overline{T(nB_X)}\\
                \Ra{\rm BCT}& \exists n_0{\rm\ s.t.\ }\overline{T(n_0B_X)}\mbox{有内点}\\
                \Rightarrow& \exists B_Y(y_0,t)\subset \overline{T(n_0B_X)}
            \end{align*}
            令$r\defeq t/n_0$,Claim:$rB_y\subset \overline{T(B_X)}$.
            \begin{align*}
                \forall y\in rB_Y,y_0\pm n_0y\in B_Y(y_0,t)\subset \overline{T(n_0 B_X)}\Rightarrow &
                \exists \{x_n\}_{n=1}^\infty,\{x_n'\}_{n=1}^\infty \subset n_0B_X{\rm\ s.t.\ }\\
                &Tx_n\rightarrow y_0+n_0y,Tx_n'\rightarrow y_0-n_0y\\
                \Rightarrow& T\left( \frac{x_n-x_n'}{2n_0} \right)\rightarrow y\\
                \Rightarrow& y\in \overline{T(B_X)} 
            \end{align*}

            \textbf{Step2}:令$\delta\defeq r/3$,则$\delta B_Y\subset T(B_X)$,即
            \begin{equation*}
                \forall y\in\delta B_Y,\exists x\in B_X{\rm\ s.t.\ }Tx=y
            \end{equation*}
            \begin{align*}
                {\rm Step1}\Rightarrow &\forall y\in \delta B_Y,3y\in rB_Y\subset\overline{T(B_X)}\\
                \Rightarrow &\exists \tilde{x}_1\in B_X{\rm\ s.t.\ }||3y-T\tilde{x}_1||_Y<\delta
            \end{align*}
            令$x_1\defeq \tilde{x}_1/3$,则$||y-Tx_1||_Y<\delta/3$.
            令$y_1\defeq y-Tx_1$,则
            \begin{align*}
                y_1\in \frac{\delta}{3}B_Y\Rightarrow& 9y_1\in rB_Y\subset \overline{T(B_X)}\\
                \Rightarrow& \exists x_2\in \frac{1}{3^2}B_X{\rm\ s.t.\ }||y_1-Tx_2||_Y<\frac{\delta}{3^2}\\
                &\vdots\\
                &y_n\defeq y_{n-1}-Tx_n\in\frac{\delta}{3^n}B_Y\\
                &\exists x_{n+1}\in \frac{1}{3^{n+1}}B_X{\rm\ s.t.\ }||y_n-Tx_{n+1}||_Y<\frac{\delta}{3^{n+1}}\\
                \Rightarrow& \ms{ \sum_{k=n+1}^{n+p}x_k }_{X}<\frac{\frac{1}{3^{n+1}}}{1-\frac{1}{3}}<\frac{1}{2^n}\\
                \Rightarrow& \left\{ \sum_{k=1}^n x_k \right\}_{n=1}^\infty \mbox{是}X\mbox{中的基本列}\\
                \Ra{X\ Banach}&\exists x\in X{\rm\ s.t.\ }\sum_{k=1}^n x_k\rightarrow x{\rm\ and\ }||x||_X\leqslant \ms{ x-\sum_{k=1}^N x_k}_{X}+\ms{ \sum_{k=1}^N x_k}_{X}<1\\
                \Rightarrow& x\in B_X{\rm\ and\ }\frac{\delta}{3^n}>||y_n||_Y=||y_{n-1}-Tx_n||_Y=\cdots=||y-T(x_1+\cdots+x_n)||_Y\\
                \Rightarrow& T\left( \sum_{k=1}^n x_k \right)\rightarrow y\\
                \Rightarrow& Tx=y
            \end{align*}
        \end{proof}
    \end{lemma}
    \begin{proof}
        设$U$是$X$上的开集,$\forall y\in T(U)$,$\exists x\in U{\rm\ s.t.\ }Tx=y$,令$V\defeq U-x$,
        \begin{align*}
            0\in V\mathop{\subset}\limits^{\rm open}X
            \Rightarrow & \exists t>0{\rm\ s.t.\ }tB_X\subset V\\
            \Ra{\rm Lemma} & \exists \delta>0{\rm\ s.t.\ }\delta B_Y\subset T(B_X)\subset \frac{1}{t}T(V)\\
            \Rightarrow & 0\mbox{是$T(V)$的内点}\\
            T(U)=T(V)+Tx \Rightarrow& y=Tx\mbox{是$T(U)$的内点}
        \end{align*}
    \end{proof}
    
    \begin{theorem}[逆算子定理,IMT]
        $X,Y$是Banach空间,$T\in \mathcal{L}(X,Y)$,则
        $T$是双射$\Rightarrow T^{-1}\in\mathcal{L}(Y,X)$.
    \end{theorem}
    \begin{proof}
        $f:X\rightarrow Y$连续$\Leftrightarrow \forall $开集
        $U\subset Y,f^{-1}(U)$是$X$上的开集。

        $T^{-1}:Y\rightarrow X$连续$\Leftrightarrow \forall $开集
        $U\subset X$,由OMT,$(T^{-1})^{-1}(U)=T(U)$是$Y$上的开集。
    \end{proof}
\subsection*{应用和推论}
    \begin{theorem}[Lax-Milgram]
        $H$是Hilbert空间,如果共轭双线性函数$a(\cdot,\cdot)$满足
        \begin{enumerate}
            \item 连续:$\exists C>0$使得$|a(x,y)|\leqslant C||x||||y||,\forall x,y\in H$.
            \item 强制(coersive):$\exists\delta>0$使得$\delta ||x||^2\leqslant a(x,x),\forall x\in H$.
        \end{enumerate}
        则存在唯一$A\in\mathcal{L}(H)$使得
        \begin{enumerate}[$1^\circ$]
            \item $a(x,y)=\ag{x,Ay},x,y\in H$.
            \item $A^{-1}$存在、有界且$||A^{-1}||\leqslant \frac{1}{\delta}$.
        \end{enumerate}
    \end{theorem}
    \begin{proof}
        Claim1:$A$是双射:
        \begin{enumerate}[(1).]
            \item $A$是单射:对于线性映射而言,单射$\Leftrightarrow {\rm Ker}(A)=A^{-1}(\{0\})=\{0\}$.
                \begin{align*}
                    Ay=0\Rightarrow& a(x,y)=\ag{x,Ay}=0,\forall x\in H\\
                    \Rightarrow& 0=a(y,y)\geqslant \delta||y||^2\\
                    \Rightarrow& y=0
                \end{align*}
            \item $A$是满射:先证明${\rm Ran}(A)$闭,设${\rm Ran}(A)\ni Ax_n\rightarrow y$,
                \begin{align*}
                    &\delta||x_n-x_m||^2\leqslant a(x_n-x_m,x_n-x_m)
                    =\ag{x_n-x_m,Ax_n-Ax_m}\leqslant ||x_n-x_m||\cdot ||Ax_n-Ax_m||\\
                    \Rightarrow&
                    ||x_n-x_m||\leqslant \frac{1}{\delta}||Ax_n-Ax_m||\rightarrow 0{\rm\ as\ }n,m\rightarrow\infty\\
                    \Rightarrow&\{x_n\}_{n=1}^\infty\mbox{是基本列}
                \end{align*}
                设$x_n\rightarrow x\in H$,
                $A$连续所以$Ax_n\rightarrow Ax$,而$Ax_n\rightarrow y$,
                所以$y=Ax\in{\rm Ran}(A)$,那么$H={\rm Ran}(A)\oplus {\rm Ran}(A)^\perp$.
                为了证明$A$满射,只需证明${\rm Ran}(A)^\perp=\{0\}$,
                \begin{equation*}
                    y\in {\rm Ran}(A)^\perp\Rightarrow 
                    \ag{y,Ax}=a(y,x)=0,\forall x\in H
                \end{equation*}
                特别地,考虑$x=y$,则$0=a(y,y)\geqslant \delta||y||^2\Rightarrow y=0$.
        \end{enumerate}
        那么由IMT,$A^{-1}\in\mathcal{L}(H)$,
        \begin{align*}
            \delta||x||^2\leqslant a(x,x)=\ag{x,Ax}\leqslant ||x||\cdot ||Ax||
            \Rightarrow& ||x||\leqslant \frac{1}{\delta}||Ax||,\forall x\in H\\
            \Rightarrow& ||A^{-1}y||\leqslant \frac{1}{\delta}||y||,\forall y\in H\\
            \Rightarrow& ||A^{-1}||\leqslant \frac{1}{\delta}
        \end{align*}
    \end{proof}

    \begin{theorem}[等价范数定理]
        $(X,\ms{\cdot}_{1}),(X,\ms{\cdot}_{2})$都是Banach空间,
        则$\ms{\cdot}_{2}\lesssim \ms{\cdot}_{1}\Rightarrow 
        \ms{\cdot}_{1}\cong \ms{\cdot}_{2}$.
    \end{theorem}
    \begin{proof}
        考虑恒等映射$I:(X,\ms{\cdot}_{1})\rightarrow (X,\ms{\cdot}_{2}),x\mapsto x$.
        \begin{align*}
            \exists C>0{\rm\ s.t.\ }\ms{x}_{2}\leqslant C\ms{x}_{1},\forall x\in X
            \Rightarrow& \frac{||I(x)||_2}{||x||_1}\leqslant C \\
            \Rightarrow& I\in\mathcal{L}( (X,\ms{\cdot}_{1}),(X,\ms{\cdot}_{2}) )\\
            \Ra{\rm IMT}&I^{-1}\in\mathcal{L}( (X,\ms{\cdot}_{1}),(X,\ms{\cdot}_{2}) )\\
            \Rightarrow&\exists C'>0{\rm\ s.t.\ }\ms{x}_{1}\leqslant C'\ms{x}_{2},\forall x\in X\\
            \Rightarrow&\frac{1}{C'}\ms{x}_{1}\leqslant \ms{x}_{2}\leqslant C\ms{x}_{1},\forall x\in X
        \end{align*}
    \end{proof}


\section{闭图像定理}
    \begin{definition}[乘积度量空间]
        $(X,\ms{\cdot}_{X}),(Y,\ms{\cdot}_{Y})$是两个度量空间,定义:
        \begin{equation*}
            \ms{ (x,y) }_{X\times Y}\defeq \ms{x}_{X}+\ms{y}_{Y}
        \end{equation*}
        可以证明,$( X\times Y,\ms{\cdot}_{X\times Y} )$构成一个新的度量空间,称为乘积空间。
    \end{definition}
    \begin{corollary}
        $X,Y$都是Banach空间,则其乘积空间$X\times Y$也是Banach空间。
    \end{corollary}
    \begin{definition}[闭算子]
        $T:X\rightarrow Y$是线性映射,
        \begin{equation*}
            {\rm Gr}(T)\defeq \{ (x,Tx):x\in {\rm Dom}(T) \}
        \end{equation*}
        称为$T$的图像,其中${\rm Dom}(T)\subset X$是指$T$的定义域,为$X$的子集.

        如果${\rm Gr}(T)$是$X\times Y$的闭子空间,则称$T$是闭算子。
    \end{definition}
    \begin{lemma}
        $T$是闭算子当且仅当
        \begin{equation*}
            {\rm Dom(T)}\ni x_n\rightarrow x,Tx_n\rightarrow y
        \end{equation*}
        蕴含(imply)
        \begin{equation*}
            x\in {\rm Dom}(T),y=Tx
        \end{equation*}
        即,如果$T$将${\rm Dom}(T)$上的收敛列$\{x_n\}$映为收敛列$\{y_n=Tx_n\}$,
        则收敛列$\{x_n\}$的极限$x\in {\rm Dom}(T)$且收敛列$\{y_n=Tx_n\}$的极限为$y=Tx$.
    \end{lemma}
    \begin{proof}
        ${\rm Gr}(T)$闭当且仅当:
        \begin{equation*}
            (x_n,Tx_n)\rightarrow (x,y)\Rightarrow (x,y)\in G_r(T)
        \end{equation*}
    \end{proof}

    \begin{remark}
        ${\rm Dom}(T)$不一定是闭集。
    \end{remark}

    \begin{example}[无界的闭算子]
        \begin{equation*}
            T=\frac{\d}{\d t}:C[0,1]\rightarrow C[0,1],\ {\rm Dom}(T)=C^1[0,1]
        \end{equation*}
    \end{example}

    \begin{proposition}
        $A$有界,$D={\rm Dom}(A)$闭,则$A$闭。
    \end{proposition}
    \begin{proof}
        设$\{x_n\}\subset D$收敛到$x$,$Ax_n\rightarrow y$,则
        $D$闭$\Rightarrow x\in D$,$A$连续$\Rightarrow Ax_n\rightarrow Ax$,所以$Ax=y$,$A$是闭算子。
    \end{proof}

\subsection*{定理内容}
    \begin{theorem}[B.L.T.]
        $X$是赋范空间,$Y$是Banach空间,
        任一$T\in \mathcal{ L }( {\rm Dom}(T),Y )$可唯一地、保范数地延拓为
        $\tilde{T}\in \mathcal{L}( \overline{{\rm Dom}(T)},Y )$.即
        \begin{equation*}
            \tilde{T}|_{ {\rm Dom}(T) }=T{\rm\ and\ }||\tilde{T}||=||T||.
        \end{equation*}
    \end{theorem}
    \begin{proof}
        \begin{align*}
            &\forall x\in 
            \overline{{\rm Dom}(T)},\exists x_n\in {\rm Dom}(T),n=1,2,\cdots{\rm\ s.t.\ }x_n\rightarrow x\\
            \Ra{ T\mbox{有界}}&||Tx_n-Tx_m ||\leqslant ||T|| ||x_n-x_m||\rightarrow 0{\rm\ as\ }n,m\rightarrow\infty\\
            \Rightarrow& \{Tx_n\}_{n=1}^\infty \mbox{是$Y$中的基本列}\\
            \Ra{Y\rm\ Banach}& \exists y\in Y{\rm\ s.t.\ }Tx_n\rightarrow y
        \end{align*}
        定义映射
        \begin{equation*}
            \tilde{T}:\overline{{\rm Dom}(T)}\rightarrow Y,x\mapsto y
        \end{equation*}
        容易验证良定,且$\tilde{T}$是线性映射。下面证明$\tilde{T}$有界:
        \begin{align*}
            &\forall x\in \overline{ {\rm Dom}(T) }, 
            ||\tilde{T}x||=||y||=\fun{lim}{n\rightarrow \infty}||Tx_n||
            \leqslant \fun{lim}{n\rightarrow\infty} ||T||||x_n||=||T||||x||\\
            \Rightarrow& \tilde{T}\in\mathcal{ L }(\overline{ {\rm Dom}(T) },Y){\rm\ and\ }||\tilde{T}||\leqslant ||T||
        \end{align*}
        另一方面,平凡地,$||\tilde{T}||\geqslant ||T||$.
    \end{proof}

    \begin{example}[Fourier变换]
        \begin{equation*}
            \hat{f}(\xi)\defeq \int_{\R^n}f(x){\rm e}^{-2\pi {\rm i}x\cdot \xi}\d x
        \end{equation*}
        其中$f\in L^1$,如何在$L^2$上定义?

        $L^1\cap L^2$在$L^2$上稠密,$||\hat{f}||_2=||f||_2,\forall f\in L^1\cap L^2$(Planchered),
        由B.L.T.可得$\mathcal{F}:f\mapsto \hat{f}$可唯一地、保范数地延拓到$L^2$上。
    \end{example}
    
    \begin{remark}
        由B.L.T和命题2.6.1,有界算子可以将其定义域延拓为闭集,从而成为闭算子。
    \end{remark}

    \begin{theorem}[闭图像定理,CGT]
        $X,Y$是Banach空间,$T:X\rightarrow Y$是闭线性算子,如果
        ${\rm Dom}(T)$是闭集,则$T$有界。
    \end{theorem}
    \begin{proof}
        ${\rm Gr}(T)$是$X\times Y$的闭子空间,所以$({\rm Gr}(T),||\cdot||_{X\times Y})$是Banach空间,定义:
        \begin{equation*}
            \Pi_1:{\rm Gr}(T)\rightarrow {\rm Dom}(T),\ (x,Tx)\mapsto x
        \end{equation*}
        \begin{equation*}
            \Pi_2:{\rm Gr}(T)\rightarrow Y,\ (x,Tx)\mapsto Tx
        \end{equation*}
        并设$T=\Pi_2\prod \Pi_1^{-1}$,
        \begin{equation*}
            \Pi_1\mbox{是双射}\mathop{\Rightarrow}\limits_{ {\rm Dom}(T)\mbox{闭用在这里} }^{ \rm IMT }
            \Pi^{-1}\mbox{有界}\Rightarrow T=\Pi_1\circ\Pi_1^{-1}\mbox{有界}
        \end{equation*}
    \end{proof}
    \begin{proof}
        $( {\rm Dom}(T),||\cdot||_X )$是Banach空间,令
        \begin{equation*}
            ||x||_G\defeq ||x||_X+||Tx||_Y,\ x\in {\rm Dom}(T)
        \end{equation*}
        {\rm Claim}:$( {\rm Dom}(T),||\cdot||_G )$也是Banach空间。实际上,
        \begin{equation*}
            ||x_n-x_m||_G=||x_n-x_m||_X+||Tx_n-Tx_m||_Y\rightarrow 0{\rm\ as\ }n,m\rightarrow\infty
        \end{equation*}
        $X,Y$是Banach,所以存在$x_n\rightarrow x,Tx_n\rightarrow y$,因为$T$闭所以
        $x\in{\rm Dom}(T),y=Tx$.于是
        \begin{equation*}
            ||x_n-x||_G=||x_n-x||_X+||Tx_n-Tx||_Y\rightarrow 0
        \end{equation*}
        那么$||\cdot||_X$弱于$||\cdot||_G$,因此$||\cdot||_X$等价于$||\cdot||_G$,
        所以存在$C>0$使得
        \begin{equation*}
            ||Tx||_Y\leqslant ||x||_G\leqslant C||x||_X,\forall x\in {\rm Dom}(T)
        \end{equation*}
    \end{proof}
\subsection*{应用和推论}
    \begin{example}[Hellinger-Toeplitz]
        $H$是Hilbert空间,如果$T:H\rightarrow H$满足
        $\ag{Tx,y}=\ag{x,Ty},\forall x,y\in H$,则$T$有界。
    \end{example}
    \begin{proof}
        先证明$T$闭:设$x_n\rightarrow x$,$Tx_n\rightarrow y$,存在$\delta\in H$使得
        \begin{equation*}
            \ag{\delta,Tx}=\ag{T\delta,x}
            =\fun{lim}{n\rightarrow\infty}\ag{T\delta,x_n}
            =\fun{lim}{n\rightarrow\infty}\ag{\delta,Tx_n}
            =\ag{\delta,y}
        \end{equation*}
        所以$Tx=y$.于是由CGT可知$T$有界。
    \end{proof}


\section{Hahn-Banach定理}
\subsection{代数形式——线性泛函的延拓}
    \begin{definition}
        $X$是向量空间,如果函数$p:X\rightarrow \R$使得
        \begin{enumerate}
            \item 正齐性:$p(tx)=tp(x),\forall x\in X,t>0$.
            \item 次可加性:$p(x+y)\leqslant p(x)+p(y),\forall x,y\in X$.
        \end{enumerate}
        则称$p$是$X$上的一个次线性泛函。

        如果$p$还满足齐次性,即
        \begin{equation*}
            p(\lambda x)=|\lambda |p(x),\forall x\in X,\forall \lambda\in \K
        \end{equation*}
        则称$p$是一个半范数。
    \end{definition}
    \begin{remark}
        \begin{enumerate}
            \item 半范数非负:$\forall x\in X,2p(x)=p(x)+p(-x)\geqslant p(0)=0$.
            \item 如果半范数$p$满足$p(x)=0\Rightarrow x=0$,则$p$是一个范数。
        \end{enumerate}
    \end{remark}
    
    \begin{theorem}[HBT for real version]
        设$X$为实向量空间,$p$是$X$上次线性泛函,$M$是$X$的子空间,
        $f$是$M$上的线性泛函,并满足$f(x)\leqslant p(x),\forall x\in M$.
        则存在$X$上的线性泛函$F$满足
        \begin{enumerate}
            \item $F|_M=f$.
            \item $F(x)\leqslant p(x),\forall x\in X$.
        \end{enumerate}
    \end{theorem}
    \begin{lemma}
        在定理条件下,设$x_0\in X\backslash M$,定义
        \begin{equation*}
            \tilde{M}\defeq M\oplus {\rm span}{x_0}
        \end{equation*}
        则存在线性映射$\tilde{f}:\tilde{M}\rightarrow\R$,满足
        \begin{enumerate}
            \item $\tilde{f}|_M=f$.
            \item $\tilde{f}(x)\leqslant p(x),\forall x\in \tilde{M}$.
        \end{enumerate}
        \begin{proof}
            \begin{align*}
                &\forall x,y\in M,f(x)+f(y)=f(x+y)\leqslant p(x+y)\leqslant p(x-x_0)+p(y+x_0)\\
                \Rightarrow&
                f(x)-p(x-x_0)\leqslant p(y+x_0)-f(y)\\
                \Rightarrow& \fun{sup}{x\in M}[ f(x)-p(x-x_0) ]
                \leqslant \fun{inf}{y\in M}[ p(y+x_0)-f(y) ]\\
                \Rightarrow& \exists \beta\in\R{\rm\ s.t.\ }
                f(x)-p(x-x_0)\leqslant \beta\leqslant p(y+x_0)-f(y),\forall x,y\in M\tag{*}
            \end{align*}
            令
            \begin{equation*}
                \tilde{f}:\tilde{M}\rightarrow \R,x+\lambda x_0\mapsto f(x)+\lambda\beta
            \end{equation*}
            于是$\tilde{f}$是线性映射,且$\tilde{f}|_M=f$.下面证明:
            \begin{equation*}
                \tilde{f}(x+\lambda x_0)\leqslant p(x+\lambda x_0),\forall x\in M,\forall \lambda\in\R
            \end{equation*}
            $\lambda=0$显然,$\lambda\neq 0$时,
            不妨设$\lambda>0$,(*)式中$x,y$均代以$\frac{x}{\lambda}$可得
            \begin{align*}
                & f(\frac{x}{\lambda})-p(\frac{x}{\lambda}-x_0)\leqslant \beta\leqslant p(\frac{x}{\lambda}+x_0)-f(\frac{x}{\lambda})\\
                \Rightarrow &
                f(x)-p(x-\lambda x_0)\leqslant \lambda\beta\leqslant p(x+\lambda x_0)-f(x)\\
                \Rightarrow &
                \left\{ \begin{array}{l}
                    f(x)-\lambda \beta=\tilde{f}(x-\lambda x_0)\leqslant p(x-\lambda x_0)\\
                    f(x)+\lambda \beta=\tilde{f}(x+\lambda x_0)\leqslant p(x+\lambda x_0)
                \end{array} \right.
            \end{align*}
        \end{proof}
    \end{lemma}
    \begin{proof}
        对两个线性泛函$g,h$,如果${\rm Dom}(g)$是
        ${\rm Dom}(h)$的闭子空间,且
        $h_{ {\rm Dom}(g) }=g$,则称$h$是$g$的一个延拓。令
        \begin{equation*}
            \mathcal{F}\defeq \{ g:g\mbox{是}f\mbox{的延拓,}g(x)\leqslant p(x),\forall x\in {\rm Dom}(g) \}
        \end{equation*}
        引入偏序:
        \begin{equation*}
            g\leqslant h\Leftrightarrow h\mbox{是}g\mbox{的延拓}
        \end{equation*}
        设$\mathcal{T}$是$\mathcal{F}$的任一全序子集,令
        \begin{equation*}
            Y\defeq \bigcup_{g\in \mathcal{T}}{\rm Dom}(g)
        \end{equation*}
        于是$Y$是$X$的闭子空间,令
        \begin{equation*}
            G:Y\rightarrow\R,x\mapsto g(x){\rm\ if\ }x\in {\rm Dom}(g)
        \end{equation*}
        $\mathcal{T}$全序$\Rightarrow G$良定且是$\mathcal{T}$的一个上界,
        由Zorn引理可得$\mathcal{F}$有极大元$F$,下面证明${\rm Dom}(F)=X$,从而$F$即为所求。

        假设不然,即存在$x_0\in X\backslash {\rm Dom}(F)$,那么由引理可得
        存在${\rm Dom}(F)\oplus {\rm span}\{x_0\}$上的线性泛函$\tilde{F}$,满足
        \begin{enumerate}
            \item $\tilde{F}|_{{\rm Dom}(F)}=F$.
            \item $\tilde{F}(x)\leqslant p(x),\forall x\in {\rm Dom}(F)\oplus {\rm span}\{x_0\}$.
        \end{enumerate}
        于是$\tilde{F}\in \mathcal{F}$且$F\leqslant \tilde{F}$,这与$F$的极大性矛盾。
    \end{proof}

    \begin{theorem}[HBT for complex version]
        设$X$为复向量空间,$p$是$X$上次线性泛函,$M$是$X$的子空间,
        $f:M\rightarrow \C$是$M$上的线性泛函,
        并满足$|f(x)|\leqslant p(x),\forall x\in M$.
        则存在$X$上的线性泛函$F:X\rightarrow\C$满足
        \begin{enumerate}
            \item $F|_M=f$.
            \item $|F(x)|\leqslant p(x),\forall x\in X$.
        \end{enumerate}
    \end{theorem}
    \begin{proof}
        \textbf{Step1}:先把$X$看作实向量空间,令
        $g\defeq {\rm Re\ }f$,则$g$是$M$上的实线性泛函,且满足
        \begin{equation*}
            g(x)\leqslant |f(x)|\leqslant p(x),\forall x\in M
        \end{equation*}
        那么由实HBT,存在线性映射$G:X\rightarrow \R$,满足
        \begin{enumerate}
            \item $G|_M=g$.
            \item $G(x)\leqslant p(x),\forall x\in X$.
        \end{enumerate}        

        \textbf{Step2}:复化。令
        \begin{equation*}
            F(x)\defeq G(x)-\i G( \i x )
        \end{equation*}
        显然有
        \begin{align*}
            &F(x+y)=F(x)+F(y)\\
            &F(\alpha x)=\alpha F(x),\forall x\in X,\forall \alpha\in \R
        \end{align*}
        所以
        \begin{equation*}
            F( (\alpha_1+\i \alpha_2)x )
            =F(\alpha_1 x)+F(\i \alpha_2 x)
            =\alpha_1 F(x)+\alpha_2 F(\i x),\forall x\in X,\forall \alpha_1,\alpha_2\in\R
        \end{equation*}
        故只需证明$F(\i x)=\i F(x)$,
        \begin{equation*}
            F(\i x)=G(\i x)-\i G(-x)
            =G(\i x)+\i G(x)
            =\i (-\i G(\i x)+G(x))
            =\i F(x)
        \end{equation*}
        得证。

        \textbf{Step3}:证明$F|_M=f$.
        \begin{align*}
            \forall x\in M,F(x)&=G(x)-\i G(\i x)\\
            &=g(x)-\i g(\i x)\\
            &={\rm Re\ }f(x)-\i\cdot {\rm Re\ }f(\i x)\\
            &={\rm Re\ }f(x)-\i\cdot {\rm Re\ }\{\i f(x)\}\\
            &={\rm Re\ }f(x)+\i\cdot {\rm Im\ }f(x)=f(x)
        \end{align*}

        \textbf{Step4}:证明$|F(x)|\leqslant p(x),\forall x\in X$.
        $F(x)=0$时显然,设$F(x)\neq 0$,存在$\theta\in\R$使得$|F(x)|=\e^{-\i \theta}F(x)$,于是
        \begin{equation*}
            |F(x)|=F(\e^{-\i \theta}x)
            =G(\e^{-\i \theta}x)-\i G( \i\e^{-\i \theta}x )
            =G(\e^{-\i \theta}x)\leqslant p(\e^{-\i \theta}x)
            =p(x)
        \end{equation*}
    \end{proof}

    \begin{theorem}[HBT]
        $X$为度量空间,$M$是其子空间,则
        \begin{equation*}
            \forall f\in M^*,\exists F\in X^*{\rm\ s.t.\ }
            F|_M=f{\rm\ and\ }||F||=||f||
        \end{equation*}
        这称为保范延拓。
    \end{theorem}
    \begin{proof}
        令
        \begin{equation*}
            p(x)\defeq ||f||\cdot ||x||,x\in X
        \end{equation*}
        则$|f(x)|\leqslant ||f||\cdot ||x||$,由复HBT可得
        存在$X$上线性泛函$F:X\rightarrow \C$满足
        $F|_M=f$且$|F(x)|\leqslant p(x)$,进而
        $|F(x)|\leqslant ||F||\cdot ||x||$,因此
        $F$是$X$上有界线性泛函,且$||F||\leqslant ||f||$.
        同时显然有$||F||\geqslant ||f||$.
    \end{proof}

    \begin{example}[HBT中延拓不唯一]
        $X=(\R^2,||\cdot||_1)$,其中$||(x_1,x_2)||_1\defeq |x_1|+|x_2|$,
        取$M=\R\times \{0\}$,$f:M\rightarrow\R,(x,0)\mapsto x$,那么$f$
        是$M$上有界线性泛函,且$||f||=1$.

        令
        \begin{equation*}
            F_t:\R^2\rightarrow\R,(x_1,x_2)\mapsto x_1+tx_2
        \end{equation*}
        那么$F_t|_M=f$,而且对于$\forall t\in (-1,1)$,
        \begin{equation*}
            |F_t(x_1,x_2)|=|x_1+tx_2|\leqslant |x_1|+|t||x_2|\leqslant ||(x_1,x_2)||_1
            \Rightarrow ||F_t||\leqslant 1
        \end{equation*}
    \end{example}
\subsection*{应用和推论}
    \begin{corollary}
        $\forall x_0\in X$,$\exists f\in X^*$满足$||f||=1$且$f(x_0)=||x_0||$.
    \end{corollary}
    \begin{proof}
        令$M\defeq {\rm span}\{x_0\}$,
        \begin{equation*}
            f_0:M\rightarrow\K,x=\lambda x_0\mapsto \lambda||x_0||
        \end{equation*}
        于是$|f_0(x)|\leqslant |\lambda| \cdot ||x_0||=||x||$,
        进而$f_0\in M^*$且$||f_0||=1$,由HBT可得存在$f\in X^*$满足
        $f|_M=f_0$,即$f(x_0)=f_0(x_0)=||x_0||$,同时$||f||=||f_0||=1$.
    \end{proof}

    \begin{corollary}
        $X\neq \{0\}\Rightarrow X^* \neq \{0\}$.
    \end{corollary}
    \begin{proof}
        取$0\neq x\in X$,由推论2.7.1可得存在$f\in X^*$满足$||f||=1$且
        $f(x)=||x||\neq 0$,此即为$0\neq f\in X^*$.
    \end{proof}

    \begin{corollary}
        $x,y\in X,x\neq y\Rightarrow \exists f\in X^*{\rm\ s.t.\ }f(x)\neq f(y)$.
    \end{corollary}
    \begin{proof}
        取$x_0=x-y\neq 0$,由推论2.7.1可得存在$f\in X^*$使得
        $f(x-y)=||x-y||\neq 0\Rightarrow f(x)\neq f(y)$.
    \end{proof}

    \begin{corollary}
        $f(x)=0,\forall f\in X^*\Rightarrow x=0$
    \end{corollary}
    \begin{proof}
        假设$x\neq 0$,由推论2.7.3可得存在$f\in X^*$使得
        $f(x)\neq f(0)=0$,矛盾。
    \end{proof}

    \begin{corollary}
        $||x||=\fun{sup}{f\in X^*,||f||=1}  |f(x)|$.
    \end{corollary}
    \begin{proof}
        $\forall f\in X^*$满足$||f||=1$,
        \begin{equation*}
            |f(x)|\leqslant ||f||\cdot ||x||=||x||
        \end{equation*}
        于是
        \begin{equation*}
            \fun{sup}{f\in X^*,||f||=1}  |f(x)|\leqslant ||x||
        \end{equation*}
        另一方面,存在$f\in X^*$满足$||f||=1$且$f(x)=||x||$,得证。
    \end{proof}

    \begin{theorem}
        $X$是度量空间,$M$是其子空间,$x_0\in X$满足
        $d={\rm dist}(x_0,M)>0$,则存在
        $f\in X^*$满足$||f||=1$且
        \begin{equation*}
            f(M)=\{0\},f(x_0)=d
        \end{equation*}
    \end{theorem}
    \begin{proof}
        令$\tilde{M}\defeq M\oplus {\rm span}\{x_0\}$,定义
        \begin{equation*}
            f_0:\tilde{M}\rightarrow \K,x=y+\lambda x_0\mapsto \lambda d
        \end{equation*}
        于是$f_0(M)=\{0\}$且$f(x_0)=d$,而且对于
        $\forall x=y+\lambda x_0$,其中$y\in M$,
        \begin{enumerate}[$1^\circ$]
            \item 如果$\lambda=0$,则$f_0(x)=0$.
            \item 如果$\lambda\neq 0$,
                \begin{equation*}
                    |f_0(x)|=|\lambda d|=|\lambda|\cdot d
                    \leqslant |\lambda |\cdot \ms{x_0+\frac{y}{\lambda}}
                    =\ms{y+\lambda x_0}=\ms{x}.
                \end{equation*}
                于是$f_0\in \tilde{M}^*$且$\ms{f_0}=1$,由HBT可得存在$f\in X^*$满足
                $f|_{\tilde{M}^*}=f_0$且$||f||=||f_0||\leqslant 1$,则
                $f(M)=\{0\},f(x_0)=d$.
        \end{enumerate}
        只需证明$||f||\geqslant 1$.
        \begin{align*}
            d=\fun{inf}{y\in M}\ms{x_0-y}\Rightarrow&
            \forall n,\exists y_n\in M{\rm\ s.t.\ }\ms{x_0-y_n}<d+\frac{1}{n}\\
            \Rightarrow& 
            \frac{ |f(x_0-y_n)| }{\ms{x_0-y_n}}
            =\frac{|f(x_0)|}{\ms{x_0-y_n}}>
            \frac{d}{d+\frac{1}{n}}\rightarrow 1{\rm\ as\ }n\rightarrow \infty\\
            \Rightarrow& \fun{sup}{n}\frac{ |f(x_0-y_n)| }{\ms{x_0-y_n}}\geqslant 1\\
            \Rightarrow& ||f||\geqslant 1
        \end{align*}
    \end{proof}

    \begin{theorem}
        $X$是度量空间,$M$是其子空间,$0\neq x_0\in X$,那么
        \begin{equation*}
            x_0\in \overline{ {\rm span\ }M }
            \Leftrightarrow 
            f(x_0)=0,\forall f\in X^*{\rm\ with\ }f(M)=\{0\}
        \end{equation*}
    \end{theorem}
    \begin{proof}
        必要性:设$x_0\in \overline{ {\rm span\ }M }$,对于$\forall f\in X^*$满足$f(M)=\{0\}$,有:
        \begin{equation*}
            f({\rm span\ }M)=f(\overline{ {\rm span\ }M })=\{0\}
        \end{equation*}
        从而$f(x_0)=0$.

        充分性:假设$x_0\notin \overline{ {\rm span\ }M }$,
        则
        \begin{equation*}
            d\defeq {\rm dist}(x_0,\overline{ {\rm span\ }M })>0
        \end{equation*}
        由定理2.7.4,存在$f\in X^*$满足$||f||=1$且
        \begin{equation*}
            f( \overline{ {\rm span\ }M} )=\{0\},f(x_0)=d>0
        \end{equation*}
        矛盾。
    \end{proof}
\subsection{几何形式——凸集分离}
    \begin{definition}
        $X$是向量空间,$C\subset X$,
        \begin{enumerate}
            \item 如果$-C=C$,称$C$对称。
            \item 如果$\forall x,y\in C$,$\forall t\in [0,1]$,都有$tx+(1-t)y\in C$,称$C$是凸集(Convex set)。
            \item 如果$\forall x\in X$,存在$t>0$使得$\frac{x}{t}\in C$,称$C$是吸收的。
        \end{enumerate}
    \end{definition}
    \begin{proposition}
        任一族凸集之交仍然是凸集。
    \end{proposition}
    \begin{definition}
        对于集合$A$,包含$A$的所有凸集之交称为$A$的凸包,记作
        \begin{equation*}
            {\rm conv}(A)\defeq 
            \bigcap_{{\rm Convex\ }C\supset A} C
        \end{equation*}
    \end{definition}
    \begin{definition}
        对于
        \begin{equation*}
            \sum_{k=1}^n \lambda_k=1,\{ \lambda_k \}_{k=1}^n
        \end{equation*}
        称
        \begin{equation*}
            \sum_{k=1}^n \lambda_k x_k
        \end{equation*}
        为$x_1,\cdots,x_n$的一个凸组合。
    \end{definition}

    \begin{proposition}
        ${\rm conv}(A)$就是$A$中向量的凸组合全体。
    \end{proposition}

    \begin{definition}
        $X$是向量空间,$C$是包含$0$的凸集,广义实值函数$P_C:X\rightarrow [0,+\infty]$,
        \begin{equation*}
            P_C(x)\defeq {\rm inf}\{ t>0:\frac{x}{t}\in C \}
        \end{equation*}
        称为$C$的Minkowski泛函。
        \begin{equation*}
            P_C(x)=+\infty\Leftrightarrow \{ t>0:\frac{x}{t}\in C \}=\varnothing
        \end{equation*}
    \end{definition}
    \begin{proposition}
        关于$P_C$,
        \begin{enumerate}
            \item $P_C(0)=0$.
            \item 正齐次性:$P_C(tx)=tP_C(x),\forall x\in X,\forall t>0$.
            \item 次可加性:$P_C(x+y)\leqslant P_C(x)+P_C(y)$
        \end{enumerate}
        注意$P_C$可能取$+\infty$,不一定是次线性泛函。
    \end{proposition}
    \begin{proof}
        只说明3.不妨设$P_C(x),P_C(y)\in \R$,
        \begin{equation*}
            \forall \varepsilon>0,\lambda\defeq P_C(x)+\frac{\varepsilon}{2},
            \mu\defeq P_C(y)+\frac{\varepsilon}{2}
        \end{equation*}
        于是
        \begin{align*}
            \frac{x}{\lambda},\frac{y}{\mu}\in C\Rightarrow&
            \frac{x+y}{\lambda+\mu}=\frac{\lambda}{\lambda+\mu}\frac{x}{\lambda}+\frac{\mu}{\lambda+\mu}\frac{y}{\mu}\in C\\
            \Rightarrow& \lambda+\mu\geqslant P_C(x+y)\\
            \Rightarrow& P_C(x+y)\leqslant P_C(x)+P_C(y)+\varepsilon
        \end{align*}
        令$\varepsilon\rightarrow 0$即得到结论。
    \end{proof}

    \begin{definition}
        $X$是复向量空间,$C$是包含$0$的凸集,
        如果$\forall x\in C,\forall \theta\in\R$,都有$\e^{\i \theta}x\in C$,则称$C$均衡。
    \end{definition}

    \begin{proposition}
        复向量空间中每个均衡、吸收凸集都决定了一个半范数。
    \end{proposition}
    \begin{proof}
        吸收$\Rightarrow P_C$是次线性泛函,均衡$\Rightarrow $齐次性。
    \end{proof}

    \begin{definition}
        $X$是实向量空间,$M$是闭子空间,
        称$M$是$X$的极大子空间是指
        任一$X$的闭子空间$Y$,若满足$M\subsetneqq Y$,则$Y=X$,
    \end{definition}

    \begin{proposition}
        $M$是极大子空间$\Leftrightarrow \exists x_0\in X$使得
        $X=M\oplus {\rm span}\{x_0\}$,也就是${\rm codim\ }M={\rm dim}(X/M)=1$.
        (习题2.4.8.)
    \end{proposition}
    \begin{proof}
        $(\Leftarrow)$:
        ${\rm dim}(X/M)=1$,则$\forall x_0\notin M$
        \begin{equation*}
            X/M=\{\lambda [x_0]:\lambda\in \K\}
        \end{equation*}
        假设存在线性子空间$S$使得$M\subsetneqq S$,则取
        $x_0\in S\backslash M$,从而
        \begin{equation*}
            {\rm span}{x_0}={\rm span}{x_0}\oplus M\subset S
        \end{equation*}
        而实际上$[\lambda x_0]=\{ \lambda x_0+m:m\in M \}$,因此
        \begin{equation*}
            X
            =\{ \lambda x_0+m:m\in M,\lambda\in\K \}
            ={\rm span}\{x_0\}\oplus M\subset S
        \end{equation*}
        只能$X=S$,与极大性矛盾。

        $(\Rightarrow )$:取$x_0\notin M$,则
        \begin{equation*}
            M\subsetneqq \bigcup_{\lambda\in\K}(ax_0+M)
            ={\rm span}\{x_0\}\oplus M
        \end{equation*}
        从而后者$=X$,所以${\rm dim}(X/M)=1$.
    \end{proof}

    \begin{definition}
        超平面是指极大子空间的平移。

        对于$X$上线性泛函$f$和$r\in\R$,
        \begin{equation*}
            H_f^r\defeq f^{-1}(\{r\})=\{ x\in X:f(x)=r \}
        \end{equation*}
    \end{definition}

    \begin{proposition}
        $L$是超平面$\Leftrightarrow $存在某个$f$和$r$,使得$L=H_f^r$.
    \end{proposition}
    \begin{proof}
        充分性:注意到$H_f^0={\rm Ker}(f)$,Claim:$H_f^0$是极大子空间。
        取$x_0\in X\backslash H_f^0$,
        \begin{align*}
            f(x-\frac{f(x)}{f(x_0)}x_0)=0,\forall x\in X
            \Rightarrow& f-\frac{f(x)}{f(x_0)}x_0\in H_f^0,\forall x\in X\\
            \Rightarrow& X=H_f^0\oplus {\rm span}\{x_0\}
        \end{align*}
        Claim得证。令$r\defeq f(x_0)$,则
        \begin{equation*}
            x\in H_f^r\Leftrightarrow f(x-x_0)=f(x)-f(x_0)=0
            \Leftrightarrow x-x_0\in H_f^0\Leftrightarrow x\in H_f^0+x_0
        \end{equation*}
        所以$H_f^r$是极大子空间$H_f^0$的平移,为超平面。

        必要性:设$L=M+a$,$M$是极大子空间,那么存在某个$x_0$使得
        $X=M\oplus {\rm span}\{x_0\}$,令
        \begin{equation*}
            f:X\rightarrow \R,x=y+\lambda x_0\mapsto \lambda
        \end{equation*}
        于是$f(M)=\{ 0\}$且$f(x_0)=1$,进而
        $M\subset H_f^0$,由$M$极大$\Rightarrow M=H_f^0$,因此$L=H_f^r{\rm\ with\ }r=1$.
    \end{proof}

    \begin{proposition}
        $f\in X^*\Rightarrow \forall r\in \R,H_f^r$是闭超平面。
    \end{proposition}

    \begin{definition}
        设$X$是实向量空间,$A,B\subset X$,
        \begin{enumerate}
            \item  称$H_f^r$分离$A,B$是指:
                \begin{equation*}
                    \fun{sup}{x\in X}f(x)\leqslant r\leqslant \fun{inf}{y\in B}f(y)
                \end{equation*}
                或者
                \begin{equation*}
                    \fun{sup}{y\in B}f(y)\leqslant r\leqslant \fun{inf}{x\in A}f(x)
                \end{equation*}
            \item 称$H_f^r$严格分离$A,B$是指上述不等号严格成立。
        \end{enumerate}
    \end{definition}

    \begin{theorem}
        $X$是实赋范空间,$C$是有内点的凸集,$x_0\notin C\Rightarrow \exists f\in X^*,\exists r\in \R$满足
        $H_f^r$分离$x_0$和$C$.
    \end{theorem}
    \begin{proof}
        不妨设$0$是$C$的内点,$P_C$是$C$的Minkowski泛函,
        由习题1.5.1,
        $P_C$是次线性泛函,并且
        \begin{equation*}
            \overline{C}=\{ x\in X:P_C(x)\leqslant 1 \}
        \end{equation*}
        $x_0\notin C\Rightarrow P_C(x_0)\geqslant 1$,$0$是$C$的内点
        $\Rightarrow \exists \varepsilon>0{\rm\ s.t.\ }B(0,\varepsilon)\subset C$,于是
        \begin{equation*}
            \forall x\in X,x\neq 0,\varepsilon\frac{x}{||x||}\in \overline{B(0,\varepsilon)}\subset\overline{C}
        \end{equation*}
        进而
        \begin{equation*}
            P_C(\varepsilon\frac{x}{||x||})\leqslant 1,\forall 0\neq x\in X
        \end{equation*}
        也就是
        \begin{equation*}
            P_C(x)\leqslant \frac{1}{\varepsilon}||x||,\forall x\in X
        \end{equation*}
        
        令$M={\rm span}\{x_0\}$,定义
        \begin{equation*}
            f_0:M\rightarrow\R,x=\lambda x_0\mapsto \lambda P_C(x_0)
        \end{equation*}
        则$f_0(x)\leqslant P_C(x),\forall x\in M$,由实HBT可得
        存在$f:X\rightarrow \R$满足$f|_M=f_0$且$f(x)\leqslant P_C(x),\forall x\in X$,进而
        \begin{equation*}
            f(x_0)=f_0(x_0)=P_C(x_0)\geqslant 1,f(x)\leqslant P_C(x)\leqslant 1,\forall x\in C
        \end{equation*}
        因此$H_f^1$分离$x_0$和$C$.

        只剩下证明$f\in X^*$,
        \begin{align*}
            f(x)\leqslant P_C(x)\leqslant \frac{1}{\varepsilon}||x||,\forall x\in X
            \Rightarrow & -f(x)\leqslant \frac{1}{\varepsilon}||x||\\
            \Rightarrow & |f(x)|\leqslant \frac{1}{\varepsilon}||x||,\forall x\in X\\
            \Rightarrow & f\in X^*
        \end{align*}
    \end{proof}

    \begin{theorem}[Hahn-Banach 凸集分离定理]
        $X$是实赋范空间,$A$是开凸集,$B$是凸集,若$A\cap B=\varnothing$,
        则存在$H_f^r$闭,并分离$A,B$.
    \end{theorem}
    \begin{proof}
        令$C=A-B$,即
        \begin{equation*}
            C=\{ x-y:x\in A,y\in B \}
            =\bigcup_{y\in B} (A-y)
        \end{equation*}
        那么$C$是凸开集,而且$0\notin C$,
        由定理2.7.6,$H_f^0$分离$C$和$\{0\}$,
        即存在$f\in X^*$使得
        \begin{equation*}
            \fun{sup}{\delta\in C}f(\delta)\leqslant 0=f(0)
        \end{equation*}
        而
        \begin{equation*}
            \fun{sup}{\delta\in C}f(\delta)
            =\fun{sup}{x\in A,y\in B}[f(x)-f(y)]
            =\fun{sup}{x\in A}f(x)-\fun{inf}{y\in B}f(y)
        \end{equation*}
        因此
        \begin{equation*}
            \fun{sup}{x\in A}f(x)\leqslant r\leqslant \fun{inf}{y\in B}f(y)
        \end{equation*}
        这里
        \begin{equation*}
            r\defeq \frac{1}{2}[ \fun{sup}{x\in A}f(x)+\fun{inf}{y\in B}f(y) ]
        \end{equation*}
    \end{proof}

    \begin{theorem}[Hahn-Banach 凸集分离定理2]
        $X$是实赋范空间,$A$是闭凸集,$B$是紧凸集,若$A\cap B=\varnothing$,
        则存在$H_f^r$闭,并严格分离$A,B$.
    \end{theorem}
    \begin{proof}
        $A$闭$B$紧且不交,所以${\rm dist}(A,B)>0$,
        令$\varepsilon=\frac{1}{4}{\rm dist}(A,B)$,
        \begin{equation*}
            A_\varepsilon\defeq A+B(0,\varepsilon),
            B_\varepsilon\defeq B+B(0,\varepsilon)
        \end{equation*}
        这两个都是开凸集且不交,由定理2.7.7可知$\exists f\in X^*,\exists r\in \R$满足
        \begin{equation*}
            \fun{sup}{x\in A_\varepsilon}f(x)\leqslant r\leqslant 
            \fun{inf}{y\in B_\varepsilon}f(y)
        \end{equation*}
        所以
        \begin{equation*}
            f(x+\varepsilon\delta)\leqslant r\leqslant f(y+\varepsilon\delta),\forall x\in A,\forall y\in B,\forall \delta\in B(0,1)
        \end{equation*}
        \begin{align*}
            \Rightarrow & -f(\delta)\leqslant \frac{f(y)-r}{\varepsilon}\\
            \Rightarrow & ||f||=\fun{sup}{\delta\in B(0,1)}f(-\delta)\leqslant \frac{f(y)-r}{\varepsilon}\\
            \Rightarrow & r\leqslant f(y)-\varepsilon||f||,\forall y\in B\\
            \Rightarrow & r\leqslant \fun{inf}{y\in B}f(y)-\varepsilon||f||<\fun{inf}{y\in B}f(y)
        \end{align*}
        同理,
        \begin{equation*}
            \fun{sup}{x\in A}f(x)<\fun{sup}{x\in A}f(x)+\varepsilon ||f||\leqslant r
        \end{equation*}
    \end{proof}
\subsection*{应用和推论}
    \begin{corollary}[Ascoli]
        $X$是实赋范空间,$C$是闭凸集,若$x_0\notin C$,则
        \begin{equation*}
            \exists f\in X^*,\exists r\in \R{\rm\ s.t.\ }
            \fun{sup}{x\in C}f(x)<r<f(x_0)
        \end{equation*}
    \end{corollary}

    \begin{corollary}
        $X$是实赋范空间,$M$是其子空间,
        \begin{equation*}
            \overline{M}\neq X\Leftrightarrow \exists f\in X^*,f\neq 0{\rm\ s.t.\ }f(M)=\{0\}
        \end{equation*}
        等价地,
        \begin{equation*}
            \overline{M}=X\Leftrightarrow \forall f\in X^*{\rm\ with\ }f(M)=\{0\}\Rightarrow f=0
        \end{equation*}
    \end{corollary}
    \begin{proof}
        假设存在$x_0\in X\backslash \overline{M}$,由Ascoli,
        \begin{equation*}
            \exists f\in X^*,\exists r\in \R{\rm\ s.t.\ }
            \fun{sup}{x\in\overline{M}}f(x)<r<f(x_0)
        \end{equation*}
        进而$f|_{\overline{M}}=0\Rightarrow f(M)=\{0\}$,所以$0<r<f(x_0)$,$f\neq 0$,矛盾。所以$X=\overline{M}$.
    \end{proof}

    \begin{corollary}[Mazur]
        $X$是实赋范空间,$C$是开凸集,$F$是线性子流形(子空间的平移),
        若$C\cap F=\varnothing$,则存在$H_f^r$闭满足$F\subset H_f^r$且
        $\fun{sup}{x\in C}f(x)\leqslant r$.
    \end{corollary}
    \begin{proof}
        设$F=M+x_0$,$M$是子空间,由分离定理
        \begin{equation*}
            \exists f\in X^*,\exists s\in\R
            {\rm\ s.t.\ }
            \fun{sup}{x\in C}f(x)\leqslant s\leqslant \fun{inf}{y\in F}f(y)
            =\fun{inf}{\delta\in M}f(\delta)+f(x_0)
        \end{equation*}
        进而
        \begin{align*}
            \fun{inf}{\delta\in M}f(\delta)\geqslant s-f(x_0)
            \Rightarrow & f|_M=0\\
            \Rightarrow & M\subset H_f^0\\
            \Rightarrow & F\subset H_f^r{\rm\ with\ }r=f(x_0)
        \end{align*}
        同时,
        \begin{equation*}
            \fun{sup}{x\in C}f(x)\leqslant s\leqslant f(x_0)=r
        \end{equation*}
    \end{proof}

    \begin{definition}
        称超平面$L=H_f^r$是凸集$C$在$x_0$处的承托超平面是指:
        $C$完全落在$L$的一侧,且$x_0\in \overline{C}\cap L$,即
        \begin{equation*}
            \fun{sup}{x\in C}f(x)\leqslant r=f(x_0)
        \end{equation*}
        或者
        \begin{equation*}
            \fun{inf}{x\in C}f(x)\geqslant r=f(x_0)
        \end{equation*}
    \end{definition}

    \begin{theorem}
        $X$是实赋范空间,$C$是有内点的闭凸集,$\forall x_0\in \partial C$均有$C$的一个承托超平面。
    \end{theorem}
    \begin{proof}
        令$E=C^\circ$,即$C$的全体内点,$F=\{x_0\}$,由Mazur可得
        \begin{equation*}
            \exists f\in X^*,\exists r\in\R{\rm\ s.t.\ }
            \fun{sup}{x\in E}f(x)\leqslant r{\rm\ and\ }\{x_0\}\subset H_f^r
        \end{equation*}
        由连续性
        \begin{equation*}
            \fun{sup}{x\in C}f(x)\leqslant r=f(x_0)
        \end{equation*}
    \end{proof}

    \begin{example}
        $C=B(0,r)$,$\forall x_0\in \partial B(0,r)$,均有承托超平面。
    \end{example}
    \begin{proof}
        $\exists f\in X^*,||f||=1$使得$f(x_0)=||x_0||=r$,而
        \begin{equation*}
            \fun{sup}{x\in C}f(x)\leqslant ||f||\fun{sup}{x\in C}||x||=r
        \end{equation*}
    \end{proof}

    \begin{example}
        设$\sum_{k=1}^\infty x_k$是Banach空间$X$中绝对收敛级数,
        $\{y_k\}_{k=1}^\infty$是$\{x_k\}_{k=1}^\infty$
        的任一重排,则
        \begin{equation*}
            \sum_{k=1}^\infty y_k=\sum_{k=1}^\infty x_k
        \end{equation*}
    \end{example}
    \begin{proof}
        $\forall f\in X^*$,
        \begin{equation*}
            \sum_{k=1}^\infty |f(x_k)|\leqslant ||f||\sum_{k=1}^\infty ||x_k||<\infty
        \end{equation*}
        于是$\sum_{k=1}^\infty f(x_k)$是$\K$上的绝对收敛级数,
        重排不变,所以
        \begin{equation*}
            \sum_{k=1}^\infty f(y_k)=\sum_{k=1}^\infty f(x_k)
            \Rightarrow f(\sum_{k=1}^\infty y_k)
            =f(\sum_{k=1}^\infty x_k)
            ,\forall f\in X^*
            \mathop{\Rightarrow}\limits^{ \rm Cor\ 2.7.4 }
            \sum_{k=1}^\infty y_k=\sum_{k=1}^\infty x_k
        \end{equation*}
    \end{proof}


\section{对偶空间、自反空间、弱收敛}
%Lec21
\subsection{对偶空间}
    \begin{definition}
        $X$的对偶空间$X^*$是指$X$上所有线性泛函组成的空间,即
        \begin{equation*}
            X^*\defeq \mathcal{L}(X,\K)
        \end{equation*}
    \end{definition}    

    回顾:
    $(X,m,\mu)$是一可测空间,
    $\Omega$是$X$上可测函数全体,
    \begin{equation*}
        L^p=L^p(\Omega,m,\mu)\defeq \{ f\in\Omega:||f||_p=(\int_X |f|^p\d \mu)^\frac{1}{p}<\infty  \}
    \end{equation*}
    对于$1\leqslant p\leqslant \infty$,对偶空间$(L^p)^*$是什么?
    \begin{theorem}[Riesz]
        设$1\leqslant p<\infty$,则$(L^p)^*=L^{q}$,其中
        \begin{equation*}
            q=\left\{ \begin{array}{ll}
                \frac{p}{p-1}&,1<p<\infty\\
                \infty&,p=1
            \end{array} \right.
        \end{equation*}
    \end{theorem}
    \begin{proof}
        我们希望构造出一个线性等距同构$J:L^q\rightarrow (L^p)^*$,如下:
        \begin{equation*}
            \forall g\in L^q,\Lambda_g:L^p\rightarrow \K,f\mapsto \Lambda_g(f)\defeq \int fg
        \end{equation*}
        \begin{equation*}
            J:L^q\rightarrow (L^p)^*,g\mapsto \Lambda_g
        \end{equation*}
        需要证明:
        \begin{enumerate}[$1^\circ$]
            \item $\Lambda_g\in (L^p)^*$.
            \item $J$线性。
            \item $||\Lambda_g||=||g||_q$,即$J$等距。
            \item $\forall \Lambda\in (L^p)^*$,存在唯一$g\in L^{p'}$使得$\Lambda=\Lambda_g$,即$J$是双射。
        \end{enumerate}
        
        \textbf{Proof of }$1^\circ-3^\circ$:
        先考虑$1<p<\infty$,
        \begin{equation*}
            \forall f\in L^p,|\Lambda_g(f)|=|\int fg|\leqslant ||g||_q\cdot ||f||_p
        \end{equation*}
        于是$\Lambda_g\in (L^p)^*$,且$||\Lambda_g||\leqslant ||g||_q$.取
        \begin{equation*}
            \tilde{f}\defeq |g|^{q-1}{\rm sgn}(g)
        \end{equation*}
        则
        \begin{align*}
            \left\{ \begin{array}{ll}
                |\tilde{f}|^p=|g|^{(q-1)p}=|g|^q\\
                \tilde{f}\cdot g=|g|^q
        	\end{array} \right.\Rightarrow& 
            \left\{ \begin{array}{ll}
                |\tilde{f}|_p^p=||g||_q^q\\
                \Lambda_g(\tilde{f})=||g||_q^q
        	\end{array} \right.\\
            \Rightarrow&
            \frac{|\Lambda_g(\tilde{f})|}{||\tilde{f}||_p}
            =\frac{||g||^q_q}{||g||_q^{\frac{q}{p}}}=
            ||g||_q^{q(1-\frac{1}{p})}=||g||_q\\
            \Rightarrow& ||\Lambda_g||\geqslant ||g||_q
        \end{align*}
        接下来考虑$p=1$,此时需假设$\mu$是$\sigma$-有限的,不妨设$\mu$有限,
        由
        \begin{equation*}
            |\Lambda_g(f)|\leqslant ||g||_\infty ||f||_1
        \end{equation*}
        得$\Lambda_g\in (L^1)^*$且$||\Lambda_g||\leqslant ||g||_\infty$,
        令
        \begin{equation*}
            E_k\defeq \{ t\in\Omega:|g(t)|>||\Lambda_g||+\frac{1}{k} \},
            f_k\defeq \chi_{E_k}\cdot {\rm sgn}(g)
        \end{equation*}
        \begin{align*}
            \Rightarrow& ||f_k||_1=\int_{E_k} |{\rm sgn}(g)|\d \mu\leqslant \mu(E_k)\\
            \Rightarrow& ||\Lambda_g||\mu(E_k)\geqslant 
            ||\Lambda g||\geqslant |\Lambda_g(f_k)|=\int \chi_{E_k}{\rm sgn}(g)g\d \mu=\int_{E_k}|g|\d \mu\geqslant
            ( ||\Lambda_g||+\frac{1}{k} )\mu(E_k)\\
            \Rightarrow& \mu(E_k)=0,\forall k\\
            \Rightarrow& \{ t\in \Omega:|g(t)|>||\Lambda_g|| \}=\bigcup_{k=1}^\infty E_k\mbox{零测}\\
            \Rightarrow& ||g||_\infty\leqslant ||\Lambda_g||
        \end{align*}

        \textbf{Proof of }$4^\circ$:
        以下假设$\Omega=[0,1]$,$\mu=m$,
        \begin{lemma}
            设$g\in L^1$,如果存在$C>0$满足
            \begin{equation*}
                |\int fg|\leqslant C||f||_p,\forall f\in L^\infty
            \end{equation*}
            则$g\in L^q$且$||g||_q\leqslant C$.
            \begin{proof}
                
            \end{proof}
        \end{lemma}
        后面也太复杂了,不抄了。
    \end{proof}
%Lec22
    那么,$(L^\infty)^*$是$L^1$吗?答案是否定的。
    \begin{theorem}
        $L^1\subsetneqq (L^\infty)^*$.
    \end{theorem}
    \begin{proof}
        对于$\forall g\in L^1$,
        \begin{equation*}
            |\Lambda_g(f)|=|\int fg|\leqslant ||g||_1||f||_\infty\Rightarrow\Lambda_g\in (L^\infty)^*
            \Rightarrow L^1\subset (L^\infty)^*
        \end{equation*}
        注意到$C[0,1]$是$L^\infty$的闭子空间,取$f_0\in L^\infty\backslash C[0,1]$,
        于是$d\defeq {\rm dist}(f_0,C[0,1])>0$,由HBT可知
        存在$\Lambda\in (L^\infty)^*,||\Lambda||=1$且
        \begin{equation*}
            \Lambda(C[0,1])=\{0\},\Lambda(f_0)=d
        \end{equation*}
        假设存在$g\in L^1$满足$\Lambda=\Lambda_g$,即
        \begin{equation*}
            \Lambda(f)=\int fg,\forall f\in L^\infty
        \end{equation*}
        那么对于$f\in C[0,1]$,$\Lambda(f)=\int fg=0$,
        取$\{ f_n \}_{n=1}^\infty \subset C[0,1]$,
        满足
        \begin{equation*}
            ||f_n-{\rm sgn}(g)||_1\rightarrow 0{\rm\ as\ }n\rightarrow\infty
        \end{equation*}
        于是有子列$f_{n_k}\mathop{\rightarrow}\limits^{\rm a.e.} 
        {\rm sgn}(g)$,由MCT得
        \begin{equation*}
            \int |g|=\fun{lim}{k\rightarrow\infty}\int f_{n_k}g=0\Rightarrow g\mathop{=}\limits^{\rm a.e.} 0
            \Rightarrow \Lambda=\Lambda_g=0
        \end{equation*}
        这与$\Lambda(f_0)=d>0$矛盾。
    \end{proof}

    接下来讨论$C[a,b]$的对偶空间。
    \begin{definition}
        对$f:[a,b]\rightarrow \mathbb{C}$和$[a,b]$的划分$P$,定义
        \begin{equation*}V(f,P)\mathop{=}\limits^{\rm def}\sum_{k=1}^N |f(t_k)-f(t_{k-1})|\end{equation*}
        如果$\mathop{\rm sup}\limits_{P} V(f,P)<\infty$,则称$f$是有界变差的。
        \begin{equation*}V_a^b(f)\mathop{=}\limits^{\rm def} \mathop{\rm sup}\limits_{P} V(f,P)\end{equation*}
        称为$f$在$[a,b]$上的全变差。
        \begin{equation*}BV[a,b]\mathop{=}\limits^{\rm def}\{ [a,b]\mbox{上的有界变差函数全体} \}\end{equation*}

        我们进一步给出$BV[a,b]$上的范数:
        \begin{equation*}
            ||f||_{BV}\defeq |f(a)|+V_a^b(f)
        \end{equation*}
        那么$(BV[a,b],||\cdot||_{BV})$是Banach空间,定义:
        \begin{equation*}
            BV_0[a,b]\defeq \{ f\in BV[a,b]:f\mbox{在$(a,b)$上右连续,}f(a)=0 \}
        \end{equation*}
        那么$BV_0[a,b]$是$BV[a,b]$的闭子空间,也是Banach空间。
    \end{definition}

    \begin{definition}[Riemann-Stieltjes积分]
        设$f,g$是$[a,b]$上实值函数,$I\in\R$,
        对$[a,b]$进行划分:
        \begin{equation*}
            \sigma(\Delta,\xi)\defeq \sum_{k=1}^n f(\xi_k)[ g(t_k)-g(t_{k-1}) ]
        \end{equation*}
        其中
        \begin{equation*}
            \xi=\{\xi_k\}_{k=1}^\infty,\xi_k\in [t_{k-1},t_k]
        \end{equation*}
        如果$\sigma(\Delta,\xi)\rightarrow I{\rm\ as\ }||\Delta||\rightarrow 0$,
        则记
        \begin{equation*}
            I=\int_a^b f\d g
        \end{equation*}
        称为$f$关于$g$的Riemann-Stieltjes积分,简记为R-S积分。
    \end{definition}

    \begin{theorem}[Riesz]
        $C[a,b]^*=BV_0[a,b]$.

        思路:$\forall g\in BV_0[a,b]$,取
        \begin{equation*}
            \Lambda_g(f)\defeq \int_a^b f\d g,f\in C[a,b]
        \end{equation*}
        证明$g\mapsto \Lambda_g$是线性等距同构。
    \end{theorem}
\subsection{自反空间}
    \begin{definition}
        $X^{**}=\mathcal{L}(X^*,\K)$,
        称为$X$的二次对偶或第二共轭空间。
        
        设$x\in X$,定义映射:
        \begin{equation*}
            x^{**}:X^*\rightarrow \K,f\mapsto f(x)
        \end{equation*}
        则
        \begin{equation*}
            |x^{**}(f)|=|f(x)|\leqslant ||x||\cdot ||f||,\forall f\in X^*
        \end{equation*}
        因此$x^{**}\in X^{**}$且$||x^{**}||\leqslant ||x||$.
        另一方面由推论2.7.1,存在$f_0\in X^*{\rm\ with\ }||f_0||=1{\rm\ s.t.\ }f_0(x)=||x||$,即
        \begin{equation*}
            x^{**}(f_0)=f_0(x)=||x||
        \end{equation*}
        于是$||x^{**}||\geqslant |x^{**}(f_0)|/||f_0||=||x||$,
        于是映射
        \begin{equation*}
            i:X\rightarrow X^{**},x\mapsto x^{**}
        \end{equation*}
        是线性等距嵌入,
        称为$X$到$X^{**}$的自然映射或自然嵌入(canonical map).
    \end{definition}

    \begin{definition}
        如果自然映射$i$是满射,从而是线性等距同构,则称$X$自反,记作$X^{**}=X$.
    \end{definition}

    \begin{example}
        不完备的空间一定不自反、
        有限维赋范空间一定自反(习题2.5.4)。
    \end{example}

    \begin{example}
        Hilbert空间自反。(作业)
    \end{example}
    \begin{proof}
        由 Riesz 表示定理,有等距同构 $\varphi:H\to H^*,x\mapsto\langle\cdot,x\rangle.\varphi$ 诱导了$H^*$ 上的内积 $\langle f,g\rangle=\overline{\langle\varphi^{-1}(f),\varphi^{-1}(g)\rangle}$。对偶空间自然完备,因此$H^*$ 也 
        是 Hilbert 空间,由 Riesz 表示定理,又有等距同构 $\Phi:H^*\to H^{**},f\mapsto$ $\overline{{\langle\varphi^{-1}(\cdot),\varphi^{-1}(f)\rangle}}.\:计算得$
        \begin{equation*}
            \Phi\circ\varphi(x)(f)=\overline{\langle\varphi^{-1}(f),x\rangle}=\langle x,\varphi^{-1}(f)\rangle=f(x),\forall x\in H,f\in H^{*}
        \end{equation*}
        故$\Phi\circ\varphi$ 就是 $H\to H^{**}$ 的自然映射 $\phi$, 因此 $\phi$ 也是等距同构,从而$H$自反。
    \end{proof}

    \begin{theorem}
        当$1<p<\infty$时,$L^p$自反。
    \end{theorem}
    \begin{proof}
        即证明:$\forall \Lambda\in (L^p)^{**}$,存在$u\in L^p$使得
        \begin{equation*}
            \Lambda(f)=f(u),\forall f\in (L^p)^*
        \end{equation*}
        这是因为
        \begin{equation*}
            i:L^p\rightarrow (L^p)^{**} \mbox{满}
            \Leftrightarrow \forall \Lambda\in (L^p)^{**},\exists u\in L^p{\rm\ s.t.\ }
            u^{**}=\Lambda ( \Lambda(f)=u^{**}(f)=f(u) )
        \end{equation*}
        回顾定理2.8.1,
        \begin{equation*}
            J:L^q\rightarrow (L^p)^*,v\mapsto f_v
        \end{equation*}
        是线性等距同构,这里$f_v(u)=\int uv$.取
        \begin{equation*}
            \varphi\defeq \Lambda\circ J
        \end{equation*}
        则$\varphi\in (L^q)^*=L^p$,
        存在唯一$u\in L^p$满足$\varphi(v)=\int uv,\forall v\in L^q$.
        那么对于$\forall f\in (L^p)^*$,令
        \begin{equation*}
            v_f\defeq J^{-1}(f)
        \end{equation*}
        那么
        \begin{equation*}
            \Lambda(f)=\Lambda( J(v_f) )=\varphi(v_f)=\int v_fu=f(u)
        \end{equation*}
    \end{proof}
    
    \begin{theorem}
        $C[a,b]$不自反。
    \end{theorem}
    \begin{proof}
        假设自反,则
        $\forall \Lambda\in C[a,b]^{**}$,存在$u\in C[a,b]$满足
        \begin{equation*}
            \Lambda(f)=f(u),\forall f\in C[a,b]^*\tag{*}
        \end{equation*}
        根据$C[a,b]^*=BV_0[a,b]$,
        \begin{equation*}
            \exists ! v_f\in BV_0[a,b]{\rm\ s.t.\ }
            f(u)=\int_a^b u\d v_f,\forall u\in C[a,b]{\rm\ and\ }
            ||v_f||_{BV}=||f||
        \end{equation*}
        令$c=\frac{a+b}{2}$,定义
        \begin{equation*}
            F_c:C[a,b]^*\rightarrow\R,f\mapsto v_f(c+0)-v_f(c-0)
        \end{equation*}
        那么
        \begin{equation*}
            |F_c(f)|\leqslant V_a^b (v_f)=||v_f||_{BV}=||f||\Rightarrow F_c\in C[a,b]^{**}
        \end{equation*}
        根据(*),存在$u_c\in C[a,b]$满足
        \begin{equation*}
            F_c(f)=f(u_c)=\int_a^b u_c\d v_f,\forall f\in C[a,b]^*
        \end{equation*}
        令
        \begin{equation*}
            v(f)\defeq \int_a^t u_c(s)\d s
        \end{equation*}
        那么$v\in BV_0[a,b]$,令$f_v\defeq J(v)$,这里
        \begin{equation*}
            J:BV_0[a,b]\rightarrow C[a,b]^*,v\mapsto f_v,f_v(u)=\int_a^b u \d v_f
        \end{equation*}
        $f_v$对应的$v_{f_v}=v$连续,于是$F_c(f_v)=0$,所以
        \begin{equation*}
            0=F_c(f_v)=\int_a^b u_c\d v=\int_a^b u_c^2\d t
        \end{equation*}
        所以$u_c=0$,进而$F_c=0$,矛盾。
    \end{proof}
%Lec23
    \begin{theorem}[Banach]
        $X^*$可分$\Rightarrow X$可分。逆命题不成立,例如$L^1$可分但$L^\infty$不可分。
    \end{theorem}
    \begin{proof}
        \textbf{Step1}:证明$X^*$中单位球面$S_1^*$可分。
        $X^*$可分,取$\{f_n\}_{n=1}^\infty$为其稠密子集,
        不妨$f_n\neq 0$,令
        \begin{equation*}
            g_n\defeq \frac{f_n}{||f_n||}\in S_1^* 
        \end{equation*}
        那么对于$\forall g\in S_1^*$,存在$f_{n_k}\rightarrow g$.
        \begin{align*}
            \Rightarrow ||g-g_{n_k}||
            \leqslant & ||g-f_{n_k}||+||f_{n_k}-g_{n_k}||\\
            =& ||g-f_{n_k}||+||( ||f_{n_k}||-1 )\frac{f_{n_k}}{||f_{n_k}||}||\\
            =& ||g-f_{n_k}||+\left| ||f_{n_k}||+1 \right|\rightarrow 0{\rm\ as\ }k\rightarrow\infty
        \end{align*}

        \textbf{Step2}:证明存在$\{x_n\}_{n=1}^\infty\subset X$,
        其中$\forall ||x_n||=1$,且
        \begin{equation*}
            \overline{ {\rm span}\{x_n\}_{n=1}^\infty }=X
        \end{equation*}
        注意到
        \begin{equation*}
            ||g_n||=\fun{sup}{x\in X,||x||=1}|g_n(x)|=1
        \end{equation*}
        所以存在$\{x_n\}_{n=1}^\infty\subset X$,
        其中$\forall ||x_n||=1$,且$|g_n(x_n)|>\frac{1}{2}$.令$M\defeq {\rm span}\{x_n\}_{n=1}^\infty$,
        假设$\overline{M}\neq X$,取$x_0\in X\backslash \overline{M}$,由HBT可得
        \begin{equation*}
            \exists f\in X^*,||f||=1{\rm\ s.t.\ }f(\overline{M})=\{0\}{\rm\ and\ }
            f(x_0)={\rm dist}(x_0,\overline{M})>0
        \end{equation*}
        于是
        \begin{equation*}
            ||g_n-f||= \fun{sup}{x\in X,||x||=1}|g_n(x)-f(x)|>|g_n(x_n)-f(x_n)|=|g_n(x_n)|>\frac{1}{2}
        \end{equation*}
        这与Step1矛盾。

        \textbf{Step3}:证明$\overline{ {\rm span}^{\mathbb{Q}}\{x_n\}_{n=1}^\infty }=X$.
    \end{proof}

    \begin{theorem}
        当$1\leqslant p<\infty$,$L^p[0,1]$可分。
    \end{theorem}
    \begin{proof}
        \begin{equation*}
            \left\{ \sum_{k=0}^{2^n-1} r_k\chi_{ [ \frac{k}{2^n},\frac{k+1}{2^n} ) }:r_k\in\Q,n\in\N_0 \right\}
        \end{equation*}
        是$L^p[0,1]$的可数稠密子集。
    \end{proof}

    \begin{theorem}
        $L^\infty$不可分。
    \end{theorem}
    \begin{proof}
        假设存在稠密子集$\{f_n\}_{n=1}^\infty$,
        \begin{equation*}
            \forall t\in (0,1),\exists f_{n_t}\in B(\chi_{[0,t]},\frac{1}{3})
        \end{equation*}
        而$t\neq s$时,${\rm dist}(\chi_{[0,t]},\chi_{[0,s]})=1$,
        因此不同的$B(\chi_{[0,t]},\frac{1}{3})$不相交,所以$\varphi:(0,1)\rightarrow\N,t\mapsto n_t$是单射,
        于是$(0,1)$可数,矛盾。
    \end{proof}

    \begin{theorem}
        $L^1$不自反。
    \end{theorem}
    \begin{proof}
        $(L^1)^*=L^\infty$,假设$L^1$自反,
        则$(L^\infty)^*\cong L^1$,$L^1$可分
        $\Rightarrow L^\infty $可分,矛盾。
    \end{proof}

    \begin{theorem}[共轭算子]
        $X,Y$是赋范空间,$T\in \L(X,Y)\Rightarrow \exists T^*\in \L( Y^*,X^* )$使得
        \begin{equation*}
            (T^* f)(x)=f(Tx),\forall f\in Y^*,\forall x\in X
        \end{equation*}
        $T^*$称为$T$的共轭算子。进而映射$*:\L(X,Y)\rightarrow \L(Y^*,X^*),T\mapsto T^*$
        是一个线性等距嵌入。
    \end{theorem}
    \begin{proof}
        设$f\in Y^*$,定义映射:
        \begin{equation*}
            \Lambda_f:X\rightarrow\K,x\mapsto f(Tx)
        \end{equation*}
        则
        \begin{equation*}
            |\Lambda_f(x)|=|f(Tx)|\leqslant ||f||\cdot ||Tx||\leqslant ||f||\cdot ||T||\cdot ||x||,\forall x\in X
        \end{equation*}
        于是$\Lambda_f\in X^*$,且$||\Lambda_f||\leqslant ||f||\cdot||T||$,定义映射
        \begin{equation*}
            T^*:Y^*\rightarrow X^*,f\mapsto \Lambda_f
        \end{equation*}
        于是$T^*$线性而且$||T^*f||=||\Lambda_f||\leqslant ||T||\cdot ||f||$,
        从而$T^*$有界且$||T^*||\leqslant ||T||$.

        对于$\forall x\in X$,不妨$x\neq 0$,由HBT,
        \begin{equation*}
            \exists f\in Y^*,||f||=1,f(Tx)=||Tx||
        \end{equation*}
        于是
        \begin{equation*}
            ||Tx||=|f(Tx)|=|(T^*f)(x)|
            \leqslant ||T^*f||\cdot ||x||\leqslant 
            ||T^*||\cdot ||f||\cdot ||x||=||T^*||\cdot ||x||
        \end{equation*}
        所以$||T||\leqslant ||T^*||$.
    \end{proof}

    \begin{example}
        $T:\K^n \rightarrow \K^m,x\mapsto Ax{\rm\ with\ }A=(a_{ij})_{m\times n}$,
        $T^*:\K^m \rightarrow \K^n,y\mapsto \overline{A^T}y$.
    \end{example}

    \begin{theorem}[pettis]
        自反空间的闭子空间自反。
    \end{theorem}
    \begin{proof}
        设$X$自反,$Y$是其闭子空间,只需证明:
        \begin{equation*}
            \forall a\in Y^{**},\exists y\in Y{\rm\ s.t.\ }
            a(f)=f(y),\forall f\in Y^*
        \end{equation*}
        定义映射
        \begin{equation*}
            T:X^*\rightarrow Y^*,f\mapsto f|_Y
        \end{equation*}
        则$T$是有界线性映射,于是取$T^*\in\mathcal{L}(Y^{**},X^{**})$满足
        \begin{equation*}
            (T^*a)(f)=a(Tf),\forall f\in X^*
        \end{equation*}
        $X$自反,所以自然映射$i_X$是满射,$T^*a\in X^{**}\Rightarrow \exists y\in X{\rm\ s.t.\ }T^*a=y^{**}$,
        所以
        \begin{equation*}
            (T^*a)(f)=y^{**}(f)=f(y),\forall f\in X^*
        \end{equation*}

        下面证明$y\in Y$,假设不然,则存在$\tilde{f}\in X^*$使得
        $\tilde{f}(Y)=0$,
        \begin{equation*}
            T(\tilde{f})=\tilde{f}|_Y=0\Rightarrow \tilde{f}(y)
            =(T^*a)(\tilde{f})=a(T(\tilde{f}))=0
        \end{equation*}
        这与$\tilde{f}(y)={\rm dist}(y,Y)>0$矛盾。

        最后说明:$a(f)=f(y),\forall f\in Y^*$.
        对于$\forall f\in Y^*$,由HBT,
        存在$F\in X^*$使得$f=TF$,
        所以
        \begin{equation*}
            a(f)=a(TF)=( T^*a )(F)=F(y)=f(y)
        \end{equation*}
    \end{proof}
%Lec24
\subsection{弱收敛}
    \begin{definition}
        $X$是赋范空间,称$\{ x_n \}_{n=1}^\infty \subset X$弱收敛于$x_0\in X$是指:
        \begin{equation*}
            f(x_n)\rightarrow f(x_0),\forall f\in X^*
        \end{equation*}
        记为$x_n\mathop{\rightarrow}\limits^{w} x_0$或者$x_n\rightarrow x_0$,称
        $x_0$为$\{ x_n \}_{n=1}^\infty $的弱极限。

        依范数拓扑收敛即为强收敛。
    \end{definition}
    \begin{proposition}
        强收敛$\Rightarrow $弱收敛。
    \end{proposition}
    \begin{proof}
        \begin{equation*}
            ||x_n-x_0||\rightarrow 0\Rightarrow |f(x_n)-f(x_0)|=| f(x_n-x_0) |
            \leqslant ||f||\cdot ||x_n-x_0||\rightarrow 0,\forall f\in X^*
        \end{equation*}
    \end{proof}

    \begin{example}
        $X=L^2(\Pi)$, $e_k(t)\defeq \e^{ -2\pi \i kt },k\in \Z$,则
        $e_k\mathop{\rightarrow}\limits^{w} 0{\rm\ as\ }|k|\rightarrow\infty$.即:
        \begin{equation*}
            \forall f\in X^*,\exists v\in L^2(\Pi){\rm\ s.t.\ }
            f(u)=\int_{\Pi} uv,u\in L^2(\Pi)
        \end{equation*}
        因此
        \begin{equation*}
            f(e_k)=\int_0^1 v(t) \e^{-2\pi \i kt}\d t=\hat{v}(k)\rightarrow 0{\rm\ as\ }|k|\rightarrow\infty
        \end{equation*}
    \end{example}

    \begin{theorem}
        ${\rm dim}X<\infty\Rightarrow$弱收敛与强收敛等价。
    \end{theorem}
    \begin{proof}
        设${\rm dim}(X)=m$,设$\{e_1,\cdots,e_m\}$是$X$的一组基,由HBT(习题2.4.7)知存在
        对偶基$f_1,\cdots,f_m\in X^*$满足
        \begin{equation*}
            f_k(e_j)=\delta_{kj},1\leqslant k,j\leqslant m
        \end{equation*}
        设$x_n\wto x_0$,即
        \begin{equation*}
            \sum_{j=1}^m \alpha_j^{(n)}e_j\wto \sum_{j=1}^m \alpha_j^{(0)}e_j
        \end{equation*}
        于是
        \begin{equation*}
            f_j(x_n)\rightarrow f_k(x_0),k=1,2,\cdots,m
        \end{equation*}
        因此$||x_n-x_0||_{\infty}\rightarrow 0{\rm\ with\ }||x||_{\infty}=\fun{max}{1\leqslant k\leqslant m}|\alpha_i|$,
        由有限维空间范数等价可得$||x_n-x_0||\rightarrow 0$.
    \end{proof}

    \begin{theorem}[Mazur]
        $x_n\mathop{\rightarrow}\limits^{w} x_0\Rightarrow x_0\in \overline{ {\rm conv}( \{ x_n \}_{n=1}^\infty ) }$
    \end{theorem}
    \begin{proof}
        令$C=\overline{ {\rm conv}( \{ x_n \}_{n=1}^\infty ) }$,假设$x_0\notin C$,则
        由Ascoli,存在$f\in X^*,\exists \alpha\in\R$使得
        \begin{equation*}
            \fun{sup}{x\in C}f(x)<\alpha<f(x_0)
        \end{equation*}
        $\Rightarrow f(x_n)<\alpha <f(x_0),n=1,2,\cdots$,
        与$f(x_n)\rightarrow f(x_0)$矛盾。
    \end{proof}

    \begin{definition}
        称$\{f_n\}_{n=1}^\infty \subset X^*$弱*收敛于$f\in X^*$是指
        \begin{equation*}
            f_n(x)\rightarrow f(x),\forall x\in X
        \end{equation*}
        记作$f_n\mathop{\rightarrow}\limits^{w^*}f $.
    \end{definition}
    $X^*$中,强收敛$\Rightarrow $弱收敛$\Rightarrow $弱*收敛。
    \begin{align*}
        f_n\mathop{\rightarrow}\limits^{w} f
        \mathop{\Leftrightarrow}\limits^{\rm def}& \Lambda(f_n)\rightarrow \Lambda(f),\forall \Lambda\in X^{**}\\
        \Rightarrow& x^{**}(f_n)\rightarrow x^{**}(f),\forall x\in X\\
        \Leftrightarrow& f_n(x)\rightarrow f(x),\forall x\in X\\
        \mathop{\Leftrightarrow}\limits^{\rm def}& f_n\mathop{\rightarrow}\limits^{w^*}f
    \end{align*}

    \begin{proposition}
        $X$自反$\Rightarrow X^*$中弱*收敛与弱收敛等价。
    \end{proposition}

    \begin{theorem}
        $X$是度量空间,
        \begin{equation*}
            x_n\mathop{\rightarrow}\limits^{w} x_n
            \Leftrightarrow
            \left\{ \begin{array}{l}
                \fun{sup}{n}||x_n||<\infty\\
                \exists \mathcal{F}\mathop{\subset }^{\rm dense} X^*{\rm\ s.t.\ }
                f(x_n)\rightarrow f(x_0),\forall f\in\mathcal{F}
            \end{array} \right.
        \end{equation*}
    \end{theorem}
    \begin{proof}
        \begin{align*}
            x_n\wto x_0\Leftrightarrow& f(x_n)\rightarrow f(x_0),\forall f\in X^*\\
            \Leftrightarrow& x_n^{**}(f)\rightarrow x_0^{**}(f),\forall f\in X^*\\
            \mathop{\Rightarrow}\limits^{\rm B-S}&
            \left\{ \begin{array}{ll}
                \fun{sup}{n}||x_n^{**}||<\infty\\
                \exists \mathcal{F}\mathop{\subset}\limits^{\rm dense}X^*{\rm\ s.t.\ }
                x_n^{**}(f)\rightarrow x_0^{**}(f),\forall f\in\mathcal{F}
            \end{array} \right.
        \end{align*}
    \end{proof}

    \begin{theorem}
        $X$是Banach空间,则
        \begin{equation*}
            f_k\mathop{\rightarrow}\limits^{w^*} f
            \Leftrightarrow
            \left\{ \begin{array}{l}
                \fun{sup}{n}||f_n||<\infty\\
                \exists M\mathop{\subset }^{\rm dense} X{\rm\ s.t.\ }
                f_n(x)\rightarrow f(x),\forall x\in M
            \end{array} \right.
        \end{equation*}
    \end{theorem}

    \begin{definition}
        称$M\subset X$弱列紧是指$M$中任一序列均有弱收敛子列;
        称$\mathcal{F}\subset X^*$弱*列紧是指$\mathcal{F}$中任一序列均有弱*收敛子列。
    \end{definition}

    \begin{theorem}[可分Banach-Alaoglu]
        $X$可分$\Rightarrow X^*$中有界集弱*列紧。
    \end{theorem}
    \begin{proof}
        设$\{f_n\}_{n=1}^\infty\subset X^*$有界,记$C=\fun{sup}{n}||f_n||$,
        $X$可分$\Rightarrow \exists \{x_n\}_{n=1}^\infty\mathop{\subset}\limits^{\rm dense}X$.

        对于$\forall m$,$\{f_n(x_m)\}_{n=1}^\infty$是有界数列,有收敛子列,
        由对角线法,$\{f_n\}_{n=1}^\infty$有子列
        $\{f_{n_k}\}_{k=1}^\infty$使得
        $\{f_{n_k}(x_m)\}_{k=1}^\infty$收敛,Claim:
        \begin{equation*}
            \exists f\in X^*{\rm\ s.t.\ }f_{n_k}\w*to f
        \end{equation*}
        对于$\forall x\in X,\forall \varepsilon>0$,存在$x_m\in \{x_n\}_{n=1}^\infty{\rm\ s.t.\ }||x-x_m||<\frac{\varepsilon}{3C}$,于是
        \begin{equation*}
            |f_{n_{k+p}}(x)-f_{n_k}(x)|
            \leqslant 
            |f_{n_{k+p}}(x)-f_{n_{k+p}}(x_m)|
            +|f_{n_{k+p}}(x_m)-f_{n_{k}}(x_m)|
            +|f_{n_{k}}(x_m)-f_{n_{k}}(x)|<\varepsilon,k\mbox{充分大,}\forall p
        \end{equation*}
        第一项$\leqslant C||x-x_m||<\frac{\varepsilon}{3}$,
        第二项在$k$充分大时$<\frac{\varepsilon}{3}$,
        第三项$\leqslant C||x_m-x||<\frac{\varepsilon}{3}$.

        所以$f(x)\defeq \fun{lim}{k\rightarrow\infty}f_{n_k}(x)$存在,
        且
        \begin{equation*}
            |f(x)|\leqslant \fun{sup}{n}|f_n(x)|\leqslant \fun{sup}{n}||f_n||\cdot ||x||
        \end{equation*}
        所以$f\in X^*$且$f_{n_k}\w*to f$.
    \end{proof}

    \begin{theorem}[Alaoglu]
        $X$是赋范空间,$X^*$中单位闭球是弱*紧的。
    \end{theorem}

    \begin{theorem}[Eberlein-Smulian]
        $X$是自反空间,则
        \begin{enumerate}
            \item $X$中有界集弱列紧;
            \item $X$中闭单位球弱自列紧。
        \end{enumerate}
    \end{theorem}
    \begin{proof}
        对于1,只需证明:$\forall R,\overline{B(0,R)}$弱列紧。
        令$Y\defeq \overline{ {\rm span}\{x_n\}_{n=1}^\infty }$,为闭子空间,
        由定理2.8.11(Pettis)知$Y$自反,同时因为$Y$可分,所以$Y^{**}=Y$可分,
        由定理2.8.6(Banach)知$Y^*$可分,再由定理2.8.16(可分B-A)知
        $Y^{**}$中有界集弱*列紧,所以
        \begin{align*}
            ||x_n^{**}||\leqslant R\Rightarrow& \{x_n^{**}\}_{n=1}^\infty \mbox{有子列}x_{n_k}^{**}\w*to x_0^{**}\in Y^{**}\\
            \Rightarrow& \forall f\in Y^*,f(x_{n_k})=x_{n_k}^{**}(f)\rightarrow x_0^{**}(f)=f(x_0)\\
            \Rightarrow& \forall F\in X^*,F(x_{n_k})
            =(F|_Y)(x_{n_k})\rightarrow
            (F|_Y)(x_0)=F(x_0)\\
            \Rightarrow& x_{n_k}\wto x_0
        \end{align*}

        对于2,
        \begin{equation*}
            x_{n_k}\wto \mathop{\Rightarrow }\limits^{\rm Ex\ 2.5.4}
            ||x_0||\leqslant \fun{liminf}{k\rightarrow\infty}||x_{n_k}||\leqslant R
        \end{equation*}
    \end{proof}




