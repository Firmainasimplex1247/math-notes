\chapter{谱理论}
%\begin{center}
%    线性算子=线性映射
%\end{center}
%\rightline{2023.10.27}
%\vspace{-5pt}
%\begin{center}
%    \pgfornament[width=0.36\linewidth,color=lsp]{88}
%\end{center}

%Lec25

\section{谱}

\subsection{谱的定义与例子}
    \begin{definition}
        $X$是复Banach空间,在$\L(X)$上引入乘法:
        \begin{equation*}
            (AB)x\defeq A(Bx)
        \end{equation*}
        则满足
        \begin{enumerate}
            \item 结合律:$(AB)C=A(BC)$.
            \item 分配律:$(A+B)C=AB+AC$,$A(B+C)=AB+AC$.
            \item $\lambda(AB)=(\lambda A)B=A(\lambda B)$.
            \item $AI=IA=A$.
            \item $||AB||\leqslant ||A||\cdot ||B||$
        \end{enumerate}
        可得$\L(X)$是一个Banach代数。
    \end{definition}

    \begin{definition}
        称$A\in\L(X)$可逆是指:存在$B\in \L(X)$使得
        \begin{equation*}
            AB=BA=I
        \end{equation*}
    \end{definition}

    \begin{definition}
        \begin{equation*}
            \sigma(A)\defeq \{ \lambda\in\C:\lambda I-A\mbox{不可逆} \}
        \end{equation*}
        称为A的谱(spectrum),$\sigma(A)$中的元素称为谱点。
        \begin{equation*}
            \rho(A)\defeq \{ \lambda\in\C:\lambda I-A\mbox{可逆} \}=\C \backslash \sigma(A)
        \end{equation*}
        称为$A$的预解集(resolvent set),$\rho(A)$中元素称为正则值。
    \end{definition}

    \begin{definition}
        如果$\lambda\in C$使得${\rm Ker}(\lambda I-A)\neq \{0\}$,
        即
        \begin{equation*}
            \exists 0\neq x\in X{\rm\ s.t.\ }Ax=\lambda x
        \end{equation*}
        则称$\lambda$为$A$的特征值,
        \begin{equation*}
            \sigma_p(A)\defeq \{ A\mbox{的特征值} \}
        \end{equation*}
        称为$A$的点谱。
    \end{definition}

    \begin{example}
        有限维线性空间上的线性映射$A\in \L(\C^n)\Rightarrow \sigma(A)=\sigma_p(A)\neq \varphi$.
    \end{example}

    \begin{example}
        设
        \begin{equation*}
            A:C[0,1]\rightarrow C[0,1],u(t)\mapsto t\cdot u(t)
        \end{equation*}
        $A$有特征值吗?
    \end{example}
    \begin{solve}
        \begin{align*}
            (\lambda I-A)u=0&\Leftrightarrow \lambda u(t)-tu(t)=0,\forall t\in [0,1]\\
            &\Leftrightarrow u(t)=0,\forall t\in [0,1]
        \end{align*}
        无特征值。
    \end{solve}

    \begin{definition}
        对于$\lambda\in \C$,
        满足${\rm Ker}(\lambda I-A)\neq\{0\}$,则有以下分类:
        \begin{enumerate}
            \item ${\rm Ran}(\lambda I-A)\neq X$,${\rm Ran}(\lambda I-A)\mathop{\subset}\limits^{\rm dense}X$,则称$\lambda $为$A$的连续谱点,其全体记为$\sigma_c(A)$,称为$A$的连续谱。
            \item $\overline{{\rm Ran}(\lambda I-A)}\neq X$,则称$\lambda $为$A$的剩余谱点,其全体$\sigma_r(A)$称为$A$的剩余谱。
            \item ${\rm Ran}(\lambda I-A)=X$,此时$\lambda\in\rho(A)$.
        \end{enumerate}
    \end{definition}

    \begin{example}
        设
        \begin{equation*}
            A:C[0,1]\rightarrow C[0,1],u(t)\mapsto t\cdot u(t)
        \end{equation*}
        则$\sigma(A)=\sigma_r(A)=[0,1]$.
    \end{example}
    \begin{proof}
        先证明:$\C\backslash [0,1]\subset \rho(A)$.
        设$\lambda\in \C\backslash [0,1]$,令
        \begin{equation*}
            T:C[0,1]\rightarrow C[0,1],u(t)\mapsto \frac{1}{\lambda-t}u(t)
        \end{equation*}
        于是$(\lambda I-A)T=I=T(\lambda I-A)$,且
        \begin{equation*}
            ||Tu||_\infty \leqslant \left[ \fun{max}{t\in [0,1]}\frac{1}{|\lambda-t|} \right]||u||_\infty
        \end{equation*}
        所以$(\lambda I-A)^{-1}=T\in\L (X)\Rightarrow \lambda\in\rho(A)$.

        再证明$[0,1]\subset \sigma_r(A)$.设$\lambda\in[0,1]$,对于
        $\forall v\in{\rm Ran}(\lambda I-A)$,存在$u\in C[0,1]$满足
        \begin{equation*}
            (\lambda-t)u(t)=v(t),t\in [0,1]
        \end{equation*}
        于是$v(\lambda)=0\Rightarrow 1\notin \overline{ {\rm Ran}(\lambda I-A) }\Rightarrow \overline{ {\rm Ran}(\lambda I-A) }\neq X$.

        最后,$[0,1]\subset\sigma_r(A)\subset\sigma(A)\subset [0,1]\Rightarrow \sigma(A)=\sigma_r(A)=[0,1]$.
    \end{proof}

    \begin{example}
        设
        \begin{equation*}
            A:L^2[0,1]\rightarrow L^2[0,1],u(t)\mapsto t\cdot u(t)
        \end{equation*}
        则$\sigma(A)=\sigma_c(A)=[0,1]$.
    \end{example}
    \begin{proof}
        与上例同理可证:$\C\backslash [0,1]\subset \rho(A)$.
        
        再证明:对于$\forall \lambda\in [0,1]$,${\rm Ran}(\lambda I-A)\neq X$.
        {\rm Claim}:$1\notin (\lambda I-A)$,否则存在$u\in L^2{\rm\ s.t.\ }1=(\lambda-t)u(t),t\in [0,1]$,
        从而
        \begin{equation*}
            \frac{1}{\lambda-t}\in u(t)\in L^2[0,1]
        \end{equation*}
        矛盾。

        最后证明:$\forall \lambda\in [0,1],{\rm Ran}(\lambda I-A)\mathop{\subset}\limits^{\rm dense}X$,
        对于$\forall v\in L^2,\forall \varepsilon>0$,定义
        \begin{equation*}
            u_\varepsilon (t)\defeq \frac{1}{\lambda-t}v(t)\cdot 
            \chi_{ [0,1]\backslash (\lambda-\varepsilon,\lambda+\varepsilon) }(t)
        \end{equation*}
        于是$u_\varepsilon\in L^2$且
        \begin{equation*}
            (\lambda I-A)u_\varepsilon=\chi_{ [0,1]\backslash(\lambda-\varepsilon,\lambda+\varepsilon) }v
            \mathop{\rightarrow}\limits^{L^2} v{\rm\ as\ }\varepsilon\rightarrow0^+
        \end{equation*}
        由积分的绝对连续性,$v\in \overline{ {\rm Ran}(\lambda I-A) }$.
    \end{proof}

\subsection{谱的基本性质}
    \begin{definition}
        算子值函数:
        \begin{equation*}
            R_\lambda(A):\rho(A)\rightarrow\L(X),\lambda\mapsto (\lambda I-A)^{-1}
        \end{equation*}
        称为$A$的预解式(resolvent).
    \end{definition}

    \begin{lemma}
        设$T\in \L(X)$,$||T||\leqslant 1$,则
        \begin{enumerate}
            \item $(I-T)^{-1}\in\L(X)$.
            \item $(I-T)^{-1}=\sum_{n=1}^\infty T^n$.(Von Neumann级数)
            \item $||(I-T)^{-1}||\leqslant (1-||T||)^{-1}$.
        \end{enumerate}
    \end{lemma}
    \begin{proof}
        \begin{enumerate}
            \item 令
                \begin{equation*}
                    S_n\defeq \sum_{k=1}^n T^k
                \end{equation*}
                则
                \begin{equation*}
                    ||S_{n+p}-S_n||=\ms{ \sum_{k=n+1}^{n+p}T^k }
                    \leqslant \sum_{k=n+1}^{n+p}||T||^k<\frac{||T||^{k+1}}{1-||T||}
                \end{equation*}
                $\L(X)$完备$\Rightarrow \exists S\in\L(X)$使得$||S_n-S||\rightarrow 0$.
                Claim:
                \begin{equation*}
                    S(I-T)=(I-T)S=I
                \end{equation*}
                注意到
                \begin{equation*}
                    ||S_n(I-T)-I||=
                    ||I-T^{n+1}-I||\leqslant ||T||^{n+1}\rightarrow 0{\rm\ as\ }n\rightarrow\infty
                \end{equation*}
                \begin{align*}
                    \Rightarrow ||S(I-T)-I||
                    \leqslant& ||S(I-T)-S_n(I-T)||+||S_n(I-T)-I||\\
                    \leqslant& ||S-S_n||\cdot ||I-T||+||S_n(I-T)-I||\rightarrow 0
                \end{align*}
                从而$S(I-T)=I$,同理$(I-T)S=I$.
            \item $||S_n-S||\rightarrow 0\Rightarrow (I-T)^{-1}=\sum_{k=0}^\infty T^n$.
            \item $||S||\leqslant \fun{sup}{n}||S_n||$.
            \end{enumerate}
    \end{proof}
    
    \begin{theorem}
        $\rho(A)$是开集,进而$\sigma(A)$是闭集。
    \end{theorem}
    \begin{proof}
        设$\lambda_0\in \rho(A)$,
        \begin{equation*}
            \lambda I-A=\lambda_0 I-A+(\lambda-\lambda_0)I
            =(\lambda_0 I-A)[ I+(\lambda-\lambda_0)(\lambda_0I-A)^{-1} ]
        \end{equation*}
        由引理3.1.1,当$|\lambda-\lambda_0|<||(\lambda_0I-A)^{-1}||^{-1}$时,
        \begin{equation*}
            B\defeq [ I+(\lambda-\lambda_0)(\lambda_0I-A)^{-1} ]^{-1}\in\L(X)
        \end{equation*}
        于是
        \begin{equation*}
            (\lambda I-A)^{-1}
            =B\cdot R_{\lambda_0}(A)\in\L(X)\Rightarrow \lambda\in\rho(A)
            \Rightarrow \mathbb{D}(\lambda_0,\frac{1}{||(\lambda_0I-A)^{-1}||})\subset \rho(A)
        \end{equation*}
    \end{proof}

    \begin{proposition}
        $A\in \L(X)\Rightarrow \sigma(A)\subset \overline{ \mathbb{D}(0,||A||) }$.
    \end{proposition}
    \begin{proof}
        即证明:$\forall \lambda\in\C{\rm\ with\ }|\lambda|>||A||,(\lambda I-A)^{-1}\in\L(X)$,
        \begin{align*}
            |\lambda|>||A||\Rightarrow& \ms{\frac{A}{\lambda}}<1\\
            \mathop{\Rightarrow}\limits^{\rm Lem3.1.1}&\left(I-\frac{A}{\lambda}\right)^{-1}\in\L(X)\\
            \Leftrightarrow& (\lambda I-A)^{-1}\in\L(X)
        \end{align*}
    \end{proof}

    \begin{corollary}
        $\sigma(A)$是$\C$中紧集。
    \end{corollary}

    \begin{definition}
        $X$是复Banach空间,$\Omega$是$\C$上的开集,称算子值函数
        \begin{equation*}
            T:\Omega\rightarrow\L(X),\lambda\mapsto T_\lambda
        \end{equation*}
        在$\lambda_0\in\Omega$全纯是指:
        存在$\lambda_0$的邻域$U$使得
        \begin{equation*}
            \forall \lambda\in U,\exists S_\lambda\in\L(X){\rm\ s.t.\ }
            \ms{ \frac{T_{\lambda+\delta}-T_\lambda}{\delta}-S_\lambda }\rightarrow 0{\rm\ as\ }\delta\rightarrow 0
        \end{equation*}
    \end{definition}
%Lec26
    \begin{theorem}
        $\lambda\mapsto R_\lambda(A)$是$\rho(A)$上的算子值全纯函数。
    \end{theorem}
    \begin{proof}
        \begin{lemma}[Resolvent Identity]
            \begin{equation*}
                R_\lambda(A)-R_\mu(A)=(\mu-\lambda)R_\lambda(A)R_\mu(A),\forall \lambda,\mu\in\rho(A)
            \end{equation*}
            \begin{proof}
                \begin{align*}
                    R_\lambda(A)&=(\lambda I-A)^{-1}(\mu I-A)(\mu I-A)^{-1}\\
                    &=(\lambda I-A)^{-1} [ \lambda I-A+(\mu-\lambda)I ](\mu I-A)^{-1}\\
                    &=R_\mu(A)+(\mu-\lambda)R_\lambda(A)R_\mu(A)
                \end{align*}        
            \end{proof}
        \end{lemma}

        \textbf{Step1}:连续性。
            对于$\forall \lambda_0\in\rho(A)$,
            \begin{equation*}
                \lambda I-A=(\lambda_0I-A)[ I+(\lambda-\lambda_0)(\lambda_0I-A)^{-1} ]
            \end{equation*}
            当$|\lambda-\lambda_0|<||(\lambda_0I-A)^{-1}||^{-1}$时,
            \begin{equation*}
                R_\lambda(A)=[ I+(\lambda-\lambda_0)R_{\lambda_0}(A) ]^{-1}R_{\lambda_0}(A)
            \end{equation*}
            于是当$|\lambda-\lambda_0|<(2||R_{\lambda_0}(A)||)^{-1}$时,
            \begin{equation*}
                ||R_{\lambda}(A)||\leqslant 
                ||[I+(\lambda-\lambda_0)R_{\lambda_0}(A)]^{-1}||\cdot ||R_{\lambda_0}(A)||
                \leqslant \frac{1}{1-\frac{1}{2}}||R_{\lambda_0}(A)||
                =2||R_{\lambda_0}(A)||
            \end{equation*}
            由引理3.1.2可知,
            \begin{equation*}
                ||R_{\lambda}(A)-R_{\lambda_0}(A)||\leqslant |\lambda-\lambda_0|\cdot
                ||R_{\lambda}(A)||\cdot ||R_{\lambda_0}(A)||
                \leqslant 2||R_{\lambda_0}(A)||^2\cdot |\lambda-\lambda_0|
            \end{equation*}
        \textbf{Step2}:全纯性。
        \begin{align*}
            \ms{ \frac{R_{\lambda}(A)-R_{\lambda_0}(A)}{\lambda-\lambda_0}+R_{\lambda_0}(A)^2}
            \mathop{=}\limits^{\rm R.I.}&
            \ms{ -R_{\lambda}(A)R_{\lambda_0}(A)+R_{\lambda_0}(A)^2 }\\
            \leqslant&||R_{\lambda_0}(A)||\cdot ||R_{\lambda}(A)-R_{\lambda_0}(A)||\rightarrow 0{\rm\ as\ }\lambda\rightarrow\lambda_0
        \end{align*}
    \end{proof}

    \begin{theorem}[Gelfand,谱不空定理]
        $0\neq A\in\L(X)\Rightarrow \sigma(A)\neq \varnothing$.
    \end{theorem}
    \begin{proof}
        假设$\sigma(A)=\varnothing$,则
        $\rho(A)=\C$,说明$\lambda\mapsto R_{\lambda}(A)$时算子值整函数,于是
        \begin{equation*}
            \forall f\in\L(X)^*,
            u_f(\lambda)\defeq f(R_{\lambda}(A)),\lambda\in\C
        \end{equation*}
        是整函数,因为
        \begin{equation*}
            \left| \frac{u_f(\lambda)-u_f(\lambda_0)}{\lambda-\lambda_0}+f(R_{\lambda_0}(A)^2) \right|
            \leqslant ||f||\cdot \ms{\frac{R_{\lambda}(A)-R_{\lambda_0}(A)}{\lambda-\lambda_0}+R_{\lambda_0}(A)^2}
            \rightarrow 0{\rm\ as\ }\lambda\rightarrow\lambda_0
        \end{equation*}
        另一方面,当$|\lambda|>2||A||$时,
        \begin{equation*}
            ||R_{\lambda}(A)||\leqslant \frac{1}{|\lambda|}\frac{1}{1-\ms{\frac{A}{\lambda}}}
            =\frac{1}{|\lambda|-||A||}\leqslant \frac{1}{||A||}
        \end{equation*}
        而$\lambda\mapsto R_\lambda(A)$连续,在$\overline{ \mathbb{D}(0,2||A||) }$上有界,
        于是存在$C>0$使得$||R_{\lambda}(A)||\leqslant C,\forall \lambda\in \C$,
        \begin{equation*}
            |u_f(\lambda)|\leqslant ||f||\cdot ||R_{\lambda}(A)||\leqslant C||f||,\forall \lambda\in\C
        \end{equation*}
        由Liouvillle定理,$u_f$是常函数。从而
        \begin{equation*}
            f(R_{\lambda}(A))=f(R_{\mu}(A)),\forall \lambda,\mu\in\C,\forall f\in\L(X)^*
        \end{equation*}
        由HBT,$R_{\lambda}(A)=R_{\mu}(A)$,这与R.I.矛盾。
    \end{proof}

    \begin{definition}
        对于$A\in\L(X)$,
        \begin{equation*}
            r_\sigma(A)\defeq {\rm sup}\{ |\lambda|:\lambda\in\sigma(A) \}
        \end{equation*}
        称为$A$的谱半径。
    \end{definition}
    \begin{theorem}[Gelfand,谱半径公式]
        \begin{equation*}
            r_\sigma(A)=\fun{lim}{n\rightarrow\infty}||A^n||^{\frac{1}{n}}
        \end{equation*}
    \end{theorem}
    \begin{proof}
        \textbf{Step1}:先证明右式极限存在,令$r\defeq \fun{inf}{n}||A^n||^{\frac{1}{n}}$,
        则
        \begin{equation*}
            \fun{liminf}{n\rightarrow\infty}||A^n||^{\frac{1}{n}}\geqslant r
        \end{equation*}
        另一方面,$\forall \varepsilon>0$,存在$m$使得
        \begin{equation*}
            ||A^m||^{\frac{1}{m}}<r+\varepsilon
        \end{equation*}
        所以对于$\forall n\in \N$,有唯一分解$n=p_nm+q_n{\rm\ with\ }0\leqslant q_n<m$,
        所以
        \begin{equation*}
            ||A^n||^{\frac{1}{n}}\leqslant ||A^{p_nm}||^{\frac{1}{n}}
            ||A^{q_n}||^{\frac{1}{n}}
            \leqslant ||A^m||^{\frac{q_n}{n}}||A||^{\frac{q_n}{n}}
            <(r+\varepsilon)^{\frac{p_nm}{n}}||A||^{\frac{q_n}{n}}
        \end{equation*}
        当$n\rightarrow\infty$时,$\frac{q_n}{n}\rightarrow 0,\frac{p_nm}{n}\rightarrow 1$,所以
        \begin{equation*}
            \fun{limsup}{n\rightarrow\infty}||A^n||^{\frac{1}{n}}\leqslant r+\varepsilon
        \end{equation*}
        令$\varepsilon\rightarrow0 $则
        \begin{equation*}
            \fun{limsup}{n\rightarrow\infty}||A^n||^{\frac{1}{n}}\leqslant r
        \end{equation*}

        \textbf{Step2}:证明$r_\sigma(A)\leqslant \fun{lim}{n\rightarrow\infty}||A^n||^{\frac{1}{n}}$.我们知道
        幂级数
        \begin{equation*}
            \sum_{n=0}^\infty ||A^n||z^n
        \end{equation*}
        的收敛半径为
        \begin{equation*}
            r=\frac{1}{\fun{lim}{n\rightarrow\infty}||A^n||^{\frac{1}{n}}}
        \end{equation*}
        令$z=\frac{1}{\lambda}$,可知当$|\lambda|>\fun{lim}{n\rightarrow\infty}||A^n||^{\frac{1}{n}}$
        时(收敛圆内绝对收敛)
        \begin{equation*}
            \sum_{n=0}^\infty \ms{\frac{A^n}{\lambda^{n+1}}}<\infty
        \end{equation*}
        $\L(X)$完备,根据引理1.4.2,级数
        \begin{equation*}
            \sum_{n=0}^\infty \ms{\frac{A^n}{\lambda^{n+1}}}
        \end{equation*}
        也收敛。另一方面,
        \begin{equation*}
            \ms{ \left( \sum_{n=1}^N \frac{A^n}{\lambda^{n+1}} \right)(\lambda I-A)-I }
            =\ms{ I-\frac{A^{N+1}}{\lambda^{N+1}-I} }\rightarrow 0
        \end{equation*}
        所以
        \begin{equation*}
            \sum_{n=1}^\infty \frac{A^n}{\lambda^{n+1}}=(\lambda I-A)^{-1}=R_{\lambda}(A)
        \end{equation*}
        从而$R_\lambda(A)\in\L(X)\Rightarrow \lambda\in\rho(A)\rightarrow r_\sigma(A)\leqslant \fun{lim}{n\rightarrow\infty}||A^n||^{\frac{1}{n}}$.

        \textbf{Step3}:证明$r_\sigma(A)\geqslant \fun{lim}{n\rightarrow\infty}||A^n||^{\frac{1}{n}}$.
        设$|\lambda|>r_\sigma(A)$,则$\lambda\in\rho(A)\Rightarrow \forall f\in\L(X)^*,f(R_\lambda(A))$在$\lambda$全纯,
        从而$f(R_\lambda(A))$在圆环$|\lambda|>r_\sigma(A)$内全纯,故可展为收敛的Laurent级数。
        另一方面,由Step2,当$|\lambda|>\fun{lim}{n\rightarrow\infty}||A^n||^{\frac{1}{n}}$时,
        \begin{equation*}
            R_\lambda(A)=\sum_{n=0}^\infty \frac{A^n}{\lambda^{n+1}}
            \Rightarrow f(R_\lambda(A))=\sum_{n=0}^\infty \frac{f(A^n)}{\lambda^{n+1}}
        \end{equation*}
        Laurent展式唯一,所以这一展式在$|\lambda|>r_\sigma(A)$上也成立。
        在内部绝对收敛,所以
        \begin{equation*}
            \forall \varepsilon>0,\sum_{n=1}^\infty \frac{ |f(A^n)| }{(r_\sigma(A)+\varepsilon)^{n+1}}<\infty
        \end{equation*}        
        记
        \begin{equation*}
            T_n\defeq \frac{A^n}{(r_\sigma(A)+\varepsilon)^{n+1}}
        \end{equation*}
        收敛级数通项有界,所以
        \begin{equation*}
            \fun{sup}{n}|f(T_n)|<\infty,\forall f\in\L(X)^*
        \end{equation*}
        由UBP,$C\defeq \fun{sup}{n}||T_n||<\infty$,从而
        \begin{equation*}
            ||A^n||\leqslant C( r_\sigma(A)+\varepsilon )^{n+1}
            \Rightarrow 
            \fun{lim}{n\rightarrow\infty}||A^n||^{\frac{1}{n}}
            \leqslant r_\sigma(A)+\varepsilon
        \end{equation*}
        令$\varepsilon\rightarrow0$得证。
    \end{proof}

%Lec27
    $\sigma(A)=\sigma_p(A)\cup\sigma_c(A)\cup\sigma_r(A)$.

    \begin{example}
        右移位算子:
        \begin{equation*}
            A:l^2\rightarrow l^2,(x_1,x_2,\cdots)\mapsto (0,x_1,x_2,\cdots)
        \end{equation*}
        则$\sigma_p(A)=\varnothing$,$\sigma_c(A)=\partial \mathbb{D}$,
        $\sigma_r(A)=\mathbb{D}$.
    \end{example}
    \begin{proof}
        $||A||=1\Rightarrow \sigma(A)=\overline{\mathbb{D}}$,
        先证明:$\sigma_p(A)=\varnothing$,否则$\exists\lambda\in\C,\exists 0\neq x\in \ell^2$使得
        \begin{equation*}
            (0,x_1,x_2,\cdots)=Ax=\lambda x=(\lambda x_1,\lambda x_2,\cdots)
        \end{equation*}
        $\lambda=0\Rightarrow x=0$,$\lambda \neq 0\Rightarrow x_1=0\Rightarrow x_2=0\Rightarrow \cdots\Rightarrow x=0$.矛盾。

        再证明:$\mathbb{D}\subset \sigma_r(A)$,
        设$\lambda\in\mathbb{D}$,Claim:
        $\overline{ {\rm Ran}(\lambda I-A) }\neq \ell^2$.
        这等价于${\rm Ran}(\lambda I-A)^\perp\neq \{0\}$.令
        $z=(1,\overline{\lambda},\overline{\lambda}^2,\cdots)$,则
        \begin{align*}
            \ag{(\lambda I-A)x,z}&=\ag{ (\lambda x_1,\lambda x_2-x_2,\lambda x_3-x_2,\cdots),(1,\overline{\lambda},\overline{\lambda}^2,\cdots) }\\
            &=\lambda x_1+\lambda^2 x_2-\lambda x_1+\lambda^3 x_3-\lambda^2 x_2+\cdots=0\\
            \Rightarrow 0\neq z&\in {\rm Ran}(\lambda I-A)^\perp
        \end{align*}

        然后证明:$\partial \mathbb{D}\subset\sigma_c(A)$.
        设$\lambda\in\partial\mathbb{D}$,
        \begin{enumerate}[$1^\circ$]
            \item 证明${\rm Ran}(\lambda I-A)\neq \ell^2$.
                \begin{align*}
                    {\rm Ran}(\lambda I-A)\ni y=(\lambda I-A)x
                    \Rightarrow& y_1=\lambda x_1,y_k=\lambda x_k-x_{k-1},k\geqslant 2\\
                    \Rightarrow& y_1=\lambda x_1,\lambda^{k-1}y_k=\lambda^k x_k-
                    \lambda^{k-1}x_{k-1},k\geqslant 2\\
                    \Rightarrow& \sum_{k=1}^n \lambda^{k-1}y_j=\lambda^n x_n
                \end{align*}
                假设${\rm Ran}(\lambda I-A)=\ell^2$,令$y=e_1$,
                \begin{align*}
                    \exists x\in\ell^2{\rm\ s.t.\ }e_1=(\lambda I-A)x
                    \Rightarrow& \lambda^n x_n=1,n=1,2,\cdots\\
                    \Rightarrow& x=( \frac{1}{\lambda},\frac{1}{\lambda^2},\cdots )
                \end{align*}
                $|\lambda|=1$,不收敛,这与$x\in \ell^2$矛盾。
            \item 证明$\overline{ {\rm Ran}(\lambda I-A) }=\ell^2$.只需证明
                ${\rm Ran}(\lambda I-A)^\perp=\{0\}$.对于$\forall x\in {\rm Ran}(\lambda I-A)^\perp$,
                \begin{equation*}
                    0=\ag{z,(\lambda I-A)e_n}=\overline{\lambda}z_n-z_{n+1},\forall n
                \end{equation*}
                所以$z_{n+1}=\overline{\lambda}z_n\Rightarrow |z_{n+1}|=|z_n|\Rightarrow z=0$.
        \end{enumerate}

        最后:$\overline{\mathbb{D}}\subset \sigma_c(A)\cup \sigma_r(A)\subset \sigma(A)\subset \overline{\mathbb{D}}$,
        由前面结论可得$\sigma_c(A)=\partial \mathbb{D}$,$\sigma_r(A)=\mathbb{D}$.
    \end{proof}


\section{紧算子的谱}
    
\subsection{紧算子}
\begin{definition}
    $X,Y$是Banach空间,$A\in \L(X,Y)$,
    \begin{enumerate}
        \item 如果$A$把每个有界集映为列紧集,称$A$紧,记作$A\in \mathcal{T}(X,Y)$.
        \item 如果$A$把$X$中每个弱收敛序列映为$Y$中强收敛序列,称$A$全连续。
        \item 如果${\rm dim}( {\rm Ran}(A) )<\infty$,则称$A$是有限秩算子,记作$A\in\mathcal{F}(X,Y)$.
    \end{enumerate}
\end{definition}

\begin{proposition}
    $\mathcal{F}(X,Y)\subset \mathcal{T}(X,Y)$,且是闭子空间。
\end{proposition}
\begin{proof}
    有限维线性空间里的有界集列紧。
\end{proof}

\begin{example}
    $I\in \mathcal{T}(X)\Leftrightarrow {\rm dim}(X)<\infty$.
\end{example}

\begin{example}
    设$K(\cdot,\cdot)$在$[a,b]^2$上连续,
    \begin{equation*}
        (Tu)(s)\defeq \int_a^b K(s,t)u(t)\d t
    \end{equation*}
    则$T:C[a,b]\rightarrow C[a,b]$紧。
\end{example}
\begin{proof}
    设$\mathcal{F}\subset C[a,b]$有界,
    记$M=\fun{sup}{u\in\mathcal{F}}||u||$,则
    \begin{equation*}
        ||Tu||\leqslant ||T||\cdot M,\forall u\in \mathcal{F}
    \end{equation*}
    所以$T(\mathcal{F})$一致有界。

    同时,$\forall \varepsilon>0,\forall u\in\mathcal{F}$,
    $K(\cdot,\cdot)$一致连续$\Rightarrow \exists \delta>0{\rm\ s.t.}$
    \begin{equation*}
        |K(s',t)-K(s'',t)|<\frac{\varepsilon}{M(b-a)},\forall s',s''\in [a,b]{\rm\ with\ }|s'-s''|<\delta,\forall t\in [a,b]
    \end{equation*}
    所以
    \begin{equation*}
        |(Tu)(s')-(Tu)(s'')|\leqslant \int_a^b 
        | K(s',t)-K(s'',t) ||u(t)|\d t
        <\varepsilon,\forall s',s''{\rm\ with\ }|s'-s''|<\delta,\forall u\in\mathcal{F}
    \end{equation*}
    所以$T(\mathcal{F})$等度连续,进而列紧。
\end{proof}

\begin{proposition}
    $\mathcal{T}(X,Y)\subset \mathcal{L}(X,Y)$,且是闭子空间。
\end{proposition}
\begin{proof}
    设$A_n\in\mathcal{T}(X,Y)$使得$||A_n-A||\rightarrow 0$,下证$A$紧。
    设$M\subset X$有界,$C\defeq \fun{sup}{n}||x||<\infty$,Claim:$A(M)$列紧。
    对于$\forall \varepsilon$,取$N$充分大使得
    \begin{equation*}
        ||A_N-A||<\frac{\varepsilon}{3C}
    \end{equation*}
    $A_N(M)$列紧,
    \begin{equation*}
        \exists x_1,\cdots,x_m\in M{\rm\ s.t.\ }
        A_N(M)\subset \bigcup_{k=1}^m B(A_N x_k,\frac{\varepsilon}{3})
    \end{equation*}
    所以$\forall x\in M,\exists k\in \{1,2,\cdots,m\}{\rm\ s.t.\ }$
    \begin{equation*}
        ||A_Nx-A_Nx_{k}||<\frac{\varepsilon}{3}
    \end{equation*}
    从而
    \begin{equation*}
        ||Ax-Ax_k||\leqslant 
        ||Ax-A_Nx||+||A_Nx-A_Nx_k||+||A_Nx_k-Ax_k||<\varepsilon
    \end{equation*}
    于是$\{ Ax_1,\cdots,Ax_m \}$是$A(M)$的有穷$\varepsilon$网。
\end{proof}

\begin{proposition}
    紧算子的值域可分。
\end{proposition}
\begin{proof}
    \begin{equation*}
        {\rm Ran}(A)=\bigcup_{k=1}^\infty A(B(0,n))
    \end{equation*}
    列紧$\Rightarrow $可分,设$M_n$是$A(B(0,n))$的可数稠密子集,取$M_n$的并即为
    ${\rm Ran}(A)$的可数稠密子集。
\end{proof}

\begin{proposition}
    紧算子与有界算子的(两种)复合是紧算子。
\end{proposition}
\begin{proof}
    设$T$有界,$A$紧,
    若$\{x_n\}_{n=1}^\infty\subset X$有界,$A$紧所以
    $\{ Ax_n \}_{n=1}^\infty$列紧,有子列$\{Ax_{n_k}\}_{k=1}^\infty$收敛,
    $T$有界所以
    $\{ TAx_{n_k} \}_{k=1}^\infty$收敛,因此$TA$是紧算子。

    若$M$有界,则$T(M)$有界,$A$紧所以$AT(M)$列紧,$AT$是紧算子。
\end{proof}

\begin{theorem}
    对于$A\in\L(X,Y)$,
    \begin{enumerate}
        \item 紧$\Rightarrow$全连续;
        \item 如果$X$自反,则$A$紧$\Leftrightarrow A$全连续。
    \end{enumerate}
\end{theorem}
\begin{proof}
    对于1,假设$A$紧而不全连续,即存在$x_n\wto x_0$但$||Ax_n-Ax_0||\nrightarrow 0$:
    存在$\varepsilon_0>0$,存在子列$\{x_{n_k}\}_{k=1}^\infty$使得$||Ax_{n_k}-Ax_0||\geqslant \varepsilon_0$.

    那么$x_{n_k}\wto x_0$,由UBP知$\{ x_{n_k} \}_{k=1}^\infty$有界,
    $A$紧所以$\{ Ax_{n_k} \}_{k=1}^\infty$有收敛子列,
    不妨设$Ax_{n_k}\rightarrow y$.

    另一方面,$\forall f\in Y^*$,
    \begin{equation*}
        f(Ax_{n_k}-Ax_0)=(A^*f)(x_{n_k}-x_0)\rightarrow 0
    \end{equation*}
    则$Ax_{n_k}\wto Ax_0\Rightarrow Ax_0=y\Rightarrow ||Ax_{n_k}-Ax_0||\rightarrow 0$,矛盾。

    对于2,设$\{x_n\}$有界,由定理2.8.18(Eberlein‑Smulian),
    $X$自反$\Rightarrow $有子列$x_{n_k}\wto x_0$,
    $A$全连续$\Rightarrow ||Ax_{n_k}-Ax_0||\rightarrow 0$.
\end{proof}
%Lec28

\subsection{Riesz-Fredholm定理}

\begin{definition}
    对于$\mathcal{F}\subset X^*$,
    \begin{equation*}
        \mathcal{F}^\perp\defeq \{ x\in X:f(x)=0,\forall f\in \mathcal{F} \}
    \end{equation*}
    称之为$\mathcal{F}$在$X$中的零化子。
\end{definition}

\begin{theorem}[Riesz-Fredholm]
    设$A\in\mathcal{T}(X)$,$T=I-A$,则
    \begin{enumerate}
        \item ${\rm dim}({\rm Ker}(T))<\infty$.
        \item ${\rm Ran}(T)$闭。
        \item Fredholm Alternative,二择一律:$T$单$\Leftrightarrow T$满。
        \item ${\rm Ran}(T)={\rm Ker}(T^*)$.
        \item ${\rm dim}({\rm Ker}(T))={\rm dim}({\rm Ker}(T^*))$.
    \end{enumerate}
\end{theorem}
\begin{proof}
    1.记$M={\rm Ker}(T)$,$S_M$为$M$中的单位球面,
            $S_X$为$X$中的单位球面,于是
            \begin{equation*}
                x\in S_M\Leftrightarrow x\in S_X,(I-A)x\in 0
                \Leftrightarrow x\in S_X,x=Ax\in A(S_X)
            \end{equation*}
            所以$S_M\subset A(S_X)$,后者列紧,
            从而$S_M$列紧,因此${\rm dim}(M)<\infty$.
\end{proof}
\begin{proof}
    2.设${\rm Ran}(T)\ni y_n\rightarrow y$,
    其中$y_n=Tx_n=x_n-Ax_n$.

    \textbf{Case 1}:$\{x_n\}_{n=1}^\infty$有界,
    $A$紧$\Rightarrow \{Ax_n\}_{n=1}^\infty$有收敛子列
    $\{Ax_{n_k}\}_{k=1}^\infty$,设$Ax_{n_k}\rightarrow u$,
    \begin{align*}
        x_{n_k}=y_{n_k}+Ax_{n_k}\rightarrow y+u\Rightarrow& 
        y_{n_k}=Tx_{n_k}\rightarrow T(y+u)\\
        \Rightarrow&y=T(y+u)\in {\rm Ran}(T)
    \end{align*}

    \textbf{Case 2}:$\{x_n\}_{n=1}^\infty$无界,
    令$d_n\defeq {\rm dist}(x_n,{\rm Ker}(T))$,
    存在$z_n\in {\rm Ker}(T)$满足$||x_n-z_n||=d_n$,Claim:
    $\{x_n-z_n\}_{n=1}^\infty$有界。假设不然,
    不妨$d_n\rightarrow +\infty$,令
    \begin{equation*}
        v_n\defeq \frac{x_n-z_n}{||x_n-z_n||}
    \end{equation*}
    于是
    \begin{equation*}
        Tv_n=\frac{Tx_n-Tz_n}{d_n}=\frac{y_n}{d_n}\rightarrow 0
    \end{equation*}
    由于$||v_n||=1$,所以$\{Av_n\}_{n=1}^\infty$有收敛子列,
    设$Ax_{n_k}\rightarrow w$,则$v_{n_k}=Av_{n_k}+Tv_{n_k}\rightarrow w
    $,而$Tv_{n_k}\rightarrow 0$,所以$Tw=0$,$w\in {\rm Ker}(T)$,
    \begin{equation*}
        ||v_n-z||=\frac{1}{d_n}||x_n-(z_n+d_n z)||\geqslant \frac{d_n}{d_n}=1
    \end{equation*}
    这与$v_{n_k}\rightarrow w\in {\rm Ker}(T)$矛盾。
    从而$\{x_n-z_n\}$有界且$T(x_n-z_n)=Tx_n=y_n$,约化为Case1.
\end{proof}
\begin{proof}
    3.
    \begin{lemma}
        \begin{enumerate}[(1)]
            \item ${\rm Ker}(T)\subset {\rm Ker}(T^2)\subset \cdots$
            \item $\exists n{\rm\ s.t.\ }{\rm Ker}(T^n)={\rm Ker}(T^{n+1})$.
        \end{enumerate}
        \begin{proof}
            (1)显然,只说明(2):
            假设不成立,即$\forall n,{\rm Ker}(T^{n})\subsetneqq {\rm Ker}(T^{n+1})$,
            由Riesz阴历,存在$x_n\in {\rm Ker}(T^{n+1}),||x_n||=1{\rm\ s.t.\ }$
            ${\rm dist}(x_n,{\rm Ker}(T^n))>\frac{1}{2}$.

            对于$\forall n,m$,不妨设$n>m$,
            \begin{equation*}
                T^n(Tx_n+Ax_m)=T^{n+1}x_n+T^nAx_m=A(T^nx_m)=0
            \end{equation*}
            所以$Tx_n+Ax_m\in Tx_n+Ax_m(T^n)$,从而
            \begin{equation*}
                ||Ax_n-Ax_m||=||x_n-(Tx_n+Ax_m)||>\frac{1}{2}
            \end{equation*}
            这说明$\{Ax_n\}$无收敛子列,与$A$紧矛盾。
        \end{proof}
    \end{lemma}

    假设$T$满但不单,也就是${\rm Ker}(T)\neq \{0\}$,
    取$0\neq x_0\in {\rm Ker}(T)$,因为$T$是满射,存在$Tx_1=x_0$、
    $Tx_2=x_1\cdots$
    \begin{equation*}
        0\neq x_0=Tx_1=T^2x_2=\cdots
        \Rightarrow T^n x_n\neq 0,T^{n+1}x_n=0
    \end{equation*}
    从而$x_n\in {\rm Ker}(T^{n+1})\backslash {\rm Ker}(T^n)$.这与引理矛盾。

    假设$T$单而不满,令$X_1=T(X)={\rm Ran}(T)$,
    $X_1$是$X$的闭真子空间,取$X_2=T(X_1)$,则$X_2$为$X_1$的
    闭真子空间,
    否则$T(X_1)=X_1$,取$x_0\in X\backslash X_1$,
    \begin{equation*}
        Tx_0\in T(X)=X_1=T(X_1)\Rightarrow Tx_0'=Tx_0
    \end{equation*}
    与$T$单射矛盾,因此可以取出一系列$X_n=T^n(X)$满足
    $X_{n+1}$是$X_n$的真闭子空间,由Riesz,
    \begin{equation*}
        \exists x_n\in X_n,||x_n||=1{\rm\ s.t.\ }
        {\rm dist}(x_n,X_{n+1})>\frac{1}{2}
    \end{equation*}
    那么对于$\forall n,m$,不妨$n>m$,
    \begin{equation*}
        Ax_m-Ax_n=-(x_m-Ax_m)+(x_n-Ax_n)+x_m-x_n
        =x_m-(x_n+Tx_n-Tx_n)\in X_{m+1}
    \end{equation*}
    于是
    \begin{equation*}
        ||Ax_m-Ax_n||\geqslant {\rm dist}(x_m,X_{m+1})>\frac{1}{2}
    \end{equation*}
    从而$\{Ax_n\}$无收敛子列,与$A$紧矛盾。
\end{proof}
\subsection{Riesz-Schauder定理}
\begin{theorem}[Riesz-Schauder]
    设$A\in\mathcal{T}(X)$,则
    \begin{enumerate}
        \item 如果${\rm dim}(X)=\infty$,则$0\in\sigma(A)$.
        \item $\sigma(A)\backslash \{0\}=\sigma_p(A)\backslash \{0\}$.
        \item 非零特征值的特征子空间一定是有限维的。
        \item 不同特征值的特征向量线性无关。
        \item $0$是$\sigma(A)$的唯一可能的极限点。
    \end{enumerate}
\end{theorem}
\begin{proof}
    1.假设$0\in\rho(A)$,则$A^{-1}\in\L(X)$,$I=A^{-1}\circ A$是有界算子和紧算子的复合,也是紧算子,从而${\rm dim}(X)<\infty$.
\end{proof}
\begin{proof}
    2.只需证明:
    \begin{equation*}
        \forall \lambda\notin \sigma_p(A),\lambda\neq 0\Rightarrow \lambda\in\rho(A)
    \end{equation*}
    实际上,$\lambda\notin \sigma_p(A)\Rightarrow \lambda I-A$单
    ,由F.A.$\Rightarrow \lambda I-A$是双射,
    由IMT$\Rightarrow (\lambda I-A)^{-1}\in\L(X)$.
\end{proof}
\begin{proof}
    3.对于$\forall 0\neq \lambda \in\sigma_p(A)$,
    \begin{equation*}
        {\rm Ker}(\lambda I-A)={\rm Ker}(I-\frac{1}{\lambda}A)
        \mathop{\Rightarrow }\limits^{\rm Riesz-Fredholm} {\rm dim}(\lambda I-A)<\infty
    \end{equation*}
\end{proof}
\begin{proof}
    5.假设$\sigma(A)$有极限点$\lambda_0\neq 0$,则
    存在$\lambda_n\in\sigma(A)$使得$\lambda_n\rightarrow\lambda_0$.
    不妨设$\{\lambda_n\}$互不相同,$\lambda_0\Rightarrow n$充分大时
    $\lambda_n\neq 0$,故不妨设所有$\lambda_n\neq 0$.
    \begin{equation*}
        \frac{1}{\lambda_n}\rightarrow \frac{1}{\lambda_0}\Rightarrow
        \fun{sup}{n}\left|\frac{1}{\lambda_n}\right|<\infty
    \end{equation*}
    取$x_n\in{\rm Ker}(\lambda_n I-A)$,由4.可得
    $\{x_n\}_{n=1}^\infty$线性无关,令$X_n\defeq {\rm span}\{x_1,\cdots,x_n\}$,则
    $X_n$是$X_{n+1}$的真闭子空间,由Riesz引理$\Rightarrow \exists y_n\in X_n,||y_n||=1$使得
    \begin{equation*}
        {\rm dist}(y_n,X_{n-1})>\frac{1}{2}
    \end{equation*}
    而
    \begin{equation*}
        y_n=\sum_{k=1}^n \alpha_k x_k\Rightarrow
        (\lambda_n I-A)y_n=\sum_{k=1}^n \alpha_k(\lambda_n-\lambda_k) x_k
    \end{equation*}
    对于$\forall n,m$,不妨$n>m$,
    \begin{equation*}
        \ms{A\left(\frac{y_n}{\lambda_n}\right)-A\left(\frac{y_m}{\lambda_m}\right)}
        =\ms{ y_n-\mathop{\left[ y_n-A\left(\frac{y_n}{\lambda_n}\right)\right]}\limits_{\in X_{n-1}}-
        \mathop{\left[A\left(\frac{y_m}{\lambda_m}\right)\right]}\limits_{\in X_{m}\subset X_{n-1}} }
        \geqslant {\rm dist}(y_n,X_{n-1})>\frac{1}{2}
    \end{equation*}
    另一方面,$\{ 
        \frac{y_n}{\lambda_n}
     \}$是有界集,从而$\{A\left(\frac{y_n}{\lambda_n}\right)\}$有收敛子列,矛盾。
\end{proof}

\begin{corollary}
    $A\in\mathcal{T}(X)\Rightarrow \sigma(A)$至多可数。
\end{corollary}
\begin{proof}
    令
    \begin{equation*}
        E_k\defeq \sigma_p(A)\cap \{ \lambda\in\C:|\lambda|>\frac{1}{k} \}
    \end{equation*}
    则
    \begin{equation*}
        \sigma(A)\backslash
        \{0\}=\sigma_p(A)\backslash\{0\}=\bigcup_{k=1}^\infty E_k
    \end{equation*}
    只需证明$E_k$元素个数有限。假设不然,由B-W$\Rightarrow E_k$有极限点,
    而且${\rm dist}(0,E_k)\geqslant \frac{1}{k}$,从而$\lambda_0\neq 0$,
    但$\sigma(A)$只可能以$0$作为极限点,矛盾。
\end{proof}

\begin{corollary}
    如果$X$无穷维,$A\in\mathcal{T}(X)$,则只有三种情形:
    \begin{enumerate}[(1).]
        \item $\sigma(A)=\{0\}$,例如取$A=0$.
        \item $\sigma(A)=\{0,\lambda_1,\cdots,\lambda_n\}$.
        \item $\sigma(A)=\{0,\lambda_1,\lambda_2,\cdots\}$且$\lambda_n\rightarrow 0$.
    \end{enumerate}
    这里面的$\lambda_i\in \sigma_p(A)$.
\end{corollary}
\begin{proof}
    令
    \begin{align*}
        F_0\defeq& \sigma(A)\cap\{\lambda\in\C:|\lambda|\geqslant 1\}\\
        F_k\defeq& \sigma(A)\cap\{\lambda\in\C:\frac{1}{k+1}\leqslant |\lambda|<\frac{1}{k}\},k=1,2,\cdots
    \end{align*}
    则$\sigma(A)\subset \bigcup_{k=1}^\infty F_k$,且$F_k$中元素个数有限,
    则按照$F_k$顺次排列$\lambda_1,\lambda_2$即可。
\end{proof}

\begin{example}
    给定$\lambda_1,\cdots,\lambda_n\in \C$,令
    \begin{equation*}
        A_n:\ell^2\rightarrow\ell^2,(x_1,x_2,\cdots)\mapsto 
        (\lambda_1x_1,\lambda_2x_2,\cdots,\lambda_nx_n,0,\cdots)
    \end{equation*}
    则$A_n$是有界有限秩算子,从而是紧算子,且
    \begin{equation*}
        \{ 0,\lambda_1,\cdots,\lambda_n \}\subset \sigma_p(A)
    \end{equation*}
    而$\forall \lambda\in\C\backslash \{ 0,\lambda_1,\cdots,\lambda_n \}$有:
    \begin{align*}
        (\lambda I-A_n)x=0\Leftrightarrow& 
        ( (\lambda-\lambda_1)x_1,\cdots,(\lambda-\lambda_n)x_n,\lambda x_{n+1},\cdots )=0\\
        \Leftrightarrow& x=0\\
        \Rightarrow& \lambda I-A_n\mbox{单}\\
        \Rightarrow& \lambda I-A_n\mbox{是双射}\\
        \Rightarrow& \lambda\in\rho(A_n)\Rightarrow \sigma(A_n)=\sigma_p(A_n)=\{0,\lambda_1,\cdots,\lambda_n\}
    \end{align*}
\end{example}

\begin{example}
    设$\{\lambda_n\}_{n=1}^\infty \subset\C\backslash \{0\}$,$\lambda_n\rightarrow 0$,
    \begin{equation*}
        A:\ell^2\rightarrow \ell^2,(x_1,x_2,\cdots)\mapsto 
        (\lambda_1x_1,\lambda_2x_2,\cdots)
    \end{equation*}
    则根据收敛列有界,$A$是有界算子。同时
    \begin{equation*}
        ||A-A_n||={\rm sup}{||x||_2=1}
        \left( \sum_{k=n+1}^\infty |\lambda_k|^2|x_k|^2 \right)^\frac{1}{2}\rightarrow 0
    \end{equation*}
    $A_n$是紧算子$\Rightarrow A$是紧算子(经典方法)。

    考虑$Ae_k=\lambda_k e_k$,$\{ \lambda_k \}\subset \sigma_p(A)$,
    而且$0\notin\sigma_p(A)$.
    \begin{equation*}
        \forall \lambda\in\backslash \{0,\lambda_1,\lambda_2,\cdots\}\Rightarrow \fun{inf}{k}|\lambda-\lambda_k|>0
    \end{equation*}
    令
    \begin{equation*}
        T:\ell^2\rightarrow\ell^2,(x_1,x_2,\cdots)\mapsto 
        ( \frac{x_1}{\lambda-\lambda_1},\frac{x_2}{\lambda-\lambda_2},\cdots )
    \end{equation*}
    则$T=(\lambda I-A)^{-1}$,且
    \begin{equation*}
        ||Tx||_2\leqslant ||x||_2\cdot \fun{sup}{k}\frac{1}{|\lambda-\lambda_k|}
    \end{equation*}
    于是$T\in\L(X)$,从而$\lambda\in\rho(A)$,这说明
    $\sigma(A)=\{0,\lambda_1,\lambda_2,\cdots\}$.
\end{example}